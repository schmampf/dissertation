
\chapter{Appendix}

% Reference \cite{Schertel2019}
% Reference \cite{Schertel2020}

\begin{figure}
    \centering
    \import{theory/plot test}{test.pgf}
    \caption{Complex susceptibility}
    \label{fig:test_plot}
\end{figure}

Figure \ref{fig:test_plot}


\begin{figure}
    \centering
    \import{theory/inkscape test}{test.pdf_tex}
    \caption{blablabla.}
    \label{fig:test_inkscape}
\end{figure}

Figure \ref{fig:test_inkscape}


\begin{program}
    \caption{Listing Caption is above.}
    \label{program:test}
    \lstinputlisting[language=Python]{appendix/test.py}
\end{program}

Program \ref{program:test}


\chapter{Theory}
In this Chapter in order to understand the ferromagnetic resonance (FMR) technique I want to recall some basic concepts of magneto dynamics. Furthermore, I also want to give a short excursion into the coherence lengths of singlet and triplet Cooper pairs, since they are needed to define my samples dimensions reasonably. Finally, I want to present you the current state of the art, in the field of FMR measurements on thin ferromagnetic films.

%%%%%%%%%%%%%%%%%%%%%%%%%%%%%%%%%%%%%%%%%%%%%%%%%%%%%%%%%%%%%%%%%%%%%%%%%%%%%%%%%%%%%%%%%%%%%
\section{Landau-Lifshitz-Gilbert Model}
The dynamic magnetic properties of a ferromagnet can be described by damped precessional motion of the exchange-coupled magnetic moments.
If the externally applied magnetic fields and excitation energies are significantly smaller than the exchange coupling energy, we can describe the whole ferromagnet as a macroscopic magnetic spin by the Landau-Lifshitz-Gilbert (LLG) model. 

The LLG differential equation for the magnetization $\vec{M}$ is given by
\begin{align}
\frac{\text{d}\vec{M}}{\text{d}t}=\underbrace{-\left|\gamma\right| \left(\vec{M}\times \mu_0 \vec{H}_\text{eff}\right)}_{(1)}
+\underbrace{\frac{\alpha}{\left|\vec{M}\right|}\left( \vec{M}\times \frac{\text{d}\vec{M}}{\text{d}t}\right)}_{(2)}\,,\label{formula:LLG}
\end{align}
where $\gamma$ is the gyro-magnetic ratio, $\alpha$ is the Gilbert damping parameter and $\mu_0$ is the vacuum permeability. The effective magnetic field is given by $\vec{H}_\text{eff}=\vec{H}+\vec{H}_\text{ani}$, with $\vec{H}$ being the externally applied magnetic field and $\vec{H}_\text{ani}$ the anisotropy field. 

For better understanding of the LLG model, we first neglect the Gilbert damping term (2) in equation \ref{formula:LLG}. With this neglection the LLG equation simply describes the precession of the magnetization $\vec{M}$ around the effective field $\vec{H}_\text{eff}$. Here the gyro-magnetic ratio is given by $\gamma_\text{e}=\frac{g_\text{e}\mu_\text{B}}{\hbar}=\frac{g_\text{e}q_\text{e}}{2m_\text{e}}$, where $\hbar$ is the reduced Planck constant, $\mu_\text{B}$ the Bohr magneton, $g_\text{e}$, $q_\text{e}$ and $m_\text{e}$ describe the $g$-factor, charge and mass of the free electron respectively. Since the magnetism in ferromagnets, especially in Co, is caused by the electron spins, we can assume a $g$-factor $g\approx 2$ and thus a gyromgnetic ratio of $\frac{\gamma}{2\pi}=28.025$\,GHzT$^{-1}$.
The precession frequency $\omega_\text{res}$ is given by
\begin{align}
    \omega_\text{res}=\gamma\mu_0|\vec{H}_\text{eff}|\,,
\end{align}
resulting in typical resonance frequencies $\omega_\text{res}/2\pi$ in the GHz range, for a few Tesla of applied external magnetic field.

Now we consider the phenomenological Gilbert term (2) from equation \ref{formula:LLG}. After the displacement of $\vec{M}$, it precesses in a spiral trajectory back to its equilibrium position. This relaxation is due to the scattering of phonons and magnons and can be characterised by the relaxation rate $\kappa$. This rate is related to the frequency linewidth by $\Delta\omega=\frac{1}{2\kappa}$, where the linewidth $\Delta\omega$ is usually defined as full width at half maximum (FWHM) and $\kappa$ as the inverse of the half width at half maximum (HWHM).
Now the linewidth $\Delta\omega$ is  connected to the Gilbert parameter $\alpha$ by
\begin{align}
    \Delta\omega=2\alpha\omega_\text{res}+\Delta\omega_0\,.
\end{align}
The inhomogeneous broadening $\Delta\omega_0$ is caused by magnetic inhomogeneities or surface effects.

In order to solve the LLG differential equation \ref{formula:LLG} we will neglect for now the field anisotropy $\vec{H}_\text{ani}=0$. To this end we split the applied magnetic field $\vec{H}$ and magnetization $\vec{M}$ into static ($\vec{H}_0, \vec{M}_0$) and dynamic ($\vec{h}, \vec{m}$) components.
\begin{align}
    \vec{H}&=\vec{H}_0+\vec{h}(t)\label{formula:LLG_field}\\
    \vec{M}&=\vec{M}_0+\vec{m}(t)
\end{align}
Next we assume a static magnetic field in $z$-direction and a dynamic field in $x$- and $y$-direction $\vec{H}=\left(h_x(t),h_y(t),H_0\right)$. If the dynamic magnetic field is much smaller than the static magnetic field, we can also write for the magnetization $\vec{M}=\left(m_x(t),m_y(t),M_0\right)$. Here is $M_0$ the absolute static magnetization. The solution is given by $\vec{m}=\boldsymbol{\chi}\vec{h}$. The two-dimensional Polder tensor $\boldsymbol{\chi}$ is given by
\begin{align}
    \boldsymbol{\chi}=
    \left(
    \begin{array}{cc}
        \chi_{11} & i\chi_{12} \\
        -i\chi_{12} & \chi_{22}
    \end{array}
    \right)\,.
\end{align}

Since we neglect magnetic field anisotropy, the diagonal elements are the same $\chi=\chi_{11}=\chi_{22}$. Further, we neglect every higher damping therm $\mathcal{O}(\alpha^2)$, in order to get the linear response of a ferromagnet in an external field. Finally, the Polder susceptibility is then given by
\begin{align}
    \chi(\omega,H_0)=\frac{\omega_M\left(\gamma\mu_0H_0-i\Delta\omega\right)}
    {\left( \omega_\text{res}(H_0)\right)^2-\omega^2-i\omega\Delta\omega}\,. \label{eq:chi}
\end{align}
Here is the magnetization frequency  $\omega_M=\gamma\mu_0 M_0$ and the resonance frequency $\omega_\text{res}$. In Figure \ref{fig:theo_chi}, you can see the qualitative behaviour of the Polder susceptibility $\chi(\omega)$ around the resonance frequency $\omega_\text{res}$.
\section{Kittel Formula}\label{sec:theo_Kittel}
Most macroscopic ferromagnets have two opposite favorable directions in which they are most easily magnetized. The so-called easy-axis, which is parallel to the two directions, can be of different origin. In the following simplest case the sample shape anisotropy is treated. Here the magnetic field $\vec{H}$ can be written as follows:
\begin{align}
    \vec{H}=\vec{H}_0+\vec{H}_\text{demag}+\vec{h}(t)\,.
\end{align}

The demagnetization field is given by $\vec{H}_\text{demag}=\vec{N}\cdot\vec{M}$, where the spatially independent demagnetization tensor $\vec{N}$ is in diagonal form, with elements $N_{x,y,z}$.
The resonance frequency can be written as
\begin{align}
    \omega_\text{res}=\gamma\mu_0\sqrt{\left(H_0+(N_y-N_z)M_0\right)\left(H_0+(N_x-N_z)M_0\right)}\,,\label{formula:kittel}
\end{align}
if the applied magnetic field is in the direction of the $z$-axis.

These demagnetisation factors $N_{x,y,z}$ are strongly dependent on the sample geometry. The three most common geometries are discussed below. 
\begin{enumerate}
    \item spherical geometry ($N_{x,y,z}=1/3$).
    \begin{align}
        \omega_\text{res}^\circ=\gamma\mu_0H_0\label{formula:kittel1}
    \end{align}
    \item thin out-of-plane magnetized film ($N_{x,y}=0, N_z=1$).
    \begin{align}
        \omega_\text{res}^\perp=\gamma\mu_0(H_0-M_0)\label{formula:kittel2}
    \end{align}
    \item thin in-plane magnetized film ($N_{x,z}=0, N_y=1$).
    \begin{align}
        \omega_\text{res}^\parallel=\gamma\mu_0\sqrt{H_0(H_0+M_0)}\label{formula:kittel3}
    \end{align}
\end{enumerate}
The ferromagnetic resonance frequencies are simulated in Figure \ref{fig:theo_wres}. 

If there are no other anisotropies than shape anisotropy, $M_0$ is replaced by the saturation magnetization $M_\text{s}$.

If uniaxial field anisotropy is present, $H_0$ is replaced by $H_\text{eff}=H_0+H_\text{ani}$. For thin in-plane magnetized films, the uniaxial field anisotropy is typically on the order of a few mT. 

At this point, it should be noted that we do not know with certainty whether the easy-axis in thin Co films is in-plane, because out-of-plane easy-axis is also possible. With a suitable substrate, such as Au or Pt, single-digit monolayers of Co, and a suitable cap, such as Au or Ag, out-of-plane magnetization of Co may be present. In addition, there are studies of Co films as thin as $40\,$nm that both at normal and oblique incidence of atomic current during electron beam evaporation Co always exhibit in-plane easy-axis.
At this point, I suspect that Co exhibits robust in-plane magnetization and only shows out-of-plane magnetization under very special conditions.

Furthermore, it should be considered that superconductors, such as aluminum, have a critical magnetic field of only a few mT. Only for very thin films, in the single-digit nanometer range, aluminum can achieve an in-plane critical field in the Tesla range. To use a finite ferromagnetic resonant frequency, within the critical magnetic field, an in-plane geometry must be used. 

%%%%%%%%%%%%%%%%%%%%%%%%%%%%%%%%%%%%%%%%%%%%%%%%%%%%%%%%%%%%%%%%%%%%%%%%%%%%%%%%%%%%%%%%%%%%%
\section{Co-planar Waveguide}
In order to generate an alternating magnetic field distribution $\vec{h}$ I use a co-planar wave guide (CPW). The CPW consists of an inner conductor, with width $w$, an infinitesimal height and two neighboring ground pads. For further descriptions I use the laboratory reference frame, shown in Figure \ref{fig:CPW_schematic}, with $x$ perpendicular to the CPW plane, $z$ in direction of the inner conductor and $y$ being perpendicular to $x$ and $z$.