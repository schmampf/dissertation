% !TEX root = ../thesis.tex

\chapter{Appendix}
\label{chapter:appendix}


    \section{washboard potential analogy ???}
        This expression can be interpreted as the equation of motion of a fictitious particle with an effective mass $m_\mathrm{eff} = \hbar C / 2e$, moving in a tilted washboard potential
        \begin{equation}
            U(\phi) = -E_\mathrm{J}\left(\cos\phi + \frac{I}{I_\mathrm{C}}\phi\right)\,,
            \label{eq:macro:washboard}
        \end{equation}
        where $E_\mathrm{J} = \hbar I_\mathrm{C} / 2e$ is the Josephson coupling energy. In this analogy:
        \begin{itemize}
            \item the particle position corresponds to the phase $\phi$,
            \item its velocity $\dot{\phi}$ corresponds to the voltage across the junction,
            \item the damping term $\hbar/2eR$ corresponds to viscous friction,
            \item and the tilt of the potential is controlled by the bias current $I$.
        \end{itemize}
        For $I < I_\mathrm{C}$, the particle is trapped in one of the potential wells, corresponding to the zero-voltage (superconducting) state. When $I > I_\mathrm{C}$, the potential minima disappear, the phase starts to run, and a finite voltage develops.

% !TEX root = ../thesis.tex


\section{Fabrication in Laboratory} 
\label{subsection:realization}
    This subsection serves as a practical guide, outlining the small but critical steps necessary for successful sample preparation. 


    \subsubsection*{Wafer Preparation}
        Start with a 300\,\mum thick bronze wafer, which is typically covered by a protective foil. To remove the foil without damaging the wafer, cool the wafer in liquid nitrogen and then gently peel off the shards. This method is both effective and non-destructive. The next step is to polish the wafer. Attach a polishing head made from sewn cotton cloth to a drill. Use aluminum oxide particles in the form of a white compound block as the abrasive material. The polishing head should be adequately moistened, preferably soaked, with a solvent such as alcohol or IPA, as these have proven effective. Secure the wafer to a smooth, milled steel block using double-sided adhesive tape\footnote{e.g. Tesa Universal}. Ensure that no tape protrudes from beneath the wafer to avoid interference during polishing. Begin polishing by moving the polishing head from top to bottom. This approach prevents debris generated during the process from settling back onto the wafer, which would require additional polishing to remove. Rotate the steel block periodically, as it is easier to polish the lower section of the wafer. Polishing is complete when the wafer achieves a uniform mirror-like finish across its surface. To detach the wafer from the steel block, cool it again in liquid nitrogen. For cleaning, rinse the wafer sequentially with acetone and then IPA, ensuring the acetone is removed completely before using IPA. Finally, dry the wafer with pressurized nitrogen gas while the IPA is still wet to avoid leaving any residues behind. \cite{oberg_machinerys_2000}
        
        Another option for polishing involves using sandpaper with progressively finer grit on a glass slide, as described by Patrick Raif \cite{raif_fabrication_nodate}. However, this method has been omitted due to its significantly higher time requirements, despite its superior ability to minimize long-range surface variations.

        Next, prepare the spin coater, PA, and vacuum oven. The PA is pre-portioned into small crimp-seal vials to minimize moisture exposure and ensure faster warming. Retrieve one vial from the freezer and allow it to reach room temperature while preheating the convection oven to 135\,$^\circ$C. Line the interior of the spin coater with aluminum foil and ensure the vacuum oven is operational, with its chamber open.

        To remove any adsorbed water from the wafer, place it on a hot plate at 100\,$^\circ$C for at least one minute. For proper adhesion to the 1\,in-chuck, apply a layer of parafilm, ensuring it is securely bent around the chuck's edges. Create five small holes in the parafilm using a syringe-- one at the center and one near the edge in each cardinal direction. Allow the wafer to cool to room temperature before centering it on the chuck. Once centered, activate the vacuum pump. Before applying the PA, test the spin-coating program to confirm proper adhesion. This step is especially critical for larger wafers, as adhesion issues may not be readily apparent due to limited use.
        
        Blow the wafer with nitrogen to remove any residual dust. Pour the PA onto the wafer, ensuring it covers approximately 90\% of the surface and avoiding air bubbles. If bubbles are present, gently pop or move them using a syringe, being careful not to scratch the wafer. Start the spin-coating process with an initial spread cycle at 300\,rpm for 30\,s, followed by a final spin at 5000\,rpm for 90\,s. Transfer the wafer directly to the convection oven for a 5\,min soft bake at 135\,$^\circ$C. For hard baking, use the vacuum oven with a programmed cycle that ramps up to 400,$^\circ$C, holds for 30,min, and then cools to room temperature over several hours. Detailed instructions and the temperature profile for the vacuum oven can be found in Reference \cite{raif_fabrication_nodate}. Baking immediately after spin coating has been shown to improve results.
        
        The application of resists A4 and EL11 follows a similar process. Aluminum foil is unnecessary, as these resists can be easily cleaned with acetone. Ensure the resists are not expired, as prolonged storage significantly affects their performance. Further details are provided in the Appendix in Table \ref{tab:methods:sample_parameter}.

        The wafer is now ready to be cut into smaller pieces measuring 3\,$\times$,8\,mm. This is done using a custom-made shear sheet cutter equipped with spacers of 3\,mm and 18\,mm. Before using the cutter, clean the cutting edges thoroughly with IPA to ensure precise and contamination-free cuts. Samples taken from the wafer's edges are more likely to have variations in the thickness profiles of the PI and resists. For consistent and reliable sample fabrication, it is recommended to use pieces from the center of the wafer, where the resist layers are more uniform. Thickness variations at the edge may impact the required electron beam dose during lithography. Once the samples are cut, place them in a chip tray\footnote{e.g. Entegris} for safe handling and storage.

    \subsubsection*{Electron beam lithography}
        The next step involves performing EBL using the Crossbeam system. Begin by utilizing the jug with a Faraday cup to measure the beam current. To locate the Faraday cup efficiently, you can use a position list in the Zeiss software. Ensure that the working distance is as close as possible to 5.00\,mm for optimal results. Note that the beam current may not stabilize immediately, so it's advisable to wait for some time or return later to perform the measurements.

        Aligning the sample horizontally is critical for accurate exposure. The Focused Ion Beam (FIB) toolbox provides alignment tools that simplify this process. Zoom out to the maximum view and use the upper-right corner of the sample and the farthest left edge for initial alignment. This method is sufficient for correcting angular misalignment. While you can further refine alignment using features in the Elphy Plus software, it is unlikely to improve the angle alignment once the sample is mounted in the cryostat.

        Next, focus the beam in the upper-right corner of the sample. Use the wobbling tool to position the beam in the center of the aperture, and reduce astigmatism using the stigmator adjustments for both the $x$- and $y$-axes. Once the beam is well-focused, check the beam current at the Faraday cup. For the 30\,\mum aperture in high-current mode, the beam current should be approximately 0.5\,nA. This aperture setting is critical for small and precise structures. For the 120\,\mum  aperture, which is used for the 1000\,\mum writing fields, the beam current is less critical since the bigger leads are more tolerant of under-exposure. In fact, under-exposing larger fields can reduce writing time without compromising quality. The beam current for the 120\,\mum aperture in high-current mode should typically be around 5\,nA. Perform a final focusing step at the central position of the upper edge of the sample. Look for a dirt particle on the resist to aid in focusing, and confirm that the working distance remains at 5.00(1)\,mm. At this stage, also ensure that the beam shift is set to zero.

        Before initiating the writing process, note the current time. The design layout includes a timestamp, which can help you correlate your notes with the physical sample. If the measured beam current deviates from previous settings, recalculate the settling time for the 100,\mum field accordingly. The executed position list in the Elphy Plus software will automatically locate the writing fields, apply the corresponding settings and write the design. 
        
        Repeat the whole alignment, focus and writing process for the next samples until you have a batch size of 4-5 samples.

        If it has been several weeks since your last EBL session, it is recommended to check the alignment between the 100\,\mum and 1000\,\mum fields before writing your first sample. To do this, expose a 100\,\mum field and one surrounding 1000\,\mum field in the upper-right corner of the sample. The exposed resist will differ in appearance from the unexposed areas, allowing you to assess alignment accuracy. Be mindful that prolonged observation can unintentionally expose the resist, reducing contrast and making the alignment harder to judge. If misalignment is detected, adjust the settings accordingly. For instance, if the 100\,\mum field is shifted left, adjust $U$ by $-\Delta X$; if shifted upward, adjust $V$ by $-\Delta Y$. Repeat this process as needed until the alignment is satisfactory.

        Once the EBL is complete, the samples must be developed. Begin by preparing the necessary solutions: fill a crimp-seal vial with a mixture of one part MIBK to three parts IPA. Additionally, prepare separate crimp-seal vials containing pure IPA-- one for each sample being developed.

        To develop the sample, immerse it in the MIBK/IPA solution for 25\,s, gently swirling the vial throughout this duration to ensure uniform exposure. After 25\,s, promptly transfer the sample into a vial with pure IPA and continue swirling. Note that the MIBK/IPA solution can be reused for all samples in the same batch, but fresh IPA should be used for each individual sample to maintain cleanliness. Leave the sample in the IPA solution for 60\,s before removing it. Once removed, dry the sample carefully with a stream of nitrogen gas. Throughout the process, securely hold the sample with tweezers to avoid unnecessary handling or accidental release during the waiting or transfer steps. Finally, inspect the developed samples under a light microscope to confirm that the patterns have been developed successfully and are free of defects.

    \subsubsection*{Shadow Evaporation}
        To begin the shadow evaporation process, place the sample holder on a hot plate set to 100\,$^\circ$C for a few minutes. This step helps to remove any adsorbed water, significantly improving the vacuum quality. Verify the orientation of the sample to ensure the shadow evaporation proceeds in the correct direction. For optimal results, load the sample into the load lock and leave it under vacuum overnight or longer. This extended pumping period further enhances the vacuum quality.

        When ready, transfer the sample holder into the main chamber for the first evaporation. Using electron beam evaporation, deposit 60\,nm of aluminum at an angle of 4\,$^\circ$. Take care to heat the aluminum gradually over 5-10\,min to prevent damage to the crucible and ensure uniform evaporation. Ensure the crucible is sufficiently filled with aluminum; typically, adding 1-3\,pellets per evaporation is sufficient.

        After completing the first evaporation, transfer the sample back to the load lock and close the pumping line. Introduce pure oxygen into the load lock to achieve a pressure of 3.0(2) mbar and allow the sample to oxidize for 3 minutes. To terminate the oxidation process, reopen the pumping line. To achieve a sufficient vacuum level of approximately $10^{-6}$\,mbar, ensure the oxygen feeding line is thoroughly evacuated.

        Once the load lock vacuum is restored, transfer the sample back into the main chamber for the second evaporation. Deposit\,100 nm of aluminum at an angle of 34\,$^\circ$ to complete the shadow evaporation process.

        To ensure the formation of a well-oxidized Al$_\text{2}$O$_\text{3}$ layer, it is essential to achieve a smooth aluminum surface. However, during the process, I encountered the formation of numerous dimples. Despite trying several approaches to minimize them, no definitive solution emerged. It appears that higher evaporation rates, specifically above 4\,\AA/s, tend to result in fewer dimples, though the difference is not statistically significant. Contrary to previous recommendations, a less deep vacuum or a shorter time interval between the last vacuum break and evaporation seemed to favor a reduction in the number of dimples. One might question the extended focus on achieving a high-quality vacuum in the load lock. The working theory behind this approach is that it prevents the sample from evaporating water or other contaminants during the aluminum deposition. While a poor vacuum in the main chamber may encourage nucleation around contaminants, potentially leading to a smoother surface, the observed effects were not significant enough to draw definitive conclusions, and further investigations were not pursued.

        The lift-off process is carried out by placing the sample in a crimp-seal vial filled with pure acetone. Heat the vial on a hot plate at approximately 60\,$^\circ$C for 1-3\,h. After this, use a pipette to carefully agitate the acetone inside the vial, ensuring that all metal flakes are removed. Rinse the sample with IPA and then blow it dry with nitrogen to complete the process.

    \subsubsection*{Plasma Etching}
        As penultimate step, etching the samples using the PlasmaPro 80 ICP RIE by Oxford is performed. Although this tool is not designed for homogeneous etching, it can be adapted to achieve this by operating at low table RF and high pressures. To stabilize the plasma before entering a less stable condition that favors homogeneous etching, an ignition step is implemented. The temperature is critical in achieving the desired undercut. A temperature of 100\,$^\circ$C has proven to be optimal, as it provides a sufficient undercut while maintaining a slow enough process to control the etching depth. However, the parameter I found after some optimization, can be found in Table \ref{tab:methods:rie_parameter}.
        \begin{table}
            \centering
            \caption{RIE recipy \texttt{MCBJ SSET v5 @ 100C}}
            \label{tab:methods:rie_parameter}
            \begin{tabular}{llll}
                Parameter       & Stabilization         & Ignition      & Etching  \\
                \\\hline
                Process time    & 5\,min                & 5\,s          & 30\,min       \\
                Table heater    & 100\,$^\circ$C   & -             & -             \\
                Oxygen flow     & 50\,sccm           & -             & -             \\
    
                Pressure        & 100\,mTorr & ramp & 250\,mTorr \\       
    
                \\\hline
                Table RF &&&\\
                Power Demand            & - & 20(5)\,W              & 3(5)\,W \\
                Max Reflected Power     & - & 5\,W                  & 2\,W \\
                Tolerance Time          & - & 7\,s                  & - \\
                Min DC Bias             & - & 1\,V                  & - \\
                AMU (C1, C2)            & - & (41\,\%, 23\,\%)      & - \\
    
                \\\hline
                ICP RF &&&\\
                Power Demand            & - & 400(50)\,W            & - \\
                Max Reflected Power     & - & 100\,W                & - \\
                Tolerance Time          & - & 10\,s                 & - \\
                AMU (C1, C2)            & - & (64\,\%, 39\,\%)      & - \\
                
            \end{tabular}
        \end{table}
        
        A laser interferometer is used to monitor the etching depth. For optimal exposure, find a spot on the sample, avoiding metal structures, where the reflected intensity is as high as possible, but not saturated. During etching, the intensity follows a cosine curve, which, although imperfect, is still recognizable. The number of periods $m$ you need to wait depends on the laser wavelength $\lambda$, the desired depth $d$, and the refractive index $n=1.81$ \cite{noauthor_duramide-100_2012}. For our process, we typically optimize for 
        % \begin{align}
        %     \textit{m} = \frac{\text{2} \cdot \textit{n} \cdot \textit{d}}{\lambda} = \frac{2 \cdot 1.81 \cdot 500\,\mathrm{nm}}{632.8\,nm} = 2.86
        % \end{align}
        \begin{align}
            m = \frac{2 \cdot n \cdot d}{\lambda} = \frac{2 \cdot 1.81 \cdot 500\,\mathrm{nm}}{632.8\,\mathrm{nm}} = 2.86
        \end{align}
        periods. An example curve of the laser intensity over time is shown in Figure \ref{fig:methods:rie}.

        \begin{figure}
            \centering
            \import{methods/sample/rie}{etchdata.pgf}
            \caption{Laser intensity during etch process}
            \label{fig:methods:rie}
        \end{figure}
        
        Since the etching speed varies significantly between samples, it is recommended to etch each sample individually for optimal results. 



    \subsubsection*{Contacting}

        \coffeestainA{0.05}{0.85}{-100}{2in}{-4in}
        \coffeestainD{0.025}{0.15}{-100}{1.5in}{1in}
        Contacting the sample was initially a challenging task for me, and it took gathering advice from many current and former users. Eventually, I realized that the main issue was shaky hands. Shakiness can be caused by factors such as caffeine, caffeine deprivation, lack of sleep, stress, and poor eyesight. To address this, I stopped drinking coffee, as I realized I am quite sensitive to it. To reduce stress, I also made it a point not to contact more than one sample per day. I took my time, and if I wasn't feeling up to it, I would postpone the task until the next day. After all the previous production steps, losing samples due to rushed contacting is far more costly than taking the time to do it right.

        As for my eyesight, I discovered that I am nearsighted, which is not so bad, but I also have astigmatism, which makes things more challenging. The poor lighting around the workbench made it even worse. I found that working at the workbench in the middle of P5 offered two advantages over other spots: good lighting options and a relatively high table. This allowed me to maintain a comfortable position while being close to the sample, even when sitting on a low chair.

        To fixate the sample, I use adhesive strips\footnote{e.g. tesafilm\textregistered\ standard by Tesa} on a copper plate, which allows some movement of the plate but keeps the sample fixed in place. The next step is to prepare some insulated copper wires soldered to post connectors. You should strip the insulation off the wire tips, and the bias lines can be twisted. It's important to keep the wires as short as possible-- 4\,--\,5\,cm is usually enough. Be sure to insulate the rear part of the post connector with an insulating varnish\footnote{e.g. GE Low Temperature Varnish (59-C5-101) by Oxford Instruments}.

        Position the wire tips with the help of play dough so that they touch the pads of the sample. Then, using a sharpened toothpick, apply a small amount of silver paste onto the wire and pad. Make sure to use silver paste specifically for contacting, as its viscosity is critical for a proper connection. The paste should be just enough to wet the sample so that only the pad is connected. If the viscosity is too low, it could cause a short circuit; too high, and the wire will stick to the toothpick instead of the pad. To prevent this, use the silver paste in the lid after shaking the paste bottle, and just cover the tip of the toothpick.

        Next, use two-component epoxy to provide strain relief for the wires. The viscosity of the epoxy is crucial here as well. If you use the epoxy too soon after mixing, it will continue to flow and could touch unwanted areas. The interaction between the epoxy and silver paste is not fully understood, so it's important to avoid mixing them. I typically test the epoxy by making small test dots and waiting for the right consistency before applying it. Each side should be glued separately, as the working time is narrow.
        
        After loading the sample into the cryostat, the final step is to scratch the shortage with an engraving tool. During this process, and while closing the radiation shields, ensure that you are properly grounded by wearing a grounding bracelet. It is also recommended, to do this task barefoot and to raise the humidity with an air humidifier to prevent static buildup.

        \begin{figure}
            \centering
            \caption{blabla}
            \label{fig:methods:contacting}
            \includegraphics[width=.4\textwidth]{methods/sample/contacting/foto.jpg}
        \end{figure}
    
    

    \subsubsection*{Parameter for Sample Preparation}
        \begin{table}
            \centering
            \caption{here goes some text.}
            \label{tab:methods:sample_parameter}
            \begin{tabular}{ll}
                \hline
                Sacrificial layer   & Durimide 115A\\
                Spinning            & spread cycle: 300\,rpm for 30\,s\\
                                    & spin cycle: 5000\,rpm for 90\,s\\
                                    & ramps for 3\,s, \texttt{Program 3}\\
                Soft baking         & 135\,$^\circ$C for 5\,min in convection oven\\
                Hard baking         & 400\,$^\circ$C for 30\,min in vacuum oven\\

                \hline
                Spacing layer       & MMA(8.5)MAA EL11\\
                Spinning            & spread cycle: 500\,rpm for 5\,s\\
                                    & spin cycle: 2500\,rpm for 90\,s\\
                                    & ramps for 3\,s, \texttt{Program 1}\\
                Soft baking         & 150\,$^\circ$C for 1\,min on hot plate\\

                \hline
                Electron resist     & 950 PMMA A4\\
                Spinning            & spread cycle: 500\,rpm for 5\,s\\
                                    & spin cycle: 5000\,rpm for 60\,s\\
                                    & ramps for 3\,s, \texttt{Program 2}\\
                Hard baking         & 170\,$^\circ$C for 30\,min in convection oven\\

                \hline
                Exposure            & Crossbeam by Zeiss, Elphy Plus by Raith\\
                Settings            & 10\,kV acceleration, 5\,mm working distance,\\ 
                Aperture            & 30\,\mum and 100\,\mum aperture, high current,\\
                                    & 155\,\textmu C$/$cm$^\text{2}$ area dose\\
                Development         & 25\,s in 1 MIBK + 3 IPA\\
                                    & 60\,s in IPA\\
                
                \hline
                Shadow Evaporation  & First layer: 60\,nm Al at -4\,$^\circ$\\
                                    & Oxidation: 3\,mbar of O$_\text{2}$ for 3\,min\\
                                    & Second layer: 100\,nm Al at 34\,$^\circ$\\
                Lift-off            & 1\,h in acetone at 60\,$^\circ$C\\

                \hline
                Reactive Ion Etching & PlasmaPro 80 ICP RIE by Oxford\\
                                    & \texttt{MCBJ SSET v5 @ 100C}, see Table \ref{tab:methods:rie_parameter}\\
                \hline
                
            \end{tabular}
        \end{table}


        
        % After loading the sample in the cryostat, as a last step, the shortage get scratched with an engraving tool.
        % 100\,$^\circ$C has proven to yield a sufficient undercut and the process is slow enough to control the etching depth.
        % A single  

        % \begin{align}
        %     \textit{m} = \frac{\text{2} \cdot \textit{n} \cdot \textit{d}}{\lambda} = \frac{2 \cdot 1.81 \cdot 500\,\mathrm{nm}}{632.8\,nm} = 2.86
        % \end{align}

        % refractive index = 1.81 (cite: manual)

    % \subsection{substrate}
    %     Warum Bronze? Warum? Christian Schirm hat nur echt viele substrate getestet. Aber auch nicht erzählt warum genau diese? copper-berilium is poisonous. 
    %     % https://www.jlab.org/sites/default/files/magnet-group/materials/low_temperature_materials_properties.pdf

    %     polieren, plus dann poliamid


    %     Wichtig: SE funktioniert besser mit neuerem Wafer

    % \subsection{Sample Design}
    %     Insel übernommen von vorgängern. (thomas Susi laura) Dabei wurde optimiert dass die Insel klein genug ist für coulomb blockade effect und groß genug um nicht abzureißen. ebenso gate mit 300nm abstand scheint gut zu passen. 

    %     Neu im Game ist stripline. auch wenn nicht 50 Ohms matched so ist die Idee, dass der geringe abstand zu insel der dominierende eingestrahlte teil der strahlung ist.
    %     Lange länge um möglichst teiffrequente resonanz zu bekommen und bei höheren frequenzen besser einzustrahlen.

    %     Problematic mit pads, da vier und dann noch kurzschluss um tunnelbarriere möglichst lange zu schützen.

    % \subsection{Elektron Beam Lithography}
    %     Warum Raith? 
    %     Electron Beam Lithography alignment und stiching. Möglicht wenige polygone. positionlist.
    %     verzicht auf contamination dots, durch high-current und 30mu blende und damit hohe rückwärtige aperture. -> höhere tiefenschärfe. Nachteil ist, das nicht wirlich niedrigere dosen verwendet werden können.

%     \subsection{shadow evaporation}
%     \subsection{Reactive Ion Etching}
%     \subsection{contact}

\section{Drivers}
    into the appendix
    \subsection{p5 control}
    \subsection{GUI}
    \subsection{Femtos}
    \subsection{ADwin}
    \subsection{Bluefors Software}
    \subsection{Magnet}
    \subsection{Motor}
    \subsection{VNA}
    \subsection{Yoko}
    \subsection{Keysight}





\section{Test}
    Reference \cite{schertel_magnetic-field_2019}


    \the\textwidth
    \printinunitsof{in}\prntlen{\textwidth}
    \printinunitsof{in}\prntlen{\linewidth}
    \printinunitsof{mm}\prntlen{\textwidth}
    \printinunitsof{mm}\prntlen{\linewidth}

    \begin{figure}
        \centering
        \import{theory/stuff/plot test}{test.pgf}
        \caption{Complex susceptibility}
        \label{fig:test_plot}
    \end{figure}

    Figure \ref{fig:test_plot}


    \begin{figure}
        \centering
        \import{theory/stuff/inkscape test}{test.pdf_tex}
        \caption{blablabla.}
        \label{fig:test_inkscape}
    \end{figure}

    Figure \ref{fig:test_inkscape}


    \begin{program}
        \caption{Listing Caption is above.}
        \label{program:test}
        \lstinputlisting[language=Python]{appendix/test.py}
    \end{program}

    Program \ref{program:test}

        
Option 1 — No explicit “Miscellaneous” chapter

This is the cleanest for most theses.
You integrate every bit of content under one of the main scientific chapters or the appendices.
Use appendices for anything that is:
	•	technical but not central (calibration curves, fitting scripts, additional I–V traces),
	•	background or validation measurements (e.g., temperature calibration, microwave attenuation tests).

Then the main text stays focused and logical.

⸻

🔹 Option 2 — A short “Miscellaneous Results” chapter

If you have several small, independent results that don’t warrant full chapters — for example:
	•	supplementary microwave experiments (different frequencies, powers),
	•	tests on non-Al samples,
	•	noise measurements,
	•	comparisons between fitting methods,

— then you can group them as a compact “Additional Experiments and Analysis” or “Miscellaneous” chapter near the end.

Suggested layout:

Chapter 6 — Additional Experiments and Analyses

6.1 Temperature and Magnetic-Field Dependence
Briefly show confirmation that the devices behave as expected (e.g. gap suppression).

6.2 Frequency-Dependent Effects
Show the asymmetric I–V features and discuss possible causes (resonances, heating, impedance mismatch).

6.3 Auxiliary Measurements and Calibrations
Include details such as microwave coupling, line attenuation, or base-temperature determination.

Keep this concise — it shouldn’t look like an afterthought but rather a supporting evidence collection.

⸻

🔹 Option 3 — Miscellaneous Appendices

If your “miscellaneous” content is highly technical or data-heavy:
	•	Appendix A: Detailed fitting procedure for Tien–Gordon model
	•	Appendix B: Raw I–V data and error estimation
	•	Appendix C: Microwave calibration
	•	Appendix D: Temperature extraction via Dynes fits

That keeps your main story crisp but documents everything transparently.

⸻

🔹 Recommendation for you

Given your description so far, I’d suggest:
	•	No “Miscellaneous” chapter in the main thesis body.
	•	Instead, include a short “Additional Experiments” section at the end of your High-Transmission chapter (for asymmetric I–Vs etc.),
	•	and move purely technical or validation content to appendices.

That way, everything still feels intentional and thematically grouped.
