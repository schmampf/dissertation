% !TEX root = ../thesis.tex
\chapter{Few-channel atomic contacts}

Chapter 4 -- Few-Channel Atomic Contacts

4.1 \section{Concept of Quantum Channels in Superconducting Atomic Contacts}
	-	Short recap of Landauer formula and transmission eigenchannels.
	-	Introduce “pincode” terminology and cite Scheer et al. (1997).
	-	Explain why determining the transmission distribution is crucial for interpreting MAR and microwave effects.

4.2 \section{Reproduction of Scheer et al. Measurement (your highlight)}
	-	Describe your breaking-trace measurement and fitting routine.
	-	Show several I-V traces during elongation and extracted transmission vectors.
	-	Compare to canonical results -- emphasize that this validates your setup.
	-	Discuss possible deviations (temperature, noise, fitting uncertainty).

4.3 \section{Photon-Assisted Multiple Andreev Reflection (PAMAR)}
	-	Build on the established pincode and show how microwaves modify those MAR features.
	-	Present simulations (FCS + modified Tien-Gordon).

4.4 \section{Discussion}
	-	Comment on consistency between pincode, MAR behavior, and microwave response.

Chapter 4: Few-channel atomic contacts (intermediate transmission)

Short theory section:
	-	Full Counting Statistics (FCS) for MAR including AC drive.
	-	Modified Tien-Gordon for finite transmission.
	-	Mention that JC is neglected in simulation (justify).

Results/discussion:
	-	MCBJ conductance evolution → channel pin-code (compare to Elke Scheer).
	-	PAMAR structures and their microwave modulation.
	-	Agreement between experiment and FCS-based simulation.

Purpose: connect microscopic channel picture to measurable photon-assisted MAR signatures.
