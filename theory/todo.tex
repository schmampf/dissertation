
    \section{Experimental Realization}


    \section{TODO}
    bin am überlegen die schemata maximal im Anhang zu zeigen (so poblig aus dem Poster screenshoten.) Ich finde man muss viel und lange erklären was wann für vereinfachungen gelten und wie die Bilder gemeint sind. richtige physik lässt sich damit nur schwer machen. Lieber mehr IV zeigen!
    \begin{itemize}
        \item check consistency:
        \begin{itemize}
            \item quasi-particle
            \item Cooper-pair
            \item aluminum
            \item weak-coupling
            \item normal state
            \item - or --
            \item -- bei Modell namen (Tien--Gordon, Ginzburg--Landau)
        \end{itemize}
        \item check consistency seeblau100
    \end{itemize}

    \subsection*{theory}
    \begin{itemize}
        \item table with theories
    \end{itemize}

    \subsection*{basics figures}
    \begin{itemize}
        \item Landauer channels picture (mode counting + partial reflection)
        \item Atomic contact orbitals set channels sketch
        \item Microwave drive / phase-factor / photon sidebands figure
        \item DOS figure. parabolic and nearly constant around EF
    \end{itemize}

    \subsection*{macro}
    \begin{itemize}
        \item overlapping wave functions
        \item rscj shaltplan (mini)
        \item plots (a lot of them)
        \item do the citations
    \end{itemize}

    \subsection*{meso}
    \begin{itemize}
        \item check ABS section
        % \item AR-IV-tau
        % \item AR-IV-m
        % \item MAR-IV-tau
        % \item MAR-IV-m
        \item PAAR
        \item PAMAR
        \item do the citations
    \end{itemize}

    \subsection*{stochastic}
    \begin{itemize}
        \item alles
    \end{itemize}

    \subsection*{experiment and state of the art}
    \begin{itemize}
        \item alles
    \end{itemize}
