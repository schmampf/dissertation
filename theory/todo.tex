
    \section{TODO}
    bin am überlegen die schemata maximal im Anhang zu zeigen (so poblig aus dem Poster screenshoten.) Ich finde man muss viel und lange erklären was wann für vereinfachungen gelten und wie die Bilder gemeint sind. richtige physik lässt sich damit nur schwer machen. Lieber mehr IV zeigen!
    \begin{itemize}
        \item check consistency:
        \begin{itemize}
            \item quasiparticle
            \item Cooper-pair
            \item aluminum
            \item weak-coupling
            \item normal state
            \item - or --
            \item -- bei Modell namen (Tien--Gordon, Ginzburg--Landau)
        \end{itemize}
        \item check consistency seeblau100
    \end{itemize}

    \subsection*{theory}
    \begin{itemize}
        \item table with theories
    \end{itemize}

    \subsection*{basics figures}
    \begin{itemize}
        \item Landauer channels picture (mode counting + partial reflection)
        \item Atomic contact orbitals set channels sketch
        \item DOS figure. parabolic and nearly constant around EF
        \item Aluminum Channel distribution (Carlos)
    \end{itemize}

    \subsection*{macro}
    \begin{itemize}
        \item better RCSJ Model including actual suerconducting quasi-particle current
    \end{itemize}

    \subsection*{meso}
    \begin{itemize}
        \item Semiconductor Schaubilder?
        \item understand fSS / PAMAR / RCSJ
    \end{itemize}

    \subsection*{stochastic}
    \begin{itemize}
        \item check for consistency with rewritten basics / micro / macro / meso..
        \item citations
        \item SET Section is a bloody mess
    \end{itemize}
