% !TEX root = ../thesis.tex
%=========================================================
\chapter{Superconducting Transport}
\label{ch:superconducting-transport}
%=========================================================

    Superconductivity was first observed by Heike Kamerlingh Onnes in 1911, when the electrical resistance of mercury vanished abruptly upon cooling below 4.2\,K. Shortly thereafter, the Meissner--Ochsenfeld effect revealed that superconductors also expel magnetic fields, establishing the two defining macroscopic properties of this quantum state of matter: perfect conductivity and perfect diamagnetism. \cite{onnes_further_1991, meissner_neuer_1933}

    At its core, superconductivity originates from an effective attraction between electrons near the Fermi surface. When an electron moves through the metal, it slightly distorts the ion lattice, creating a transient region of enhanced positive charge. A second electron can be attracted to this distortion, resulting in a very weak but cooperative pairing mechanism mediated by lattice vibrations (phonons). Despite the underlying Coulomb repulsion, this mechanism allows pairs of electrons with opposite momentum and spin to form Cooper-pairs. When many such pairs form simultaneously, they condense into a coherent macroscopic quantum state capable of carrying electrical current without resistance.

    The remainder of this chapter develops a comprehensive theoretical framework for superconducting transport from four complementary viewpoints. These perspectives differ in the degrees of freedom they emphasize quasi-particles, phase dynamics, transparency channels, and environmental fluctuations. Together form a consistent and unified picture of the phenomena observed in experiment.

    The first part introduces the microscopic Bardeen--Cooper--Schrieffer theory. It describes how electrons near the Fermi surface pair in the weak-coupling limit, how this pairing leads to an energy gap and a characteristic quasi-particle density of states, and how these quantities manifest in tunneling spectroscopy and photon-assisted transport.

    The second part presents the macroscopic Ginzburg--Landau description, where the superconducting condensate is represented by a complex order parameter with a well-defined phase. Spatial variations of this phase control the behavior of weak links and give rise to the Josephson effects, Shapiro steps, and the influence of dissipation and capacitance captured by the RCSJ model.

    The third part provides the mesoscopic description, which connects microscopic quasi-particle physics with macroscopic phase coherence. Here, Andreev reflection and multiple Andreev reflections govern charge transport in highly transparent contacts, and the resulting \textit{I--V} characteristics encode the microscopic transmission properties of the junction.

    Finally, the stochastic description accounts for phase fluctuations in dissipative electromagnetic environments through \textit{P(E)}-theory. This framework captures how the impedance of the surrounding circuitry affects phase coherence, energy exchange, and the overall \textit{I--V} characteristics.

    Before turning to the detailed microscopic, macroscopic, mesoscopic, and stochastic descriptions, the following two subsections introduce the global definitions used throughout this chapter: the material platform (aluminum) and the general microwave drive model applied in all driven-transport scenarios.
    
    \subsubsection*{Aluminum as a Model BCS Superconductor}

        Throughout this chapter, aluminum serves as the representative weak-coupling BCS superconductor. It possesses a nearly free-electron-like band structure, an isotropic s-wave energy gap, and a well-understood phonon-mediated pairing mechanism. The zero-temperature gap and critical temperature are to good approximation
        \begin{equation}
            \Delta_0 \approx 180\,\mu\mathrm{eV}\,,\qquad
            T_\mathrm{C} \approx 1.2\,\mathrm{K}\,.
            \label{eq:aluminum}
        \end{equation}
        Its long coherence length of order 1-2\,\textmu m, high electronic purity, and weak spin--orbit scattering make aluminum ideally suited for tunneling spectroscopy and mesoscopic superconducting transport. The comparatively small gap places typical microwave frequencies well below the pair-breaking threshold, enabling controlled photon-assisted processes without degrading superconductivity\footnote{The pair-breaking frequency of aluminium is given by $\nu\mathrm{pb}=2\Delta_0/h\approx 100\,\mathrm{GHz}$.}. Moreover, aluminum naturally forms reproducible aluminum-oxide tunnel barriers and reliably yields stable atomic contacts. It therefore provides a consistent and unified material platform for all transport phenomena discussed in this chapter.
        
    \subsubsection*{Microwave Drive Model}
    
        Many of the transport phenomena discussed in this chapter involve the presence of a time-dependent electromagnetic drive. To avoid repeated introductions of the same setup, we summarize here the general assumptions that apply to all subsequent descriptions. The drive is treated as a classical, externally imposed field. No cavity modes or backaction of the junction onto the field are considered. The entire voltage drop is assumed to occur across the junction, a convenient gauge choice. In consequence, the electromagnetic field enters only through the scalar potential in the tunneling Hamiltonian or via the Josephson phase evolution. The drive amplitude is taken to be small enough not to heat the electrodes, allowing both to remain in local thermal equilibrium with Fermi--Dirac distributions. Finally, the drive frequencies are chosen such that photon energy is way smaller than the pair-breaking energy, ensuring that Cooper pairs remain intact while the phase dynamics are driven coherently. In practice, frequencies in the microwave range up to about 20\,GHz are used.

        The spatially uniform microwave field is then given by 
        \begin{equation}
            V(t) = V_0 + A \cos (2\pi\nu t)\,,
            \label{eq:microwave}
        \end{equation}
        with a static component $V_0$, drive amplitude $A$, and frequency $\nu$. 
        
        The charge relevant for the electromagnetic phase is $q = m e$ with $m\in\{1,\,2,\,3,\,\dots\}$. For quasi-particles $m=1$, for Cooper-pair and single Andreev processes $m=2$, and for higher-order multiple Andreev reflections the effective transferred charge is $m>2$. However, the accumulated potential, determines the relevant electromagnetic phase factor
        \begin{align}
            \exp\!\left(-\ima\phi(t)\right) &= \exp\!\left( - \frac{\ima}{\hbar} \int_0^t q V(t')\, dt' \right) \\
            &= \!\exp\left(-\ima \phi_0(t)\right)\exp\!\left(\ima \alpha \sin(2\pi\nu t) \right)\,.
            \label{eq:microwave:phase-factor}
        \end{align}
        where $\alpha=qA/h\nu$ denotes the dimensionless modulation strength and $\phi_0(t) = q V_0 t / \hbar$ is the phase difference generated by the static drive.

        The phase factor is conveniently expanded using the Jacobi--Anger identity,
        \begin{equation}
            \exp\!\left(\ima\alpha \sin(2\pi\nu t)\right) = \sum_{n=-\infty}^{\infty} J_n(\alpha)\,
            \exp\!\left(\ima n\ 2\pi\nu t\right)\,
            \label{eq:microwave:jacobi-anger}
        \end{equation}
        where $J_n(\alpha)$ is the $n$-th Bessel function of the first kind. This harmonic decomposition will be used repeatedly in this chapter.


        
    \newpage

    % !TEX root = ../thesis.tex

\section{Microscopic Description}
\label{sec:micro}

    The microscopic description of superconductivity is provided by the Bardeen--Cooper--Schrieffer (BCS) theory, which assumes that electrons near the Fermi surface form bound pairs of opposite momentum and spin, so called Cooper-pairs.
    
    At its heart, this pairing mechanism reflects a subtle interplay between electrons and phonons. When an electron moves through the metal, it slightly displaces the positively charged ions, creating a momentary region of excess positive charge. A second electron passing nearby can be attracted to this distortion, leading to an effective, though very weak attraction between the two. 
    
    This describes how electrons in a metal, which normally repel each other due to their negative charge, can nevertheless form bound pairs, known as Cooper-pairs, through an effective attraction mediated by vibrations of the crystal lattice. When many such pairs form simultaneously, they condense into a collective quantum state that carries electrical current without resistance.

    A key assumption of the BCS framework is that this effective interaction is weak compared to the electronic energy scales of the metal. This situation is referred to as the weak-coupling limit. In this regime, the attractive interaction acts only within a narrow energy shell around the Fermi surface, typically a few meV compared to Fermi energies of several eV. As a result, only a small fraction of electrons near the Fermi level participate in pairing, while the rest of the electronic structure remains essentially unaffected. Because the coupling is weak, the resulting energy gap and the critical temperature are small but can be predicted with high accuracy from the microscopic parameters.

    Among all known superconductors, aluminum is one of the best realizations of this weak-coupling scenario. Its superconducting transition temperature and energy gap are small, the electron-phonon interaction is weak and well understood, and its electronic structure is simple and free from strong correlations or magnetic effects. These features make aluminum behave almost exactly as predicted by the original BCS theory, allowing quantitative agreement between experiment and theory without the need for corrections or more advanced models.

    In summary, the weak-coupling limit describes a situation where the effective electron-phonon attraction is small compared to the Fermi energy, leading to a narrow pairing region and universal relationships between microscopic and macroscopic quantities. Aluminum exemplifies this regime and therefore serves as an archetype of a conventional BCS superconductor.

    The following subsections outline the key aspects of this microscopic picture. 
    First, the superconducting gap and its temperature dependency is introduced. This shows how the collective pairing strength evolves with thermal excitation. 
    Second, the electronic density of states is derived. The phenomenological Dynes parameter as measure of quasi-particle lifetime is introduced. The Fermi-Dirac distribution function and its temperature dependency is introduced as well.
    Third, the microscopic expressions for tunneling current are discussed, establishing the direct connection between these theoretical quantities and experimentally measurable \textit{I-V} and \textit{dI-dV} characteristics. 

    Finally, the extension to photon-assisted tunneling is presented, where an oscillating electromagnetic field enables quasi-particles to exchange discrete energy quanta during tunneling, thereby extending the static tunneling picture to the regime of externally driven superconducting transport.
    
    % \cite{bardeen_microscopic_1957}

    \subsection{Superconducting Gap}
    \label{subsec:micro:sc-gap}

        The weak-coupling approximation simplifies the theoretical treatment significantly. The normal-state density of states can be treated as constant, and the pairing interaction can be approximated as uniform within the relevant energy range. Many observable quantities therefore become universal, such as the ratio between the zero-temperature gap and the critical temperature, 
        \begin{equation}
            \Delta_0 \approx 1.764\, k_\mathrm{B} T_\mathrm{C}\,.
            \label{eq:micro:Delta0}
        \end{equation}

        The superconducting gap does not remain constant with temperature. At absolute zero, all available electrons near the Fermi surface form Cooper pairs, and the condensate is perfectly ordered. As the temperature rises, thermal excitations begin to break some of these pairs, leaving fewer electrons bound in the superconducting state. With fewer pairs contributing to the collective order, the overall pairing strength weakens, and the energy required to break a pair the gap $\Delta(T)$ gradually decreases.

        This reduction continues smoothly until the critical temperature $T_\mathrm{C}$ is reached. At that point, thermal energy becomes strong enough to completely disrupt the pairing correlations, and the superconducting state collapses, leading to $\Delta(T_\mathrm{C}) = 0$. The temperature dependence of the gap is a direct reflection of this balance between thermal disorder and the pairing interaction.

        \begin{wrapfigure}[12]{r}{0.4\textwidth}
            \captionsetup{format=plain}%
            \centering
            \vspace{-1em} % fine-tune vertical position
            \import{theory/micro}{gap-suppression.pgf}
            \caption{Temperature dependence of the superconducting gap $\Delta(T)$.}
            \label{fig:micro:gap_suppression}
        \end{wrapfigure} 
        In the BCS framework, $\Delta(T)$ follows a universal curve that results from solving the self-consistent gap equation, meaning the same functional form applies to all weak-coupling superconductors. 
        Solving the underlying integrals over the Fermi distribution in the microscopic theory numerically, results in the following Equation \ref{eq:DeltaT} or shown in Figure \ref{fig:micro:gap_suppression}.
        \begin{equation}
            \frac{\Delta(T)}{\Delta_0} \approx \tanh\left(1.74\,\sqrt{\frac{T_\mathrm{C}}{T}-1}\right)
            \label{eq:micro:DeltaT}
        \end{equation}

        % cite: Mühlschlegel (1959) für const2


    \subsection{Density of States}
    \label{subsec:micro:dos}

        In the following, all energies are expressed relative to the Fermi energy $E_\mathrm{F}$, such that $E=0$ corresponds to the Fermi level around which superconducting correlations develop.
        
        In the normal state, the density of states (DOS) near the Fermi level varies only weakly with energy. Over the narrow range relevant to superconductivity this variation can be neglected, and the DOS can be treated as constant. The corresponding value at the Fermi energy is denoted
        \begin{equation}
            N_0 \equiv N_\mathrm{N}(E_\mathrm{F}) = \frac{1}{2\pi^2} \left(\frac{2m}{\hbar^2}\right)^{3/2} \sqrt{E_\mathrm{F}}\,,
            \label{eq:micro:DOS-N0}
        \end{equation}
        representing the normal state DOS per spin at the Fermi level.

        When superconductivity sets in, pairing correlations reorganize this otherwise flat spectrum. A gap of width $2\Delta$ opens around $E_\mathrm{F}$ where single-particle excitations are absent in the ideal BCS limit, and the missing spectral weight is redistributed to the gap edges. These edges appear as sharp coherence peaks in the superconducting DOS, reflecting the high density of available quasi-particle states at the threshold for pair breaking. The resulting expression reads
        \begin{equation}
            \frac{N_\mathrm{S}(E)}{N_0} = 
                \left\{
                \begin{array}{@{}r@{\quad}l@{}}
                    0 & (|E| < \Delta)\\
                    \dfrac{|E|}{\sqrt{E^2-\Delta^2}} & (|E| \ge \Delta)
                \end{array}
                \right.\,.
            \label{eq:micro:DOS-BCS}
        \end{equation}
        Thus, in the ideal BCS limit the quasi-particle DOS is strictly zero within the energy gap and diverges at its edges, $E=\pm\Delta$. Increasing temperature alone does not cause a smearing of the density of states, but merely compresses its features in energy as the gap closes.

        However, real spectra are never perfectly sharp. A simple and very effective phenomenology is the broadening by Dynes parameter. It is implemented by the substitution $E\to E+\ima\Gamma$, while just considering the real part
        \begin{equation}
            \frac{N_\mathrm{S}(E)}{N_0} = \Re\!\left(\frac{E+\ima\Gamma}{\sqrt{(E+\ima\Gamma)^2-\Delta^2}}\right) \quad (|E| \ge \Delta)\,.
            \label{eq:micro:DOS-Dynes}
        \end{equation}
        Such broadening arises from finite quasi-particle lifetimes due to inelastic scattering, spatial inhomogeneity, pair breaking by magnetic impurities, or non-equilibrium effects, all of which smear the ideal BCS singularities. The DOS for a variaty of $\Gamma$ is shown in Figure \ref{fig:micro:dos-fermi}. 
        \begin{figure}[t]
            \centering
            \import{theory/micro}{dos-fermi.pgf}
            \caption{Superconducting quasi-particle density of states $N_\mathrm{S}(E)$ and Fermi--Dirac distribution $f(E)$ for different Dynes parameters $\Gamma/\Delta_0$ and temperatures $T/T_\mathrm{C}$. Increasing $\Gamma$ broadens the coherence peaks of $N_\mathrm{S}(E)$, while increasing temperature smooths the Fermi edge. Parameters correspond to aluminum with $\Delta_0 = 180\,$\textmu eV and $T_\mathrm{C} = 1.18\,\mathrm{K}$, representative of a weak-coupling BCS superconductor.}
            \label{fig:micro:dos-fermi}
        \end{figure}

        Whereas the DOS specifies where states exist, the Fermi--Dirac distribution encodes how single-particle states are occupied at a given temperature and chemical potential. In the context of metals and conventional superconductors treated in this thesis, the chemical potential can be identified with the Fermi energy to very good approximation, $\mu\approx E_\mathrm{F}$, because thermal corrections are small on the scale of eV. So the energy scale is still relative to the Fermi energy, as before.

        In equilibrium the occupation probability of a state at energy $E$ is given by        
        \begin{equation}
            f(E) = \frac{1}{1+\exp\left(\frac{E}{k_\mathrm{B}T}\right)}\,.
            \label{eq:micro:fermidirac}
        \end{equation}        
        At zero temperature the distribution reduces to a sharp step $ \theta(E)$.

        However, all states below the Fermi function are filled and all states above are empty. At finite temperature the step is thermally broadened over an energy scale of order $k_\mathrm{B}T$, as shown in Figure \ref{fig:micro:dos-fermi}. 

    
    \subsection{Tunnel Current}
    \label{subsec:micro:tunnel-current}

        Tunneling spectroscopy offers a powerful means to probe the quasi-particle spectrum of superconductors in a controlled and conceptually simple way. The key idea is that if two electrodes are separated by a sufficiently thin insulating barrier, quasi-particles can quantum-mechanically tunnel between them, even though classically forbidden. 
        
        In the tunneling limit, the barrier is high and wide enough that the process is incoherent and each quasi-particle tunnels independently, so momentum conservation is effectively relaxed. 
        
        Under these conditions, the tunnel current from material $1$ to $2$ is given by
        \begin{equation}
            I_{1\to2}(V) \propto \int_{-\infty}^\infty \left(\frac{N_1(E)}{N_0} f_1(E) \right)\cdot \left(\frac{N_2(E+eV)}{N_0} \left(1-f_2(E+eV)\right)\right)\mathrm{d}E\,,
            \label{eq:micro:tunnel-1to2}
        \end{equation}
        where $eV$ is an externally applied voltage bias. The first part in Equation \ref{eq:micro:tunnel-1to2}, is given by the occupied states in material 1, the second part is given by the unoccupied states in material 2. However, in order to get the total tunnel current, one have to substract the tunnel current from material 2 to 1, resulting in
        \begin{equation}
            I(V) \propto \int_{-\infty}^\infty \frac{N_1(E)}{N_0} \frac{N_2(E+eV)}{N_0} \left(f_1(E)-f_2(E+eV)\right)\mathrm{d}E\,.
            \label{eq:micro:tunnel}
        \end{equation}
        
        The difference $f_1(E)-f_2(E+eV)$ accounts for the imbalance in occupation between the two electrodes induced by the applied bias voltage $eV$, ensuring that current flows only when filled states on one side overlap with empty states on the other. In this picture, the densities of states $N_1(E)$ and $N_2(E)$ define where electrons can tunnel, while the Fermi--Dirac distributions define which of those states are populated. The convolution of these terms thus directly connects the microscopic electronic structure to the measurable \textit{I-V} characteristics.

        In the case of an all normal conducting tunnel barrier, the current is given by
        \begin{equation}
            I_\mathrm{NN}(V) = \frac{G_\mathrm{N}}{e} \int_{-\infty}^{\infty}f(E) - f(E + eV)\mathrm{d}E\,.
            \label{eq:micro:tunnel-nn}
        \end{equation}
        All geometric factors of the tunnel barrier, along with $N_0$ are collapsing into the normal conductance $G_\mathrm{N}$. 
        Given the two electrodes are in thermal equilibrium, effectively collapses the equation to Ohm's law
        \begin{equation}
            I_\mathrm{NN}(V) =  G_\mathrm{N}V\,.
            \label{eq:micro:ohms-law}
        \end{equation}

        In the case of a junction between a normal metal and a superconductor, the tunneling current is given by
        \begin{equation}
            I_\mathrm{NS}(V) = \frac{G_\mathrm{N}}{e} \int_{-\infty}^{\infty} \frac{N_\mathrm{S}(E)}{N_0} \left[f(E) - f(E + eV)\right] \mathrm{d}E\,,
            \label{eq:micro:tunnel-ns}
        \end{equation}
        where $N_\mathrm{S}(E)$ denotes the superconducting quasi-particle density of states and $f(E)$ the Fermi--Dirac distribution. The differential conductance then follows as
        \begin{equation}
            \frac{\mathrm{d}I_\mathrm{NS}(V)}{\mathrm{d}V} = G_\mathrm{N} \left[ \frac{N_\mathrm{S}(E)}{N_0} \otimes -\frac{\partial f(E)}{\partial E}\right]_{E=eV}\,,
            \label{eq:micro:tunnel-ns-didv}
        \end{equation}
        showing that the measured \textit{dI-dV} corresponds to the superconducting density of states thermally broadened by the derivative of the Fermi function. This derivative is a symmetric, bell-shaped function whose width scales with temperature. As $T \to 0$, it approaches a delta function, and the convolution becomes negligible. At finite temperature, however, the \textit{dI-dV} smeares out, by the convolution. The energy resolution is then limited by thermal broadening, given approximately by $\Delta E_\text{th} \approx 3.5\,k_\mathrm{B}T$, which sets the smallest energy scale over which spectral features can be resolved.

        The expressions derived above not only describe the origin of tunneling currents but also provide a direct link to experimental observables. In scanning tunneling microscopy (STM), for example, a normal-metal tip above a superconducting surface forms an NIS junction. Measuring the \textit{I-V} or \textit{dI-dV} characteristics at sufficiently low temperature thus enables a direct mapping of the energy-resolved quasi-particle spectrum of the superconductor. 

        In case of all superconducting tunnel barrier, both electrodes contribute gapped densities of states. Their convolution, together with thermal and lifetime broadening, determines the observed shape of the \textit{I-V} and \textit{dI-dV} curves. 
        \begin{equation}
            I_\mathrm{SS}(V) = \frac{G_\mathrm{N}}{e}\int_{-\infty}^{\infty} \frac{N_\mathrm{S}(E)}{N_0} \cdot \frac{N_\mathrm{S}(E+eV)}{N_0} \cdot \left[f(E) - f(E + eV)\right] \mathrm{d}E
            \label{eq:micro:tunnel-ss}
        \end{equation}
        Importantly, temperature $T$ and Dynes broadening $\Gamma$ influence the spectra in distinct ways. Finite temperature broadens the Fermi edges via the Fermi--Dirac distribution, while $\Gamma$ introduces intrinsic smearing of the quasi-particle DOS itself. By fitting measured \textit{I-V} or \textit{dI-dV} data, with both parameters as variables, one can disentangle thermal effects from genuine lifetime or inelastic processes.

        These interpretations form the foundation of tunneling spectroscopy as a quantitative probe of superconductivity, allowing the extraction of $\Delta_0$, $T$, and $\Gamma$ from experimental data with high accuracy.

        \begin{figure}
            \centering
            \import{theory/micro}{tunnel-current.pgf}
            \caption{Tunnel current, smearing, dos, fermidirac}
            \label{fig:micro:tunnel-current}
        \end{figure}

        \textbf{cite: Giaever}

    \subsection{Photon-Assisted Tunneling}
    \label{subsec:micro:pat}

        While the previous section described static tunneling, the application of a time-dependent voltage enables quasi-particles to exchange discrete energy quanta with an external electromagnetic field, giving rise to photon-assisted tunneling (PAT).

        In practice, this effect is typically studied using electromagnetic radiation in the microwave range, since photon energies $h\nu$ in this regime are comparable to the superconducting energy gap $\Delta_0$ and can therefore induce measurable sidebands without breaking the Cooper pairs.

        Possible energies are given by
        \begin{equation}
            E_n = n h\nu\quad (n \in \mathbb{Z})\,,
            \label{eq:micro:pat-En}
        \end{equation}
        where each photon carries an energy $h\nu$ and $n$ is the number of photons absorbed ($n>0$) or emitted ($n<0$). As a result, additional tunneling channels open at these energies, producing characteristic replicas of the coherence peaks in \textit{dI-dV}.

        The Tien--Gordon model provides a simple and intuitive framework to describe this effect. It assumes that the junction operates in the tunneling limit, as described before in Section~\ref{subsec:micro:tunnel-current}. The electromagnetic field is treated classically and is represented by a spatially uniform, time-dependent voltage across the junction,
        \begin{equation}
            V(t) = V_0 + A \cos (2\pi\,\nu t)\,.
            \label{eq:micro:pat-V(t)}
        \end{equation}
        Here $V_0$ is the applied voltage bias, $A$ the amplitude, and $\nu$ the frequency of the microwave field across the junction. The field is assumed to remain unaltered by the tunneling current, meaning no cavity or backaction effects, and the drive is weak enough not to heat the electrodes, allowing both to stay in thermal equilibrium.

        In a gauge where the entire voltage drop appears as a scalar potential, a tunneling electron acquires a time-dependent factor in its wavefunction,
        \begin{equation}
            \psi(t) \propto \exp\!\left(-\frac{i}{\hbar}\int_0^t e V(t')\, dt' \right)\,.
            \label{eq:micro:pat-wf0}
        \end{equation}

        For the harmonic drive of Eq.~\eqref{eq:micro:pat-V(t)}, the integral can be solved by
        \begin{equation}
            \int_0^t e V(t')\, dt' = eV_0 t + \frac{eA}{2\pi\nu}\sin(2\pi\nu t)\,.
            \label{eq:micro:pat-int}
        \end{equation}
        This equantion is further solved by the Jacobi--Anger identity,
        \begin{equation}
            \exp\left(i\alpha \sin(2\pi\nu t)\right) = \sum_{n=-\infty}^{\infty} J_n(\alpha)\, \exp\left(i n 2\pi\nu t\right)\,,
        \end{equation}
        where $\alpha = eA/h\nu$ denotes the dimensionless modulation strength and $J_n(\alpha)$ is the $n$-th Bessel function of first kind.

        The wavefunction becomes then
        \begin{equation}
            \psi(t) \propto \sum_{n=-\infty}^{\infty} J_n(\alpha)\, \exp\!\left( -\frac{i}{\hbar}(eV_0 - n h\nu)\, t  \right)\,,
            \label{eq:micro:pat-wf}
        \end{equation}
        revealing that an electronic state subjected to an AC voltage becomes a superposition of components at energies shifted by $n h\nu$. Each component corresponds to the absorption ($n>0$) or emission ($n<0$)
        of $|n|$ photons. Because tunneling probabilities are proportional to the squared amplitude, these channels carry weights $J_n^2(\alpha)$.

        This perspective permits a particularly transparent interpretation in terms of an effective, photon-dressed density of states. Each shifted wavefunction component contributes a copy of the local DOS displaced by $n h\nu$. The resulting effective DOS is
        \begin{equation}
            N_\mathrm{PAT}(E) = \sum_{n=-\infty}^{\infty} J_n^2(\alpha)\, N(E + n h\nu)\,,
            \label{eq:micro:pat-dos}
        \end{equation}
        which replaces $N(E)$ in the tunneling expression. The PAT-modified current therefore becomes a weighted sum of shifted copies of the static \textit{I-V} curve.

        Putting these contributions together leads to the Tien--Gordon formula,
        \begin{equation}
            I(V_0) = \sum_{n=-\infty}^{\infty} J_n^2\!\left( \frac{eA}{h\nu}\right) \cdot I_0\!\left(V_0 - \frac{n h\nu}{e}\right)\,.
            \label{eq:micro:pat-tien-gordon}
        \end{equation}
        which expresses the total current as the incoherent sum of all photon-assisted tunneling channels.  Each channel contributes a copy of the static \textit{I-V} curve shifted by $n h\nu/e$ and weighted by the probability $J_n^2(\alpha)$ for absorbing or emitting $n$ photons, producing the characteristic sideband structure observed in experiment.

        The same relation also holds for the differential conductance, meaning that the photon-assisted replicas $\mathrm{d}I_0(V_0)/\mathrm{d}V$ appear identically as in the current, providing a direct experimental link to the superconducting density of states.

        In practice, the amplitude $A$ can be determined experimentally by comparing the relative heights of the sidebands with the expected Bessel function weights. The spacing $h\nu$ between sidebands provides a direct and robust calibration of the frequency. Photon-assisted tunneling thus offers a straightforward semiclassical description of how an oscillating field modifies quasi-particle transport, serving as a bridge between static tunneling spectroscopy and driven quantum dynamics.

        
    \newpage

    % !TEX root = ../thesis.tex

%=========================================================
\section{Macroscopic Description}
\label{sec:macro}
%=========================================================

    \begin{wrapfigure}[13]{r}{0.4\textwidth}
        \captionsetup{format=plain}%
        \centering
        \vspace{-1.5em}
        \import{theory/macro}{delta-r.pgf}
        \caption{
            Spatial profile of $\Delta_1(r)$ (\legend{seeblau100}) and $\Delta_2(r)$ (\legend{seegrau80}) across a tunnel junction (\legend{seegrau65}). Their respective magnitude $|\Delta|$ (shaded) varies only weakly accross the barrier. Coherent coupling is governed by the macroscopic phase difference $\phi$.
            }
        \label{fig:macro:delta-r}
    \end{wrapfigure}
    While the previous section described dissipative transport in terms of single-particle tunneling, the complementary low-energy limit is governed by coherent Cooper-pair tunneling. In this regime, the quasiparticle spectrum plays no direct role. Instead, transport is determined solely by the phase of the superconducting gap introduced in Eq.~\ref{eq:micro:complex-delta}. Because the magnitude $|\Delta|$ varies only weakly across a weak link, the relevant dynamical variable is the phase difference
    \begin{equation}
        \phi = \phi_1 - \phi_2\,,
        \label{eq:macro:phase-difference}
    \end{equation}
    which fully determines the supercurrent. The Josephson effect discussed here is treated in the weak-coupling (tunneling) limit, analogous to the quasiparticle tunneling regime introduced in the previous section.
    Figure~\ref{fig:macro:delta-r} shows a schematic illustration of this situation.
 
    A weak link, such as an insulating barrier, a metallic constriction, or a short normal region, allows the pairing potentials of the two superconductors to overlap. Since their amplitudes remain essentially constant on the junction scale, the coupling depends only on the relative phase. This phase difference is the essential quantity governing coherent Cooper-pair tunneling and forms the basis of the Josephson effect.

    The following subsections introduce the two Josephson relations, discuss their physical implications, and develop the framework required to describe microwave-driven junctions and the RCSJ model.

    %=========================================================
    \subsection{Josephson Effect}
    \label{subsec:macro:josephson}
    %=========================================================
        
        \begin{figure}[t]
            \captionsetup{format=plain}%
            \centering
            \subfigure[
                Current-phase relation of a weakly coupled Josephson junction (Eq.~\ref{eq:macro:dc}).
            ]{\import{theory/macro}{josephson-iphi.pgf}}
            \hspace{3mm}
            \subfigure[
                Josephson current over time for a constant voltage $V_0 = \Delta_0/e$ (Eq.~\ref{eq:macro:dc} \& \ref{eq:macro:phi-t}).
            ]{\import{theory/macro}{josephson-it.pgf}}
            \subfigure[
                Combined supercurrent and quasiparticle contribution to the \textit{I--V} characteristic.
            ]{\import{theory/macro}{josephson-iv.pgf}}
            \hspace{3mm}
            \subfigure[
                Temperature dependence of the critical current $I_\mathrm{C}(T)$ (Eq.~\ref{eq:macro:critical-current}).
            ]{\import{theory/macro}{critical-current.pgf}}
            \caption{
                Josephson current over phase (a) and time (b). \textit{I--V} characteristic (c) and temperature dependence of the critical current (d).
            }
            \label{fig:macro:josephson}
        \end{figure}

        When two superconductors are weakly coupled through a thin insulating barrier, constriction, or short metallic link, the amplitudes of their order parameters vary only minimally across the junction. The relevant degree of freedom is therefore the phase, which changes from $\phi_1$ to $\phi_2$ across the weak link. This phase difference $\phi$ governs the supercurrent flowing through the junction, and its maximum magnitude is set by the critical current $I_\mathrm{C}$.

        This perspective emphasizes that the Josephson effect arises from coherent phase coupling between the two superconductors, not from changes in the order-parameter amplitude. The resulting phase-driven transport is captured by two fundamental relations, known as the Josephson equations.

        % DC Josephson
        The DC Josephson effect is described by the current-phase relation (CPR)
        \begin{equation}
            I_\mathrm{J} = I_\mathrm{C}\sin\phi\,,
            \label{eq:macro:dc}
        \end{equation}
        where $I_\mathrm{C}$ denotes the critical current of the weak link, as shown in Figure~\ref{fig:macro:josephson} (a). This CPR states that a dissipationless supercurrent can flow at zero voltage, driven solely by the phase difference between the two superconductors and is a direct consequence of coherent Cooper-pair tunneling. 

        % AC Josephson
        The AC~Josephson relation links the temporal evolution of the phase to the voltage across the junction,
        \begin{equation}
            \frac{\mathrm{d}\phi}{\mathrm{d}t} = \frac{2e}{\hbar}\,V_0\,,
            \label{eq:macro:ac}
        \end{equation}
        implying that a constant voltage causes the phase difference to increase uniformly in time.

        Consequently, the phase evolves linearly,
        \begin{equation}
            \phi(t) = \phi_0 + 2\pi \nu_0t\,,\quad
            \nu_0 = \frac{2e}{h}\,V_0\,,
            \label{eq:macro:phi-t}
        \end{equation}
        which results in an oscillating supercurrent with frequency $\nu_0$. This is illustrated in Fig.~\ref{fig:macro:josephson} (b), where the time-dependent current $I(t)$ is shown together with the corresponding time scale $\Delta t = 1/\nu_0$ set by the Josephson frequency.
        
        % IV Josephson
        In a real junction, quasiparticle tunneling appears in parallel to the phase-driven supercurrent. The resulting \textit{I--V} characteristic, shown in Fig.~\ref{fig:macro:josephson} (c), features a dissipationless supercurrent branch at zero voltage and a dissipative quasiparticle branch that onsets above the gap. This combined response forms the characteristic transport signature of a weakly coupled Josephson junction.
        
        % Ambegaokar–Baratoff
        The critical current is set microscopically by the Ambegaokar--Baratoff (AB) relation,
        \begin{equation}
            I_\mathrm{C}(T) = \frac{\pi}{2}\,\frac{G_\mathrm{N}\Delta(T)}{e}\,
            \tanh\!\left(\frac{\Delta(T)}{2k_\mathrm{B}T}\right),
            \label{eq:macro:critical-current}
        \end{equation}
        shown in Fig.~\ref{fig:macro:josephson} (d). It reflects the BCS temperature dependence of the superconducting gap (Eq.~\ref{eq:micro:DeltaT}) together with the thermal occupation of quasiparticle states. At zero temperature, the expression simplifies to the well known result $I_\mathrm{C} = (\pi/2)\,G_\mathrm{N}\Delta_0/e$.

        % Josephson Coupling Energy
        The strength of phase coupling in a Josephson junction is quantified by the Josephson energy
        \begin{equation}
            E_\mathrm{J}=\frac{\hbar I_\mathrm{C}}{2e}\,,
            \label{eq:macro:josephson-energy}
        \end{equation}
        which sets the energetic stiffness of the phase and thus the robustness of coherent Cooper-pair tunneling. A large $E_\mathrm{J}$ corresponds to a well-defined phase, while a small $E_\mathrm{J}$ makes the junction susceptible to fluctuations.

        Together with the thermal energy $E_\mathrm{T}=k_\mathrm{B}T$ and the charging energy $E_\mathrm{C}=e^2/(2C)$, the Josephson energy sets the scale on which the phase behaves either classically or becomes susceptible to fluctuations. In the regime $E_\mathrm{J}\gg\{E_\mathrm{T},E_\mathrm{C}\}$, the phase is stiff and the Josephson relations hold in their ideal form, yielding a stable supercurrent. At finite temperature, thermally activated phase slips lead to a rounding of the sharp switching expected in an ideal \textit{I--V} curve. This behavior is naturally described within the classical RCSJ framework, which incorporates thermal noise through the resistive branch. In the charge-dominated regime $E_\mathrm{C} \gtrsim E_\mathrm{J}$, quantum and fully stochastic descriptions of phase dynamics go beyond the classical RCSJ model and are discussed in Section~\ref{sec:stochastic}.

        Having established the phase dynamics in static conditions, we now turn to the interplay between the intrinsic Josephson oscillation and external microwave fields.


    %=========================================================
    \subsection{Shapiro Steps}
    \label{subsec:macro:shapiro}
    %=========================================================
        \begin{figure}[t]
            \centering
            \import{theory/macro}{shapiro-ideal.pgf}
            \caption{
                Microwave-driven transport in a weakly coupled Josephson junction. Parameters correspond to aluminum (Sec.~\ref{subsec:basics:aluminum}), with $T=0$, $\gamma=0$, and $\nu=10.0\,\mathrm{GHz}$.
                }
            \label{fig:macro:shaprio-iv}
        \end{figure}

        When a Josephson junction is exposed to an external microwave field, the phase dynamics become modulated by the drive introduced in Section~\ref{subsec:basics:micro-wave}. The Josephson frequency $\nu_0$ mixes with the external frequency $\nu$, and the phase takes the form
        \begin{equation}
            \phi(t)=\phi_0 + 2\pi\nu_0 t + \alpha\sin(2\pi\nu t)\,,
            \label{eq:macro:shapiro-phase}
        \end{equation}
        with $\alpha = 2eA/h\nu$ the dimensionless drive strength.

        Inserting this expression into the Josephson current-phase relation (Eq.~\ref{eq:macro:dc}) and expanding the phase modulation using the Jacobi-Anger identity (Eq.~\ref{eq:microwave:jacobi-anger}) yields harmonics at frequencies $\nu_0 - n\nu$. Whenever the resonance condition $\nu_0 = n\nu$ is satisfied, the $n$-th harmonic becomes stationary and contributes a finite dc component to the current. This produces quantized voltage plateaus at
        \begin{equation}
            V_n = \frac{nh\nu}{2e}\,,
            \label{eq:macro:shapiro-step}
        \end{equation}
        known as Shapiro steps.

        The amplitude of each step is governed by the Bessel weight $J_n(\alpha)$, which reflects the strength of phase modulation by the microwave field. In contrast to the Tien--Gordon description of quasiparticle tunneling, the Bessel functions appear here without being squared because the Josephson current depends linearly on the phase modulation rather than on squared transition probabilities. Consequently, the step amplitudes scale as $J_n(\alpha)$ instead of $J_n^2(\alpha)$. The time-averaged \textit{I--V} characteristic can therefore be written in direct analogy to the Tien--Gordon description,
        \begin{equation}
            I(V_0)
                = \sum_{n=-\infty}^{\infty}
                    J_n(\alpha)\,
                    I_{0}\!\left(V_0 - \frac{nh\nu}{2e}\right),
            \label{eq:macro:shapiro-iv}
        \end{equation}
        where $I_0(V_0)$ denotes the static Josephson \textit{I--V} curve in the absence of microwaves.

        In a real junction, quasiparticle tunneling occurs in parallel with the phase-driven supercurrent. Under microwave irradiation, both Shapiro steps and PAT appear simultaneously. This combined response constitutes the characteristic microwave-driven transport signature of a weakly coupled Josephson junction.
        As illustrated in Fig.~\ref{fig:macro:shaprio-iv}, the interplay of phase locking and photon-assisted tunneling produces a rich structure in both the \textit{I--V} and differential conductance characteristics.


    %=========================================================
    \subsection{RCSJ Model}
    \label{subsec:macro:rcsj}
    %=========================================================    

        The ideal Josephson relations describe how the supercurrent depends on the phase and how the phase evolves under a constant voltage. However, real junctions exhibit dissipation, capacitance, thermal fluctuations, and microwave-driven phase dynamics that cannot be    captured by the ideal equations alone.

        A convenient way to incorporate these additional contributions is to represent the junction as an effective circuit. To describe these effects, the Josephson element    must be embedded into an effective circuit model.

        Throughout this section, the junction is assumed to operate in the tunneling limit and in the classical regime $E_\mathrm{J} \gg {E_\mathrm{C}, E_\mathrm{T}}$, such that phase dynamics are well described by the RCSJ equation without quantum corrections.

        %========================================================= 
        % \subsubsection*{Schaltplan}
        %=========================================================

            \begin{wrapfigure}[12]{r}{35mm}
                \captionsetup{format=plain}%
                \centering
                \vspace{-.5em}
                \import{theory/macro}{rcsj-model.pdf_tex}
                \caption{
                    RCSJ model of a real Josephson junction.
                }
                \label{fig:macro:rcsj}
            \end{wrapfigure}
            The resistively and capacitively shunted junction (RCSJ) model provides the minimal dynamical description of a real Josephson junction. It augments the ideal Josephson element by a normal resistance representing quasiparticle tunneling and a capacitance associated with the junction electrodes, as sketched in Fig.~\ref{fig:macro:rcsj}. These additions capture the essential dissipative and inertial mechanisms that govern the phase dynamics.

            The damping regime encoded in the quality factor $Q$ strongly affects the visibility and stability of microwave-driven features such as Shapiro steps. Overdamped junctions exhibit clean, well-defined plateaus, whereas underdamped junctions show hysteresis and residual phase oscillations that distort the step structure.

            In the following, we derive the RCSJ equation of motion for the phase and introduce the dynamical regimes that arise from the interplay of Josephson nonlinearity, dissipation, and inertia.

        %========================================================= 
        \subsubsection*{Current Bias}
        %=========================================================

            To obtain the phase dynamics of the junction, the currents through the three parallel elements are summed according to Kirchhoff's law,
            \begin{equation}
                I_\mathrm{bias} = I_\mathrm{C}\sin(\phi)
                    + \frac{V}{R}
                    + C\frac{\mathrm{d}V}{\mathrm{d}t}\,.
                \label{eq:macro:rcsj-ibias}
            \end{equation}
            We use $I_\mathrm{bias}$ to denote the externally applied current, distinguishing it from the Josephson supercurrent $I_\mathrm{J}$, the resistive current, and the capacitive displacement current that appear in parallel in the RCSJ model.

            Using the AC~Josephson voltage-phase relation, this expression can be written entirely in terms of the phase,
            \begin{equation}
                I_\mathrm{bias} = I_\mathrm{C} \sin(\phi)
                + \frac{\hbar}{2eR}\,\frac{\mathrm{d}\phi}{\mathrm{d}t}
                + \frac{\hbar C}{2e}\,\frac{\mathrm{d}^2\phi}{\mathrm{d}t^2}
                \,.
                \label{eq:macro:rcsj-cpr}
            \end{equation}
            This is the RCSJ equation of motion and forms the basis for all classical descriptions of Josephson phase dynamics. It contains an inertial term, a damping term, and the nonlinear Josephson restoring force, producing the characteristic tilted-washboard potential and the associated dynamical regimes discussed below.

        %========================================================= 
        \subsubsection*{Dimensionless RCSJ Model}
        %=========================================================

            To analyze the dynamics implied by Eq.~\ref{eq:macro:rcsj-cpr}, it is useful to cast the equation into a dimensionless form. This separates the contributions of inertia, damping, and the nonlinear Josephson term.

            The natural time scale of the junction is set by the Josephson plasma frequency,
            \begin{equation}
                \omega_\mathrm{p}
                    = \sqrt{\frac{2eI_\mathrm{C}}{\hbar C}}\,,
                \label{eq:macro:rcsj-plasma-frequency}
            \end{equation}
            which determines the small-oscillation frequency of the phase in a potential minimum. It is distinct from the AC Josephson frequency, which reflects voltage-driven phase evolution, and instead characterizes the intrinsic resonance of the phase in the absence of bias.

            Introducing the rescaled time $t'=\omega_\mathrm{p} t$ and the normalized current $i = I_\mathrm{bias}/I_\mathrm{C}$ gives the compact dimensionless form
            \begin{equation}
                i
                = \sin\phi
                + \frac{1}{Q}\,\frac{\mathrm{d}\phi}{\mathrm{d}t'}
                + \frac{\mathrm{d}^2\phi}{\mathrm{d}t'^2}\,,
                \label{eq:macro:rcsj-cpr-norm}
            \end{equation}
            where the quality factor
            \begin{equation}
                Q = \sqrt{\frac{2eI_\mathrm{C}R^2C}{\hbar}}
                \label{eq:macro:rcsj-damping}
            \end{equation}
            quantifies the damping of the phase dynamics.

            In this normalized form, the physical regimes become transparent. The second derivative describes inertial motion, $(1/Q)\mathrm{d}\phi/\mathrm{d}t'$ represents viscous damping due to the shunt resistor, and $\sin\phi$ provides the nonlinear restoring force from the Josephson coupling. The single parameter $Q$ therefore distinguishes underdamped ($Q\gg 1$) from overdamped ($Q\ll 1$) junctions and sets the stage for the dynamical behavior discussed next.

        %========================================================= 
        \subsubsection*{Tilted Washboard Potential}
        %=========================================================

            The normalized RCSJ equation~\ref{eq:macro:rcsj-cpr-norm}, admits a mechanical interpretation that provides an intuitive picture of the phase dynamics. The equation is mathematically equivalent to the motion of a particle of unit mass in a tilted, periodic potential subject to viscous damping. 

            The normalized potential is obtained by identifying the restoring force with
            \begin{equation}
                -\partial u/\partial\phi = \sin\phi - i\,,
                \quad
                u = U/ E_\mathrm{J}\,,
            \end{equation}
            which yields
            \begin{equation}
                u(\phi)
                = -\cos\phi \;-\; i\,\phi\,.
                \label{eq:macro:washboard-potential}
            \end{equation}
            In the untilted case ($i=0$) the barrier height is given by $\Delta U = 2E_\mathrm{J}$. This ''tilted washboard potential'' consists of a cosine landscape whose overall slope is controlled by the normalized bias current $i$, as shown in Figure~\ref{fig:macro:rcsj-u-phi}.
            
            \begin{figure}[t]
                \centering
                \import{theory/macro}{u-phi.pgf}
                \caption{
                    Tilted washboard potential of the RCSJ model (Eq.~\ref{eq:macro:washboard-potential}) for increasing normalized bias current $i$. For $i=0$ (\legend{seeblau100}) the phase is trapped in a stable minimum. At intermediate tilt, $0<i<1$ (\legend{seeblau65}), the reduced barrier enables thermally activated or noise-driven phase slips (\legend{seegrau35}) between adjacent minima. At $i=1$ (\legend{seeblau35}) the barriers vanish and the phase runs downhill, corresponding to the finite-voltage state.
                    }
                \label{fig:macro:rcsj-u-phi}
            \end{figure}

            For small bias, $i<1$, the potential exhibits a series of metastable minima. In this regime, the phase can remain localized in one of these wells, corresponding to the zero-voltage superconducting branch of the \textit{I--V} characteristic. Small oscillations of the phase around the minimum occur at the plasma frequency defined in Eq.~\ref{eq:macro:rcsj-plasma-frequency}, but the average voltage remains zero.

            As the bias is increased, the tilt of the potential grows and the barriers separating adjacent minima are reduced. At $i=1$, the barriers vanish and the phase becomes free to run down the potential without encountering any local minima. This running solution corresponds to the finite-voltage, resistive state of the junction. In the intermediate regime, fluctuations, either thermal or current-induced, can cause the phase to escape from a metastable minimum even for $i<1$, giving rise to switching from the superconducting to the resistive branch.
        
            A finite phase momentum acquired during escape might, once the bias is reduced again, retrap the phase only at a smaller tilt, leading to a retrapping back to the superconducting branch.

            The washboard picture thus provides a unified interpretation of phase localization, escape, and running dynamics. It naturally explains the origin of the switching current, the emergence of retrapping current, and the role of thermal activation and noise, which are explored in the following subsections.

        %========================================================= 
        \subsubsection*{Switching Dynamics and Phase Diffusion}
        %=========================================================

            \begin{wrapfigure}[19]{r}{0.4\textwidth}
                \captionsetup{format=plain}%
                \centering
                \vspace{-1em} % fine-tune vertical position
                \import{theory/macro}{rcsj-iv.pgf}
                \caption{
                    \textit{I--V} characteristic of a Josephson junction in the intermediate damping regime ($Q\sim 1$), illustrating the combined effects of phase diffusion and residual inertia. Thermal and current noise produce a finite slope in the zero-voltage branch, while moderate damping leads to a hysteretic transition into the running state.
                }
                \label{fig:macro:rcsj-iv}
            \end{wrapfigure}
            The switching current $I_\mathrm{SW}$ is the experimentally observed current at which a Josephson junction leaves the zero-voltage state and enters the resistive, running-phase regime. Unlike the intrinsic critical current $I_\mathrm{C}$, which is a microscopic parameter set by the Josephson coupling energy, the switching current is a stochastic quantity. Its value is determined by the competition between the decreasing barrier height of the tilted washboard potential and fluctuations that drive premature escape. Thermal activation, current noise from the electromagnetic environment, and the rate at which the bias current is ramped all increase the likelihood of early escape and thus reduce ths switching current below the critical current. Only in the ideal, noise-free limit does the switching current approach the intrinsic critical current.

            In real Josephson junctions the supercurrent branch acquires a finite slope due to thermal and noise-induced fluctuations of the phase. Even for currents below the intrinsic critical value, stochastic forces drive small excursions of the phase within the tilted washboard potential, leading to a finite average phase velocity and hence a small, non-zero voltage. In the RCSJ model this effect is enhanced by the resistive branch in parallel with the Josephson element, which provides a dissipative path whenever the phase is not perfectly static. As a result, the ideal vertical supercurrent step is replaced by a broadened zero-voltage region whose slope grows with temperature, current noise and environmental damping.

        %========================================================= 
        \subsubsection*{Retrapping Dynamics}
        %=========================================================

            After the phase has escaped into the running state, the junction develops a finite average voltage and the phase accelerates down the tilted washboard potential. When the bias current is reduced again, the junction does not immediately return to the zero-voltage state. Instead, the phase continues to move until sufficient kinetic energy has been dissipated for it to fall back into a metastable minimum. The current at which this return to the superconducting branch occurs is the retrapping current $I_\mathrm{R}$.

            In contrast to the switching current, set by escape from a minimum, the retrapping current reflects how efficiently damping removes the kinetic energy accumulated during the running state. Thermal and current noise smooth this transition, producing a broadened return to the superconducting branch rather than a sharp jump.

            A representative \textit{I--V} trace for this intermediate damping regime is shown in Fig.~\ref{fig:macro:rcsj-iv}. It highlights the finite slope of the supercurrent branch due to phase diffusion, the broadened switching transition, and the smooth return at $I_\mathrm{R}$ characteristic of junctions with $Q\!\sim\!1$.

            In the overdamped limit ($Q\ll 1$), the phase has essentially no inertia. As soon as the bias is lowered below the point where the washboard potential recreates minima, the phase immediately relocks. The retrapping current equals the switching current.
            
            In the underdamped limit ($Q\gg 1$), the phase retains significant kinetic energy after escape and cannot relock until the tilt is nearly removed. The retrapping current therefore approaches zero, giving rise to strong hysteresis in the \textit{I--V} characteristic.

        %========================================================= 
        \subsubsection*{Shapiro Steps in the RCSJ Model}
        %=========================================================

            In the full RCSJ model, Shapiro steps arise through frequency locking between the intrinsic Josephson oscillation and an external microwave drive, as described in Section~\ref{subsec:basics:micro-wave}. The normalized RCSJ equation becomes
            \begin{equation}
                i + a\sin(2\pi\nu' t')
                = \sin\phi
                + \frac{1}{Q}\,\frac{\mathrm{d}\phi}{\mathrm{d}t'}
                + \frac{\mathrm{d}^2\phi}{\mathrm{d}t'^2}\,,
                \label{eq:macro:rcsj-driven}
            \end{equation}
            where $i$ is the normalized bias current. The normalized amplitude $a$ and frequency $\nu$ are given by
            \begin{equation}
                a = A/I_\mathrm{C} \sqrt{R^{-2} + \left(2\pi\nu C\right)^2}\,,\qquad
                \nu' = \nu/\nu_\mathrm{p}\,.
            \end{equation}

            The washboard potential then becomes time dependent
            \begin{equation}
                u(\phi)
                = -\cos\phi \;-\; i\,\phi - a\,\phi\sin(2\pi\nu' t')\,.
                \label{eq:macro:rcsj-washboard-potential}
            \end{equation}
            
            Weak damping makes the junction sensitive to residual oscillations, which can distort Shapiro steps and reduce their visibility under microwave irradiation.
            
            In the overdamped regime, damping suppresses inertial phase oscillations and stabilizes the frequency-locked state, yielding clean, well-defined Shapiro steps. Thermal fluctuations broaden the step edges but do not compromise their visibility. This regime most closely reflects the behavior of the atomic aluminum contact studied in this work.

            Figure~\ref{fig:macro:rcsj-shapiro-iv} shows the resulting \textit{I--V} characteristics in the under- and overdamped limit. The phase locks robustly to the external drive, producing a sequence of Shapiro plateaus.
            \begin{figure}
                \centering
                \subfigure[
                    Overdamped junction ($Q\ll 1$), switching current parameter $I_\mathrm{SW}=0.5\,I_\mathrm{C}$.
                    ]{\import{theory/macro/}{shapiro-under.pgf}}
                \subfigure[
                    Underdamped junction ($Q\gg 1$), switching current parameter $I_\mathrm{SW}=0.2\,I_\mathrm{C}$.
                    ]{\import{theory/macro/}{shapiro-over.pgf}}
                \caption{
                    Microwave-driven dc transport of a Josephson junction within the RCSJ model for (a) overdamped and (b) underdamped phase dynamics. Shown is the normalized bias current $I_\mathrm{bias}/(G_\mathrm{N}\Delta_0/e)$ versus the normalized voltage $eV/\Delta_0$ for drive amplitudes $eA/\Delta_0\in\{0,0.3,0.6\}$ at $\nu=10\,\mathrm{GHz}$. Parameters correspond to aluminum (Sec.~\ref{subsec:basics:aluminum}) with $T=0$ and $\gamma=0$.
                    }
                \label{fig:macro:rcsj-shapiro-iv}
            \end{figure}

            The RCSJ framework bridges the gap between the ideal Josephson prediction of perfectly sharp Shapiro plateaus and the experimentally observed \textit{I--V} characteristics, where damping, noise, and capacitive dynamics shape the visibility and stability of the steps.
    \newpage

    % !TEX root = ../thesis.tex

%=========================================================
\section{Mesoscopic Description}
\label{sec:meso}
%=========================================================

    Summarize this section here.

    %=========================================================
    \subsection{Bogoliubov--de~Gennes Formalism}
    \label{subsec:meso:bdg}
    %=========================================================

    To describe superconductivity in mesoscopic and spatially inhomogeneous structures, we employ the Bogoliubov--de~Gennes (BdG) formalism, i.e. the real-space mean-field formulation of BCS theory. Starting from the BCS pairing Hamiltonian \cite{bardeen_microscopic_1957} and applying the canonical Bogoliubov transformation \cite{bogoljubov_new_1958} in the electron--hole (Nambu) representation \cite{nambu_quasi-particles_1960}, one obtains a quadratic quasiparticle Hamiltonian and an effective eigenvalue problem for the two-component Nambu spinor, as presented systematically by de~Gennes \cite{de_gennes_superconductivity_1966}. 
    
    The BdG framework provides a microscopic description of electron--hole conversion at normal--superconductor interfaces (Andreev reflection) \cite{andreev_thermal_1964} and underlies scattering approaches to transport in hybrid junctions, including the BTK model and its extensions \cite{blonder_transition_1982,beenakker_quantum_1992}.

    
        %=========================================================
        \subsubsection*{Single Particle Hamiltonian}
        %=========================================================

            To establish notation, we start from the normal-state single-particle description in the grand-canonical ensemble, where energies are measured relative to the chemical potential. The corresponding Hamiltonian reads
            \begin{equation}
                \hat{H}_\mathrm{N}(\vec{r}) = -\frac{\hbar^2}{2m}\nabla^2 + U(\vec{r}) - \mu\,,
                \label{eq:meso:h-n}
            \end{equation}
            where the (possibly spatially varying) potential $U(\vec r)$ models confinement, tunnel barriers, or disorder.

            The eigenfunctions of $\hat{H}_\mathrm{N}$ define the natural mode basis of the normal conductor,
            \begin{equation}
                \hat{H}_\mathrm{N}(\vec{r})\, \Phi_k(\vec{r}) = \xi_k\, \Phi_k(\vec{r})\,,\quad
                \xi_k=\varepsilon_k-\mu\,,
                \label{eq:meso:psi-n}
            \end{equation}
            where $\xi_k$ denotes the single-particle energies measured from the chemical potential. 
            
            In a homogeneous metal ($U=0$), the eigenfunctions and eigenvalues are given by
            \begin{equation}
                \Phi_{\vec q}(\vec r) \propto e^{\ima \vec q\cdot \vec r}\,,\quad
                \varepsilon_{\vec q}=\tfrac{\hbar^2 q^2}{2m}\,.
            \end{equation}

            In this translationally invariant limit, the abstract mode label $k$ can be identified with the wave vector $\vec q$. Throughout this chapter we reserve $k$ for a generic mode index and use explicit vectors such as $\vec q$ whenever a plane-wave momentum label is meant. More generally, the label $k$ should be understood as a compact index for the normal-state eigenmodes.

        %=========================================================
        \subsubsection*{BCS Hamiltonian}
        %=========================================================

            Upon second quantization, we introduce fermionic field operators $\hat{\psi}_\sigma(\vec r)$ and $\hat{\psi}_\sigma^\dagger(\vec r)$ that annihilate/create an electron at position $\vec{r}$ with spin $\sigma$. In the normal-state eigenbasis they admit the expansion $\hat\psi_\sigma(\mathbf r)=\sum_k \Phi_k(\mathbf r)\,c_{k\sigma}$, but we will work in real space in the following.

            Superconductivity is incorporated at the mean-field level by decoupling an effective attractive interaction in the spin-singlet pairing channel. This introduces the (generally position-dependent) pair potential $\Delta(\vec r)$, which couples time-reversed states and yields the quadratic mean-field (BCS) Hamiltonian
            \begin{equation}
                \begin{aligned}
                \hat{H}_\mathrm{BCS}(\vec{r}) 
                &= \int \mathrm{d}r^3 \sum_{\sigma=\uparrow, \downarrow} \hat{\psi}_\sigma^\dagger(\vec{r})\, \hat{H}_\mathrm{N}(\vec{r})\, \hat{\psi}_\sigma(\vec{r})\\
                &+ \int \mathrm{d}r^3 \left( 
                    \Delta(\vec{r})\, \hat{\psi}_\uparrow^\dagger(\vec{r})\, \hat{\psi}_\downarrow^\dagger(\vec{r})
                    + \Delta^\ast(\vec{r})\, \hat{\psi}_\downarrow(\vec{r})\, \hat{\psi}_\uparrow(\vec{r})
                    \right)
                \end{aligned}
                \label{eq:meso:h-bcs}
            \end{equation}

            Here $\Delta$ is the complex superconducting order parameter, which in general may depend on position. In the simplest $s$-wave, spin-singlet case considered throughout this thesis, it can be written as $\Delta(\vec r) = |\Delta| \, e^{\ima\phi(\vec r)}$, as introduced in Section~\ref{sec:micro}. It represents the amplitude and phase of the Cooper-pair condensate. The mean-field decoupling generates an additional condensation term, proportional to $|\Delta|^2$ and the inverse pairing interaction. Since it does not affect the quasiparticle eigenproblem, we omit it here.

        %=========================================================
        \subsubsection*{Bogoliubov Transformation (Nambu Formalism)}
        %=========================================================
        
            The bilinear structure of Eq.~\eqref{eq:meso:h-bcs} suggests working in an electron--hole representation, where quasiparticles appear as coherent superpositions of particle and hole degrees of freedom. In the spin-singlet \textit{s}-wave case, and in the absence of spin-dependent fields, the problem separates into two equivalent $2\times 2$ blocks. We work in the reduced Nambu basis
            \begin{equation}
                \hat\Psi(\vec r) =
                \begin{psmallmatrix}
                    \hat{\psi}_{\uparrow}(\vec r) \\
                    \hat{\psi}^{\dagger}_{\downarrow}(\vec r)
                \end{psmallmatrix}\,.
                \label{eq:meso:nambu-operator}
            \end{equation}

            We now diagonalize the quadratic Hamiltonian (Eq.~\eqref{eq:meso:h-bcs}) by introducing a set of fermionic quasiparticle operators $\hat\gamma_k$ labeled by a mode index $k$. Each mode is characterized by two position-dependent $c$-number amplitudes $u_k(\vec r)$ and $v_k(\vec r)$, which encode the electron- and hole-like components of the corresponding Bogoliubov quasiparticle.

            A Bogoliubov quasiparticle mode $k$ is then created by an operator of the form
            \begin{equation}
                \hat{\gamma}_k^{\dagger} = \int \mathrm{d}^3 r\,\left(u_k(\vec r)\,\hat{\psi}^{\dagger}_{\uparrow}(\vec r) + v_k(\vec r)\,\hat{\psi}_{\downarrow}(\vec r)\right),
                \label{eq:meso:bogoliubov-operator}
            \end{equation}
            where the amplitudes $u_k(\vec r)$ and $v_k(\vec r)$ quantify the electron- and hole-like components of the quasiparticle wave function. For compactness, we collect these amplitudes into the Nambu spinor
            \begin{equation}
                \Psi_k(\vec r) =
                \begin{psmallmatrix}
                    u_k(\vec r) \\
                    v_k(\vec r)
                \end{psmallmatrix}\,.
                \label{eq:meso:nambu-spinor}
            \end{equation}

        %=========================================================
        \subsubsection*{Bogoliubov--de~Gennes Equation}
        %=========================================================

            In this reduced Nambu basis, the quadratic mean-field Hamiltonian of Eq.~\eqref{eq:meso:h-bcs} can be written as the Bogoliubov--de~Gennes (BdG) Hamiltonian,
            \begin{equation}
                \hat{H}_\mathrm{BdG}(\vec r) =
                \begin{pmatrix}
                    \hat{H}_\mathrm{N}(\vec r) & \Delta(\vec r) \\
                    \Delta^\ast(\vec r) & -\hat{H}_\mathrm{N}^\ast(\vec r)
                \end{pmatrix}
                \,,
                \label{eq:meso:h-bdg}
            \end{equation}
            whose off-diagonal pairing potential $\Delta(\vec r)$ explicitly couples electron and hole amplitudes.

            The quasiparticle modes are then obtained from the BdG eigenvalue problem
            \begin{equation}
                \hat{H}_\mathrm{BdG}(\vec r) \, \Psi_k(\vec r) = E_k\, \Psi_k(\vec r)\,.
                \label{eq:meso:BdG}
            \end{equation}
            Here the index $k$ labels the resulting quasiparticle eigenmodes, while the explicit argument $\vec r$ denotes their spatial dependence. Because the BdG Hamiltonian possesses an intrinsic particle--hole symmetry, every solution at energy $+E_k$ is accompanied by a partner at $-E_k$ (with exchanged electron and hole components). In the normal-state limit $\Delta\to 0$, Eq.~\eqref{eq:meso:BdG} reduces to the eigenproblem of $\hat{H}_\mathrm{N}$ in Eq.~\eqref{eq:meso:h-n}.

        %=========================================================
        \subsubsection*{Uniform \textit{s}-wave Superconductor}
        %=========================================================

            As a solvable reference case, consider a homogeneous bulk superconductor with constant $U$ and a uniform pair potential $\Delta=|\Delta|e^{\ima\phi}$. In this case the BdG eigenmodes can be labeled by a wave vector $\vec q$ and an electron-like or hole-like branch, and take the plane-wave form
            \begin{equation}
                \Psi_{\vec q}(\vec r) = \Psi_{0}(E_{\vec q})\, e^{\ima \vec q \cdot \vec r}\,,\qquad
                \xi_{\vec q}=\frac{\hbar^2 q^2}{2m}-\mu\,,\qquad
                E_{\vec q}=\sqrt{\xi_{\vec q}^2+|\Delta|^2}\,.
                \label{eq:meso:BdG-plane-wave}
            \end{equation}
            Here $\Psi_{0}(E_{\vec q})$ is a two-component spinor containing the electron and hole amplitudes for the quasiparticle at energy $E_{\vec q}$. Choosing a gauge where $\phi$ is constant, a convenient phase convention is
            \begin{equation}
                \Psi_{0}(E_{\vec q})\equiv
                \begin{psmallmatrix}
                    u_0(E_{\vec q}) \\
                    v_0(E_{\vec q})\,e^{\ima\phi}
                \end{psmallmatrix}\,,
            \end{equation}
            so that the condensate phase appears explicitly in the hole component.
        
            In the bulk description above, quasiparticle states are naturally labeled by the mode index (here the wave vector $\vec q$). In scattering problems, by contrast, one typically works at fixed quasiparticle energy $E$ (set by bias and the reservoir distributions) and distinguishes electron-like and hole-like branches through the sign of the normal-state energy.

            Accordingly, we reparametrize the bulk dispersion relation $E^2=\xi^2+|\Delta|^2$ in terms of
            \begin{equation}
                \xi_{\pm}(E) = \pm\sqrt{E^2-|\Delta|^2}\,.
                \label{eq:meso:xi-branches}
            \end{equation}
            For $|E|<|\Delta|$ the quantity $\xi_{\pm}(E)$ is purely imaginary, reflecting that bulk solutions are evanescent rather than propagating.

            The corresponding coherence factors of a uniform $s$-wave superconductor can then be written in their standard form,
            \begin{equation}
                \begin{aligned}
                    u_0(E) &= \sqrt{\tfrac{1}{2}\left(1+\xi_{\pm}(E)\,/\,E\right)}\,,\\
                    v_0(E) &= \sqrt{\tfrac{1}{2}\left(1-\xi_{\pm}(E)\,/\,E\right)}\,.
                \end{aligned}
                \label{eq:meso:coherence-factors}
            \end{equation}
            Here the choice of branch $\xi_{\pm}(E)$ corresponds to electron-like ($+$) or hole-like ($-$) propagation. These energy-domain expressions will be used throughout the following scattering formulations (e.g. BTK), where one works at fixed $E$ rather than fixed $\vec q$.

        The absence of propagating subgap quasiparticles in a homogeneous superconductor is the microscopic origin of Andreev conversion at an N--S interface. In particular, the spectrum $E^2=\xi^2+|\Delta|^2$ implies the familiar BCS density of states, Equation~\eqref{eq:micro:dos-bcs}, while for $|E|<|\Delta|$ one must describe transport in terms of coherent electron--hole conversion processes.
    
    %=========================================================
    \subsection{Andreev Reflection}
    \label{subsec:meso:ar}
    %=========================================================

        At a normal--superconductor (N--S) interface, quasiparticles with energies $|E|<|\Delta|$ cannot propagate in the superconductor because the BCS quasiparticle continuum starts only at $|E|=|\Delta|$. Instead, the corresponding BdG solutions in the superconducting electrode are evanescent. Subgap transport across the interface is therefore mediated by coherent electron--hole conversion. An incident electron from the normal side can be reflected as a hole while a charge $2e$ is transferred into the condensate. This process is known as Andreev reflection and constitutes the elementary mechanism by which normal-state carriers couple to superconducting correlations \cite{andreev_thermal_1964}.
        
        Andreev reflection provides the dominant subgap transport mechanism in N--S junctions and serves as the microscopic building block for superconducting transport in mesoscopic weak links. In a two-terminal S--N--S geometry, successive Andreev conversions at both interfaces lead to discrete Andreev bound states (ABS) in equilibrium and to multiple Andreev reflection (MAR) under finite bias. The following sections build directly on this Andreev-based picture, developing ABS and MAR from the same underlying mechanism.

        More generally, and in particular for atomic-scale contacts, the same physics is most naturally formulated in terms of scattering channels characterized by their normal-state transmissions $\tau_i$. To make this qualitative picture quantitative and to compute the corresponding \textit{I--V} characteristics across the full transparency range, we now turn to the Blonder--Tinkham--Klapwijk (BTK) model \cite{blonder_transition_1982,beenakker_quantum_1992}.

        %=========================================================
        \subsubsection*{Blonder--Tinkham--Klapwijk Model}
        %=========================================================

            Building on the qualitative picture of Andreev reflection introduced above, the Blonder--Tinkham--Klapwijk (BTK) model provides a quantitative scattering description of an N--S interface with arbitrary transparency $\tau$. It yields the energy-resolved  probabilities for Andreev reflection ($A$), normal reflection ($B$), and transmission into the superconductor ($C$). To incorporate finite quasiparticle lifetimes, we use the phenomenological broadening introduced by Pleceník \textit{et al.}, replacing the quasiparticle energy by $E \rightarrow |E| + \ima\gamma$ \cite{plecenik_finite-quasiparticle-lifetime_1994}. 

            Using the normalization factor $d$, the Andreev and normal reflection amplitudes ($a$, $b$) acquire the compact form,
            \begin{equation}
                \begin{aligned}
                    A(E) &= aa^*\,,\quad
                    a= u_0v_0/d\,,\quad
                    d = \left(u^2_0-(1-\tau)v_0^2\right)/\tau\,,\\
                    B(E) &= bb^*\,,\quad
                    b= -(u_0^2-v_0^2)\left( (1-\tau) + \ima \sqrt{\tau(1-\tau)} \right)\,/\,d\,,
                \end{aligned}
                \label{eq:meso:btk-parameter}
            \end{equation}
            with $u_0$ and $v_0$ as coherence factors, given by Eq.~\eqref{eq:meso:coherence-factors}

            For subgap energies, transmission into the quasi-particle continuum vanishes ($C\to0$), so $A(E) + B(E) = 1$. It is therefore convenient to define the energy-resolved spectral weights $\rho$ of the single-particle (1e) and two-particle (2e) processes,
            \begin{equation}
                \begin{aligned}
                    \rho_\mathrm{1e}(E) &= 1 - A(E) - B(E)\,,\\
                    \rho_\mathrm{2e}(E) &= 2 \, A(E) \,,
                \end{aligned}
                \label{eq:meso:btk-dos}
            \end{equation}
            which quantify the relative importance of normal and Andreev processes. For finite lifetime broadening ($\gamma>0$) this strict relation is relaxed and the effective single-particle weight $\rho_{\mathrm{1e}}$ acquires finite subgap contributions. Although these functions resemble densities of states, they should not be interpreted as the physical BCS DOS. Instead, they represent the BTK spectral kernels entering the current integral (Eq.~\ref{eq:meso:btk-iv}). 

            The current through an N--S junction then follows from the BTK kernel,
            \begin{equation}
                \begin{aligned}
                    I_\mathrm{1e}(V) &= \frac{G_0}{e} \int_{-\infty}^{\infty} \rho_\mathrm{1e}(E) \left(f(E) - f(E + eV)\right) \mathrm{d}E\,,\\
                    I_\mathrm{2e}(V) &= \frac{G_0}{e} \int_{-\infty}^{\infty} \rho_\mathrm{2e}(E) \left(f(E) - f(E + eV)\right) \mathrm{d}E\,,\\
                    I_\mathrm{NS}(V) &= I_\mathrm{1e}(V) + I_\mathrm{2e}(V)\,.
                \end{aligned}
                \label{eq:meso:btk-iv}
            \end{equation}
            Increasing the transparency enhances the weight of Andreev processes, leading to a pronounced subgap conductance and a gradual reduction of the coherence-peak height. In the tunneling limit ($\tau \ll 1$), Eq.~\eqref{eq:meso:btk-iv} reduces to the conventional quasiparticle-tunneling expression, while for $\tau \approx 1$ the transport approaches the Andreev limit, where charge is transferred predominantly in units of $2e$.

            \begin{figure}[t]
                \centering
                \subfigure[Spectral weight of single-particle process]{\import{theory/meso}{btk-1e-dos.pgf}}
                \hfill
                \subfigure[Spectral weight of two-particle process]{\import{theory/meso}{btk-2e-dos.pgf}}
                \subfigure[\textit{I--V} characteristic]{\import{theory/meso}{btk-iv.pgf}}
                \hfill
                \subfigure[\textit{dI--dV} characteristics]{\import{theory/meso}{btk-didv.pgf}}
                \caption{
                    BTK spectral weights of the single-particle (1e) and two-particle (2e) channels and the corresponding \textit{I--V} and \textit{dI--dV} characteristics for various channel transparencies $\tau$. Increasing $\tau$ shifts spectral weight from normal to Andreev processes, leading to enhanced subgap conductance and reduced coherence-peak height. Parameters correspond to aluminum (Sec.~\ref{subsec:basics:aluminum}), with $T=0$ and $\gamma = 0$.
                    }
                \label{fig:meso:btk}
            \end{figure}

            Figure~\ref{fig:meso:btk} summarizes the BTK description of an N--S interface across the full transparency range. Panels (a) and (b) show the energy-resolved spectral weights $\rho_{\mathrm{1e}}(E)$ and $\rho_{\mathrm{2e}}(E)$ (Eq.~\eqref{eq:meso:btk-dos}), which enter the current integral in Eq.~\eqref{eq:meso:btk-iv}. In the tunneling regime ($\tau\ll 1$), the response is dominated by the single-particle channel and exhibits pronounced coherence peaks at $|E|\approx\Delta$. With increasing transparency, spectral weight is transferred to the Andreev channel, yielding enhanced subgap conductance and a reduced peak height, as reflected in the corresponding $I$--$V$ and $\mathrm{d}I/\mathrm{d}V$ curves in panels (c) and (d).

    %=========================================================
    \subsection{Andreev Bound States}
    \label{subsec:meso:abs}
    %=========================================================
    
        \begin{wrapfigure}[18]{r}{0.4\textwidth}
            \captionsetup{format=plain}%
            \centering
            \vspace{-1.5em}
            \import{theory/macro}{delta-r.pgf}
            \textbf{This is a placeholder!}
            \caption{
                Spatial profile of $\Delta_1(r)$ (\legend{seeblau100}) and $\Delta_2(r)$ (\legend{seegrau100}) across a tunnel junction (\legend{seegrau65}). Their respective magnitude $|\Delta|$ (\legend{seeblau35}/\legend{seegrau35}) varies only weakly accross the barrier. Coherent coupling is governed by the macroscopic phase difference $\phi$.
                }
            \label{fig:meso:delta-r}
        \end{wrapfigure}
        Andreev reflection becomes phase-coherent and spectrally quantized when two superconductors are connected by a mesoscopic weak link. In an S--N--S geometry, a subgap quasiparticle ($|E|<|\Delta|$) cannot escape into the superconducting continua and instead undergoes successive Andreev conversions at both interfaces, alternating between electron- and hole-like character while acquiring the superconducting phase difference 
        \begin{equation}
            \phi = \phi_1 - \phi_2\,.
            \label{eq:meso:phase-difference}
        \end{equation}
        The resulting constructive-interference condition quantizes the motion into discrete subgap eigenstates localized around the junction, known as Andreev bound states (ABS) \cite{andreev_thermal_1964,kulik_effect_1969,ishii_josephson_1970}. 
        
        These ABS form the microscopic origin of the dc Josephson effect in short and mesoscopic junctions and their phase-dependent energies $E_n(\phi)$ determine the equilibrium supercurrent via their occupation \cite{beenakker_josephson_1991, beenakker_quantum_1992}.

        %=========================================================
        \subsubsection*{Andreev Approximation and Limit}
        %=========================================================

            A transparent route from BdG to ABS is to treat Andreev reflection as an energy-dependent boundary condition at the superconducting leads, while representing the weak link by its normal-state scattering properties.

            In the Andreev approximation, $|E|,|\Delta|\ll E_\mathrm{F}$, only quasiparticles in a narrow shell around the Fermi surface contribute. For an isotropic parabolic band, the dispersion can be linearized in the radial direction,
            \begin{equation}
                \xi(k) \approx \tfrac{\hbar^2 }{m} \left( {k}_\mathrm{F} \cdot ({k}-{k}_\mathrm{F}) \right)\,,
            \end{equation}
            so that electron-like and hole-like components of a quasiparticle have nearly equal momentum magnitudes. For an excitation at energy $E$ the corresponding wave numbers satisfy
            \begin{equation}
                2\,\delta k = k_\mathrm{e} - k_\mathrm{h} \approx k_\mathrm{F}\, E/ E_\mathrm{F}\,,
            \end{equation}
            implying that the Andreev-reflected hole approximately retraces the incoming trajectory, so called retroreflection, up to corrections of order $E/E_\mathrm{F}$.

            In the Andreev limit of a clean, specular, and highly transparent N--S interface with negligible normal reflection, subgap conversion occurs with approximately unit probability. The corresponding Andreev reflection amplitude for an electron being reflected as a hole at the electrode can then be written as a pure phase factor,
            \begin{equation}
                r_\mathrm{A}(E) = e^{-\ima\phi_\mathrm{A}(E)}\,e^{-\ima\phi}\,,\qquad
                \phi_\mathrm{A}(E)\equiv\arccos\!\left(E/|\Delta|\right)\,.
                \label{eq:meso:andreev-leads}
            \end{equation}
            Here $\phi_\mathrm{A}(E)$ is the energy-dependent Andreev phase, while $\phi$ denotes the condensate phase of the superconducting electrode. For nonideal interfaces, finite backscattering reintroduces normal reflection and reduces the Andreev amplitude below unit.

        %=========================================================
        \subsubsection*{Energy Spectra and Current Phase Relation}
        %=========================================================

            \begin{figure}[t]
                \centering
                \subfigure[
                    Energy Spectra
                    ]{\import{theory/meso}{abs-Ephi.pgf}}
                \hfill
                \subfigure[
                    Current Phase Relation
                    ]{\import{theory/meso}{abs-Iphi.pgf}}
                \caption{
                    Andreev bound states and equilibrium current--phase relation of a short single-channel contact for several transparencies $\tau$ (light: tunnel-like, dark: near ballistic) at $T=0$. 
                    (a) Phase dispersion of the Andreev levels $E_\pm(\phi)$. With increasing $\tau$, the levels acquire a stronger phase dependence and the minimal splitting at $\phi=\pi$ decreases, approaching a level crossing in the ballistic limit.
                    (b) Corresponding equilibrium supercurrent $I(\phi)$ obtained from the occupied branch via Eq.~\eqref{eq:meso:cpr-single}. Increasing $\tau$ drives the current--phase relation away from the sinusoidal tunnel limit towards a strongly skewed, cusp-like form near $\phi=\pi$, reflecting the increasingly sharp phase dispersion of the bound state.
                }
                \label{fig:meso:abs}
            \end{figure}

            A complete Andreev cycle through the junction converts an electron into a hole at one interface and back into an electron at the other. A bound state forms when this closed electron--hole trajectory reproduces itself, i.e. when the total phase accumulated in one round trip is an integer multiple of $2\pi$.

            In the short-junction limit, the dwell time through the normal region is much shorter than $\hbar/|\Delta|$. The energy dependence of the normal-region scattering can then be neglected for quasiparticle energies $|E|\lesssim|\Delta|$, and the weak link is fully characterized by the transmission eigenvalues $\tau_i$ of its normal-state scattering matrix.

            For a single channel of transmission $\tau$, the constructive-interference condition reduces to the compact form
            \begin{equation}
                \sin^2\!\phi_\mathrm{A}(E) = \tau\,\sin^2\!\left(\phi/2\right)\,,
                \label{eq:meso:abs-quantization}
            \end{equation}
            which directly yields the familiar Andreev bound-state spectrum 
            \begin{equation}
                E_{\pm}(\phi)=\pm|\Delta|\sqrt{1-\tau\sin^2\!\left(\phi/2\right)}\,.
                \label{eq:meso:abs-spectrum}
            \end{equation}
            Figure~\ref{fig:meso:abs}(a) shows the spectura for different transmissions.

            The phase dependence of the ABS spectrum implies that these states carry a nondissipative supercurrent. In equilibrium, the current follows from the derivative of the junction free energy with respect to the superconducting phase difference.

            Equivalently, one may express the contribution of each ABS branch through its occupation, which yields the standard relation
            \begin{equation}
                I(\phi)=-2\pi \frac{G_0}{e} \frac{\partial E(\phi)}{\partial \phi}\,\tanh\!\left(\frac{E(\phi)}{2k_\mathrm{B}T}\right)\,.
                \label{eq:meso:cpr-general}
            \end{equation}
            Here $E(\phi)$ denotes the positive-energy ABS spectrum, and the hyperbolic tangent accounts for thermal occupation.

            Inserting Eq.~\eqref{eq:meso:abs-spectrum} yields
            \begin{equation}
                I(\phi)=\frac{\pi}{2}\frac{G_0}{e}|\Delta|\,
                \frac{\tau\sin\phi}{\sqrt{1-\tau\sin^2\!\left(\phi/2\right)}}\,
                \tanh\!\left(\frac{|\Delta|\sqrt{1-\tau\sin^2\!\left(\phi/2\right)}}{2k_\mathrm{B}T}\right)\,.
                \label{eq:meso:cpr-single}
            \end{equation}
            In the zero-temperature limit, $\tanh\!\left(E/2k_\mathrm{B}T\right)\to 1$ and the current is set solely by the phase dispersion of the bound state, as shown in Fig.~\ref{fig:meso:abs}(b).

            In the tunneling limit ($\tau\ll 1$), the current--phase relation becomes sinusoidal and thus reduces to the standard Josephson form discussed in Sec.~\ref{sec:macro}. In the same limit, the corresponding critical current is consistent with the Ambegaokar--Baratoff result when expressed in terms of the normal-state resistance.

            In the opposite ballistic limit ($\tau\to 1$), the bound-state spectrum becomes particularly simple,
            \begin{equation}
                E_{\pm}(\phi)=\pm|\Delta|\,\left|\cos\!\left(\phi/2\right)\right|\,.
                \label{eq:meso:abs-ballistic}
            \end{equation}
            The corresponding current at $T=0$ follows from Eq.~\eqref{eq:meso:cpr-single},
            \begin{equation}
                I(\phi)=\tfrac{\pi}{2}\tfrac{G_0}{e}|\Delta|\,\sin\!\left(\phi/2\right)\,\mathrm{sgn}\!\left(\cos\!\left(\phi/2\right)\right)\,.
                \label{eq:meso:cpr-ballistic}
            \end{equation}
            Here, the current exhibits a cusp at $\phi=\pi$ originating from the crossing of the ABS branches. Any deviation from perfect transmission opens a finite minimum gap $E_\mathrm{min}=|\Delta|\sqrt{1-\tau}$ at $\phi=\pi$ and smooths this feature, while finite temperature further rounds it through the occupation factor.

        %=========================================================
        \subsubsection*{Multi-Channel Contacts}
        %=========================================================

            Real atomic contacts generally support several conduction channels. In the short-junction limit, each channel $i$ is characterized by a transmission eigenvalue $\tau_i$ and supports a pair of Andreev branches $E_{i,\pm}(\tau_i,\phi)$ given by Eq.~\eqref{eq:meso:abs-spectrum}. The equilibrium supercurrent follows by summing the single-channel contribution (Eq.~\eqref{eq:meso:cpr-single}) over all channels,
            \begin{equation}
                I(\phi,T)=\sum_i I(\tau_i,\phi,T)\,.
                \label{eq:meso:cpr-multichannel}
            \end{equation}
            In the many-mode limit, it is often convenient to replace the discrete set $\{\tau_i\}$ by a statistical distribution $\rho(\tau)$ and approximate the channel sum by an average,
            \begin{equation}
                I(\phi,T)\simeq \int_0^1 \rho(\tau)\, I(\tau,\phi,T)\,\mathrm d\tau\,.
                \label{eq:meso:cpr-multichannel-rho}
            \end{equation}
            In this representation, $\rho(\tau)$ is normalized such that $G_\mathrm N=G_0\int_0^1 \rho(\tau)\,\tau\,\mathrm d\tau$ \cite{beenakker_josephson_1991,beenakker_quantum_1992}.

            In the dirty short-contact limit, the transmission eigenvalues of a quasi-one-dimensional diffusive wire follow the Dorokhov--Mello--Pereyra--Kumar (DMPK) distribution \cite{dorokhov_transmission_1982},
            \begin{equation}
                \rho(\tau) = \frac{G_\mathrm{N}}{G_0} \frac{1}{2\tau \sqrt{1-\tau}}\,,\qquad 0<\tau<1\,.
                \label{eq:meso:dmpk}
            \end{equation}
            The density is bimodal, with integrable divergences at $\tau\to 0$ and $\tau\to 1$, as shown in Fig.~\ref{fig:meso:abs-rhotau}. This indicates that a diffusive conductor can be viewed statistically as a mixture of many almost-closed and a few almost-open channels.
            \begin{figure}[ht]
                \centering
                \includegraphics[width=.42\textwidth]{theory/meso/abs-rhotau.png}
                \caption{DMPK distribution $\rho(\tau)$ of transmission eigenvalues for a short diffusive wire. The distribution is bimodal with integrable divergences at $\tau\to 0$ and $\tau\to 1$, and is normalized such that $G_\mathrm N=G_0\int_0^1 \rho(\tau)\,\tau\,\mathrm d\tau$.}
                \label{fig:meso:abs-rhotau}
            \end{figure}
                        
            Averaging the short-junction single-channel result over Eq.~\eqref{eq:meso:dmpk} yields the first Kulik--Omel'yanchuk (KO--1) current--phase relation in the dirty (diffusive) limit. At $T=0$, it reduces to the closed form
            \begin{equation}
                I_\mathrm{KO1}(\phi) = \tfrac{\pi}{2} \tfrac{G_\mathrm{N}}{e}|\Delta|\,\cos\!\left(\phi/2\right)\,\operatorname{artanh}\!\left(\sin\!\left(\phi/2\right)\right)\,.
                \label{eq:meso:ko-1}
            \end{equation}
            Although $\operatorname{artanh}[\sin(\phi/2)]$ diverges as $\phi\to\pi$, the prefactor $\cos(\phi/2)$ vanishes such that the product, and therefore the supercurrent, remains finite.

            In the complementary clean short-contact limit, transport is dominated by (nearly) open channels and the current--phase relation (KO--2) takes the compact finite-temperature form
            \begin{equation}
                I_\mathrm{KO2}(\phi, T) = \pi\,\frac{G_\mathrm{N}}{e}\,\Delta(T)\,\sin\!\left(\frac{\phi}{2}\right)\,
                    \tanh\!\left(\frac{\Delta(T)\cos\!\left(\phi/2\right)}{2k_\mathrm{B}T}\right)\,.
                \label{eq:meso:ko-2}
            \end{equation}
            Both limits provide widely used reference models for short superconducting weak links \cite{kulik_contribution_1975,golubov_current-phase_2004}.

            Figure~\ref{fig:meso:abs-ko} compares the temperature dependence of the critical current obtained from the Ambegaokar--Baratoff result and from the KO--1/KO--2 current--phase relations by applying $I_\mathrm{C}(T)=\max_{\phi\in[0,2\pi]} I(\phi,T)$ with a BCS-like gap $\Delta(T)$.
            \begin{figure}[ht]
                \centering
                \includegraphics[width=.55\textwidth]{theory/meso/abs-ko.png}
                \caption{Temperature dependence of the critical current $I_\mathrm{C}(T)$ for three canonical short-junction limits: Ambegaokar--Baratoff (tunnel junction), KO--1 (diffusive short contact), and KO--2 (clean short contact). The critical current is obtained by maximizing the corresponding current--phase relation with respect to $\phi$ using a BCS-like gap $\Delta(T)$.}
                \label{fig:meso:abs-ko}
            \end{figure}


        % %=========================================================
        % \subsubsection*{Fractional Shapiro Steps}
        % %=========================================================

            % For intermediate and increasing transmission ($0<\tau<=1$), the progressive skewing of the current--phase relation, can equivalently be expressed as an enhanced higher-harmonic content in a Fourier expansion $I(\phi)=\sum_{n\ge 1} I_n\sin(n\phi)$. This harmonic content will be central when discussing microwave-driven phase dynamics and fractional Shapiro steps below.
    \newpage

    % !TEX root = ../thesis.tex

\section{Stochastic Description}
\label{sec:stochastic}
    
    In the previous sections superconducting transport was described in three complementary ways. The microscopic picture treats it as incoherent quasiparticle tunneling, the macroscopic picture describes it as coherent Cooper-pair dynamics governed by the superconducting phase, and the mesoscopic picture considers coherent electron-hole processes in high-transmission weak links. All of these descriptions rely on a well-defined superconducting phase and on the coherence of successive tunneling events. When the electromagnetic environment generates strong voltage fluctuations, this phase coherence is lost. The junction then enters the stochastic transport regime.

    In the stochastic regime, the loss of phase coherence plays a central role. The superconducting phase $\phi$ is conjugate to the charge $q$ on the junction,
    \begin{equation}
        \left[\phi, q\right] = 2e\ima\,,
        \label{eq:stochastic:phase-charge}
    \end{equation}
    implying that a well-defined phase requires charge to be delocalized, whereas localized charge leads to strong phase fluctuations. Any electromagnetic environment connected to the junction produces voltage fluctuations $\delta V(t)$, which translate into fluctuations of the phase via the AC Josephson relation (Eq.~\ref{eq:macro:ac}),
    \begin{equation}
        \delta \phi(t) = \frac{2e}{\hbar}\int_0^t \delta V(t')\,\mathrm{d}t'\,.
    \end{equation}

    Large phase fluctuations randomize the phase so strongly that the average of the phase-dependent factor vanishes
    \begin{equation}
        \left\langle \exp(\ima\phi(t))\right\rangle = 0\,.
    \end{equation}
    This loss of a non-zero phase average signals the destruction of long-range phase coherence. Consequently, the coherent Josephson effect, Andreev bound states, and multiple Andreev reflections no longer exist. Transport becomes a sequence of  incoherent tunneling events whose rates are governed by the energy exchanged with the electromagnetic environment. The crossover between coherent and incoherent regimes is controlled by the strength of the environment. Weak damping ($\mathrm{Re}\,Z(\omega) \ll R_\mathrm{Q}$) preserves phase coherence, whereas strong damping ($\mathrm{Re}\,Z(\omega) \sim R_\mathrm{Q}$) leads to phase diffusion and marks the onset of the stochastic transport regime.

    So, charge transfer no longer proceeds through coherent condensate dynamics or well-defined quasiparticle trajectories, but instead through discrete and statistically independent tunneling events. The electromagnetic environment can absorb or emit energy during each event, so that tunneling rates are determined not only by the electronic density of states but also by the impedance of the surrounding circuit. This interplay is captured by the \textit{P(E)}--theory, which provides the universal framework for describing energy exchange between a tunnel junction and its environment.

    The stochastic description therefore complements the microscopic, macroscopic, and mesoscopic frameworks by covering the fully incoherent limit of superconducting transport. The following sections introduce the origin of phase fluctuations, the \textit{P(E)}--formalism, and the resulting phenomena: dynamical Coulomb blockade of single-electron tunneling\footnote{In the microscopic description, tunneling is elastic and proceeds between well-defined BCS quasi-particle states, which we refer to as quasi-particle tunneling. In contrast, the stochastic description considers single-electron tunneling events dressed by environmental fluctuations, where energy exchange with the environment renders the process inelastic and probabilistic.}, incoherent Cooper-pair tunneling and its photon-assisted counterpart, incoherent Andreev reflection in the absence of phase coherence, and finally the superconducting single-electron transistor as a device in which these processes combine in a controlled and experimentally relevant manner.
    

    \subsection[P(E)--Theory]{\textit{P(E)}--Theory}
    \label{subsec:stochastic:pe-theory}

        In the presence of strong phase fluctuations, charge transport through a junction occurs as a sequence of independent tunneling events. Because the electromagnetic environment can absorb or emit energy during such a process, the tunneling rate is determined not only by the electronic density of states but also by the probability $P(E)$ that the environment exchanges an energy $E$ with the junction. This renders the tunneling process inelastic even at zero temperature and forms the basis of the stochastic description of superconducting transport.

        The statistical properties of these fluctuations are determined by the environmental impedance $Z(\omega)$. Within linear-response theory, the phase correlation function can be written as
        \begin{equation}
            J(t) = \frac{2}{R_\mathrm{Q}} \int_0^\infty \frac{\mathrm{Re}\,Z(\omega)}{\omega}
            \coth\!\left(\frac{\hbar\omega}{2k_\mathrm{B}T}\right) \left(\cos(\omega t) - 1\right)
            - \ima\sin(\omega t)\,\mathrm{d}\omega\,,
            \label{eq:stochastic:Jt}
        \end{equation}
        where $R_\mathrm{Q}$ is the superconducting resistance quantum\footnote{The superconducting resistance quantum is $R_\mathrm{Q} = h/(2e)^2 \approx 6.453\,\mathrm{k\Omega}$, which appears when phase fluctuations couple to Cooper-pair charge $2e$. The quantity $R_0 = h/2e^2 = 12.9\,\mathrm{k\Omega}$ is the resistance quantum for a single spin-resolved electronic channel. In this thesis the conductance quantum is defined as $G_0 = 2e^2/h = 77.48\,\mu\mathrm{S}$, which includes spin degeneracy and corresponds to the conductance of a fully transmitting normal-state channel. $R_\mathrm{Q}$ governs the strength of phase fluctuations in the stochastic description, while $R_0$ and $G_0$ appear in mesoscopic transport and Landauer-type expressions.}.

        The structure of Eq.~\ref{eq:stochastic:Jt} reflects how the electromagnetic environment shapes the phase dynamics. The factor $\mathrm{Re}\,Z(\omega)$ captures the dissipative part of the environment, which determines how strongly voltage fluctuations couple to the junction. It appears divided by $\omega$ because the phase is the time integral of the voltage, so that low-frequency components contribute most strongly to its fluctuations. The thermal occupation of each environmental mode enters through the factor $\coth(\beta\hbar\omega/2)$, ensuring that both quantum and thermal noise are included. The combination $\cos(\omega t)-1$ forms the real part of $J(t)$ and describes phase diffusion, reflecting the loss of phase memory induced by fluctuating voltages. In contrast, the term $-\ima\sin(\omega t)$ yields the imaginary part of $J(t)$ and encodes the phase winding generated by the environment. Altogether, these ingredients ensure that the long-time or low-frequency behavior of the environmental impedance dominates the asymptotic form of $J(t)$.

        The function $P(E)$ is defined as the Fourier transform of $J(t)$, 
        \begin{equation}
            P(E) = \frac{1}{2\pi\hbar} \int_{-\infty}^{\infty} \exp\left(J(t) + \ima E t/\hbar\right)\,\mathrm{d}t\,.
            \label{eq:stochastic:pedef}
        \end{equation}
        This formulation relies on the assumption of a linear, Gaussian environment, so that all phase fluctuations are fully captured by the correlator $J(t)$.
        The shape of $P(E)$ therefore reflects the spectral properties of the environment in a universal way.
        
        An important universal property of the $P(E)$ function is its normalization,
        \begin{equation}
            \int_{-\infty}^{\infty} P(E)\,\mathrm{d}E = 1\,,
            \label{eq:stochastic:pe-normalization}
        \end{equation}
        which follows directly from the definition in Eq.~\ref{eq:stochastic:pedef} and the condition $J(t\!\to\!0)=0$. This ensures that environmental fluctuations redistribute spectral weight among different energy-exchange channels without altering the total tunneling probability.

        In thermal equilibrium the environment additionally satisfies the detailed-balance relation
        \begin{equation}
            P(-E) = e^{-E/k_\mathrm{B}T} P(E)\,,
            \label{eq:stochastic:pe-detailedbalance}
        \end{equation}
        which guarantees thermodynamic consistency of energy exchange processes.

        The rate for a tunneling event\footnote{Throughout this work we distinguish between the Dynes broadening parameter $\gamma$, which enters the quasi-particle density of states in the microscopic tunneling description, and the tunneling rates $\Gamma$ that appear in the stochastic description of incoherent charge transfer. The two quantities are unrelated and refer to different physical mechanisms.},
        \begin{equation}
            \Gamma(V) = \int_{-\infty}^\infty P(E)\,F(E, V)\,\mathrm{d}E\,,
            \label{eq:stochastic:rate}
        \end{equation}
        is obtained from Fermi's golden rule as a convolution of $P(E)$ with the electronic part of the problem $F(E, V)$. It collects the relevant density of states and Fermi functions.
        This expression is completely general and applies to single-electron tunneling, Cooper-pair tunneling, and Andreev processes alike\footnote{The only distinction is the transferred charge $q$ in a tunneling event. The environment couples to the energy $qV$, with $q=e$ for single-electron tunneling and $q=2e$ for Cooper-pair tunneling and Andreev reflection.}.

        Several limiting forms of $P(E)$ are particularly useful for understanding the stochastic transport regime. The \textit{P(E)}-framework applies strictly to tunneling processes, where individual transfer events are well separated and described by Fermi's golden rule. It does not capture coherent multi-particle trajectories such as multiple Andreev reflections, which require a mesoscopic description and finite transparency.
        For a weak electromagnetic environment with, phase fluctuations are small and $P(E)$ becomes sharply peaked around $E=0$, approaching
        \begin{equation}
            P(E) \approx \delta(E) \qquad (\mathrm{Re}\,Z(\omega) \ll R_\mathrm{Q})\,.
            \label{eq:stochastic:pe-coherent}
        \end{equation}
        In this limit tunneling is effectively elastic and coherent transport is recovered.

        In contrast, a strong electromagnetic environment with $\mathrm{Re}\,Z(\omega)\sim R_\mathrm{Q}$ produces large phase fluctuations and a broad $P(E)$, such that tunneling events must exchange energy with the environment. This regime marks the onset of incoherent, environment-assisted charge transfer and underlies phenomena such as dynamical Coulomb blockade and incoherent Cooper-pair tunneling.

        For an Ohmic environment, the low-energy behavior of $P(E)$ exhibits a universal power law,
        \begin{equation}
            P(E) \propto E^{2R/R_\mathrm{Q} - 1} \qquad (E > 0,\ \mathrm{Re}\,Z(\omega) = R)\,,
            \label{eq:stochastic:pe-ohmic}
        \end{equation}
        which reflects the suppression of small-energy exchange by quantum fluctuations. 
        This result follows from the low-frequency limit of an environment with a frequency-independent real impedance, $\mathrm{Re}\,Z(\omega)=R$, and is therefore a property of the $P(E)$ kernel itself rather than a feature of any specific transport process.
        
        The specific consequences of these limiting forms for single-electron tunneling, Cooper-pair tunneling, and subgap Andreev processes are discussed in the subsequent sections.


    \subsection{Dynamical Coulomb Blockade}
    \label{subsec:stochastic:dcb}

        Dynamical Coulomb blockade (DCB) describes the suppression of single-electron tunneling at low bias due to the combined effects of charge quantization and the electromagnetic environment. In contrast to the microscopic description, where quasi-particle tunneling is elastic and governed solely by the electronic density of states, DCB arises when the environment possesses a sufficiently large real impedance such that the transfer of an electron across the junction requires the environment to absorb a finite amount of energy. If this energy is not available, the tunneling event is suppressed.

        The physical origin of dynamical Coulomb blockade can be understood by considering the energetic requirements of a single tunneling event. When an electron traverses the junction, it must transiently raise the voltage across the junction capacitor, which requires an electrostatic energy of the order of $E_\mathrm{C} = e^2/2C$. Because this charge transfer takes place within a quantum circuit, the required energy cannot arise from the junction itself, but must instead be supplied by the surrounding electromagnetic environment. Whether the environment can provide this energy is determined by the probability distribution $P(E)$, whose low-energy weight reflects the extent to which environmental modes can exchange small amounts of energy.
        
        If the real part of the environmental impedance is appreciable at low frequencies, the corresponding suppression of $P(E\!\approx\!0)$ makes it unlikely that the environment can provide the small energy quanta required for low-energy tunneling processes. This not only reduces the differential conductance near zero voltage but more generally suppresses all tunneling events that rely on small energy exchange with the environment. The resulting modification of the \textit{I-V} characteristics is therefore governed entirely by the low-frequency properties of the impedance. This mechanism, rooted in the discrete transfer of charge across the junction and the energetic constraints imposed by the environment, captures the universal and modelindependent essence of dynamical Coulomb blockade.
        
        Whenever the low-frequency part of the environmental impedance is Ohmic, $\mathrm{Re}\,Z(\omega)=R$ for $\omega \ll \omega_c$, the suppression of small-energy exchange produces a characteristic power-law dependence of the differential conductance,
        \begin{equation}
            \frac{\mathrm{d}I(V)}{\mathrm{d}V} \propto V^{2R/R_\mathrm{Q}} \qquad (eV \ll \hbar\omega_\mathrm{c})\,,
            \label{eq:stochastic:dcb-powerlaw}
        \end{equation}
        where $\omega_\mathrm{c}$ denotes the effective high-frequency cutoff of the environment. For a simple RC model one obtains $\omega_\mathrm{c}=1/RC$, but in general $\omega_\mathrm{c}$ is set by the fastest environmental mode for which the Ohmic approximation remains valid. This ''zero-bias anomaly'' directly reflects the low-energy behavior of $P(E)$ and constitutes the most prominent experimental signature of DCB. While Eq.~\ref{eq:stochastic:dcb-powerlaw} applies to a purely Ohmic environment, a zero-bias suppression of the conductance is a generic feature whenever the low-frequency part of the environmental impedance provides dissipation, i.e.\ whenever $\mathrm{Re}\,Z(\omega\to 0)>0$. In this case $P(E\to 0)\to 0$ and a zero-bias anomaly emerges with an exponent determined by the low-frequency behavior of $\mathrm{Re}\,Z(\omega)$.

        For environments with a real impedance that exceeds $R_\mathrm{Q}$ at low frequencies the tunneling electron cannot draw the small amounts of energy needed for charge transfer. The conductance then becomes exponentially small at low bias. In this strong damping limit the charge remains localized on the junction capacitor and the junction behaves as an effective insulator.

        DCB is not restricted to normal-metal junctions. The expression for the tunneling current remains
        \begin{equation}
            I(V) \propto \int_{-\infty}^\infty P(E)\, \frac{N_1(E)}{N_0}\, \frac{N_2(E+eV)}{N_0}\, \left(f_1(E)-f_2(E+eV)\right)\,\mathrm{d}E\,.
            \label{eq:stochastic:dcb-current}
        \end{equation}
        but the electronic factor inherits the density of states of the electrodes. For NN junctions the density of states is constant, for NS junctions it contains the superconducting BCS form on one side, and for SS junctions both electrodes contribute superconducting densities of states. Regardless of the microscopic details, the environment always suppresses single-electron tunneling at low bias through the same mechanism.
        In SIS junctions the single-electron current appears only above the pair-breaking threshold $eV \gtrsim 2\Delta$, but once quasi-particle states are available, the environmental suppression of small-energy exchange acts on the onset in the same universal manner.

        We now apply this general formalism to single-electron tunneling before turning to the corresponding two-electron processes.


        \subsubsection*{Dynamical Coulomb Blockade with Environmental Resonances}

            If the impedance contains discrete resonances that originate from inductive elements or cavity modes, the real part of the impedance acquires sharp features at the corresponding frequencies. These features imprint themselves onto the $P(E)$ function, which develops peaks at the energies of the environmental modes. The resulting $I(V)$ curves contain satellite structures at the same energies. These resonant features arise purely from the structure of the electromagnetic environment and should not be confused with photon-assisted dynamical Coulomb blockade, discussed below, which requires an externally applied microwave drive and produces sidebands at integer multiples of the drive frequency.

            A discrete environmental resonance at frequency $\omega_0$ produces a characteristic structure in the $P(E)$ function. The oscillatory contribution to the phase correlator leads to a Poisson series of sidebands in the energy-exchange probability,  
            \begin{equation}
                P(E) = e^{-\alpha_0} \sum_{n=0}^{\infty} \frac{\alpha_0^{\,n}}{n!}\, P_{\mathrm{Ohmic}}(E - n\hbar\omega_0)\,,
            \end{equation}
            where $\alpha_0 = R/R_\mathrm{Q}$ quantifies the coupling strength to the mode. Each term represents the absorption or emission of $n$ quanta of the environmental resonance. The measurable current inherits the same structure through the convolution with the electronic factor,  
            \begin{equation}
                I(V) = e^{-\alpha_0} \sum_{n=0}^{\infty} \frac{\alpha_0^{\,n}}{n!}\, I_0\!\left(V - \frac{n\hbar\omega_0}{e}\right),
            \end{equation}
            so that satellite peaks appear at voltages shifted by $n\hbar\omega_0/e$. These features arise solely from the internal mode of the environment and do not require an external microwave drive.


        \subsubsection*{Photon-Assisted Dynamical Coulomb Blockade}

            In the presence of microwave irradiation, a tunneling electron may absorb or emit integer multiples of the photon energy $h\nu$. In the stochastic regime this photon-assisted tunneling does not arise from a coherent phase modulation as in the Tien--Gordon description, but instead from additional energy channels in the inelastic $P(E)$-process.  
            In this regime the microwave field is treated as a classical, deterministic voltage modulation superposed on the stochastic environmental fluctuations; no phase-coherent mixing between sidebands occurs. The classical treatment of the microwave field is valid whenever the applied drive contains many photons per cycle, such that quantum fluctuations of the field are negligible compared to the deterministic modulation.

            A tunneling event may therefore exchange an energy $nh\nu$, with both the environment and the microwave field. The corresponding probability is obtained by dressing the environmental
            probability with Bessel weights,
            \begin{equation}
                P(E) = \sum_{n=-\infty}^{\infty} J_n^2\left(\frac{eA}{h\nu}\right)\, P_0(E - nh\nu)\,,
                \label{eq:stochastic:pe-padcb}
            \end{equation}
            where $A$ and $\nu$ are the amplitude and frequency of the applied microwave drive.

            The resulting photon-assisted DCB current generalizes Eq.~\ref{eq:stochastic:dcb-current} by incorporating the additional photon sidebands,
            \begin{equation}
                I(V_0) = \sum_{n=-\infty}^{\infty} J_n^2\!\left( \frac{eA}{h\nu}\right) \cdot I_0\!\left(V_0 - \frac{n h\nu}{e}\right)\,.
                \label{eq:stochastic:padcb}
            \end{equation}
            Photon-assisted DCB therefore combines environmental energy exchange with photon-assisted processes in a fully incoherent manner. In contrast to coherent PAT in the microscopic Tien--Gordon picture, no phase-coherent sideband mixing occurs. Instead, microwaves redistribute weight among the inelastic channels of the $P(E)$ kernel. This produces microwave-induced replicas of the DCB-suppressed current without generating coherent Shapiro-like features.

            DCB constitutes the single-electron $q=e$ counterpart of incoherent Cooper-pair tunneling, which is discussed next. Whereas DCB suppresses the tunneling of individual electrons, the same environmental interaction enables incoherent $q=2e$ charge transfer in superconducting junctions. This forms the basis of incoherent Cooper-pair tunneling, discussed in the following section.


    \subsection{Incoherent Two-Electron Tunneling}
    \label{subsec:stochastic:2e}

        In the stochastic regime, the superconducting phase becomes fully randomized by the electromagnetic environment, and any phase-coherent mechanism—Josephson tunneling, Andreev bound states, and multiple Andreev reflections—is suppressed. Nevertheless, the transfer of a charge $(2e)$ across a voltage-biased junction remains possible through inelastic, stochastic tunneling events. These processes form a unified class of incoherent two-electron tunneling, in which the junction exchanges an energy $(2e)V$ with its environment. Their rates are governed by the same $P(E)$ kernel as in dynamical Coulomb blockade, but evaluated at the doubled energy scale associated with the transferred charge.

        Despite their different microscopic origins, incoherent Cooper-pair tunneling (SS junctions) and incoherent Andreev reflection (NS junctions) share the same universal structure:
        \begin{equation}
            \Gamma_{2e}(V) = \int_{-\infty}^{\infty} P(E)\,F_{2e}(E,V)\,\mathrm{d}E\,,
            \label{eq:stochastic:2e-general}
        \end{equation}
        with the environment coupling to the charge $(2e)$ through the argument of the $P(E)$ function. The only distinction lies in the electronic prefactor $F_{2e}(E,V)$, which encodes the microscopic structure of the junction.
        
        The stochastic regime treats all $(2e)$ processes on equal footing: both incoherent Cooper-pair tunneling in SS junctions and incoherent Andreev reflection in NS junctions are described by the same charge-$(2e)$ energy-exchange kernel $P(E)$, with only their electronic prefactors differentiating them. The following subsections describe these two cases in detail.

        \subsubsection*{Incoherent Cooper-Pair Tunneling}

            In a SS tunnel junction the coherent Josephson effect is destroyed once environmental phase fluctuations become strong enough that the phase-dependent part of the Cooper-pair tunneling amplitude averages to zero
            \begin{equation}
                \left\langle \exp(\ima\phi(t))\right\rangle = 0\,.
            \end{equation}
            The Josephson coupling energy $E_\mathrm{J}=\hbar I_\mathrm{C}/2e$ then no longer produces a DC supercurrent but enters the tunneling rate in second order. Since a Cooper pair of charge $(2e)$ is transferred between the two condensates, the corresponding electronic factor takes the simple form
            \begin{equation}
                F_\mathrm{ICPT}(E,V) = \frac{\pi E_\mathrm{J}^{2}}{2\hbar} \left(\delta(E-2eV) - \delta(E+2eV)\right)\,,
                \label{eq:stochastic:icpt-F2e}
            \end{equation}
            which, inserted into Eq.~\ref{eq:stochastic:2e-general}, yields the standard expression for the incoherent Cooper-pair tunneling rate and by multiplying by the transferred charge gives the ICPT current,
            \begin{equation}
                I_\mathrm{ICPT}(V)
                = \frac{\pi E_\mathrm{J}^{2}}{\hbar}
                \left(1 - \exp(-2eV/k_\mathrm{B}T)\right) P(2eV)\,.
                \label{eq:stochastic:icpt-current}
            \end{equation}

            The low-bias suppression of $I_\mathrm{ICPT}(V)$ is therefore entirely governed by the behavior of $P(E)$ near $E=0$. For Ohmic environments this leads to the universal power law $I\propto V^{2R/R_\mathrm{Q}-1}$, identical to the dynamical Coulomb blockade of single-electron tunneling but evaluated at the doubled charge $(2e)$.


        \subsubsection*{Incoherent Andreev Reflection}

            At an NS interface, subgap transport is governed microscopically by Andreev reflection: an incoming electron from the normal electrode is retroreflected as a hole, while a Cooper pair is injected into the superconductor. In the coherent BTK description this process relies on well-defined electron-hole amplitudes. Once environmental phase fluctuations destroy coherence, Andreev reflection remains possible but becomes a stochastic, inelastic two-electron tunneling process.

            In the tunneling limit, the coherent BTK structure collapses to a single effective second-order tunneling parameter, and the electronic factor becomes
            \begin{equation}
                F_\mathrm{IAR}(E,V) = \frac{\tau^2 G_0}{e^{2}} \left(f(E-eV) - f(E+eV)\right)\,,
                \label{eq:stochastic:iar-F2e}
            \end{equation}
            where $f(E)$ is the Fermi function of the normal electrode. Inserting Eq.~\ref{eq:stochastic:iar-F2e} into the general 2e formula \ref{eq:stochastic:2e-general} results in the corresponding current,
            \begin{equation}
                I_\mathrm{IAR}(V) = \frac{2\tau^2 G_0}{e} \int_{-\infty}^{\infty} P(E)\,\left(f(E-eV)-f(E+eV)\right)\,\mathrm{d}E\,.
                \label{eq:stochastic:iar-current}
            \end{equation}

            As in ICPT, low-bias IAR is suppressed by $P(E\to 0)$. The only difference between IAR and ICPT lies in the electronic factor: the normal metal provides the Fermi functions $f(E\pm eV)$, whereas ICPT involves only the condensate wave functions encoded in $E_\mathrm{J}^{2}$.

            Finally, it is important to clarify why neither coherent Andreev reflection nor multiple Andreev reflections survive in a SS junction once the system enters the stochastic regime. In an SS junction both electrodes possess gapped BCS densities of states, so no normal-metal continuum is available to support elastic electron-hole conversion inside the gap. Coherent Andreev reflection relies on well-defined electron and hole amplitudes with a fixed superconducting phase, and multiple Andreev reflections require an extended sequence of such coherent conversions. Strong environmental phase fluctuations destroy this phase coherence, making the amplitudes of successive conversion events uncorrelated. As a result, MAR processes are fully suppressed, and even single Andreev reflection has no independent meaning in SS junctions. The only remaining subgap transport channel is therefore the incoherent transfer of a Cooper pair between the two condensates, described by incoherent Cooper-pair tunneling (ICPT).


        \subsubsection*{Photon-Assisted Incoherent Two-Electron Tunneling}

            When a microwave drive is applied, the phase remains fully randomized by the environment, so no Shapiro-like coherent interference occurs. Instead, the drive opens additional inelastic channels. The probability to exchange energy with both environment and microwave field becomes
            \begin{equation}
                P(E) = \sum_{n=-\infty}^{\infty} J_{n}^{2}\!\left(\frac{2eA}{h\nu}\right)P_{0}(E-nh\nu).
                \label{eq:stochastic:pe-pa2e}
            \end{equation}

            Inserting Eq.~\ref{eq:stochastic:pe-pa2e} into the 2e tunneling rate yields the photon-assisted current,
            \begin{equation}
                I_{2e}(V_{0}) = \sum_{n=-\infty}^{\infty} J_{n}^{2}\!\left(\frac{2eA}{h\nu}\right)\, I_{2e,0}\!\left(V_{0}-\frac{nh\nu}{2e}\right),
                \label{eq:stochastic:pa2e}
            \end{equation}
            where $I_{2e,0}(V)$ is either Eqs.~\ref{eq:stochastic:icpt-current} (ICPT) or \ref{eq:stochastic:iar-current} (IAR), depending on the junction type. Thus photon-assisted ICPT and photon-assisted IAR share the same structure, only their electronic prefactors differ.


    \subsection{Superconducting Single-Electron Transistor}
    \label{subsec:stochastic:sset}

        The superconducting single-electron transistor (SSET) combines two fundamental ingredients of mesoscopic charge transport: Coulomb blockade due to charge quantization on a small island, and superconductivity in the source, drain, and island electrodes. To understand its transport characteristics, it is instructive to first recall the basic concepts of static Coulomb blockade and the operation of a normal-state single-electron transistor (SET), before discussing the qualitative differences that arise once all electrodes are superconducting.

        \subsubsection*{Static Coulomb Blockade}

            When a metallic island is isolated by tunnel junctions of capacitance $C_1$ and $C_2$, adding an excess electron requires the electrostatic charging energy 
            \begin{equation}
                E_\mathrm{C} = e^2/(2C_\Sigma)\,, \qquad C_\Sigma = C_1 + C_2 + C_\mathrm{G}\,
            \end{equation}
            where $C_\mathrm{G}$ denotes the gate capacitance. 

            For the static Coulomb blockade, one need to suppress both, thermal and charge fluctuation.
            Thermal fluctuation are suppressed, when the charging energy exceeds the thermal energy $E_\mathrm{C} \gg k_\mathrm{B}T$. Charge Fluctuations are considered to be suppressed usually, when the normal state tunneling resistance exceep the resistance quantum $R_\mathrm{T} \gg R_Q$.
            Only when both criteria are met does the island retain a well-defined integer charge. Adding or removing an electron is both energetically unfavorable and quantum-mechanically unlikely. In this regime charge becomes localized on the island.

            This suppression of charge motion is known as static Coulomb blockade. The island is then characterized by well-defined charge states $n e$ and its electrostatic energy
            \begin{equation}
                E_n(V_\mathrm{G}) = E_\mathrm{C} (n - n_\mathrm{G})^2,
            \end{equation}
            with $n_\mathrm{G} = C_\mathrm{G}V_\mathrm{G}/e$ the dimensionless gate-induced charge. Degeneracy between two charge states occurs at half-integer values of $n_\mathrm{G}$.

        \subsubsection*{Normal-State SET}

            A normal-state single-electron transistor consists of two tunnel junctions in series, each with tunneling resistance $R_{T,i}$ and capacitance $C_i$, and a metallic island between them.  Transport occurs through sequential single-electron transitions $n\!\rightarrow n\pm1$ whenever the electrostatic energy decreases during the tunneling event.

            Instead of writing the general condition $\Delta E<0$, it is more transparent to express the onset of conduction directly through the Coulomb-diamond boundaries.  The island has electrostatic energy $E_n = E_C (n - n_g)^2$, and a tunneling transition becomes energetically allowed when the applied bias provides enough energy to overcome the charging cost. This yields the diamond edges
            \begin{equation}
                |eV| = 2E_C\,\left|\,2n_g - (2n+1)\,\right|,
            \end{equation}
            which delimit regions of blocked and allowed transport in the $(V,V_g)$ plane.  Inside these diamonds no sequential tunneling is possible and the SET is in Coulomb blockade; on the diamond edges, the charge states $n$ and $(n\pm1)$ become degenerate, giving rise to conductance peaks; and outside the diamonds sequential single-electron tunneling provides a finite conductance.

            Outside the Coulomb-blockaded region the SET behaves as two independent tunnel junctions in series. The sequential-tunneling conductance therefore approaches $G_{\mathrm{seq}} = (R_{T,1}+R_{T,2})^{-1}$, which reduces to $G_T/2$ for symmetric junctions.

            Away from the degeneracy points, only higher-order cotunneling processes remain, and the conductance is strongly suppressed.

        \subsubsection*{Superconducting SET (SSET): Why It Is Special}

            When the leads and the island become superconducting, the SET acquires qualitatively new transport channels:
            \begin{enumerate}
                \item \textbf{Cooper-pair tunneling} allows $2e$ charge transfer between neighboring charge states $n\!\to\!n\pm2$. In the absence of phase coherence (as treated in this chapter), these processes occur as incoherent Cooper-pair tunneling (ICPT) with rates governed by the same $P(E)$ kernel that enters single-junction ICPT.
                \item \textbf{Quasiparticle tunneling} becomes possible only above the superconducting energy gap. Single-electron transitions $n\!\to\!n\pm1$ require an energy cost of at least $\Delta$ per broken pair, modifying the Coulomb blockade diagram into superconducting Coulomb diamonds.
                \item \textbf{Parity effects} arise because the island prefers to host an even number of electrons. Near charge degeneracy, odd charge states acquire an additional energy cost of $\Delta$, which modifies the periodicity of the stability diagram and the accessible charge states at low temperature.
            \end{enumerate}
            Together, these ingredients produce a device whose transport is governed not simply by sequential tunneling, but by competing single-electron and two-electron processes, each dressed by the electromagnetic environment through stochastic energy exchange. The SSET therefore naturally combines all mechanisms introduced in this chapter: DCB of quasiparticles, incoherent Cooper-pair tunneling, and incoherent Andreev processes at the NS interfaces between island and leads.

        \subsubsection*{Transport Cycles in the SSET}

            The coexistence of quasiparticle and Cooper-pair processes gives rise to distinct transport cycles, the most prominent being the Josephson--quasiparticle (JQP) cycle, the double Josephson--quasiparticle (DJQP) cycle, and subgap cycles involving incoherent Andreev reflection. These cycles arise when a sequence of energetically allowed transitions forms a closed loop in the $(n,V)$ plane, enabling a steady-state current through the device. In the stochastic regime, the rates of the Cooper-pair steps are governed by ICPT, the quasiparticle steps follow the DCB-modified single-electron tunneling expression, and the cycle current is set by the slowest transition.

            A quantitative description is obtained by writing rate equations for the charge-state occupations and computing the steady-state current, discussed in the following sections.

        \subsubsection*{Coherent Multiple Andreev Reflections in SSETs}

In addition to incoherent processes described by the $P(E)$-framework, 
superconducting single-electron transistors can access a qualitatively 
different regime in which transport is governed by coherent multiple 
Andreev reflections (MAR). This regime appears when the electromagnetic 
environment is weak and the junction transparencies are sufficiently 
large so that the superconducting phase remains well defined over many 
successive electron-hole conversion events.

In a superconducting weak link with transmission $\tau\lesssim 1$, an 
electron incident at subgap energy undergoes repeated Andreev reflections 
at the two NS interfaces. These coherent trajectories gain energy in 
steps of $eV$ and allow charge transfer at bias voltages 
$eV<2\Delta$. The resulting MAR current produces characteristic 
subharmonic gap structures at
\[
    eV = \frac{2\Delta}{n}, \qquad n = 1,2,3,\ldots
\]
which appear as step-like onsets in the current-voltage characteristics 
and as pronounced lines of enhanced conductance in the SSET stability diagram.

\subsubsection*{Experimental observations (Lorenz and Sprenger).}
Coherent MAR in aluminium SSETs was first observed in detail by 
Lorenz~\cite{LorenzPhD} and Sprenger~\cite{SprengerPhD}. Their devices 
featured moderately transparent tunnel junctions and an electromagnetic 
environment with low dissipation, allowing the phase to remain coherent 
over several Andreev cycles. Both theses report clear subharmonic gap 
structure, MAR-induced steps in the $I(V)$ characteristics, and coherent 
interference features indicative of well-preserved superconducting 
phase dynamics.

\subsubsection*{Theoretical and hybrid-regime simulations (Sobral-Rey).}
More recently, Sobral-Rey extended this picture by studying the 
interplay of MAR and Coulomb blockade in hybrid SSET devices in a regime 
that interpolates between pure tunneling and coherent Andreev transport. 
Her simulations and measurements~\cite{SobralReyPRL2024} demonstrate 
how MAR survives in the presence of charging effects, how Coulomb 
blockade modifies the MAR subharmonic structure, and how the onset of 
higher-order MAR processes depends sensitively on junction transparency, 
parity effects, and the detuning from charge degeneracy. These results 
show that MAR remains robust well beyond the simplest weak-coupling 
limit and that charging energy and superconductivity can coexist in a 
nontrivial way in the intermediate-transparency regime.

\subsubsection*{Relation to the stochastic regime.}
Coherent MAR processes rely on a well-defined superconducting phase and 
on the correlated nature of successive electron-hole conversions. 
Therefore, they lie fundamentally outside the applicability of the 
$P(E)$-framework, which assumes statistically independent tunneling 
events. As environmental damping increases or the junction transparency 
is reduced, phase fluctuations destroy this coherence and MAR gradually 
gives way to incoherent two-electron processes (ICPT in SS junctions and 
IAR in NS junctions). MAR therefore marks the opposite, mesoscopic 
extreme of superconducting transport compared to the stochastic regime 
discussed in this chapter.
    \newpage

    \section{Experimental Realization}


    \section*{TODO}
    bin am überlegen die schemata maximal im Anhang zu zeigen (so poblig aus dem Poster screenshoten.) Ich finde man muss viel und lange erklären was wann für vereinfachungen gelten und wie die Bilder gemeint sind. richtige physik lässt sich damit nur schwer machen. Lieber mehr IV zeigen!
    \begin{itemize}
        \item do citations in theory introduction
        \item check consistency:
        \begin{itemize}
            \item quasi-particle
            \item Cooper-pair
            \item aluminum
            \item weak-coupling
            \item normal state
            \item - or --
            \item -- bei Modell namen (Tien--Gordon, Ginzburg--Landau)
        \end{itemize}
        \item check consistency seeblau100
    \end{itemize}

    \subsection*{theory}
    \begin{itemize}
        \item table with theories
    \end{itemize}

    \subsection*{macro}
    \begin{itemize}
        \item overlapping wave functions
        \item rscj shaltplan (mini)
        \item plots (a lot of them)
        \item do the citations
    \end{itemize}

    \subsection*{meso}
    \begin{itemize}
        \item check ABS section
        % \item AR-IV-tau
        % \item AR-IV-m
        % \item MAR-IV-tau
        % \item MAR-IV-m
        \item PAAR
        \item PAMAR
        \item do the citations
    \end{itemize}

    \subsection*{stochastic}
    \begin{itemize}
        \item alles
    \end{itemize}

    \subsection*{experiment and state of the art}
    \begin{itemize}
        \item alles
    \end{itemize}

    \newpage

