% !TEX root = ../thesis.tex
%=========================================================
\chapter{Foundations of Superconducting Transport}
\label{ch:foundations}
%=========================================================
% This chapter introduces the theoretical and conceptual background
% shared across all experiments. It defines the terminology and key
% models used throughout the thesis.

%---------------------------------------------------------
% Figure Plan:
% 1. Superconducting DOS with Dynes broadening
% 2. Temperature/magnetic-field dependence schematic (gap closing)
% 3. Andreev reflection diagram
% 4. Multiple Andreev reflection schematic I-V curve
% 5. Landauer picture / channel conductance sketch
% 6. Josephson effect schematic (DC & AC, phase evolution)
% 7. Shapiro step illustration
% 8. Tien-Gordon energy sideband diagram
% 9. MCBJ experimental schematic
% 10. Microwave coupling schematic (setup block diagram)
%---------------------------------------------------------

this is a placeholder

%---------------------------------------------------------
\section{Superconductivity and Quasiparticles}
\label{sec:superconductivity}
%---------------------------------------------------------

this is a placeholder

\subsection{BCS Ground State and Excitations}
% Briefly introduce the BCS Hamiltonian and quasiparticle excitation
% spectrum E_k = sqrt(xi_k^2 + Delta^2). Define coherence factors u,v.
% Mention temperature dependence Delta(T).

The microscopic description of superconductivity is provided by the Bardeen–Cooper–Schrieffer (BCS) theory, which assumes that electrons near the Fermi surface form bound pairs of opposite momentum and spin (Cooper pairs). The mean-field BCS Hamiltonian reads
\begin{equation}
    H_\mathrm{BCS} = \sum_{k,\sigma}\xi_k\, c_{k\sigma}^\dagger c_{k\sigma}
    - \sum_k \left( \Delta\, c_{k\uparrow}^\dagger c_{-k\downarrow}^\dagger + \Delta^*\, c_{-k\downarrow} c_{k\uparrow} \right),
\end{equation}
where $\xi_k = \epsilon_k - \mu$ is the single-particle energy measured from the Fermi level and $\Delta$ denotes the superconducting order parameter.  
Diagonalization via the Bogoliubov transformation yields quasiparticle excitations with energy
\begin{equation}
    E_k = \sqrt{\xi_k^2 + |\Delta|^2},
\end{equation}
and coherence factors
\begin{equation}
    u_k^2 = \frac{1}{2}\!\left(1 + \frac{\xi_k}{E_k}\right), \qquad
    v_k^2 = \frac{1}{2}\!\left(1 - \frac{\xi_k}{E_k}\right).
\end{equation}
The superconducting energy gap $\Delta$ depends on temperature and vanishes at the critical temperature $T_\mathrm{c}$. Within the weak-coupling limit this dependence is approximately
\begin{equation}
    \Delta(T) \approx \Delta_0\, \tanh\!\left[1.74\sqrt{\frac{T_\mathrm{c}}{T}-1}\right],
\end{equation}
where $\Delta_0 \simeq 1.76\,k_\mathrm{B}T_\mathrm{c}$ is the zero-temperature gap.

\subsection{Density of States and the Dynes Parameter}
% Write N_S(E) = N_0 |E| / sqrt(E^2 - Delta^2).
% Discuss the divergence at the gap edge and introduce phenomenological
% broadening via E -> E + i Gamma (Dynes parameter).
% Figure: plot of DOS with varying Gamma values.

The excitation spectrum derived above leads directly to a characteristic energy dependence of the quasiparticle density of states (DOS). Within the BCS model, the normalized superconducting DOS is given by
\begin{equation}
    \frac{N_S(E)}{N_0} = 
    \Re\!\left[\frac{|E|}{\sqrt{E^2 - \Delta^2}}\right],
\end{equation}
where $N_0$ is the normal-state DOS at the Fermi level. This expression diverges at the gap edges $E=\pm\Delta$, producing the well-known coherence peaks, and vanishes for $|E|<\Delta$, reflecting the absence of single-particle states inside the energy gap.

\begin{figure}
  \centering
  \includegraphics[width=0.75\linewidth]{theory/dynes-dos/dynes-dos.pdf}% adjust path as needed
  \caption{Superconducting density of states $N_S(E)$ with Dynes broadening for $\Gamma/\Delta\in\{0.00,0.02,0.05,0.10\}$.}
  \label{fig:dos_dynes}
\end{figure}

In real materials the coherence peaks are broadened due to inelastic scattering, finite quasiparticle lifetimes, and inhomogeneities. A convenient phenomenological description was introduced by Dynes \cite{dynes_direct_1978}, who replaced $E \rightarrow E + i\Gamma$ in the BCS expression. The resulting DOS reads
\begin{equation}
    \frac{N_S(E)}{N_0} =
    \Re\!\left[\frac{E+i\Gamma}{\sqrt{(E+i\Gamma)^2 - \Delta^2}}\right],
\end{equation}
where $\Gamma$ is the so‑called Dynes parameter. Increasing $\Gamma$ smears the singularities and introduces a small finite DOS inside the gap. The parameter is thus a useful measure of the effective quasiparticle broadening or dissipation present in an experiment. Figure~\ref{fig:dos_dynes} illustrates the effect of different $\Gamma/\Delta$ ratios on the shape of the superconducting DOS.


%---------------------------------------------------------
\section{Quasiparticle Tunneling in SIS Junctions}
\label{sec:tunneling}
%---------------------------------------------------------

this is a place holder

\subsection{Tunnel Current Formula}
% Write I(V) ∝ ∫ N_S(E) N_S(E+eV)[f(E)-f(E+eV)] dE
% Explain how this leads to suppressed current for |V| < 2Δ/e.
% Mention coherence peaks.
The tunneling current in a superconducting junction can be calculated within the framework of first-order perturbation theory, where the insulating barrier is sufficiently thick to suppress coherent Cooper-pair tunneling. In this limit, the transport is dominated by single-particle tunneling processes. The resulting current through an S--I--S junction is given by the convolution of the densities of states of both electrodes, weighted by the difference of their Fermi–Dirac occupation probabilities,
\begin{equation}
    I(V) \propto \int_{-\infty}^{+\infty} 
    N_S(E)\, N_S(E+eV)\,[f(E) - f(E+eV)]\, dE,
    \label{eq:tunnel_current}
\end{equation}
where $N_S(E)$ is the superconducting density of states and $f(E)$ denotes the Fermi distribution. At zero temperature, current can only flow when filled states in one electrode overlap with empty states in the other. Because no quasiparticle states exist within the energy gap, $|E|<\Delta$, the integral vanishes for $|eV| < 2\Delta$, resulting in a suppressed current around zero bias. Once the applied voltage exceeds $|eV| = 2\Delta$, quasiparticle states begin to overlap, and the current rises steeply. The corresponding differential conductance exhibits sharp coherence peaks at the threshold voltage, directly reflecting the singularities of the superconducting DOS.

At higher bias ($|eV| \gg 2\Delta$), the current approaches the normal-state behavior and scales linearly with $V$. Finite temperature, inelastic scattering, or lifetime broadening (represented by the Dynes parameter $\Gamma$) lead to a rounding of the coherence peaks and a small residual subgap conductance. This model forms the theoretical foundation for the analysis of the experimental $I$--$V$ characteristics presented in Chapter~\ref{ch:tunnelbarrier}.


\subsection{Temperature and Magnetic-Field Dependence}
% Qualitative discussion; schematic figure showing gap closing with T or B.
The superconducting energy gap $\Delta$ is not a fixed quantity but depends sensitively on both temperature and magnetic field. Within the BCS framework, $\Delta(T)$ decreases monotonically with increasing temperature and vanishes at the critical temperature $T_\mathrm{c}$. The temperature dependence can be approximated by the weak-coupling relation
\begin{equation}
    \Delta(T) \approx \Delta_0\, \tanh\!\left[1.74\sqrt{\frac{T_\mathrm{c}}{T}-1}\right],
\end{equation}
which reproduces the experimentally observed behavior for many conventional superconductors. As the temperature approaches $T_\mathrm{c}$, thermal excitations break Cooper pairs, reducing the superfluid density and leading to a continuous closure of the energy gap.

A similar suppression of the superconducting gap occurs under the influence of an external magnetic field. When a field is applied, screening currents generate pair-breaking effects through orbital motion and Zeeman splitting of the electron spins. In a simple phenomenological description, the field dependence of the gap can be expressed as
\begin{equation}
    \Delta(B) = \Delta_0\, \sqrt{1 - (B/B_\mathrm{c})^2},
\end{equation}
where $B_\mathrm{c}$ is the thermodynamic critical field at which superconductivity is destroyed. For thin films or nanostructures, the relevant pair-breaking field can differ from the bulk value due to geometry, disorder, and spin–orbit coupling.

\begin{figure}
  \centering
  \includegraphics[width=0.85\linewidth]{theory/gap-suppression/gap_suppression.pdf}
  \caption{Schematic suppression of the superconducting energy gap $\Delta$ with temperature and magnetic field. 
  Left: BCS-like temperature dependence $\Delta(T)/\Delta_0$. 
  Right: phenomenological field dependence $\Delta(B)/\Delta_0 = \sqrt{1-(B/B_c)^2}$.}
  \label{fig:gap_suppression}
\end{figure}

Figure \ref{fig:gap_suppression} schematically illustrates both dependencies, highlighting the continuous closure of the superconducting gap with increasing temperature or magnetic field. These relations are essential for interpreting experimental $I$--$V$ curves and for distinguishing intrinsic quasiparticle effects from thermal or magnetic suppression of superconductivity.



\subsection{Concept of Photon-Assisted Tunneling}

% Introduce Tien-Gordon qualitatively:
% I(V) = Σ_n J_n^2(α) I_0(V + n ħω/e)
% Defer application to Ch. 3 (tunnel barrier measurements).
When a superconducting tunnel junction is exposed to microwave radiation, the time-dependent voltage across the barrier can be expressed as
\begin{equation}
    V(t) = V_\mathrm{dc} + V_\mathrm{ac}\cos(\omega t),
\end{equation}
where $V_\mathrm{ac}$ and $\omega$ denote the amplitude and angular frequency of the oscillating field, respectively. The oscillatory potential periodically modulates the phase of the electron wavefunction, allowing tunneling processes that involve the absorption or emission of discrete photon energies $n \hbar \omega$.

Within the framework of Tien and Gordon \cite{tien_multiphoton_1963}, the total tunneling current under irradiation can be written as a sum over photon-assisted sidebands,
\begin{equation}
    I(V_\mathrm{dc}) = \sum_{n=-\infty}^{+\infty} J_n^2(\alpha)\,
    I_0\!\left(V_\mathrm{dc} + \frac{n\hbar\omega}{e}\right),
    \label{eq:tien_gordon}
\end{equation}
where $I_0(V)$ is the unirradiated (dark) current-voltage characteristic, $J_n(\alpha)$ are Bessel functions of the first kind, and $\alpha = eV_\mathrm{ac}/\hbar\omega$ is the dimensionless microwave amplitude. The term with index $n$ corresponds to tunneling processes involving the absorption ($n>0$) or emission ($n<0$) of $|n|$ photons.

This formalism predicts that the current-voltage curve under microwave irradiation is a weighted superposition of shifted replicas of the unperturbed $I$--$V$ curve, leading to characteristic satellite peaks in the differential conductance at voltages $V_n = V_\mathrm{dc} + n\hbar\omega/e$. For a superconducting tunnel junction, these sidebands appear symmetrically around the coherence peaks.

The Tien--Gordon model provides a simple yet powerful description of photon-assisted tunneling in the incoherent regime, where the ac field acts as a classical voltage modulation. Its quantitative application to the aluminum tunnel-barrier junctions studied in this work is discussed in Chapter~\ref{ch:tunnelbarrier}.

%---------------------------------------------------------
\section{(Multiple) Andreev Reflection}
\label{sec:mar}
%--------------------------------------------------------

this is a placeholder

\subsection{Andreev Reflection at a Normal-Superconductor Interface}

% Explain the process (electron -> hole + Cooper pair) and phase correlation.
% Figure: schematic of Andreev reflection.
At the interface between a normal metal (N) and a superconductor (S), the continuity of the quasiparticle wavefunction imposes a unique boundary condition on charge transport. When an electron in the normal metal approaches the interface with an energy smaller than the superconducting energy gap ($|E| < \Delta$), it cannot enter the superconductor as a single quasiparticle because no available states exist within the gap. Instead, the electron is reflected as a hole with opposite momentum and spin, while a Cooper pair carrying charge $2e$ is transmitted into the superconductor. This process, known as \emph{Andreev reflection}, ensures charge and momentum conservation across the interface.

In the simplest picture, the incident electron and the reflected hole are phase-correlated; they form the two components of a time-reversed pair. The amplitude of Andreev reflection depends on the transparency of the interface and is maximal for a perfectly clean contact. For a barrier of finite strength, normal reflection competes with Andreev reflection, reducing the probability of Cooper-pair transfer.

This retroreflection mechanism gives rise to a finite conductance at subgap voltages, even when quasiparticle tunneling is forbidden. The probability of Andreev reflection can be derived from the Bogoliubov--de Gennes equations and depends on both the excitation energy and the interface transparency. Andreev reflection is thus the fundamental process linking normal-metal transport to superconducting correlations.

Figure~\ref{fig:andreev_reflection} schematically illustrates the process: an electron incident from the normal metal (blue arrow) is reflected as a hole (orange arrow), while a Cooper pair (green arrow) enters the superconductor. This two-particle conversion process underlies many phenomena discussed later in this thesis, such as multiple Andreev reflection and Josephson transport in atomic-scale contacts.

\begin{figure}[h]
  \centering
  \includegraphics[width=0.8\linewidth]{theory/andreev-reflection/andreev_reflection.pdf}
  \caption{Schematic illustration of Andreev reflection at a normal-superconductor interface. 
  An incoming electron from the normal metal (blue arrow) is reflected as a hole (orange arrow), 
  while a Cooper pair (green arrow) is transmitted into the superconductor.}
  \label{fig:andreev_reflection}
\end{figure}

\subsection{Multiple Andreev Reflection in Superconducting Point Contacts}
% Explain subharmonic gap structure eV = 2Δ/n.
% Figure: schematic current-voltage characteristics showing MAR steps.

When two superconductors are connected through a constriction of atomic dimensions, the electron transport at subgap voltages is governed by multiple Andreev reflection (MAR). In this regime, an electron incident on the interface cannot tunnel directly through the gap but undergoes successive Andreev reflections between the two superconducting electrodes. Each reflection converts an electron into a hole (or vice versa) while transferring a Cooper pair to the condensate, effectively advancing the quasiparticle energy by $eV$ with every traversal of the junction.

After $n$ such reflections, the quasiparticle gains an energy of $neV$ and can finally escape into the continuum when $neV = 2\Delta$. This condition gives rise to the so‑called \emph{subharmonic gap structure} in the $I$–$V$ characteristics, with distinct features appearing at
\begin{equation}
    eV_n = \frac{2\Delta}{n}, \qquad n = 1, 2, 3, \dots
\end{equation}
These structures are a hallmark of coherent superconducting transport and provide direct insight into the transparency of the contact.

For low-transmission junctions ($\tau \ll 1$), the subgap current is weak and dominated by two‑particle tunneling ($n=1$). As the transmission increases, higher‑order MAR processes become more pronounced, producing a series of steps or peaks in the differential conductance at voltages $2\Delta/n$. In the fully transparent limit ($\tau \to 1$), these discrete features merge into a smooth subgap current approaching the Andreev limit.

Figure~\ref{fig:mar_iv} shows a schematic current–voltage characteristic highlighting the typical subharmonic gap structure resulting from multiple Andreev reflection. The evolution and relative strength of these features as a function of channel transmission form an important diagnostic tool for determining the microscopic properties of superconducting point contacts.

\begin{figure}[h]
  \centering
  \includegraphics[width=0.75\linewidth]{theory/mar-iv/mar-iv.pdf}
  \caption{Schematic current–voltage characteristics of a superconducting point contact showing 
  multiple Andreev reflection (MAR). Subharmonic features appear at voltages $eV = 2\Delta/n$, 
  corresponding to processes involving $n$ successive Andreev reflections.}
  \label{fig:mar_iv}
\end{figure}

\subsection{Transmission Channels and the Landauer Picture}
% Introduce G = G_0 Σ_i τ_i.
% Extend to superconducting case: each τ_i defines a transport channel.
% Define "pincode" {τ_i}, foreshadowing Ch. 4.

The Landauer picture provides a simple yet powerful framework for describing quantum transport through mesoscopic conductors. In this view, electrical conduction is not determined by bulk properties such as resistivity, but by the number and quality of available transport channels that connect the two electrodes. Each channel is characterized by a transmission probability $\tau_i \in [0,1]$, representing the likelihood that an electron entering from one side is transmitted to the other without reflection.

For a normal conductor, the total conductance is obtained by summing over all channels,
\begin{equation}
    G = G_0 \sum_i \tau_i, \qquad G_0 = \frac{2e^2}{h},
\end{equation}
where $G_0$ is the conductance quantum accounting for the spin degeneracy of the electrons. A perfectly transmitting single channel yields a conductance quantum $G_0$, while partially transmitting channels contribute proportionally less. This concept naturally explains the appearance of conductance quantization in atomic-sized contacts and quantum point contacts.

In superconducting contacts, the same set of transmission probabilities $\{\tau_i\}$ governs all transport processes, but the current now involves both electron and hole excitations due to Andreev reflection. The transparency of each channel determines the strength of multiple Andreev reflection features and the amplitude of the Josephson supercurrent. The full set $\{\tau_i\}$ is therefore often referred to as the \emph{pincode} of a contact, as it uniquely characterizes its microscopic configuration.

Figure~\ref{fig:landauer_cartoon} illustrates this picture for an atomic-scale contact where a few discrete conduction channels with different transmissions $\tau_i$ contribute to the total current. In later chapters, this concept forms the basis for analyzing superconducting transport in atomic contacts and for extracting the individual transmission probabilities from measured $I$--$V$ characteristics.

\begin{figure}[h]
  \centering
  \includegraphics[width=0.85\linewidth]{theory/landauer/landauer.pdf}
  \caption{Landauer picture of quantum transport through an atomic-scale contact. 
  Each colored line represents a transmission channel with probability $\tau_i$. 
  The total conductance is given by $G = G_0 \sum_i \tau_i$, where 
  $G_0 = 2e^2/h$ is the conductance quantum. Channels with higher transmission 
  contribute proportionally more to the total current.}
  \label{fig:landauer_cartoon}
\end{figure}

%---------------------------------------------------------
\section{The Josephson Effect}
\label{sec:josephson}
%---------------------------------------------------------
\subsection{DC Josephson Relation}
% Present I = I_c sin φ and Ambegaokar-Baratoff formula I_c R_N = πΔ/2e.

\subsection{AC Josephson Effect and Shapiro Steps}
% dφ/dt = 2eV/ħ, f_J = 2eV/h.
% Explain how microwave drive leads to Shapiro steps at V_n = nħω/2e.

\subsection{Coherent vs Incoherent Tunneling under Microwaves}
% Clarify difference between coherent Shapiro steps (JC) and incoherent photon-assisted tunneling.

%---------------------------------------------------------
\section{Photon-Assisted Tunneling: The Tien-Gordon Model}
\label{sec:tien-gordon}
%---------------------------------------------------------

\subsection{Voltage Modulation and Energy Sidebands}
% Derive basic expression and interpret Bessel-function weighting.
% Define α = e V_ac / ħω.

\subsection{Applicability and Limitations}
% Valid for weak coupling, incoherent regime.
% Mention that it is extended in later chapters to higher transmission.

\subsection{Comparison to AC Josephson Effect}
% Highlight physical distinction and experimental observables.

%---------------------------------------------------------
\section{Experimental Principles and Measurement Techniques}
\label{sec:measurement-principles}
%---------------------------------------------------------

\subsection{Mechanically Controllable Break Junction (MCBJ)}
% Explain mechanical principle, piezo control, stability advantages.
% Figure: schematic of MCBJ setup.

\subsection{Microwave Coupling and Calibration}
% Conceptual description of microwave delivery, coupling efficiency, and
% definition of V_ac.

\subsection{Measurement Scheme}
% Differential conductance using lock-in technique, voltage biasing scheme.
% Mention cryostat environment and filtering at a high level.
% Details deferred to the Methods chapter.

%---------------------------------------------------------
\section{Summary}
\label{sec:foundations-summary}
%---------------------------------------------------------
% Summarize the main theoretical and conceptual tools.
% Provide a table linking each concept to later chapters.

\begin{table}[h]
    \centering
    \caption{Overview of key theoretical concepts and where they are applied in the thesis.}
    \begin{tabular}{lll}
        \hline
        Concept & Later Chapter & Physical Regime \\
        \hline
        Dynes DOS + Tien-Gordon & Ch.~\ref{ch:tunnelbarrier} & Tunnel-barrier junctions \\
        FCS + MAR & Ch.~\ref{ch:atomiccontacts} & Few-channel contacts \\
        Incoherent pair tunneling & Ch.~\ref{ch:hightransmission} & High-transmission regime \\
        \hline
    \end{tabular}
    \label{tab:concepts}
\end{table}

%=========================================================
% End of Chapter 2
%=========================================================
