% !TEX root = ../thesis.tex
%=========================================================
\chapter{Superconducting Transport}
\label{ch:superconducting-transport}
%=========================================================

    Superconductivity was first discovered by Heike Kamerlingh Onnes in 1911, when he observed that the electrical resistance of mercury dropped abruptly to zero upon cooling below $4.2\,$K. This remarkable phenomenon revealed a new quantum state of matter, characterized by the complete absence of electrical resistance and the expulsion of magnetic fields, known as the Meissner effect. These two defining properties—perfect conductivity and perfect diamagnetism—set superconductors apart from ordinary metals and established a foundation for both fundamental research and technological applications.

    At its core, superconductivity arises from the collective behavior of electrons, which form bound pairs and condense into a macroscopic quantum state described by a single coherent wavefunction. The subsequent sections build a comprehensive theoretical picture of this phenomenon from complementary perspectives.

    The first section, \textbf{Microscopic Description}, introduces the Bardeen-Cooper-Schrieffer (BCS) theory, which explains how electrons near the Fermi surface pair via phonon-mediated attraction and how this pairing gives rise to an energy gap and a characteristic quasiparticle density of states. This microscopic framework establishes the foundation for understanding tunneling spectroscopy and photon-assisted transport.

    The second section, \textbf{Macroscopic Description}, develops the Josephson picture, where the collective phase of the superconducting condensate determines the behavior of weak links. Here, phase coherence manifests in phenomena such as the DC and AC Josephson effects, Shapiro steps, and the influence of dissipation and capacitance captured by the RCSJ model.

    Finally, the \textbf{Mesoscopic Description} section bridges microscopic and macroscopic transport through the theory of Andreev and multiple Andreev reflections (MAR). These processes describe how charge is transferred through highly transparent superconducting contacts and how the resulting $I(V)$ characteristics encode the microscopic transmission properties of the junction, forming the quantitative link between experiment and theory.

    \newpage

    % !TEX root = ../thesis.tex

\section{Microscopic Description}
\label{sec:micro}

    The microscopic description of superconductivity is provided by the Bardeen--Cooper--Schrieffer (BCS) theory, which assumes that electrons near the Fermi surface form bound pairs of opposite momentum and spin, so called Cooper-pairs.
    
    At its heart, this pairing mechanism reflects a subtle interplay between electrons and phonons. When an electron moves through the metal, it slightly displaces the positively charged ions, creating a momentary region of excess positive charge. A second electron passing nearby can be attracted to this distortion, leading to an effective, though very weak attraction between the two. 
    
    This describes how electrons in a metal, which normally repel each other due to their negative charge, can nevertheless form bound pairs, known as Cooper-pairs, through an effective attraction mediated by vibrations of the crystal lattice. When many such pairs form simultaneously, they condense into a collective quantum state that carries electrical current without resistance.

    A key assumption of the BCS framework is that this effective interaction is weak compared to the electronic energy scales of the metal. This situation is referred to as the weak-coupling limit. In this regime, the attractive interaction acts only within a narrow energy shell around the Fermi surface, typically a few meV compared to Fermi energies of several eV. As a result, only a small fraction of electrons near the Fermi level participate in pairing, while the rest of the electronic structure remains essentially unaffected. Because the coupling is weak, the resulting energy gap and the critical temperature are small but can be predicted with high accuracy from the microscopic parameters.

    Among all known superconductors, aluminum is one of the best realizations of this weak-coupling scenario. Its superconducting transition temperature and energy gap are small, the electron-phonon interaction is weak and well understood, and its electronic structure is simple and free from strong correlations or magnetic effects. These features make aluminum behave almost exactly as predicted by the original BCS theory, allowing quantitative agreement between experiment and theory without the need for corrections or more advanced models.

    In summary, the weak-coupling limit describes a situation where the effective electron-phonon attraction is small compared to the Fermi energy, leading to a narrow pairing region and universal relationships between microscopic and macroscopic quantities. Aluminum exemplifies this regime and therefore serves as an archetype of a conventional BCS superconductor.

    The following subsections outline the key aspects of this microscopic picture. 
    First, the superconducting gap and its temperature dependency is introduced. This shows how the collective pairing strength evolves with thermal excitation. 
    Second, the electronic density of states is derived. The phenomenological Dynes parameter as measure of quasi-particle lifetime is introduced. The Fermi-Dirac distribution function and its temperature dependency is introduced as well.
    Third, the microscopic expressions for tunneling current are discussed, establishing the direct connection between these theoretical quantities and experimentally measurable \textit{I-V} and \textit{dI-dV} characteristics. 

    Finally, the extension to photon-assisted tunneling is presented, where an oscillating electromagnetic field enables quasi-particles to exchange discrete energy quanta during tunneling, thereby extending the static tunneling picture to the regime of externally driven superconducting transport.
    
    % \cite{bardeen_microscopic_1957}

    \subsection{Superconducting Gap}
    \label{subsec:micro:sc-gap}

        The weak-coupling approximation simplifies the theoretical treatment significantly. The normal-state density of states can be treated as constant, and the pairing interaction can be approximated as uniform within the relevant energy range. Many observable quantities therefore become universal, such as the ratio between the zero-temperature gap and the critical temperature, 
        \begin{equation}
            \Delta_0 \approx 1.764\, k_\mathrm{B} T_\mathrm{C}\,.
            \label{eq:micro:Delta0}
        \end{equation}

        The superconducting gap does not remain constant with temperature. At absolute zero, all available electrons near the Fermi surface form Cooper pairs, and the condensate is perfectly ordered. As the temperature rises, thermal excitations begin to break some of these pairs, leaving fewer electrons bound in the superconducting state. With fewer pairs contributing to the collective order, the overall pairing strength weakens, and the energy required to break a pair the gap $\Delta(T)$ gradually decreases.

        This reduction continues smoothly until the critical temperature $T_\mathrm{C}$ is reached. At that point, thermal energy becomes strong enough to completely disrupt the pairing correlations, and the superconducting state collapses, leading to $\Delta(T_\mathrm{C}) = 0$. The temperature dependence of the gap is a direct reflection of this balance between thermal disorder and the pairing interaction.

        \begin{wrapfigure}[12]{r}{0.4\textwidth}
            \captionsetup{format=plain}%
            \centering
            \vspace{-1em} % fine-tune vertical position
            \import{theory/micro}{gap-suppression.pgf}
            \caption{Temperature dependence of the superconducting gap $\Delta(T)$.}
            \label{fig:micro:gap_suppression}
        \end{wrapfigure} 
        In the BCS framework, $\Delta(T)$ follows a universal curve that results from solving the self-consistent gap equation, meaning the same functional form applies to all weak-coupling superconductors. 
        Solving the underlying integrals over the Fermi distribution in the microscopic theory numerically, results in the following Equation \ref{eq:DeltaT} or shown in Figure \ref{fig:micro:gap_suppression}.
        \begin{equation}
            \frac{\Delta(T)}{\Delta_0} \approx \tanh\left(1.74\,\sqrt{\frac{T_\mathrm{C}}{T}-1}\right)
            \label{eq:micro:DeltaT}
        \end{equation}

        % cite: Mühlschlegel (1959) für const2


    \subsection{Density of States}
    \label{subsec:micro:dos}

        In the following, all energies are expressed relative to the Fermi energy $E_\mathrm{F}$, such that $E=0$ corresponds to the Fermi level around which superconducting correlations develop.
        
        In the normal state, the density of states (DOS) near the Fermi level varies only weakly with energy. Over the narrow range relevant to superconductivity this variation can be neglected, and the DOS can be treated as constant. The corresponding value at the Fermi energy is denoted
        \begin{equation}
            N_0 \equiv N_\mathrm{N}(E_\mathrm{F}) = \frac{1}{2\pi^2} \left(\frac{2m}{\hbar^2}\right)^{3/2} \sqrt{E_\mathrm{F}}\,,
            \label{eq:micro:DOS-N0}
        \end{equation}
        representing the normal state DOS per spin at the Fermi level.

        When superconductivity sets in, pairing correlations reorganize this otherwise flat spectrum. A gap of width $2\Delta$ opens around $E_\mathrm{F}$ where single-particle excitations are absent in the ideal BCS limit, and the missing spectral weight is redistributed to the gap edges. These edges appear as sharp coherence peaks in the superconducting DOS, reflecting the high density of available quasi-particle states at the threshold for pair breaking. The resulting expression reads
        \begin{equation}
            \frac{N_\mathrm{S}(E)}{N_0} = 
                \left\{
                \begin{array}{@{}r@{\quad}l@{}}
                    0 & (|E| < \Delta)\\
                    \dfrac{|E|}{\sqrt{E^2-\Delta^2}} & (|E| \ge \Delta)
                \end{array}
                \right.\,.
            \label{eq:micro:DOS-BCS}
        \end{equation}
        Thus, in the ideal BCS limit the quasi-particle DOS is strictly zero within the energy gap and diverges at its edges, $E=\pm\Delta$. Increasing temperature alone does not cause a smearing of the density of states, but merely compresses its features in energy as the gap closes.

        However, real spectra are never perfectly sharp. A simple and very effective phenomenology is the broadening by Dynes parameter. It is implemented by the substitution $E\to E+\ima\Gamma$, while just considering the real part
        \begin{equation}
            \frac{N_\mathrm{S}(E)}{N_0} = \Re\!\left(\frac{E+\ima\Gamma}{\sqrt{(E+\ima\Gamma)^2-\Delta^2}}\right) \quad (|E| \ge \Delta)\,.
            \label{eq:micro:DOS-Dynes}
        \end{equation}
        Such broadening arises from finite quasi-particle lifetimes due to inelastic scattering, spatial inhomogeneity, pair breaking by magnetic impurities, or non-equilibrium effects, all of which smear the ideal BCS singularities. The DOS for a variaty of $\Gamma$ is shown in Figure \ref{fig:micro:dos-fermi}. 
        \begin{figure}[t]
            \centering
            \import{theory/micro}{dos-fermi.pgf}
            \caption{Superconducting quasi-particle density of states $N_\mathrm{S}(E)$ and Fermi--Dirac distribution $f(E)$ for different Dynes parameters $\Gamma/\Delta_0$ and temperatures $T/T_\mathrm{C}$. Increasing $\Gamma$ broadens the coherence peaks of $N_\mathrm{S}(E)$, while increasing temperature smooths the Fermi edge. Parameters correspond to aluminum with $\Delta_0 = 180\,$\textmu eV and $T_\mathrm{C} = 1.18\,\mathrm{K}$, representative of a weak-coupling BCS superconductor.}
            \label{fig:micro:dos-fermi}
        \end{figure}

        Whereas the DOS specifies where states exist, the Fermi--Dirac distribution encodes how single-particle states are occupied at a given temperature and chemical potential. In the context of metals and conventional superconductors treated in this thesis, the chemical potential can be identified with the Fermi energy to very good approximation, $\mu\approx E_\mathrm{F}$, because thermal corrections are small on the scale of eV. So the energy scale is still relative to the Fermi energy, as before.

        In equilibrium the occupation probability of a state at energy $E$ is given by        
        \begin{equation}
            f(E) = \frac{1}{1+\exp\left(\frac{E}{k_\mathrm{B}T}\right)}\,.
            \label{eq:micro:fermidirac}
        \end{equation}        
        At zero temperature the distribution reduces to a sharp step $ \theta(E)$.

        However, all states below the Fermi function are filled and all states above are empty. At finite temperature the step is thermally broadened over an energy scale of order $k_\mathrm{B}T$, as shown in Figure \ref{fig:micro:dos-fermi}. 

    
    \subsection{Tunnel Current}
    \label{subsec:micro:tunnel-current}

        Tunneling spectroscopy offers a powerful means to probe the quasi-particle spectrum of superconductors in a controlled and conceptually simple way. The key idea is that if two electrodes are separated by a sufficiently thin insulating barrier, quasi-particles can quantum-mechanically tunnel between them, even though classically forbidden. 
        
        In the tunneling limit, the barrier is high and wide enough that the process is incoherent and each quasi-particle tunnels independently, so momentum conservation is effectively relaxed. 
        
        Under these conditions, the tunnel current from material $1$ to $2$ is given by
        \begin{equation}
            I_{1\to2}(V) \propto \int_{-\infty}^\infty \left(\frac{N_1(E)}{N_0} f_1(E) \right)\cdot \left(\frac{N_2(E+eV)}{N_0} \left(1-f_2(E+eV)\right)\right)\mathrm{d}E\,,
            \label{eq:micro:tunnel-1to2}
        \end{equation}
        where $eV$ is an externally applied voltage bias. The first part in Equation \ref{eq:micro:tunnel-1to2}, is given by the occupied states in material 1, the second part is given by the unoccupied states in material 2. However, in order to get the total tunnel current, one have to substract the tunnel current from material 2 to 1, resulting in
        \begin{equation}
            I(V) \propto \int_{-\infty}^\infty \frac{N_1(E)}{N_0} \frac{N_2(E+eV)}{N_0} \left(f_1(E)-f_2(E+eV)\right)\mathrm{d}E\,.
            \label{eq:micro:tunnel}
        \end{equation}
        
        The difference $f_1(E)-f_2(E+eV)$ accounts for the imbalance in occupation between the two electrodes induced by the applied bias voltage $eV$, ensuring that current flows only when filled states on one side overlap with empty states on the other. In this picture, the densities of states $N_1(E)$ and $N_2(E)$ define where electrons can tunnel, while the Fermi--Dirac distributions define which of those states are populated. The convolution of these terms thus directly connects the microscopic electronic structure to the measurable \textit{I-V} characteristics.

        In the case of an all normal conducting tunnel barrier, the current is given by
        \begin{equation}
            I_\mathrm{NN}(V) = \frac{G_\mathrm{N}}{e} \int_{-\infty}^{\infty}f(E) - f(E + eV)\mathrm{d}E\,.
            \label{eq:micro:tunnel-nn}
        \end{equation}
        All geometric factors of the tunnel barrier, along with $N_0$ are collapsing into the normal conductance $G_\mathrm{N}$. 
        Given the two electrodes are in thermal equilibrium, effectively collapses the equation to Ohm's law
        \begin{equation}
            I_\mathrm{NN}(V) =  G_\mathrm{N}V\,.
            \label{eq:micro:ohms-law}
        \end{equation}

        In the case of a junction between a normal metal and a superconductor, the tunneling current is given by
        \begin{equation}
            I_\mathrm{NS}(V) = \frac{G_\mathrm{N}}{e} \int_{-\infty}^{\infty} \frac{N_\mathrm{S}(E)}{N_0} \left[f(E) - f(E + eV)\right] \mathrm{d}E\,,
            \label{eq:micro:tunnel-ns}
        \end{equation}
        where $N_\mathrm{S}(E)$ denotes the superconducting quasi-particle density of states and $f(E)$ the Fermi--Dirac distribution. The differential conductance then follows as
        \begin{equation}
            \frac{\mathrm{d}I_\mathrm{NS}(V)}{\mathrm{d}V} = G_\mathrm{N} \left[ \frac{N_\mathrm{S}(E)}{N_0} \otimes -\frac{\partial f(E)}{\partial E}\right]_{E=eV}\,,
            \label{eq:micro:tunnel-ns-didv}
        \end{equation}
        showing that the measured \textit{dI-dV} corresponds to the superconducting density of states thermally broadened by the derivative of the Fermi function. This derivative is a symmetric, bell-shaped function whose width scales with temperature. As $T \to 0$, it approaches a delta function, and the convolution becomes negligible. At finite temperature, however, the \textit{dI-dV} smeares out, by the convolution. The energy resolution is then limited by thermal broadening, given approximately by $\Delta E_\text{th} \approx 3.5\,k_\mathrm{B}T$, which sets the smallest energy scale over which spectral features can be resolved.

        The expressions derived above not only describe the origin of tunneling currents but also provide a direct link to experimental observables. In scanning tunneling microscopy (STM), for example, a normal-metal tip above a superconducting surface forms an NIS junction. Measuring the \textit{I-V} or \textit{dI-dV} characteristics at sufficiently low temperature thus enables a direct mapping of the energy-resolved quasi-particle spectrum of the superconductor. 

        In case of all superconducting tunnel barrier, both electrodes contribute gapped densities of states. Their convolution, together with thermal and lifetime broadening, determines the observed shape of the \textit{I-V} and \textit{dI-dV} curves. 
        \begin{equation}
            I_\mathrm{SS}(V) = \frac{G_\mathrm{N}}{e}\int_{-\infty}^{\infty} \frac{N_\mathrm{S}(E)}{N_0} \cdot \frac{N_\mathrm{S}(E+eV)}{N_0} \cdot \left[f(E) - f(E + eV)\right] \mathrm{d}E
            \label{eq:micro:tunnel-ss}
        \end{equation}
        Importantly, temperature $T$ and Dynes broadening $\Gamma$ influence the spectra in distinct ways. Finite temperature broadens the Fermi edges via the Fermi--Dirac distribution, while $\Gamma$ introduces intrinsic smearing of the quasi-particle DOS itself. By fitting measured \textit{I-V} or \textit{dI-dV} data, with both parameters as variables, one can disentangle thermal effects from genuine lifetime or inelastic processes.

        These interpretations form the foundation of tunneling spectroscopy as a quantitative probe of superconductivity, allowing the extraction of $\Delta_0$, $T$, and $\Gamma$ from experimental data with high accuracy.

        \begin{figure}
            \centering
            \import{theory/micro}{tunnel-current.pgf}
            \caption{Tunnel current, smearing, dos, fermidirac}
            \label{fig:micro:tunnel-current}
        \end{figure}

        \textbf{cite: Giaever}

    \subsection{Photon-Assisted Tunneling}
    \label{subsec:micro:pat}

        While the previous section described static tunneling, the application of a time-dependent voltage enables quasi-particles to exchange discrete energy quanta with an external electromagnetic field, giving rise to photon-assisted tunneling (PAT).

        In practice, this effect is typically studied using electromagnetic radiation in the microwave range, since photon energies $h\nu$ in this regime are comparable to the superconducting energy gap $\Delta_0$ and can therefore induce measurable sidebands without breaking the Cooper pairs.

        Possible energies are given by
        \begin{equation}
            E_n = n h\nu\quad (n \in \mathbb{Z})\,,
            \label{eq:micro:pat-En}
        \end{equation}
        where each photon carries an energy $h\nu$ and $n$ is the number of photons absorbed ($n>0$) or emitted ($n<0$). As a result, additional tunneling channels open at these energies, producing characteristic replicas of the coherence peaks in \textit{dI-dV}.

        The Tien--Gordon model provides a simple and intuitive framework to describe this effect. It assumes that the junction operates in the tunneling limit, as described before in Section~\ref{subsec:micro:tunnel-current}. The electromagnetic field is treated classically and is represented by a spatially uniform, time-dependent voltage across the junction,
        \begin{equation}
            V(t) = V_0 + A \cos (2\pi\,\nu t)\,.
            \label{eq:micro:pat-V(t)}
        \end{equation}
        Here $V_0$ is the applied voltage bias, $A$ the amplitude, and $\nu$ the frequency of the microwave field across the junction. The field is assumed to remain unaltered by the tunneling current, meaning no cavity or backaction effects, and the drive is weak enough not to heat the electrodes, allowing both to stay in thermal equilibrium.

        In a gauge where the entire voltage drop appears as a scalar potential, a tunneling electron acquires a time-dependent factor in its wavefunction,
        \begin{equation}
            \psi(t) \propto \exp\!\left(-\frac{i}{\hbar}\int_0^t e V(t')\, dt' \right)\,.
            \label{eq:micro:pat-wf0}
        \end{equation}

        For the harmonic drive of Eq.~\eqref{eq:micro:pat-V(t)}, the integral can be solved by
        \begin{equation}
            \int_0^t e V(t')\, dt' = eV_0 t + \frac{eA}{2\pi\nu}\sin(2\pi\nu t)\,.
            \label{eq:micro:pat-int}
        \end{equation}
        This equantion is further solved by the Jacobi--Anger identity,
        \begin{equation}
            \exp\left(i\alpha \sin(2\pi\nu t)\right) = \sum_{n=-\infty}^{\infty} J_n(\alpha)\, \exp\left(i n 2\pi\nu t\right)\,,
        \end{equation}
        where $\alpha = eA/h\nu$ denotes the dimensionless modulation strength and $J_n(\alpha)$ is the $n$-th Bessel function of first kind.

        The wavefunction becomes then
        \begin{equation}
            \psi(t) \propto \sum_{n=-\infty}^{\infty} J_n(\alpha)\, \exp\!\left( -\frac{i}{\hbar}(eV_0 - n h\nu)\, t  \right)\,,
            \label{eq:micro:pat-wf}
        \end{equation}
        revealing that an electronic state subjected to an AC voltage becomes a superposition of components at energies shifted by $n h\nu$. Each component corresponds to the absorption ($n>0$) or emission ($n<0$)
        of $|n|$ photons. Because tunneling probabilities are proportional to the squared amplitude, these channels carry weights $J_n^2(\alpha)$.

        This perspective permits a particularly transparent interpretation in terms of an effective, photon-dressed density of states. Each shifted wavefunction component contributes a copy of the local DOS displaced by $n h\nu$. The resulting effective DOS is
        \begin{equation}
            N_\mathrm{PAT}(E) = \sum_{n=-\infty}^{\infty} J_n^2(\alpha)\, N(E + n h\nu)\,,
            \label{eq:micro:pat-dos}
        \end{equation}
        which replaces $N(E)$ in the tunneling expression. The PAT-modified current therefore becomes a weighted sum of shifted copies of the static \textit{I-V} curve.

        Putting these contributions together leads to the Tien--Gordon formula,
        \begin{equation}
            I(V_0) = \sum_{n=-\infty}^{\infty} J_n^2\!\left( \frac{eA}{h\nu}\right) \cdot I_0\!\left(V_0 - \frac{n h\nu}{e}\right)\,.
            \label{eq:micro:pat-tien-gordon}
        \end{equation}
        which expresses the total current as the incoherent sum of all photon-assisted tunneling channels.  Each channel contributes a copy of the static \textit{I-V} curve shifted by $n h\nu/e$ and weighted by the probability $J_n^2(\alpha)$ for absorbing or emitting $n$ photons, producing the characteristic sideband structure observed in experiment.

        The same relation also holds for the differential conductance, meaning that the photon-assisted replicas $\mathrm{d}I_0(V_0)/\mathrm{d}V$ appear identically as in the current, providing a direct experimental link to the superconducting density of states.

        In practice, the amplitude $A$ can be determined experimentally by comparing the relative heights of the sidebands with the expected Bessel function weights. The spacing $h\nu$ between sidebands provides a direct and robust calibration of the frequency. Photon-assisted tunneling thus offers a straightforward semiclassical description of how an oscillating field modifies quasi-particle transport, serving as a bridge between static tunneling spectroscopy and driven quantum dynamics.

        
    \newpage

    % !TEX root = ../thesis.tex

%=========================================================
\section{Macroscopic Description}
\label{sec:macro}
%=========================================================

    \begin{wrapfigure}[13]{r}{0.4\textwidth}
        \captionsetup{format=plain}%
        \centering
        \vspace{-1.5em}
        \import{theory/macro}{delta-r.pgf}
        \caption{
            Spatial profile of $\Delta_1(r)$ (\legend{seeblau100}) and $\Delta_2(r)$ (\legend{seegrau80}) across a tunnel junction (\legend{seegrau65}). Their respective magnitude $|\Delta|$ (shaded) varies only weakly accross the barrier. Coherent coupling is governed by the macroscopic phase difference $\phi$.
            }
        \label{fig:macro:delta-r}
    \end{wrapfigure}
    While the previous section described dissipative transport in terms of single-particle tunneling, the complementary low-energy limit is governed by coherent Cooper-pair tunneling. In this regime, the quasiparticle spectrum plays no direct role. Instead, transport is determined solely by the phase of the superconducting gap introduced in Eq.~\ref{eq:micro:complex-delta}. Because the magnitude $|\Delta|$ varies only weakly across a weak link, the relevant dynamical variable is the phase difference
    \begin{equation}
        \phi = \phi_1 - \phi_2\,,
        \label{eq:macro:phase-difference}
    \end{equation}
    which fully determines the supercurrent. The Josephson effect discussed here is treated in the weak-coupling (tunneling) limit, analogous to the quasiparticle tunneling regime introduced in the previous section.
    Figure~\ref{fig:macro:delta-r} shows a schematic illustration of this situation.
 
    A weak link, such as an insulating barrier, a metallic constriction, or a short normal region, allows the pairing potentials of the two superconductors to overlap. Since their amplitudes remain essentially constant on the junction scale, the coupling depends only on the relative phase. This phase difference is the essential quantity governing coherent Cooper-pair tunneling and forms the basis of the Josephson effect.

    The following subsections introduce the two Josephson relations, discuss their physical implications, and develop the framework required to describe microwave-driven junctions and the RCSJ model.

    %=========================================================
    \subsection{Josephson Effect}
    \label{subsec:macro:josephson}
    %=========================================================
        
        \begin{figure}[t]
            \captionsetup{format=plain}%
            \centering
            \subfigure[
                Current-phase relation of a weakly coupled Josephson junction (Eq.~\ref{eq:macro:dc}).
            ]{\import{theory/macro}{josephson-iphi.pgf}}
            \hspace{3mm}
            \subfigure[
                Josephson current over time for a constant voltage $V_0 = \Delta_0/e$ (Eq.~\ref{eq:macro:dc} \& \ref{eq:macro:phi-t}).
            ]{\import{theory/macro}{josephson-it.pgf}}
            \subfigure[
                Combined supercurrent and quasiparticle contribution to the \textit{I--V} characteristic.
            ]{\import{theory/macro}{josephson-iv.pgf}}
            \hspace{3mm}
            \subfigure[
                Temperature dependence of the critical current $I_\mathrm{C}(T)$ (Eq.~\ref{eq:macro:critical-current}).
            ]{\import{theory/macro}{critical-current.pgf}}
            \caption{
                Josephson current over phase (a) and time (b). \textit{I--V} characteristic (c) and temperature dependence of the critical current (d).
            }
            \label{fig:macro:josephson}
        \end{figure}

        When two superconductors are weakly coupled through a thin insulating barrier, constriction, or short metallic link, the amplitudes of their order parameters vary only minimally across the junction. The relevant degree of freedom is therefore the phase, which changes from $\phi_1$ to $\phi_2$ across the weak link. This phase difference $\phi$ governs the supercurrent flowing through the junction, and its maximum magnitude is set by the critical current $I_\mathrm{C}$.

        This perspective emphasizes that the Josephson effect arises from coherent phase coupling between the two superconductors, not from changes in the order-parameter amplitude. The resulting phase-driven transport is captured by two fundamental relations, known as the Josephson equations.

        % DC Josephson
        The DC Josephson effect is described by the current-phase relation (CPR)
        \begin{equation}
            I_\mathrm{J} = I_\mathrm{C}\sin\phi\,,
            \label{eq:macro:dc}
        \end{equation}
        where $I_\mathrm{C}$ denotes the critical current of the weak link, as shown in Figure~\ref{fig:macro:josephson} (a). This CPR states that a dissipationless supercurrent can flow at zero voltage, driven solely by the phase difference between the two superconductors and is a direct consequence of coherent Cooper-pair tunneling. 

        % AC Josephson
        The AC~Josephson relation links the temporal evolution of the phase to the voltage across the junction,
        \begin{equation}
            \frac{\mathrm{d}\phi}{\mathrm{d}t} = \frac{2e}{\hbar}\,V_0\,,
            \label{eq:macro:ac}
        \end{equation}
        implying that a constant voltage causes the phase difference to increase uniformly in time.

        Consequently, the phase evolves linearly,
        \begin{equation}
            \phi(t) = \phi_0 + 2\pi \nu_0t\,,\quad
            \nu_0 = \frac{2e}{h}\,V_0\,,
            \label{eq:macro:phi-t}
        \end{equation}
        which results in an oscillating supercurrent with frequency $\nu_0$. This is illustrated in Fig.~\ref{fig:macro:josephson} (b), where the time-dependent current $I(t)$ is shown together with the corresponding time scale $\Delta t = 1/\nu_0$ set by the Josephson frequency.
        
        % IV Josephson
        In a real junction, quasiparticle tunneling appears in parallel to the phase-driven supercurrent. The resulting \textit{I--V} characteristic, shown in Fig.~\ref{fig:macro:josephson} (c), features a dissipationless supercurrent branch at zero voltage and a dissipative quasiparticle branch that onsets above the gap. This combined response forms the characteristic transport signature of a weakly coupled Josephson junction.
        
        % Ambegaokar–Baratoff
        The critical current is set microscopically by the Ambegaokar--Baratoff (AB) relation,
        \begin{equation}
            I_\mathrm{C}(T) = \frac{\pi}{2}\,\frac{G_\mathrm{N}\Delta(T)}{e}\,
            \tanh\!\left(\frac{\Delta(T)}{2k_\mathrm{B}T}\right),
            \label{eq:macro:critical-current}
        \end{equation}
        shown in Fig.~\ref{fig:macro:josephson} (d). It reflects the BCS temperature dependence of the superconducting gap (Eq.~\ref{eq:micro:DeltaT}) together with the thermal occupation of quasiparticle states. At zero temperature, the expression simplifies to the well known result $I_\mathrm{C} = (\pi/2)\,G_\mathrm{N}\Delta_0/e$.

        % Josephson Coupling Energy
        The strength of phase coupling in a Josephson junction is quantified by the Josephson energy
        \begin{equation}
            E_\mathrm{J}=\frac{\hbar I_\mathrm{C}}{2e}\,,
            \label{eq:macro:josephson-energy}
        \end{equation}
        which sets the energetic stiffness of the phase and thus the robustness of coherent Cooper-pair tunneling. A large $E_\mathrm{J}$ corresponds to a well-defined phase, while a small $E_\mathrm{J}$ makes the junction susceptible to fluctuations.

        Together with the thermal energy $E_\mathrm{T}=k_\mathrm{B}T$ and the charging energy $E_\mathrm{C}=e^2/(2C)$, the Josephson energy sets the scale on which the phase behaves either classically or becomes susceptible to fluctuations. In the regime $E_\mathrm{J}\gg\{E_\mathrm{T},E_\mathrm{C}\}$, the phase is stiff and the Josephson relations hold in their ideal form, yielding a stable supercurrent. At finite temperature, thermally activated phase slips lead to a rounding of the sharp switching expected in an ideal \textit{I--V} curve. This behavior is naturally described within the classical RCSJ framework, which incorporates thermal noise through the resistive branch. In the charge-dominated regime $E_\mathrm{C} \gtrsim E_\mathrm{J}$, quantum and fully stochastic descriptions of phase dynamics go beyond the classical RCSJ model and are discussed in Section~\ref{sec:stochastic}.

        Having established the phase dynamics in static conditions, we now turn to the interplay between the intrinsic Josephson oscillation and external microwave fields.


    %=========================================================
    \subsection{Shapiro Steps}
    \label{subsec:macro:shapiro}
    %=========================================================
        \begin{figure}[t]
            \centering
            \import{theory/macro}{shapiro-ideal.pgf}
            \caption{
                Microwave-driven transport in a weakly coupled Josephson junction. Parameters correspond to aluminum (Sec.~\ref{subsec:basics:aluminum}), with $T=0$, $\gamma=0$, and $\nu=10.0\,\mathrm{GHz}$.
                }
            \label{fig:macro:shaprio-iv}
        \end{figure}

        When a Josephson junction is exposed to an external microwave field, the phase dynamics become modulated by the drive introduced in Section~\ref{subsec:basics:micro-wave}. The Josephson frequency $\nu_0$ mixes with the external frequency $\nu$, and the phase takes the form
        \begin{equation}
            \phi(t)=\phi_0 + 2\pi\nu_0 t + \alpha\sin(2\pi\nu t)\,,
            \label{eq:macro:shapiro-phase}
        \end{equation}
        with $\alpha = 2eA/h\nu$ the dimensionless drive strength.

        Inserting this expression into the Josephson current-phase relation (Eq.~\ref{eq:macro:dc}) and expanding the phase modulation using the Jacobi-Anger identity (Eq.~\ref{eq:microwave:jacobi-anger}) yields harmonics at frequencies $\nu_0 - n\nu$. Whenever the resonance condition $\nu_0 = n\nu$ is satisfied, the $n$-th harmonic becomes stationary and contributes a finite dc component to the current. This produces quantized voltage plateaus at
        \begin{equation}
            V_n = \frac{nh\nu}{2e}\,,
            \label{eq:macro:shapiro-step}
        \end{equation}
        known as Shapiro steps.

        The amplitude of each step is governed by the Bessel weight $J_n(\alpha)$, which reflects the strength of phase modulation by the microwave field. In contrast to the Tien--Gordon description of quasiparticle tunneling, the Bessel functions appear here without being squared because the Josephson current depends linearly on the phase modulation rather than on squared transition probabilities. Consequently, the step amplitudes scale as $J_n(\alpha)$ instead of $J_n^2(\alpha)$. The time-averaged \textit{I--V} characteristic can therefore be written in direct analogy to the Tien--Gordon description,
        \begin{equation}
            I(V_0)
                = \sum_{n=-\infty}^{\infty}
                    J_n(\alpha)\,
                    I_{0}\!\left(V_0 - \frac{nh\nu}{2e}\right),
            \label{eq:macro:shapiro-iv}
        \end{equation}
        where $I_0(V_0)$ denotes the static Josephson \textit{I--V} curve in the absence of microwaves.

        In a real junction, quasiparticle tunneling occurs in parallel with the phase-driven supercurrent. Under microwave irradiation, both Shapiro steps and PAT appear simultaneously. This combined response constitutes the characteristic microwave-driven transport signature of a weakly coupled Josephson junction.
        As illustrated in Fig.~\ref{fig:macro:shaprio-iv}, the interplay of phase locking and photon-assisted tunneling produces a rich structure in both the \textit{I--V} and differential conductance characteristics.


    %=========================================================
    \subsection{RCSJ Model}
    \label{subsec:macro:rcsj}
    %=========================================================    

        The ideal Josephson relations describe how the supercurrent depends on the phase and how the phase evolves under a constant voltage. However, real junctions exhibit dissipation, capacitance, thermal fluctuations, and microwave-driven phase dynamics that cannot be    captured by the ideal equations alone.

        A convenient way to incorporate these additional contributions is to represent the junction as an effective circuit. To describe these effects, the Josephson element    must be embedded into an effective circuit model.

        Throughout this section, the junction is assumed to operate in the tunneling limit and in the classical regime $E_\mathrm{J} \gg {E_\mathrm{C}, E_\mathrm{T}}$, such that phase dynamics are well described by the RCSJ equation without quantum corrections.

        %========================================================= 
        % \subsubsection*{Schaltplan}
        %=========================================================

            \begin{wrapfigure}[12]{r}{35mm}
                \captionsetup{format=plain}%
                \centering
                \vspace{-.5em}
                \import{theory/macro}{rcsj-model.pdf_tex}
                \caption{
                    RCSJ model of a real Josephson junction.
                }
                \label{fig:macro:rcsj}
            \end{wrapfigure}
            The resistively and capacitively shunted junction (RCSJ) model provides the minimal dynamical description of a real Josephson junction. It augments the ideal Josephson element by a normal resistance representing quasiparticle tunneling and a capacitance associated with the junction electrodes, as sketched in Fig.~\ref{fig:macro:rcsj}. These additions capture the essential dissipative and inertial mechanisms that govern the phase dynamics.

            The damping regime encoded in the quality factor $Q$ strongly affects the visibility and stability of microwave-driven features such as Shapiro steps. Overdamped junctions exhibit clean, well-defined plateaus, whereas underdamped junctions show hysteresis and residual phase oscillations that distort the step structure.

            In the following, we derive the RCSJ equation of motion for the phase and introduce the dynamical regimes that arise from the interplay of Josephson nonlinearity, dissipation, and inertia.

        %========================================================= 
        \subsubsection*{Current Bias}
        %=========================================================

            To obtain the phase dynamics of the junction, the currents through the three parallel elements are summed according to Kirchhoff's law,
            \begin{equation}
                I_\mathrm{bias} = I_\mathrm{C}\sin(\phi)
                    + \frac{V}{R}
                    + C\frac{\mathrm{d}V}{\mathrm{d}t}\,.
                \label{eq:macro:rcsj-ibias}
            \end{equation}
            We use $I_\mathrm{bias}$ to denote the externally applied current, distinguishing it from the Josephson supercurrent $I_\mathrm{J}$, the resistive current, and the capacitive displacement current that appear in parallel in the RCSJ model.

            Using the AC~Josephson voltage-phase relation, this expression can be written entirely in terms of the phase,
            \begin{equation}
                I_\mathrm{bias} = I_\mathrm{C} \sin(\phi)
                + \frac{\hbar}{2eR}\,\frac{\mathrm{d}\phi}{\mathrm{d}t}
                + \frac{\hbar C}{2e}\,\frac{\mathrm{d}^2\phi}{\mathrm{d}t^2}
                \,.
                \label{eq:macro:rcsj-cpr}
            \end{equation}
            This is the RCSJ equation of motion and forms the basis for all classical descriptions of Josephson phase dynamics. It contains an inertial term, a damping term, and the nonlinear Josephson restoring force, producing the characteristic tilted-washboard potential and the associated dynamical regimes discussed below.

        %========================================================= 
        \subsubsection*{Dimensionless RCSJ Model}
        %=========================================================

            To analyze the dynamics implied by Eq.~\ref{eq:macro:rcsj-cpr}, it is useful to cast the equation into a dimensionless form. This separates the contributions of inertia, damping, and the nonlinear Josephson term.

            The natural time scale of the junction is set by the Josephson plasma frequency,
            \begin{equation}
                \omega_\mathrm{p}
                    = \sqrt{\frac{2eI_\mathrm{C}}{\hbar C}}\,,
                \label{eq:macro:rcsj-plasma-frequency}
            \end{equation}
            which determines the small-oscillation frequency of the phase in a potential minimum. It is distinct from the AC Josephson frequency, which reflects voltage-driven phase evolution, and instead characterizes the intrinsic resonance of the phase in the absence of bias.

            Introducing the rescaled time $t'=\omega_\mathrm{p} t$ and the normalized current $i = I_\mathrm{bias}/I_\mathrm{C}$ gives the compact dimensionless form
            \begin{equation}
                i
                = \sin\phi
                + \frac{1}{Q}\,\frac{\mathrm{d}\phi}{\mathrm{d}t'}
                + \frac{\mathrm{d}^2\phi}{\mathrm{d}t'^2}\,,
                \label{eq:macro:rcsj-cpr-norm}
            \end{equation}
            where the quality factor
            \begin{equation}
                Q = \sqrt{\frac{2eI_\mathrm{C}R^2C}{\hbar}}
                \label{eq:macro:rcsj-damping}
            \end{equation}
            quantifies the damping of the phase dynamics.

            In this normalized form, the physical regimes become transparent. The second derivative describes inertial motion, $(1/Q)\mathrm{d}\phi/\mathrm{d}t'$ represents viscous damping due to the shunt resistor, and $\sin\phi$ provides the nonlinear restoring force from the Josephson coupling. The single parameter $Q$ therefore distinguishes underdamped ($Q\gg 1$) from overdamped ($Q\ll 1$) junctions and sets the stage for the dynamical behavior discussed next.

        %========================================================= 
        \subsubsection*{Tilted Washboard Potential}
        %=========================================================

            The normalized RCSJ equation~\ref{eq:macro:rcsj-cpr-norm}, admits a mechanical interpretation that provides an intuitive picture of the phase dynamics. The equation is mathematically equivalent to the motion of a particle of unit mass in a tilted, periodic potential subject to viscous damping. 

            The normalized potential is obtained by identifying the restoring force with
            \begin{equation}
                -\partial u/\partial\phi = \sin\phi - i\,,
                \quad
                u = U/ E_\mathrm{J}\,,
            \end{equation}
            which yields
            \begin{equation}
                u(\phi)
                = -\cos\phi \;-\; i\,\phi\,.
                \label{eq:macro:washboard-potential}
            \end{equation}
            In the untilted case ($i=0$) the barrier height is given by $\Delta U = 2E_\mathrm{J}$. This ''tilted washboard potential'' consists of a cosine landscape whose overall slope is controlled by the normalized bias current $i$, as shown in Figure~\ref{fig:macro:rcsj-u-phi}.
            
            \begin{figure}[t]
                \centering
                \import{theory/macro}{u-phi.pgf}
                \caption{
                    Tilted washboard potential of the RCSJ model (Eq.~\ref{eq:macro:washboard-potential}) for increasing normalized bias current $i$. For $i=0$ (\legend{seeblau100}) the phase is trapped in a stable minimum. At intermediate tilt, $0<i<1$ (\legend{seeblau65}), the reduced barrier enables thermally activated or noise-driven phase slips (\legend{seegrau35}) between adjacent minima. At $i=1$ (\legend{seeblau35}) the barriers vanish and the phase runs downhill, corresponding to the finite-voltage state.
                    }
                \label{fig:macro:rcsj-u-phi}
            \end{figure}

            For small bias, $i<1$, the potential exhibits a series of metastable minima. In this regime, the phase can remain localized in one of these wells, corresponding to the zero-voltage superconducting branch of the \textit{I--V} characteristic. Small oscillations of the phase around the minimum occur at the plasma frequency defined in Eq.~\ref{eq:macro:rcsj-plasma-frequency}, but the average voltage remains zero.

            As the bias is increased, the tilt of the potential grows and the barriers separating adjacent minima are reduced. At $i=1$, the barriers vanish and the phase becomes free to run down the potential without encountering any local minima. This running solution corresponds to the finite-voltage, resistive state of the junction. In the intermediate regime, fluctuations, either thermal or current-induced, can cause the phase to escape from a metastable minimum even for $i<1$, giving rise to switching from the superconducting to the resistive branch.
        
            A finite phase momentum acquired during escape might, once the bias is reduced again, retrap the phase only at a smaller tilt, leading to a retrapping back to the superconducting branch.

            The washboard picture thus provides a unified interpretation of phase localization, escape, and running dynamics. It naturally explains the origin of the switching current, the emergence of retrapping current, and the role of thermal activation and noise, which are explored in the following subsections.

        %========================================================= 
        \subsubsection*{Switching Dynamics and Phase Diffusion}
        %=========================================================

            \begin{wrapfigure}[19]{r}{0.4\textwidth}
                \captionsetup{format=plain}%
                \centering
                \vspace{-1em} % fine-tune vertical position
                \import{theory/macro}{rcsj-iv.pgf}
                \caption{
                    \textit{I--V} characteristic of a Josephson junction in the intermediate damping regime ($Q\sim 1$), illustrating the combined effects of phase diffusion and residual inertia. Thermal and current noise produce a finite slope in the zero-voltage branch, while moderate damping leads to a hysteretic transition into the running state.
                }
                \label{fig:macro:rcsj-iv}
            \end{wrapfigure}
            The switching current $I_\mathrm{SW}$ is the experimentally observed current at which a Josephson junction leaves the zero-voltage state and enters the resistive, running-phase regime. Unlike the intrinsic critical current $I_\mathrm{C}$, which is a microscopic parameter set by the Josephson coupling energy, the switching current is a stochastic quantity. Its value is determined by the competition between the decreasing barrier height of the tilted washboard potential and fluctuations that drive premature escape. Thermal activation, current noise from the electromagnetic environment, and the rate at which the bias current is ramped all increase the likelihood of early escape and thus reduce ths switching current below the critical current. Only in the ideal, noise-free limit does the switching current approach the intrinsic critical current.

            In real Josephson junctions the supercurrent branch acquires a finite slope due to thermal and noise-induced fluctuations of the phase. Even for currents below the intrinsic critical value, stochastic forces drive small excursions of the phase within the tilted washboard potential, leading to a finite average phase velocity and hence a small, non-zero voltage. In the RCSJ model this effect is enhanced by the resistive branch in parallel with the Josephson element, which provides a dissipative path whenever the phase is not perfectly static. As a result, the ideal vertical supercurrent step is replaced by a broadened zero-voltage region whose slope grows with temperature, current noise and environmental damping.

        %========================================================= 
        \subsubsection*{Retrapping Dynamics}
        %=========================================================

            After the phase has escaped into the running state, the junction develops a finite average voltage and the phase accelerates down the tilted washboard potential. When the bias current is reduced again, the junction does not immediately return to the zero-voltage state. Instead, the phase continues to move until sufficient kinetic energy has been dissipated for it to fall back into a metastable minimum. The current at which this return to the superconducting branch occurs is the retrapping current $I_\mathrm{R}$.

            In contrast to the switching current, set by escape from a minimum, the retrapping current reflects how efficiently damping removes the kinetic energy accumulated during the running state. Thermal and current noise smooth this transition, producing a broadened return to the superconducting branch rather than a sharp jump.

            A representative \textit{I--V} trace for this intermediate damping regime is shown in Fig.~\ref{fig:macro:rcsj-iv}. It highlights the finite slope of the supercurrent branch due to phase diffusion, the broadened switching transition, and the smooth return at $I_\mathrm{R}$ characteristic of junctions with $Q\!\sim\!1$.

            In the overdamped limit ($Q\ll 1$), the phase has essentially no inertia. As soon as the bias is lowered below the point where the washboard potential recreates minima, the phase immediately relocks. The retrapping current equals the switching current.
            
            In the underdamped limit ($Q\gg 1$), the phase retains significant kinetic energy after escape and cannot relock until the tilt is nearly removed. The retrapping current therefore approaches zero, giving rise to strong hysteresis in the \textit{I--V} characteristic.

        %========================================================= 
        \subsubsection*{Shapiro Steps in the RCSJ Model}
        %=========================================================

            In the full RCSJ model, Shapiro steps arise through frequency locking between the intrinsic Josephson oscillation and an external microwave drive, as described in Section~\ref{subsec:basics:micro-wave}. The normalized RCSJ equation becomes
            \begin{equation}
                i + a\sin(2\pi\nu' t')
                = \sin\phi
                + \frac{1}{Q}\,\frac{\mathrm{d}\phi}{\mathrm{d}t'}
                + \frac{\mathrm{d}^2\phi}{\mathrm{d}t'^2}\,,
                \label{eq:macro:rcsj-driven}
            \end{equation}
            where $i$ is the normalized bias current. The normalized amplitude $a$ and frequency $\nu$ are given by
            \begin{equation}
                a = A/I_\mathrm{C} \sqrt{R^{-2} + \left(2\pi\nu C\right)^2}\,,\qquad
                \nu' = \nu/\nu_\mathrm{p}\,.
            \end{equation}

            The washboard potential then becomes time dependent
            \begin{equation}
                u(\phi)
                = -\cos\phi \;-\; i\,\phi - a\,\phi\sin(2\pi\nu' t')\,.
                \label{eq:macro:rcsj-washboard-potential}
            \end{equation}
            
            Weak damping makes the junction sensitive to residual oscillations, which can distort Shapiro steps and reduce their visibility under microwave irradiation.
            
            In the overdamped regime, damping suppresses inertial phase oscillations and stabilizes the frequency-locked state, yielding clean, well-defined Shapiro steps. Thermal fluctuations broaden the step edges but do not compromise their visibility. This regime most closely reflects the behavior of the atomic aluminum contact studied in this work.

            Figure~\ref{fig:macro:rcsj-shapiro-iv} shows the resulting \textit{I--V} characteristics in the under- and overdamped limit. The phase locks robustly to the external drive, producing a sequence of Shapiro plateaus.
            \begin{figure}
                \centering
                \subfigure[
                    Overdamped junction ($Q\ll 1$), switching current parameter $I_\mathrm{SW}=0.5\,I_\mathrm{C}$.
                    ]{\import{theory/macro/}{shapiro-under.pgf}}
                \subfigure[
                    Underdamped junction ($Q\gg 1$), switching current parameter $I_\mathrm{SW}=0.2\,I_\mathrm{C}$.
                    ]{\import{theory/macro/}{shapiro-over.pgf}}
                \caption{
                    Microwave-driven dc transport of a Josephson junction within the RCSJ model for (a) overdamped and (b) underdamped phase dynamics. Shown is the normalized bias current $I_\mathrm{bias}/(G_\mathrm{N}\Delta_0/e)$ versus the normalized voltage $eV/\Delta_0$ for drive amplitudes $eA/\Delta_0\in\{0,0.3,0.6\}$ at $\nu=10\,\mathrm{GHz}$. Parameters correspond to aluminum (Sec.~\ref{subsec:basics:aluminum}) with $T=0$ and $\gamma=0$.
                    }
                \label{fig:macro:rcsj-shapiro-iv}
            \end{figure}

            The RCSJ framework bridges the gap between the ideal Josephson prediction of perfectly sharp Shapiro plateaus and the experimentally observed \textit{I--V} characteristics, where damping, noise, and capacitive dynamics shape the visibility and stability of the steps.
    \newpage

    % !TEX root = ../thesis.tex

%=========================================================
\section{Mesoscopic Description}
\label{sec:meso}
%=========================================================

    Summarize this section here.

    %=========================================================
    \subsection{Bogoliubov--de~Gennes Formalism}
    \label{subsec:meso:bdg}
    %=========================================================

    To describe superconductivity in mesoscopic and spatially inhomogeneous structures, we employ the Bogoliubov--de~Gennes (BdG) formalism, i.e. the real-space mean-field formulation of BCS theory. Starting from the BCS pairing Hamiltonian \cite{bardeen_microscopic_1957} and applying the canonical Bogoliubov transformation \cite{bogoljubov_new_1958} in the electron--hole (Nambu) representation \cite{nambu_quasi-particles_1960}, one obtains a quadratic quasiparticle Hamiltonian and an effective eigenvalue problem for the two-component Nambu spinor, as presented systematically by de~Gennes \cite{de_gennes_superconductivity_1966}. 
    
    The BdG framework provides a microscopic description of electron--hole conversion at normal--superconductor interfaces (Andreev reflection) \cite{andreev_thermal_1964} and underlies scattering approaches to transport in hybrid junctions, including the BTK model and its extensions \cite{blonder_transition_1982,beenakker_quantum_1992}.

    
        %=========================================================
        \subsubsection*{Single Particle Hamiltonian}
        %=========================================================

            To establish notation, we start from the normal-state single-particle description in the grand-canonical ensemble, where energies are measured relative to the chemical potential. The corresponding Hamiltonian reads
            \begin{equation}
                \hat{H}_\mathrm{N}(\vec{r}) = -\frac{\hbar^2}{2m}\nabla^2 + U(\vec{r}) - \mu\,,
                \label{eq:meso:h-n}
            \end{equation}
            where the (possibly spatially varying) potential $U(\vec r)$ models confinement, tunnel barriers, or disorder.

            The eigenfunctions of $\hat{H}_\mathrm{N}$ define the natural mode basis of the normal conductor,
            \begin{equation}
                \hat{H}_\mathrm{N}(\vec{r})\, \Phi_k(\vec{r}) = \xi_k\, \Phi_k(\vec{r})\,,\quad
                \xi_k=\varepsilon_k-\mu\,,
                \label{eq:meso:psi-n}
            \end{equation}
            where $\xi_k$ denotes the single-particle energies measured from the chemical potential. 
            
            In a homogeneous metal ($U=0$), the eigenfunctions and eigenvalues are given by
            \begin{equation}
                \Phi_{\vec q}(\vec r) \propto e^{\ima \vec q\cdot \vec r}\,,\quad
                \varepsilon_{\vec q}=\tfrac{\hbar^2 q^2}{2m}\,.
            \end{equation}

            In this translationally invariant limit, the abstract mode label $k$ can be identified with the wave vector $\vec q$. Throughout this chapter we reserve $k$ for a generic mode index and use explicit vectors such as $\vec q$ whenever a plane-wave momentum label is meant. More generally, the label $k$ should be understood as a compact index for the normal-state eigenmodes.

        %=========================================================
        \subsubsection*{BCS Hamiltonian}
        %=========================================================

            Upon second quantization, we introduce fermionic field operators $\hat{\psi}_\sigma(\vec r)$ and $\hat{\psi}_\sigma^\dagger(\vec r)$ that annihilate/create an electron at position $\vec{r}$ with spin $\sigma$. In the normal-state eigenbasis they admit the expansion $\hat\psi_\sigma(\mathbf r)=\sum_k \Phi_k(\mathbf r)\,c_{k\sigma}$, but we will work in real space in the following.

            Superconductivity is incorporated at the mean-field level by decoupling an effective attractive interaction in the spin-singlet pairing channel. This introduces the (generally position-dependent) pair potential $\Delta(\vec r)$, which couples time-reversed states and yields the quadratic mean-field (BCS) Hamiltonian
            \begin{equation}
                \begin{aligned}
                \hat{H}_\mathrm{BCS}(\vec{r}) 
                &= \int \mathrm{d}r^3 \sum_{\sigma=\uparrow, \downarrow} \hat{\psi}_\sigma^\dagger(\vec{r})\, \hat{H}_\mathrm{N}(\vec{r})\, \hat{\psi}_\sigma(\vec{r})\\
                &+ \int \mathrm{d}r^3 \left( 
                    \Delta(\vec{r})\, \hat{\psi}_\uparrow^\dagger(\vec{r})\, \hat{\psi}_\downarrow^\dagger(\vec{r})
                    + \Delta^\ast(\vec{r})\, \hat{\psi}_\downarrow(\vec{r})\, \hat{\psi}_\uparrow(\vec{r})
                    \right)
                \end{aligned}
                \label{eq:meso:h-bcs}
            \end{equation}

            Here $\Delta$ is the complex superconducting order parameter, which in general may depend on position. In the simplest $s$-wave, spin-singlet case considered throughout this thesis, it can be written as $\Delta(\vec r) = |\Delta| \, e^{\ima\phi(\vec r)}$, as introduced in Section~\ref{sec:micro}. It represents the amplitude and phase of the Cooper-pair condensate. The mean-field decoupling generates an additional condensation term, proportional to $|\Delta|^2$ and the inverse pairing interaction. Since it does not affect the quasiparticle eigenproblem, we omit it here.

        %=========================================================
        \subsubsection*{Bogoliubov Transformation (Nambu Formalism)}
        %=========================================================
        
            The bilinear structure of Eq.~\eqref{eq:meso:h-bcs} suggests working in an electron--hole representation, where quasiparticles appear as coherent superpositions of particle and hole degrees of freedom. In the spin-singlet \textit{s}-wave case, and in the absence of spin-dependent fields, the problem separates into two equivalent $2\times 2$ blocks. We work in the reduced Nambu basis
            \begin{equation}
                \hat\Psi(\vec r) =
                \begin{psmallmatrix}
                    \hat{\psi}_{\uparrow}(\vec r) \\
                    \hat{\psi}^{\dagger}_{\downarrow}(\vec r)
                \end{psmallmatrix}\,.
                \label{eq:meso:nambu-operator}
            \end{equation}

            We now diagonalize the quadratic Hamiltonian (Eq.~\eqref{eq:meso:h-bcs}) by introducing a set of fermionic quasiparticle operators $\hat\gamma_k$ labeled by a mode index $k$. Each mode is characterized by two position-dependent $c$-number amplitudes $u_k(\vec r)$ and $v_k(\vec r)$, which encode the electron- and hole-like components of the corresponding Bogoliubov quasiparticle.

            A Bogoliubov quasiparticle mode $k$ is then created by an operator of the form
            \begin{equation}
                \hat{\gamma}_k^{\dagger} = \int \mathrm{d}^3 r\,\left(u_k(\vec r)\,\hat{\psi}^{\dagger}_{\uparrow}(\vec r) + v_k(\vec r)\,\hat{\psi}_{\downarrow}(\vec r)\right),
                \label{eq:meso:bogoliubov-operator}
            \end{equation}
            where the amplitudes $u_k(\vec r)$ and $v_k(\vec r)$ quantify the electron- and hole-like components of the quasiparticle wave function. For compactness, we collect these amplitudes into the Nambu spinor
            \begin{equation}
                \Psi_k(\vec r) =
                \begin{psmallmatrix}
                    u_k(\vec r) \\
                    v_k(\vec r)
                \end{psmallmatrix}\,.
                \label{eq:meso:nambu-spinor}
            \end{equation}

        %=========================================================
        \subsubsection*{Bogoliubov--de~Gennes Equation}
        %=========================================================

            In this reduced Nambu basis, the quadratic mean-field Hamiltonian of Eq.~\eqref{eq:meso:h-bcs} can be written as the Bogoliubov--de~Gennes (BdG) Hamiltonian,
            \begin{equation}
                \hat{H}_\mathrm{BdG}(\vec r) =
                \begin{pmatrix}
                    \hat{H}_\mathrm{N}(\vec r) & \Delta(\vec r) \\
                    \Delta^\ast(\vec r) & -\hat{H}_\mathrm{N}^\ast(\vec r)
                \end{pmatrix}
                \,,
                \label{eq:meso:h-bdg}
            \end{equation}
            whose off-diagonal pairing potential $\Delta(\vec r)$ explicitly couples electron and hole amplitudes.

            The quasiparticle modes are then obtained from the BdG eigenvalue problem
            \begin{equation}
                \hat{H}_\mathrm{BdG}(\vec r) \, \Psi_k(\vec r) = E_k\, \Psi_k(\vec r)\,.
                \label{eq:meso:BdG}
            \end{equation}
            Here the index $k$ labels the resulting quasiparticle eigenmodes, while the explicit argument $\vec r$ denotes their spatial dependence. Because the BdG Hamiltonian possesses an intrinsic particle--hole symmetry, every solution at energy $+E_k$ is accompanied by a partner at $-E_k$ (with exchanged electron and hole components). In the normal-state limit $\Delta\to 0$, Eq.~\eqref{eq:meso:BdG} reduces to the eigenproblem of $\hat{H}_\mathrm{N}$ in Eq.~\eqref{eq:meso:h-n}.

        %=========================================================
        \subsubsection*{Uniform \textit{s}-wave Superconductor}
        %=========================================================

            As a solvable reference case, consider a homogeneous bulk superconductor with constant $U$ and a uniform pair potential $\Delta=|\Delta|e^{\ima\phi}$. In this case the BdG eigenmodes can be labeled by a wave vector $\vec q$ and an electron-like or hole-like branch, and take the plane-wave form
            \begin{equation}
                \Psi_{\vec q}(\vec r) = \Psi_{0}(E_{\vec q})\, e^{\ima \vec q \cdot \vec r}\,,\qquad
                \xi_{\vec q}=\frac{\hbar^2 q^2}{2m}-\mu\,,\qquad
                E_{\vec q}=\sqrt{\xi_{\vec q}^2+|\Delta|^2}\,.
                \label{eq:meso:BdG-plane-wave}
            \end{equation}
            Here $\Psi_{0}(E_{\vec q})$ is a two-component spinor containing the electron and hole amplitudes for the quasiparticle at energy $E_{\vec q}$. Choosing a gauge where $\phi$ is constant, a convenient phase convention is
            \begin{equation}
                \Psi_{0}(E_{\vec q})\equiv
                \begin{psmallmatrix}
                    u_0(E_{\vec q}) \\
                    v_0(E_{\vec q})\,e^{\ima\phi}
                \end{psmallmatrix}\,,
            \end{equation}
            so that the condensate phase appears explicitly in the hole component.
        
            In the bulk description above, quasiparticle states are naturally labeled by the mode index (here the wave vector $\vec q$). In scattering problems, by contrast, one typically works at fixed quasiparticle energy $E$ (set by bias and the reservoir distributions) and distinguishes electron-like and hole-like branches through the sign of the normal-state energy.

            Accordingly, we reparametrize the bulk dispersion relation $E^2=\xi^2+|\Delta|^2$ in terms of
            \begin{equation}
                \xi_{\pm}(E) = \pm\sqrt{E^2-|\Delta|^2}\,.
                \label{eq:meso:xi-branches}
            \end{equation}
            For $|E|<|\Delta|$ the quantity $\xi_{\pm}(E)$ is purely imaginary, reflecting that bulk solutions are evanescent rather than propagating.

            The corresponding coherence factors of a uniform $s$-wave superconductor can then be written in their standard form,
            \begin{equation}
                \begin{aligned}
                    u_0(E) &= \sqrt{\tfrac{1}{2}\left(1+\xi_{\pm}(E)\,/\,E\right)}\,,\\
                    v_0(E) &= \sqrt{\tfrac{1}{2}\left(1-\xi_{\pm}(E)\,/\,E\right)}\,.
                \end{aligned}
                \label{eq:meso:coherence-factors}
            \end{equation}
            Here the choice of branch $\xi_{\pm}(E)$ corresponds to electron-like ($+$) or hole-like ($-$) propagation. These energy-domain expressions will be used throughout the following scattering formulations (e.g. BTK), where one works at fixed $E$ rather than fixed $\vec q$.

        The absence of propagating subgap quasiparticles in a homogeneous superconductor is the microscopic origin of Andreev conversion at an N--S interface. In particular, the spectrum $E^2=\xi^2+|\Delta|^2$ implies the familiar BCS density of states, Equation~\eqref{eq:micro:dos-bcs}, while for $|E|<|\Delta|$ one must describe transport in terms of coherent electron--hole conversion processes.
    
    %=========================================================
    \subsection{Andreev Reflection}
    \label{subsec:meso:ar}
    %=========================================================

        At a normal--superconductor (N--S) interface, quasiparticles with energies $|E|<|\Delta|$ cannot propagate in the superconductor because the BCS quasiparticle continuum starts only at $|E|=|\Delta|$. Instead, the corresponding BdG solutions in the superconducting electrode are evanescent. Subgap transport across the interface is therefore mediated by coherent electron--hole conversion. An incident electron from the normal side can be reflected as a hole while a charge $2e$ is transferred into the condensate. This process is known as Andreev reflection and constitutes the elementary mechanism by which normal-state carriers couple to superconducting correlations \cite{andreev_thermal_1964}.
        
        Andreev reflection provides the dominant subgap transport mechanism in N--S junctions and serves as the microscopic building block for superconducting transport in mesoscopic weak links. In a two-terminal S--N--S geometry, successive Andreev conversions at both interfaces lead to discrete Andreev bound states (ABS) in equilibrium and to multiple Andreev reflection (MAR) under finite bias. The following sections build directly on this Andreev-based picture, developing ABS and MAR from the same underlying mechanism.

        More generally, and in particular for atomic-scale contacts, the same physics is most naturally formulated in terms of scattering channels characterized by their normal-state transmissions $\tau_i$. To make this qualitative picture quantitative and to compute the corresponding \textit{I--V} characteristics across the full transparency range, we now turn to the Blonder--Tinkham--Klapwijk (BTK) model \cite{blonder_transition_1982,beenakker_quantum_1992}.

        %=========================================================
        \subsubsection*{Blonder--Tinkham--Klapwijk Model}
        %=========================================================

            Building on the qualitative picture of Andreev reflection introduced above, the Blonder--Tinkham--Klapwijk (BTK) model provides a quantitative scattering description of an N--S interface with arbitrary transparency $\tau$. It yields the energy-resolved  probabilities for Andreev reflection ($A$), normal reflection ($B$), and transmission into the superconductor ($C$). To incorporate finite quasiparticle lifetimes, we use the phenomenological broadening introduced by Pleceník \textit{et al.}, replacing the quasiparticle energy by $E \rightarrow |E| + \ima\gamma$ \cite{plecenik_finite-quasiparticle-lifetime_1994}. 

            Using the normalization factor $d$, the Andreev and normal reflection amplitudes ($a$, $b$) acquire the compact form,
            \begin{equation}
                \begin{aligned}
                    A(E) &= aa^*\,,\quad
                    a= u_0v_0/d\,,\quad
                    d = \left(u^2_0-(1-\tau)v_0^2\right)/\tau\,,\\
                    B(E) &= bb^*\,,\quad
                    b= -(u_0^2-v_0^2)\left( (1-\tau) + \ima \sqrt{\tau(1-\tau)} \right)\,/\,d\,,
                \end{aligned}
                \label{eq:meso:btk-parameter}
            \end{equation}
            with $u_0$ and $v_0$ as coherence factors, given by Eq.~\eqref{eq:meso:coherence-factors}

            For subgap energies, transmission into the quasi-particle continuum vanishes ($C\to0$), so $A(E) + B(E) = 1$. It is therefore convenient to define the energy-resolved spectral weights $\rho$ of the single-particle (1e) and two-particle (2e) processes,
            \begin{equation}
                \begin{aligned}
                    \rho_\mathrm{1e}(E) &= 1 - A(E) - B(E)\,,\\
                    \rho_\mathrm{2e}(E) &= 2 \, A(E) \,,
                \end{aligned}
                \label{eq:meso:btk-dos}
            \end{equation}
            which quantify the relative importance of normal and Andreev processes. For finite lifetime broadening ($\gamma>0$) this strict relation is relaxed and the effective single-particle weight $\rho_{\mathrm{1e}}$ acquires finite subgap contributions. Although these functions resemble densities of states, they should not be interpreted as the physical BCS DOS. Instead, they represent the BTK spectral kernels entering the current integral (Eq.~\ref{eq:meso:btk-iv}). 

            The current through an N--S junction then follows from the BTK kernel,
            \begin{equation}
                \begin{aligned}
                    I_\mathrm{1e}(V) &= \frac{G_0}{e} \int_{-\infty}^{\infty} \rho_\mathrm{1e}(E) \left(f(E) - f(E + eV)\right) \mathrm{d}E\,,\\
                    I_\mathrm{2e}(V) &= \frac{G_0}{e} \int_{-\infty}^{\infty} \rho_\mathrm{2e}(E) \left(f(E) - f(E + eV)\right) \mathrm{d}E\,,\\
                    I_\mathrm{NS}(V) &= I_\mathrm{1e}(V) + I_\mathrm{2e}(V)\,.
                \end{aligned}
                \label{eq:meso:btk-iv}
            \end{equation}
            Increasing the transparency enhances the weight of Andreev processes, leading to a pronounced subgap conductance and a gradual reduction of the coherence-peak height. In the tunneling limit ($\tau \ll 1$), Eq.~\eqref{eq:meso:btk-iv} reduces to the conventional quasiparticle-tunneling expression, while for $\tau \approx 1$ the transport approaches the Andreev limit, where charge is transferred predominantly in units of $2e$.

            \begin{figure}[t]
                \centering
                \subfigure[Spectral weight of single-particle process]{\import{theory/meso}{btk-1e-dos.pgf}}
                \hfill
                \subfigure[Spectral weight of two-particle process]{\import{theory/meso}{btk-2e-dos.pgf}}
                \subfigure[\textit{I--V} characteristic]{\import{theory/meso}{btk-iv.pgf}}
                \hfill
                \subfigure[\textit{dI--dV} characteristics]{\import{theory/meso}{btk-didv.pgf}}
                \caption{
                    BTK spectral weights of the single-particle (1e) and two-particle (2e) channels and the corresponding \textit{I--V} and \textit{dI--dV} characteristics for various channel transparencies $\tau$. Increasing $\tau$ shifts spectral weight from normal to Andreev processes, leading to enhanced subgap conductance and reduced coherence-peak height. Parameters correspond to aluminum (Sec.~\ref{subsec:basics:aluminum}), with $T=0$ and $\gamma = 0$.
                    }
                \label{fig:meso:btk}
            \end{figure}

            Figure~\ref{fig:meso:btk} summarizes the BTK description of an N--S interface across the full transparency range. Panels (a) and (b) show the energy-resolved spectral weights $\rho_{\mathrm{1e}}(E)$ and $\rho_{\mathrm{2e}}(E)$ (Eq.~\eqref{eq:meso:btk-dos}), which enter the current integral in Eq.~\eqref{eq:meso:btk-iv}. In the tunneling regime ($\tau\ll 1$), the response is dominated by the single-particle channel and exhibits pronounced coherence peaks at $|E|\approx\Delta$. With increasing transparency, spectral weight is transferred to the Andreev channel, yielding enhanced subgap conductance and a reduced peak height, as reflected in the corresponding $I$--$V$ and $\mathrm{d}I/\mathrm{d}V$ curves in panels (c) and (d).

    %=========================================================
    \subsection{Andreev Bound States}
    \label{subsec:meso:abs}
    %=========================================================
    
        \begin{wrapfigure}[18]{r}{0.4\textwidth}
            \captionsetup{format=plain}%
            \centering
            \vspace{-1.5em}
            \import{theory/macro}{delta-r.pgf}
            \textbf{This is a placeholder!}
            \caption{
                Spatial profile of $\Delta_1(r)$ (\legend{seeblau100}) and $\Delta_2(r)$ (\legend{seegrau100}) across a tunnel junction (\legend{seegrau65}). Their respective magnitude $|\Delta|$ (\legend{seeblau35}/\legend{seegrau35}) varies only weakly accross the barrier. Coherent coupling is governed by the macroscopic phase difference $\phi$.
                }
            \label{fig:meso:delta-r}
        \end{wrapfigure}
        Andreev reflection becomes phase-coherent and spectrally quantized when two superconductors are connected by a mesoscopic weak link. In an S--N--S geometry, a subgap quasiparticle ($|E|<|\Delta|$) cannot escape into the superconducting continua and instead undergoes successive Andreev conversions at both interfaces, alternating between electron- and hole-like character while acquiring the superconducting phase difference 
        \begin{equation}
            \phi = \phi_1 - \phi_2\,.
            \label{eq:meso:phase-difference}
        \end{equation}
        The resulting constructive-interference condition quantizes the motion into discrete subgap eigenstates localized around the junction, known as Andreev bound states (ABS) \cite{andreev_thermal_1964,kulik_effect_1969,ishii_josephson_1970}. 
        
        These ABS form the microscopic origin of the dc Josephson effect in short and mesoscopic junctions and their phase-dependent energies $E_n(\phi)$ determine the equilibrium supercurrent via their occupation \cite{beenakker_josephson_1991, beenakker_quantum_1992}.

        %=========================================================
        \subsubsection*{Andreev Approximation and Limit}
        %=========================================================

            A transparent route from BdG to ABS is to treat Andreev reflection as an energy-dependent boundary condition at the superconducting leads, while representing the weak link by its normal-state scattering properties.

            In the Andreev approximation, $|E|,|\Delta|\ll E_\mathrm{F}$, only quasiparticles in a narrow shell around the Fermi surface contribute. For an isotropic parabolic band, the dispersion can be linearized in the radial direction,
            \begin{equation}
                \xi(k) \approx \tfrac{\hbar^2 }{m} \left( {k}_\mathrm{F} \cdot ({k}-{k}_\mathrm{F}) \right)\,,
            \end{equation}
            so that electron-like and hole-like components of a quasiparticle have nearly equal momentum magnitudes. For an excitation at energy $E$ the corresponding wave numbers satisfy
            \begin{equation}
                2\,\delta k = k_\mathrm{e} - k_\mathrm{h} \approx k_\mathrm{F}\, E/ E_\mathrm{F}\,,
            \end{equation}
            implying that the Andreev-reflected hole approximately retraces the incoming trajectory, so called retroreflection, up to corrections of order $E/E_\mathrm{F}$.

            In the Andreev limit of a clean, specular, and highly transparent N--S interface with negligible normal reflection, subgap conversion occurs with approximately unit probability. The corresponding Andreev reflection amplitude for an electron being reflected as a hole at the electrode can then be written as a pure phase factor,
            \begin{equation}
                r_\mathrm{A}(E) = e^{-\ima\phi_\mathrm{A}(E)}\,e^{-\ima\phi}\,,\qquad
                \phi_\mathrm{A}(E)\equiv\arccos\!\left(E/|\Delta|\right)\,.
                \label{eq:meso:andreev-leads}
            \end{equation}
            Here $\phi_\mathrm{A}(E)$ is the energy-dependent Andreev phase, while $\phi$ denotes the condensate phase of the superconducting electrode. For nonideal interfaces, finite backscattering reintroduces normal reflection and reduces the Andreev amplitude below unit.

        %=========================================================
        \subsubsection*{Energy Spectra and Current Phase Relation}
        %=========================================================

            \begin{figure}[t]
                \centering
                \subfigure[
                    Energy Spectra
                    ]{\import{theory/meso}{abs-Ephi.pgf}}
                \hfill
                \subfigure[
                    Current Phase Relation
                    ]{\import{theory/meso}{abs-Iphi.pgf}}
                \caption{
                    Andreev bound states and equilibrium current--phase relation of a short single-channel contact for several transparencies $\tau$ (light: tunnel-like, dark: near ballistic) at $T=0$. 
                    (a) Phase dispersion of the Andreev levels $E_\pm(\phi)$. With increasing $\tau$, the levels acquire a stronger phase dependence and the minimal splitting at $\phi=\pi$ decreases, approaching a level crossing in the ballistic limit.
                    (b) Corresponding equilibrium supercurrent $I(\phi)$ obtained from the occupied branch via Eq.~\eqref{eq:meso:cpr-single}. Increasing $\tau$ drives the current--phase relation away from the sinusoidal tunnel limit towards a strongly skewed, cusp-like form near $\phi=\pi$, reflecting the increasingly sharp phase dispersion of the bound state.
                }
                \label{fig:meso:abs}
            \end{figure}

            A complete Andreev cycle through the junction converts an electron into a hole at one interface and back into an electron at the other. A bound state forms when this closed electron--hole trajectory reproduces itself, i.e. when the total phase accumulated in one round trip is an integer multiple of $2\pi$.

            In the short-junction limit, the dwell time through the normal region is much shorter than $\hbar/|\Delta|$. The energy dependence of the normal-region scattering can then be neglected for quasiparticle energies $|E|\lesssim|\Delta|$, and the weak link is fully characterized by the transmission eigenvalues $\tau_i$ of its normal-state scattering matrix.

            For a single channel of transmission $\tau$, the constructive-interference condition reduces to the compact form
            \begin{equation}
                \sin^2\!\phi_\mathrm{A}(E) = \tau\,\sin^2\!\left(\phi/2\right)\,,
                \label{eq:meso:abs-quantization}
            \end{equation}
            which directly yields the familiar Andreev bound-state spectrum 
            \begin{equation}
                E_{\pm}(\phi)=\pm|\Delta|\sqrt{1-\tau\sin^2\!\left(\phi/2\right)}\,.
                \label{eq:meso:abs-spectrum}
            \end{equation}
            Figure~\ref{fig:meso:abs}(a) shows the spectura for different transmissions.

            The phase dependence of the ABS spectrum implies that these states carry a nondissipative supercurrent. In equilibrium, the current follows from the derivative of the junction free energy with respect to the superconducting phase difference.

            Equivalently, one may express the contribution of each ABS branch through its occupation, which yields the standard relation
            \begin{equation}
                I(\phi)=-2\pi \frac{G_0}{e} \frac{\partial E(\phi)}{\partial \phi}\,\tanh\!\left(\frac{E(\phi)}{2k_\mathrm{B}T}\right)\,.
                \label{eq:meso:cpr-general}
            \end{equation}
            Here $E(\phi)$ denotes the positive-energy ABS spectrum, and the hyperbolic tangent accounts for thermal occupation.

            Inserting Eq.~\eqref{eq:meso:abs-spectrum} yields
            \begin{equation}
                I(\phi)=\frac{\pi}{2}\frac{G_0}{e}|\Delta|\,
                \frac{\tau\sin\phi}{\sqrt{1-\tau\sin^2\!\left(\phi/2\right)}}\,
                \tanh\!\left(\frac{|\Delta|\sqrt{1-\tau\sin^2\!\left(\phi/2\right)}}{2k_\mathrm{B}T}\right)\,.
                \label{eq:meso:cpr-single}
            \end{equation}
            In the zero-temperature limit, $\tanh\!\left(E/2k_\mathrm{B}T\right)\to 1$ and the current is set solely by the phase dispersion of the bound state, as shown in Fig.~\ref{fig:meso:abs}(b).

            In the tunneling limit ($\tau\ll 1$), the current--phase relation becomes sinusoidal and thus reduces to the standard Josephson form discussed in Sec.~\ref{sec:macro}. In the same limit, the corresponding critical current is consistent with the Ambegaokar--Baratoff result when expressed in terms of the normal-state resistance.

            In the opposite ballistic limit ($\tau\to 1$), the bound-state spectrum becomes particularly simple,
            \begin{equation}
                E_{\pm}(\phi)=\pm|\Delta|\,\left|\cos\!\left(\phi/2\right)\right|\,.
                \label{eq:meso:abs-ballistic}
            \end{equation}
            The corresponding current at $T=0$ follows from Eq.~\eqref{eq:meso:cpr-single},
            \begin{equation}
                I(\phi)=\tfrac{\pi}{2}\tfrac{G_0}{e}|\Delta|\,\sin\!\left(\phi/2\right)\,\mathrm{sgn}\!\left(\cos\!\left(\phi/2\right)\right)\,.
                \label{eq:meso:cpr-ballistic}
            \end{equation}
            Here, the current exhibits a cusp at $\phi=\pi$ originating from the crossing of the ABS branches. Any deviation from perfect transmission opens a finite minimum gap $E_\mathrm{min}=|\Delta|\sqrt{1-\tau}$ at $\phi=\pi$ and smooths this feature, while finite temperature further rounds it through the occupation factor.

        %=========================================================
        \subsubsection*{Multi-Channel Contacts}
        %=========================================================

            Real atomic contacts generally support several conduction channels. In the short-junction limit, each channel $i$ is characterized by a transmission eigenvalue $\tau_i$ and supports a pair of Andreev branches $E_{i,\pm}(\tau_i,\phi)$ given by Eq.~\eqref{eq:meso:abs-spectrum}. The equilibrium supercurrent follows by summing the single-channel contribution (Eq.~\eqref{eq:meso:cpr-single}) over all channels,
            \begin{equation}
                I(\phi,T)=\sum_i I(\tau_i,\phi,T)\,.
                \label{eq:meso:cpr-multichannel}
            \end{equation}
            In the many-mode limit, it is often convenient to replace the discrete set $\{\tau_i\}$ by a statistical distribution $\rho(\tau)$ and approximate the channel sum by an average,
            \begin{equation}
                I(\phi,T)\simeq \int_0^1 \rho(\tau)\, I(\tau,\phi,T)\,\mathrm d\tau\,.
                \label{eq:meso:cpr-multichannel-rho}
            \end{equation}
            In this representation, $\rho(\tau)$ is normalized such that $G_\mathrm N=G_0\int_0^1 \rho(\tau)\,\tau\,\mathrm d\tau$ \cite{beenakker_josephson_1991,beenakker_quantum_1992}.

            In the dirty short-contact limit, the transmission eigenvalues of a quasi-one-dimensional diffusive wire follow the Dorokhov--Mello--Pereyra--Kumar (DMPK) distribution \cite{dorokhov_transmission_1982},
            \begin{equation}
                \rho(\tau) = \frac{G_\mathrm{N}}{G_0} \frac{1}{2\tau \sqrt{1-\tau}}\,,\qquad 0<\tau<1\,.
                \label{eq:meso:dmpk}
            \end{equation}
            The density is bimodal, with integrable divergences at $\tau\to 0$ and $\tau\to 1$, as shown in Fig.~\ref{fig:meso:abs-rhotau}. This indicates that a diffusive conductor can be viewed statistically as a mixture of many almost-closed and a few almost-open channels.
            \begin{figure}[ht]
                \centering
                \includegraphics[width=.42\textwidth]{theory/meso/abs-rhotau.png}
                \caption{DMPK distribution $\rho(\tau)$ of transmission eigenvalues for a short diffusive wire. The distribution is bimodal with integrable divergences at $\tau\to 0$ and $\tau\to 1$, and is normalized such that $G_\mathrm N=G_0\int_0^1 \rho(\tau)\,\tau\,\mathrm d\tau$.}
                \label{fig:meso:abs-rhotau}
            \end{figure}
                        
            Averaging the short-junction single-channel result over Eq.~\eqref{eq:meso:dmpk} yields the first Kulik--Omel'yanchuk (KO--1) current--phase relation in the dirty (diffusive) limit. At $T=0$, it reduces to the closed form
            \begin{equation}
                I_\mathrm{KO1}(\phi) = \tfrac{\pi}{2} \tfrac{G_\mathrm{N}}{e}|\Delta|\,\cos\!\left(\phi/2\right)\,\operatorname{artanh}\!\left(\sin\!\left(\phi/2\right)\right)\,.
                \label{eq:meso:ko-1}
            \end{equation}
            Although $\operatorname{artanh}[\sin(\phi/2)]$ diverges as $\phi\to\pi$, the prefactor $\cos(\phi/2)$ vanishes such that the product, and therefore the supercurrent, remains finite.

            In the complementary clean short-contact limit, transport is dominated by (nearly) open channels and the current--phase relation (KO--2) takes the compact finite-temperature form
            \begin{equation}
                I_\mathrm{KO2}(\phi, T) = \pi\,\frac{G_\mathrm{N}}{e}\,\Delta(T)\,\sin\!\left(\frac{\phi}{2}\right)\,
                    \tanh\!\left(\frac{\Delta(T)\cos\!\left(\phi/2\right)}{2k_\mathrm{B}T}\right)\,.
                \label{eq:meso:ko-2}
            \end{equation}
            Both limits provide widely used reference models for short superconducting weak links \cite{kulik_contribution_1975,golubov_current-phase_2004}.

            Figure~\ref{fig:meso:abs-ko} compares the temperature dependence of the critical current obtained from the Ambegaokar--Baratoff result and from the KO--1/KO--2 current--phase relations by applying $I_\mathrm{C}(T)=\max_{\phi\in[0,2\pi]} I(\phi,T)$ with a BCS-like gap $\Delta(T)$.
            \begin{figure}[ht]
                \centering
                \includegraphics[width=.55\textwidth]{theory/meso/abs-ko.png}
                \caption{Temperature dependence of the critical current $I_\mathrm{C}(T)$ for three canonical short-junction limits: Ambegaokar--Baratoff (tunnel junction), KO--1 (diffusive short contact), and KO--2 (clean short contact). The critical current is obtained by maximizing the corresponding current--phase relation with respect to $\phi$ using a BCS-like gap $\Delta(T)$.}
                \label{fig:meso:abs-ko}
            \end{figure}


        % %=========================================================
        % \subsubsection*{Fractional Shapiro Steps}
        % %=========================================================

            % For intermediate and increasing transmission ($0<\tau<=1$), the progressive skewing of the current--phase relation, can equivalently be expressed as an enhanced higher-harmonic content in a Fourier expansion $I(\phi)=\sum_{n\ge 1} I_n\sin(n\phi)$. This harmonic content will be central when discussing microwave-driven phase dynamics and fractional Shapiro steps below.
    \newpage


    \section*{TODO}
    \subsection*{BCS}
    \begin{itemize}
        \item schemata tunneling, nin, nis, sis
        \item $I(V, T, Gamma)$
        \item schemata pat nin, nis, sis
        \item $I(V, A, \nu)$
        \item reference the figures
        \item do the citations
    \end{itemize}

    \subsection*{Josephson}
    \begin{itemize}
        \item Josephson IV
        \item shapiro IV
        \item rscj shaltplan
        \item rcsj washboard potential
    \end{itemize}

    \subsection*{MAR}
    \begin{itemize}
        \item AR NS Schaubild
        \item AR NS I(V)
        \item MAR SS I(V)
        \item do citations
        \item Photon-Assisted MAR (PAMAR)
    \end{itemize}

    \newpage
