% !TEX root = ../thesis.tex

%=========================================================
\setcounter{section}{-1}
\section{Basic Concepts}
\label{sec:basics}
%=========================================================
    
    In this section we collect the basic microscopic and mesoscopic concepts that underlie all transport phenomena discussed in this thesis. We start from a many-electron Hamiltonian in a crystalline solid and show how its low-energy excitations can be described in terms of quasiparticles with effective mass, lifetime, and mean free path, leading to the Drude picture of diffusive transport. On this basis we introduce the mesoscopic transport regimes, where the finite size of the conductor and phase coherence across it become essential and transport is most naturally formulated in terms of transmission channels and their eigenvalues. We then specialize to aluminum, which serves as the central material platform of this work, and summarize the normal-state, superconducting, and oxide properties that make it ideally suited for atomic-scale contacts and tunnel junctions. Finally, we outline the generic microwave drive model used throughout the thesis and the associated electromagnetic phase factors that give rise to photon-assisted tunnelling and driven Josephson dynamics.

    %=========================================================
    \subsection{Microscopic Description of Quasi-Particles}
    \label{subsec:basics:quasiparticles}
    %=========================================================

        We consider a many-electron system in a crystalline solid and show how its low-energy excitations can be described in terms of quasiparticles. The full Hamiltonian (first quantization) for $N$ electrons and fixed ionic cores at positions $\{\vec{R}_\alpha\}$ is
        \begin{equation}
            \hat{H}
            = \sum_{i=1}^N \left[
                \frac{\hat{\vec{p}}_i^2}{2m}
                + V_\text{ion}(\vec{r}_i)
            \right]
            + \frac{1}{2} \sum_{i \neq j}
            \frac{e^2}{4\pi\varepsilon_0 \lvert \vec{r}_i - \vec{r}_j \rvert},
            \label{eq:micro:full-hamiltonian}
        \end{equation}
        with canonical momenta $\hat{\vec{p}}_i = -\mathrm{i}\hbar\nabla_i$ and electron mass $m$. The ionic potential $V_\text{ion}(\vec{r})$ represents the Coulomb attraction of the electrons to the positively charged lattice. To a good approximation it is periodic,
        \begin{equation}
            V_\text{ion}(\vec{r} + \vec{R})
            = V_\text{ion}(\vec{r})
        \end{equation}
        for all Bravais lattice vectors $\vec{R}$.

        The many-electron wave function $\Psi(\vec{r}_1,\dots,\vec{r}_N,t)$ obeys the time-dependent Schr\"odinger equation
        \begin{equation}
            \mathrm{i}\hbar \frac{\partial}{\partial t} \Psi
            = \hat{H}\,\Psi,
            \label{eq:micro:schroedinger}
        \end{equation}
        supplemented by antisymmetry under particle exchange due to Fermi statistics. Solving this problem exactly is not feasible. To arrive at a tractable description of electronic transport we now introduce a sequence of simplifications.

        %=========================================================
        \subsubsection*{Effective One-Electron Description}
        %=========================================================

        First, the fast electronic motion is separated from the slow ionic motion (Born--Oppenheimer approximation), and the ions are treated as fixed classical scatterers that generate the periodic potential $V_\text{ion}(\vec{r})$ plus some weak disorder. Next, electron--electron interactions are taken into account in an average way. Screening in a metal strongly reduces the long-range Coulomb repulsion, and the dominant effect of interactions can be encoded in an effective potential and a renormalized band structure.

        On this level, the complicated many-electron problem is replaced by an effective one-electron Schr\"odinger equation
        \begin{equation}
            \hat{H}_\text{eff}\,\psi(\vec{r})
            = E\,\psi(\vec{r}),
            \qquad
            \hat{H}_\text{eff}
            = \frac{\hat{\vec{p}}^2}{2m} + V_\text{per}(\vec{r}),
            \label{eq:micro:heff}
        \end{equation}
        where $V_\text{per}(\vec{r})$ is a periodic effective potential that already includes the mean-field action of the other electrons. Electron--electron interactions beyond this mean field will later reappear as finite quasiparticle lifetimes and effective masses.

        The stationary Schr\"odinger equation with a periodic potential,
        \begin{equation}
            \left[
                -\frac{\hbar^2}{2m}\nabla^2 + V_\text{per}(\vec{r})
            \right] \psi(\vec{r})
            = E\,\psi(\vec{r}),
            \label{eq:micro:schroedinger-periodic}
        \end{equation}
        is the starting point for band theory.

        %=========================================================
        \subsubsection*{Bloch States and Band Structure}
        %=========================================================

        Because $V_\text{per}(\vec{r})$ has the periodicity of the lattice, Bloch's theorem applies. The eigenstates of $\hat{H}_\text{eff}$ can be chosen as Bloch functions
        \begin{equation}
            \psi_{n\vec{k}}(\vec{r})
            = \mathrm{e}^{\mathrm{i}\vec{k}\cdot\vec{r}}\,
            u_{n\vec{k}}(\vec{r}),
            \label{eq:micro:bloch}
        \end{equation}
        where $n$ is a band index, $\vec{k}$ lies in the first Brillouin zone, and $u_{n\vec{k}}(\vec{r}+\vec{R}) = u_{n\vec{k}}(\vec{r})$ is lattice-periodic. Inserting Eq.~\eqref{eq:micro:bloch} into Eq.~\eqref{eq:micro:schroedinger-periodic} yields for each $\vec{k}$ a discrete set of eigenvalues
        \begin{equation}
            E_n(\vec{k}),
        \end{equation}
        the \emph{band structure}.

        The Bloch functions form an orthonormal basis. In the independent-electron approximation, the many-electron ground state is obtained by filling the one-particle eigenstates up to the Fermi energy $E_\mathrm{F}$, with at most two electrons (spin up and spin down) per state. For a given electron density, the filling of the bands fixes $E_\mathrm{F}$ and the corresponding Fermi surface defined by
        \begin{equation}
            E_n(\vec{k}) = E_\mathrm{F}.
        \end{equation}

        In simple metals one or a few bands cross the Fermi level. For transport at low temperature and low bias only states in a narrow energy window $\lvert E - E_\mathrm{F} \rvert \sim \max(k_\mathrm{B}T, eV)$ are relevant. In this regime, the detailed band structure away from the Fermi surface is irrelevant and can be compressed into a small set of effective parameters such as the Fermi velocity and the density of states at the Fermi energy.

        %=========================================================
        \subsubsection*{Semiclassical Dynamics and Scattering}
        %=========================================================

        To connect this microscopic picture to transport, we consider how Bloch electrons move in weak external fields. On length and time scales large compared to the lattice spacing and inverse band gaps, the dynamics of an electron wave packet built from states in a single band $n$ can be described semiclassically by
        \begin{equation}
            \hbar \dot{\vec{k}} = -e\bigl[
                \vec{E} + \dot{\vec{r}} \times \vec{B}
            \bigr],
            \qquad
            \dot{\vec{r}} = \frac{1}{\hbar}\,\nabla_{\vec{k}} E_n(\vec{k}),
            \label{eq:micro:semiclassical}
        \end{equation}
        where $\dot{\vec{r}}$ is the group velocity, given by the gradient of the band energy, and $\vec{E}$, $\vec{B}$ are external electric and magnetic fields.

        In a perfect, infinite crystal without interactions, Bloch electrons would evolve according to Eq.~\eqref{eq:micro:semiclassical} and, for a uniform electric field, undergo Bloch oscillations rather than exhibit steady Ohmic transport. Real materials contain impurities, lattice defects, and phonons, and electrons also interact with each other. These processes scatter electrons between Bloch states and relax the current.

        Microscopically, such scattering can be encoded in a finite lifetime $\tau$ for a quasiparticle state $\lvert n,\vec{k}\rangle$. The semiclassical picture then assumes that between scattering events electrons move as Bloch wave packets, and at random times (with rate $1/\tau$) they are scattered into new states. This is the basis of the Boltzmann transport equation and recovers simple results such as the Drude conductivity when one further approximates the band as parabolic.

        %=========================================================
        \subsubsection*{Quasiparticles and Effective Single-Band Description}
        %=========================================================

        The independent-electron band picture treated interactions only at a mean-field level. In reality, the charged fermions in the Hamiltonian, which we will refer to as \emph{electrons}, interact strongly with each other and with lattice vibrations. The low-energy excitations of a metal are therefore not bare electrons but \emph{quasiparticles}. In Landau's Fermi-liquid picture these quasiparticles are long-lived fermionic modes that carry charge $-e$ and spin $1/2$, but are dressed by their interaction with the surrounding Fermi sea. One may think of a quasiparticle as an electron together with its screening cloud in the medium.

        Close to the Fermi surface, the dispersion of a given band can be simplified by expanding it to quadratic order around a Fermi wave vector $\vec{k}_\mathrm{F}$. In an isotropic approximation this yields
        \begin{equation}
            E(\vec{k}) \approx E_\mathrm{F}
            + \frac{\hbar^2 \lvert \vec{k} - \vec{k}_\mathrm{F} \rvert^2}{2m^\ast},
            \label{eq:micro:parabolic}
        \end{equation}
        which defines the \emph{effective mass} $m^\ast$. The effective mass generally differs from the bare electron mass $m$ and encodes both the band-structure curvature and the renormalization due to interactions. Within this single-band, parabolic approximation, the quasiparticle group velocity is
        \begin{equation}
            \vec{v}(\vec{k})
            = \frac{1}{\hbar} \nabla_{\vec{k}} E(\vec{k})
            \approx \frac{\hbar(\vec{k} - \vec{k}_\mathrm{F})}{m^\ast},
        \end{equation}
        and the Fermi velocity is $v_\mathrm{F} = \hbar k_\mathrm{F} / m^\ast$.

        The density of states per unit volume for a three-dimensional parabolic band reads
        \begin{equation}
            N(E) = \frac{1}{2\pi^2} \left( \frac{2m^\ast}{\hbar^2} \right)^{3/2} \sqrt{E}\,,
        \end{equation}
        which varies only weakly on the meV scale around the Fermi energy. It is therefore convenient to approximate it by a constant
        \begin{equation}
            N(E) \approx N_0 = N(E_\mathrm{F})\,,
        \end{equation}
        in the energy range relevant for low-temperature transport. This constant $N_0$ together with $v_\mathrm{F}$, $m^\ast$, and the elastic scattering time $\tau$ fully characterizes normal-state transport in simple metals in the regime of interest.

        Microscopically, a quasiparticle at momentum $\vec{k}$ and band index $n$ is thus characterized by the renormalized dispersion $\varepsilon_{n\vec{k}}$ (encoded in $m^\ast$), by a finite lifetime $\tau_{n\vec{k}}$, and by a mean free path $\ell = v_\mathrm{F}\tau$. Residual quasiparticle--quasiparticle interactions remain, but they are weak and can be treated perturbatively for excitations sufficiently close to the Fermi energy. As long as we restrict ourselves to a small energy window around $E_\mathrm{F}$, these quasiparticles are well defined and provide a faithful description of the low-energy dynamics of the interacting electron system.

        In the following, we explicitly distinguish between electrons and quasiparticles: ``electrons'' denote the bare microscopic fermions entering the Hamiltonian~\eqref{eq:micro:full-hamiltonian}, while ``quasiparticles'' denote the dressed low-energy excitations that carry current and appear in transport theory. All microscopic transport models in this thesis should therefore be understood as models for quasiparticles near the Fermi level, characterized by an effective mass $m^\ast$, a Fermi velocity $v_\mathrm{F}$, a constant normal density of states $N_0 = N(E_\mathrm{F})$, and a finite lifetime $\tau$ set by scattering off impurities, phonons, and other quasiparticles.

        %=========================================================
        \subsubsection*{Drude Model from the Microscopic Picture}
        %=========================================================

        Within the semiclassical picture the simplest transport model is the Drude description of a homogeneous metal. We treat the quasiparticles introduced above as classical point-like carriers with charge $-e$, effective mass $m^\ast$, and density $n$, which experience random momentum-relaxing collisions with an average time $\tau$. In a weak, static electric field $\vec{E}$ the equation of motion for the average quasiparticle momentum reads
        \begin{equation}
            \frac{d\langle \vec{p} \rangle}{dt}
            = -e \vec{E} - \frac{\langle \vec{p} \rangle}{\tau}\,,
        \end{equation}
        where the second term is a simple relaxation-time approximation for the effect of scattering. In the steady state, $d\langle \vec{p} \rangle / dt = 0$ and $\langle \vec{p} \rangle = - e \tau \vec{E}$. The average drift velocity is therefore
        \begin{equation}
            \vec{v}_\mathrm{drift}
            = \frac{\langle \vec{p} \rangle}{m^\ast}
            = - \frac{e \tau}{m^\ast} \vec{E}\,.
        \end{equation}
        Multiplying by the carrier density $n$ gives the current density
        \begin{equation}
            \vec{j} = - n e \vec{v}_\mathrm{drift}
            = \frac{n e^2 \tau}{m^\ast} \vec{E}\,,
        \end{equation}
        which identifies the Drude conductivity
        \begin{equation}
            \sigma = \frac{n e^{2} \tau}{m^\ast}\,.
        \end{equation}
        In this bulk limit the system is assumed to be spatially homogeneous, and spatial variations of the electric field and current density are slow on the scale of the microscopic scattering lengths, such that transport is fully captured by the local constitutive relation $\vec{j} = \sigma \vec{E}$.

        The same microscopic parameters also fix the diffusive motion of quasiparticles. The randomization of momentum over a time $\tau$ defines a mean free path $\ell = v_\mathrm{F}\tau$, where $v_\mathrm{F}$ is the Fermi velocity, and the coarse-grained dynamics of the carrier density are diffusive with diffusion constant
        \begin{equation}
            D = \frac{v_\mathrm{F}^{2}\tau}{3}\,.
        \end{equation}
        These quantities $\sigma$, $\ell$, and $D$ summarize the influence of microscopic scattering on transport in the macroscopic, diffusive limit and will be the natural reference point for the mesoscopic regimes discussed in the following subsection.
                
    %=========================================================
    \subsection{Mesoscopic Transport of Quasiparticles}
    \label{subsec:basics:mesoscopic}
    %=========================================================

        An additional important length scale is the phase-coherence length $L_\phi$. It is set by inelastic and dephasing processes such as electron--electron and electron--phonon scattering and measures how far a quasiparticle can propagate while retaining a well-defined quantum phase. In a diffusive conductor one may write $L_\phi = \sqrt{D\tau_\phi}$ in terms of the diffusion constant $D$ and the dephasing time $\tau_\phi$, whereas in the ballistic limit $L_\phi \approx v_\mathrm{F}\tau_\phi$. For $L \lesssim L_\phi$ quantum interference between different scattering paths is observable and leads to mesoscopic phenomena such as weak localization and universal conductance fluctuations, while for $L \gg L_\phi$ these effects are averaged out and transport reduces to the classical Drude description.

        Mesoscopic transport refers to the regime in which at least one characteristic length scale of the conductor becomes comparable to microscopic electronic length scales. Relevant quantities are the system size $L$, the elastic mean free path $\ell = v_\mathrm{F}\tau$, the phase-coherence length $L_\phi$, and, in diffusive systems, the thermal length $L_T = \sqrt{\hbar D / k_\mathrm{B}T}$. For $L \gg \ell$ electrons undergo many scattering events and transport is well described by the bulk Drude picture. When $L$ approaches $\ell$ or $L_\phi$, the finite size of the conductor and phase coherence across it begin to dominate transport, and a fully microscopic, quantum description becomes necessary.

        %=========================================================
        \subsubsection*{Diffusive Mesoscopic Transport}
        %=========================================================

        In the mesoscopic diffusive regime the mean free path remains much shorter than the system size, $\ell \ll L$, but phase coherence can persist over many scattering events, $L \lesssim L_\phi$. On average, transport follows the classical drift-diffusion picture with conductance and diffusion constant, as introduced in the Drude model above. Quantum mechanically, however, the multiple scattering paths interfere. This gives rise to small but characteristic corrections to the Drude conductance, such as weak localization and universal conductance fluctuations, and in strongly disordered or low-dimensional systems may ultimately lead to Anderson localization. In the present work we only require the diffusive regime as a reference for the ballistic limit realized in atomic-scale contacts.

        %=========================================================
        \subsubsection*{Ballistic Transport and Transmission Channels}
        %=========================================================

        In the ballistic regime the length of the constriction is shorter than the mean free path, $L \ll \ell$, such that quasiparticles traverse the contact without momentum-relaxing scattering. Resistance can no longer be attributed to dissipative processes distributed along the conductor and instead arises from partial reflection at the contact region that connects two metallic reservoirs. A convenient description is then provided by the scattering matrix of the contact. At a given energy $E$ the constriction supports a finite number of transverse modes that can be combined into transmission eigenchannels with energy-dependent transmission probabilities $\tau_i(E)$.

        In the linear-response regime at low temperature the conductance of such a phase-coherent contact is given by the Landauer formula
        \begin{equation}
            G = G_0 \sum_i \tau_i, \qquad G_0 = \frac{2e^2}{h}\,,
        \end{equation}
        where $\tau_i$ are the transmission eigenvalues at the Fermi energy. Each channel contributes a fraction $\tau_i$ of the conductance quantum $G_0$. The full information about normal-state transport through the contact is thus encoded in the set of transmission eigenvalues $\{\tau_i\}$, often referred to as the \emph{pin code} of the contact.

        %=========================================================
        \subsubsection*{Atomic-Scale Contacts}
        %=========================================================

        Metallic atomic contacts provide a paradigmatic realization of ballistic, phase-coherent transport. The constriction consists of only one or a few atoms, far shorter than any microscopic scattering length, so that $L \ll \ell$ and $L \ll L_\phi$ are automatically fulfilled. In contrast to semiconductor quantum point contacts, where the number of channels is controlled by geometric confinement, the transmission channels of an atomic contact are determined by the valence orbitals of the atoms forming the narrowest part of the constriction. For aluminum, with three relevant valence orbitals, this typically results in three dominant transport channels: one $sp_z$-like channel and two transverse $p$-like channels.

        Experimentally, the set of transmission probabilities $\{\tau_i\}$ can be extracted from superconducting transport measurements using multiple Andreev reflection and related techniques. In this thesis we will consistently use the Landauer picture and characterize each atomic contact by its transmission eigenvalues. These normal-state pin codes then serve as input for the microscopic description of superconducting transport phenomena discussed in the following sections.

        For comparison, quantum point contacts in high-mobility two-dimensional electron gases realize a similar ballistic, phase-coherent constriction, but with many channels whose number can be tuned electrostatically. There, the same Landauer formula yields the well-known quantized conductance staircase $G \approx N G_0$ when each occupied mode is almost perfectly transmitted, $\tau_i \simeq 1$. While such semiconductor QPCs provide a clean didactic example of channel quantization, the focus of this work lies on metallic atomic contacts, where the number and transparency of channels are dictated by atomic-scale structure and chemistry rather than by lithographic geometry.

    %=========================================================
    \subsection{Aluminum}
    \label{subsec:basics:aluminum}
    %=========================================================

        Aluminum serves as the central material platform of this thesis. It provides both the normal-state metallic electrodes and the weak-coupling BCS superconductor in all transport experiments discussed in the following.

        Aluminum is a simple $sp$-metal with three valence electrons per atom and an fcc crystal structure. In the effective single-band picture introduced above, its low-energy normal-state properties are encoded in the Fermi wave vector, Fermi velocity, effective mass, and normal-state density of states. For bulk aluminum the band structure is nearly free-electron-like, such that $m^\ast \simeq m$ provides a good approximation. Typical values are a Fermi velocity of about $2 \times 10^6\,\mathrm{m/s}$, corresponding to a Fermi wavelength of order $0.5\,\mathrm{nm}$, and a Fermi energy in the $10\,\mathrm{eV}$ range. In high-purity thin films the elastic mean free path reaches tens of nanometres, well above all atomic length scales, placing aluminum firmly in the regime of good metals with weak disorder and among the best conducting elemental metals (after silver, copper, and gold).

        For aluminum thin films used in typical mesoscopic experiments, the phase-coherence length $L_\phi$ is not a fixed material constant but depends sensitively on disorder, geometry, and temperature. At sub-kelvin temperatures it is commonly of order micrometres, sufficiently large to treat transport through atomic-scale contacts as fully phase coherent and to justify the ballistic picture used throughout this thesis. At the atomic scale, the conduction channels in an aluminum contact directly reflect the symmetry of the three valence orbitals near the Fermi energy, typically resulting in one predominantly $sp_z$-like channel and two transverse $p$-like channels that reappear in the channel-resolved analysis of superconducting transport below.

        %=========================================================
        \subsubsection*{Superconductor}
        %=========================================================

        In all superconducting parts of this thesis, aluminum acts as a prototypical weak-coupling BCS superconductor. It has a nearly free-electron-like band structure, an isotropic $s$-wave energy gap\footnote{
            Here and in the following, ``$s$-wave'' refers to the isotropic angular dependence of the superconducting order parameter on the Fermi surface, classified by spherical harmonics, and should not be confused with atomic $s$ orbitals. In aluminum the conduction band has predominantly $sp$-hybridized character, but the Cooper-pair wave function is $s$-wave in the sense of an isotropic pairing symmetry.
            }, 
        and a well-understood phonon-mediated pairing mechanism. 
            
        The zero-temperature gap and critical temperature are to good approximation
        \begin{equation}
            \Delta_0 \approx 180\,\mu\mathrm{eV}\,,\qquad
            T_\mathrm{C} \approx 1.2\,\mathrm{K}\,.
            \label{eq:basics:aluminum}
        \end{equation}
        The comparatively small gap places typical microwave frequencies well below the pair-breaking threshold, enabling controlled photon-assisted processes without degrading superconductivity\footnote{
            The pair-breaking frequency of aluminum is given by $\nu_\mathrm{pb}=2\Delta_0/h\approx 87\,\mathrm{GHz}$.
            }. 

        In the clean limit the superconducting coherence length is given by $\xi_0 \simeq \hbar v_\mathrm{F}/\pi\Delta_0$, yielding a value of order $1$--$2\,\text{\textmu m}$ for aluminum. Its long coherence length, high electronic purity, and weak spin--orbit scattering make aluminum ideally suited for tunneling spectroscopy and mesoscopic superconducting transport. 

        %=========================================================
        \subsubsection*{Aluminum Oxide}
        %=========================================================

        A key practical advantage of aluminum is its native oxide. When exposed to oxygen, aluminum rapidly forms a thin insulating layer of aluminum oxide (AlO$_x$) on its surface. This layer is chemically stable and strongly bound to the underlying metal. Under ambient conditions the native oxide thickness is on the nanometre scale; by controlled oxidation in vacuum one can reproducibly tune the oxide thickness and thereby the normal-state tunnel resistance over several orders of magnitude. The resulting Al--AlO$_x$--Al junctions exhibit large barrier heights, low defect densities, and excellent reproducibility, making aluminum oxide the standard tunnel barrier material in superconducting electronics. In the context of this thesis, the native and artificially grown aluminum oxide layers define the tunnel barriers in our SIS junctions and provide mechanical stability for the atomic contacts formed in aluminum break junctions.
        
    %=========================================================
    \subsection{Microwave Drive Model}
    \label{subsec:basics:micro-wave}
    %=========================================================
    
        Many of the transport phenomena discussed in this chapter involve the presence of a time-dependent electromagnetic drive. To avoid repeated introductions of the same setup, we summarize here the general assumptions that apply to all subsequent descriptions. The drive is treated as a classical, externally imposed field. No cavity modes or backaction of the junction onto the field are considered. The entire voltage drop is assumed to occur across the junction, a convenient gauge choice. In consequence, the electromagnetic field enters only through the scalar potential in the tunneling Hamiltonian or via the Josephson phase evolution. The drive amplitude is taken to be small enough not to heat the electrodes, allowing both to remain in local thermal equilibrium with Fermi--Dirac distributions. Finally, the drive frequencies are chosen such that photon energy is way smaller than the pair-breaking energy, ensuring that Cooper pairs remain intact while the phase dynamics are driven coherently. In practice, frequencies in the microwave range up to about 20\,GHz are used.

        The spatially uniform microwave field is then given by 
        \begin{equation}
            V(t) = V_0 + A \cos (2\pi\nu t)\,,
            \label{eq:microwave}
        \end{equation}
        with a static component $V_0$, drive amplitude $A$, and frequency $\nu$. 

        %=========================================================
        \subsubsection*{Electromagnetic Phase}
        %=========================================================
        
        A time-dependent voltage does not only change the electrochemical potential of the electrodes but also imprints a time-dependent phase on any charged excitation that tunnels across the junction. In the gauge used here, the entire drive enters through the scalar potential $V(t)$ and appears as a phase factor multiplying the tunnelling amplitude. The relevant quantity is the charge $q$ carried by the process under consideration. For single-quasiparticle tunnelling one has $q = e$, for Cooper-pair tunnelling and single Andreev reflection processes $q = 2e$, and for higher-order multiple Andreev reflections the effective transferred charge is an integer multiple $q = m e$ with $m>2$.

        In general, the accumulated electromagnetic phase between time $0$ and $t$ is given by
        \begin{equation}
            \phi(t) = \frac{1}{\hbar} \int_0^t q V(t')\, dt'\,,
        \end{equation}
        and enters the tunnelling matrix element as a phase factor.

        Inserting the harmonic drive from Eq.~\eqref{eq:microwave}, and performing the time integration separates the phase into a static and an oscillatory contribution,
        \begin{align}
            \exp\!\left(-\ima\phi(t)\right)
            &= \exp\!\left(-\ima \phi_0(t)\right)\,
               \exp\!\left(\ima \alpha \sin(2\pi\nu t) \right)\,,
            \label{eq:microwave:phase-factor}
        \end{align}
        where
        \begin{equation}
            \phi_0(t) = \frac{q V_0 t}{\hbar}
            \qquad\text{and}\qquad
            \alpha = \frac{q A}{h\nu}
        \end{equation}
        denote the phase accumulated from the static bias and the dimensionless modulation strength, respectively. Both depend linearly on the transferred charge $q$ and therefore distinguish quasiparticle, Cooper-pair, and higher-order Andreev processes within the same formalism.

        The oscillatory phase factor is conveniently expanded using the Jacobi--Anger identity,
        \begin{equation}
            \exp\!\left(\ima\alpha \sin(2\pi\nu t)\right)
            = \sum_{n=-\infty}^{\infty} J_n(\alpha)\,
              \exp\!\left(\ima n\, 2\pi\nu t\right)\,,
            \label{eq:microwave:jacobi-anger}
        \end{equation}
        where $J_n(\alpha)$ is the $n$-th Bessel function of the first kind. This harmonic decomposition will be used repeatedly in this chapter to describe photon-assisted tunnelling processes for different effective charges $q = m e$.
