% !TEX root = ../thesis.tex

%=========================================================
% \setcounter{section}{-1}
\section{Basic Concepts}
\label{sec:basics}
%=========================================================
    
    In this section we collect the basic microscopic and mesoscopic concepts that underlie all transport phenomena discussed in this thesis. We start from a many-electron Hamiltonian in a crystalline solid and show how its low-energy excitations can be described in terms of quasiparticles with effective mass, lifetime, and mean free path, leading to the Drude picture of diffusive transport. On this basis we introduce the mesoscopic transport regimes, where the finite size of the conductor and phase coherence across it become essential and transport is most naturally formulated in terms of transmission channels and their eigenvalues. We then specialize to aluminum, which serves as the central material platform of this work, and summarize the normal-state, superconducting, and oxide properties that make it ideally suited for atomic-scale contacts and tunnel junctions. Finally, we outline the generic microwave drive model used throughout the thesis and the associated electromagnetic phase factors that give rise to photon-assisted tunneling and driven Josephson dynamics.

    In this chapter we deliberately focus on a minimal set of effective parameters and working concepts; more detailed microscopic derivations and material-specific refinements are deferred to the specialized chapters where they become relevant. \cite{ashcroft_solid_2012, gross_festkorperphysik_2014}

    %=========================================================
    \subsection{Microscopic Quasiparticles}
    \label{subsec:basics:micro}
    %=========================================================

        We consider a many-electron system in a crystalline solid and show how its low-energy excitations can be described in terms of quasiparticles. The full Hamiltonian (first quantization) for $N$ electrons and fixed ionic cores at positions $\{\vec{r}_i\}$ is
        \begin{equation}
            \hat{H}
            = \sum_{i=1}^N \left(
                \frac{\hat{\vec{p}}^{\,2}_i}{2m}
                + V_\text{ion}(\vec{r}_i)
            \right)
            + \frac{1}{2} \sum_{i \neq j}
            \frac{e^2}{4\pi\varepsilon_0 \lvert \vec{r}_i - \vec{r}_j \rvert}\,,
            \label{eq:micro:full-hamiltonian}
        \end{equation}
        with canonical momenta $\hat{\vec{p}}_i = -\ima\hbar\nabla_i$ and electron mass $m$. The ionic potential $V_\text{ion}(\vec{r})$ represents the Coulomb attraction of the electrons to the positively charged lattice. To a good approximation it is periodic,
        \begin{equation}
            V_\text{ion}(\vec{r} + \vec{R})
            = V_\text{ion}(\vec{r})
        \end{equation}
        for all Bravais lattice vectors $\vec{R}$.

        The many-electron wave function $\Psi(\vec{r}_1,\dots,\vec{r}_N,t)$ obeys the time-dependent Schrödinger equation
        \begin{equation}
            \ima\hbar \frac{\partial}{\partial t} \Psi
            = \hat{H}\,\Psi\,,
            \label{eq:micro:schroedinger}
        \end{equation}
        supplemented by antisymmetry under particle exchange due to Fermi statistics. Solving this problem exactly is not feasible. To arrive at a tractable description of electronic transport we now introduce a sequence of simplifications. \cite{ashcroft_solid_2012, gross_festkorperphysik_2014}

        %=========================================================
        \subsubsection*{Effective One-Electron Description}
        %=========================================================

            First, according to the Born--Oppenheimer approximation, the fast electronic motion is separated from the slow ionic motion, and the ions are treated as fixed classical scatterers that generate the periodic potential $V_\text{ion}(\vec{r})$ plus some weak disorder. Next, electron--electron interactions are taken into account in an average way. Screening in a metal strongly reduces the long-range Coulomb repulsion, and the dominant effect of interactions can be encoded in an effective potential and a renormalized band structure. 
            On this level, the complicated many-electron problem is reduced to an effective one-electron eigenvalue problem
            \begin{equation}
                \hat{H}_\text{eff}\,\psi(\vec{r})
                = E\,\psi(\vec{r})\,,
                \qquad
                \hat{H}_\text{eff}
                = \frac{\hat{\vec{p}}^{\,2}}{2m} + V_\text{per}(\vec{r})\,,
                \label{eq:micro:heff}
            \end{equation}
            where $V_\text{per}(\vec{r})$ is a periodic effective potential that already includes the mean-field action of the other electrons. 
            Residual interactions and disorder manifest at low energies through renormalized band parameters and relaxation processes that we summarize below by a small set of effective quasiparticle parameters.

            The stationary eigenvalue problem with a periodic potential,
            \begin{equation}
                \left[
                    -\frac{\hbar^2}{2m}\nabla^2 + V_\text{per}(\vec{r})
                \right] \psi(\vec{r})
                = E\,\psi(\vec{r})\,,
                \label{eq:micro:schroedinger-periodic}
            \end{equation}
            is the starting point for band theory. \cite{born_zur_1927, ashcroft_solid_2012,bloch_uber_1929}

        %=========================================================
        \subsubsection*{Bloch States and Band Structure}
        %=========================================================

            Because $V_\text{per}(\vec{r})$ has the periodicity of the lattice, Bloch's theorem applies. The eigenstates of $\hat{H}_\text{eff}$ can be chosen as Bloch functions
            \begin{equation}
                \psi_{n\vec{k}}(\vec{r})
                = \mathrm{e}^{\ima\vec{k}\cdot\vec{r}}\,
                u_{n\vec{k}}(\vec{r})\,,
                \label{eq:micro:bloch}
            \end{equation}
            where $n$ is a band index, $\vec{k}$ lies in the first Brillouin zone, and $u_{n\vec{k}}(\vec{r}+\vec{R}) = u_{n\vec{k}}(\vec{r})$ is lattice-periodic. Inserting Eq.~\eqref{eq:micro:bloch} into Eq.~\eqref{eq:micro:schroedinger-periodic} yields for each $\vec{k}$ a discrete set of eigenvalues $E_n(\vec{k})$ that define the band structure.

            The Bloch functions form an orthonormal basis. In the independent-electron approximation, the many-electron ground state is obtained by filling the one-particle eigenstates up to the Fermi energy $E_\mathrm{F}$, with at most two electrons (spin up and spin down) per state. For a given electron density, the filling of the bands fixes $E_\mathrm{F}$ and the corresponding Fermi surface defined by
            \begin{equation}
                E_n(\vec{k}) = E_\mathrm{F}\,.
            \end{equation}

            In simple metals one or a few bands cross the Fermi level. For transport at low temperature and low bias, only states in a narrow energy window $\lvert E - E_\mathrm{F} \rvert \sim \max(k_\mathrm{B}T, eV)$ are relevant. In this regime, the detailed band structure away from the Fermi surface is irrelevant and can be compressed into a small set of effective parameters such as the Fermi velocity and the density of states at the Fermi energy. \cite{ashcroft_solid_2012,bloch_uber_1929}

        %=========================================================
        \subsubsection*{Semiclassical Dynamics and Scattering}
        %=========================================================

            To connect this microscopic picture to transport, we consider how Bloch electrons move in weak external fields. On length and time scales large compared to the lattice spacing and inverse band gaps, the dynamics of an electron wave packet built from states in a single band $n$ can be described semiclassically by
            \begin{equation}
                \hbar \frac{\mathrm{d}\vec{k}}{\mathrm{d}t} = -e\left(
                    \vec{E} + \frac{\mathrm{d}\vec{r}}{\mathrm{d}t} \times \vec{B}
                \right)\,,
                \qquad
                \frac{\mathrm{d}\vec{r}}{\mathrm{d}t} = \frac{1}{\hbar}\,\nabla_{\vec{k}} E_n(\vec{k})\,,
                \label{eq:micro:semiclassical}
            \end{equation}
            where $\mathrm{d}\vec{r}/\mathrm{d}t$ is the group velocity, given by the gradient of the band energy, and $\vec{E}$, $\vec{B}$ are external electric and magnetic fields. \cite{ashcroft_solid_2012}

            In a perfect, infinite crystal without interactions, Bloch electrons would evolve according to Eq.~\eqref{eq:micro:semiclassical} and, for a uniform electric field, undergo Bloch oscillations rather than exhibit steady Ohmic transport. Real materials contain impurities, lattice defects, and phonons, and residual electron-electron interactions. These processes scatter carriers between Bloch states and thereby relax the current. \cite{bloch_uber_1929,zener_theory_1934}

            For the transport regimes relevant in this thesis it is sufficient to summarize all momentum-relaxing processes by a single (energy-averaged) momentum-relaxation time $t_\mathrm{qp}$. The semiclassical picture then assumes that between scattering events electrons propagate as Bloch wave packets, while scattering events occur stochastically and randomize the momentum distribution. In the next subsubsection we recast this description in terms of quasiparticles and introduce the small set of effective parameters that will be used repeatedly. \cite{ashcroft_solid_2012, gross_festkorperphysik_2014}

        %=========================================================
        \subsubsection*{Quasiparticles and Effective Single-Band Description}
        %=========================================================

            The independent-electron band picture treated interactions only at a mean-field level. In reality, the charged fermions in the Hamiltonian, which we will refer to as \emph{electrons}, interact with each other and with lattice vibrations. The low-energy excitations of a metal are therefore not bare electrons but \emph{quasiparticles}. In Landau's Fermi-liquid picture these quasiparticles are long-lived fermionic modes that carry charge $-e$ and spin $1/2$, but are dressed by their interaction with the surrounding Fermi sea. \cite{landau_theory_1956}

            Throughout this thesis we characterize normal-state quasiparticles in a narrow energy window around $E_\mathrm{F}$ by a small set of effective parameters: an effective mass $m^\ast$, a Fermi velocity $v_\mathrm{F}$, a (constant) normal-state density of states $N_0 = N(E_\mathrm{F})$, and a momentum-relaxation time $t_\mathrm{qp}$. The corresponding elastic mean free path is $\ell_\mathrm{qp}= v_\mathrm{F} t_\mathrm{qp}$. Here, $t_\mathrm{qp}$ is a transport (momentum-relaxation) time and need not coincide with the single-particle lifetime that governs spectral broadening.

            In an isotropic single-band approximation we approximate the dispersion by a parabolic form,
            \begin{equation}
                E(\vec{k}) \approx \frac{\hbar^2 \lvert \vec{k} \rvert^2}{2m^\ast}\,,
                \qquad
                E_\mathrm{F} = \frac{\hbar^2 k_\mathrm{F}^2}{2m^\ast}\,.
                \label{eq:micro:parabolic}
            \end{equation}
            The effective mass generally differs from the bare electron mass $m$ and encodes both the band-structure curvature and renormalization due to interactions. The quasiparticle group velocity is then
            \begin{equation}
                \vec{v}(\vec{k})
                = \frac{1}{\hbar} \nabla_{\vec{k}} E(\vec{k})
                \approx \frac{\hbar\vec{k}}{m^\ast}\,,
            \end{equation}
            and the Fermi velocity is $v_\mathrm{F} = \hbar k_\mathrm{F} / m^\ast$.

            The density of states per unit volume for a three-dimensional parabolic band (with $E$ measured from the band edge) reads
            \begin{equation}
                N(E) = \frac{1}{2\pi^2} \left( \frac{2m^\ast}{\hbar^2} \right)^{3/2} \sqrt{E}\,,
                \label{eq:basics:n-dos}
            \end{equation}
            and evaluated at the Fermi energy yields $N_0 = N(E_\mathrm{F})$. Since $E_\mathrm{F}$ is of order eV whereas the relevant excitation energies are on the meV scale, $N(E)$ varies only weakly in the window of interest. It is therefore convenient to approximate it by a constant
            \begin{equation}
                N(E) \approx N_0 = N(E_\mathrm{F})\,,
                \label{eq:basics:n-dos-approx}
            \end{equation}
            This approximation is accurate in the energy range relevant for low-temperature transport. Together with $m^\ast$ and $v_\mathrm{F}$, the momentum-relaxation time $t_\mathrm{qp}$ sets the characteristic transport scales (mean free path $\ell_\mathrm{qp}$ and, in the diffusive limit, the diffusion constant), and provides the link between the microscopic quasiparticle picture and the macroscopic Drude description derived below. \cite{ashcroft_solid_2012, gross_festkorperphysik_2014}

        %=========================================================
        \subsubsection*{Drude Model from the Microscopic Picture}
        %=========================================================

            Within the semiclassical picture the simplest transport model is the Drude description of a homogeneous metal. We treat the quasiparticles introduced above as classical point-like carriers with charge $-e$, effective mass $m^\ast$, and density $n$, which experience random momentum-relaxing collisions characterized by the relaxation time $t_\mathrm{qp}$ introduced above. In a weak, static electric field $\vec{E}$ the equation of motion for the average quasiparticle momentum reads
            \begin{equation}
                \frac{\mathrm{d}\langle \vec{p}\, \rangle}{\mathrm{d}t}
                = -e \vec{E} - \frac{\langle \vec{p}\, \rangle}{t_\mathrm{qp}}\,,
            \end{equation}
            where the second term is a simple relaxation-time approximation for the effect of scattering. In the steady state, $\mathrm{d}\langle \vec{p} \,\rangle / \mathrm{d}t = 0$ and $\langle \vec{p} \rangle = - e t_\mathrm{qp} \vec{E}$. The average drift velocity is therefore
            \begin{equation}
                \vec{v}_\mathrm{drift}
                = \frac{\langle \vec{p}\, \rangle}{m^\ast}
                = - \frac{e t_\mathrm{qp}}{m^\ast} \vec{E}\,.
            \end{equation}
            Multiplying by the carrier density $n$ gives the current density
            \begin{equation}
                \vec{j} = - n e \vec{v}_\mathrm{drift}
                = \frac{n e^2 t_\mathrm{qp}}{m^\ast} \vec{E}\,,
            \end{equation}
            which identifies the Drude conductivity
            \begin{equation}
                \sigma = \frac{n e^{2} t_\mathrm{qp}}{m^\ast}\,.
            \end{equation}
            In this bulk limit the system is assumed to be spatially homogeneous, and spatial variations of the electric field and current density are slow on the scale of the microscopic scattering lengths, such that transport is fully captured by the local constitutive relation $\vec{j} = \sigma \vec{E}$.

            Integrating the local constitutive relation $\vec{j}=\sigma\vec{E}$ for a homogeneous wire of length $L$ and cross-sectional area $A$ yields the familiar macroscopic Ohm's law. In the conductance form that will be used throughout this thesis, one writes
            \begin{equation}
                I = G \ V\,,\qquad
                G = \sigma\,\frac{A}{L}\,.
                \label{eq:basics:ohms-law}
            \end{equation}
            Equivalently, $V=IR$ with $R = 1/G$ and one may introduce the resistivity $\rho= 1/\sigma$ such that $R=\rho L/A$. While this bulk relation provides a convenient reference, it breaks down once the conductor becomes phase coherent and comparable in size to microscopic length scales, where transport must be described in terms of transmission channels as discussed below. \cite{ohm_galvanische_1827}

            Ohm's law provides a macroscopic relation between the measurable quantities $I$ and $V$. The Drude picture goes one step further by linking this response to microscopic scattering. The same relaxation time $t_\mathrm{qp}$ that enters $\sigma$ also sets the characteristic length and time scales of quasiparticle motion.

            Momentum randomization over a time $t_\mathrm{qp}$ defines an elastic mean free path
            \begin{equation}
                \ell_\mathrm{qp}= v_\mathrm{F} t_\mathrm{qp}\,,
            \end{equation}
            where $v_\mathrm{F}$ is the Fermi velocity, and on longer scales the coarse-grained dynamics of the carrier density become diffusive with diffusion constant
            \begin{equation}
                D = \frac{v_\mathrm{F}^{2} t_\mathrm{qp}}{3} = \frac{v_\mathrm{F}\ell_\mathrm{qp}}{3}\,.
            \end{equation}
            These quantities $\sigma$, $\ell_\mathrm{qp}$, and $D$ summarize the influence of microscopic scattering on transport in the bulk, diffusive limit and will serve as the reference point for the mesoscopic regimes discussed in the following subsection. \cite{drude_zur_1900}
                
    %=========================================================
    \subsection{Mesoscopic Quasiparticles}
    \label{subsec:basics:meso}
    %=========================================================

        A key new ingredient of mesoscopic transport is the phase-coherence length $\ell_\phi$. It is set by inelastic and dephasing processes such as electron--electron and electron--phonon scattering and quantifies how far a quasiparticle can propagate while retaining a well-defined quantum phase. At finite temperature, quantum interference is additionally limited by thermal averaging, which is commonly characterized by the thermal length $\ell_T$ (defined below).

        Mesoscopic transport refers to the regime in which the conductor size $L$ becomes comparable to one or more microscopic length scales. The relevant hierarchy is set by the elastic mean free path $\ell_\mathrm{qp}$, which distinguishes ballistic from diffusive motion, and the phase-coherence length $\ell_\phi$, which distinguishes coherent from incoherent transport.

        For macroscopic conductors, $L \gg \ell_\mathrm{qp}$ and $L \gg \ell_\phi$, quasiparticles undergo many scattering events and lose phase coherence long before traversing the sample. Transport is then well described by the bulk Drude picture. In contrast, when $L \lesssim \ell_\phi$ phase coherence extends across the conductor and quantum interference between different scattering paths becomes observable. Whether this coherent regime is ballistic or diffusive is then decided by $\ell_\mathrm{qp}$. For $L \ll \ell_\mathrm{qp}$ transport is ballistic and naturally described in terms of transmission channels. Whereas for $\ell_\mathrm{qp} \ll L \lesssim \ell_\phi$ motion is diffusive but phase coherent and exhibits mesoscopic corrections to the Drude conductance. Figure~\ref{fig:basics:meso} illustrates the relevant transport regimes. \cite{ashcroft_solid_2012, gross_festkorperphysik_2014}

        \begin{figure}
            \centering
            \subfigure[
                Classic Regime
                ]{\import{theory/schema}{classic.pdf_tex}}
            \hfill
            \subfigure[
                Diffusive Regime
                ]{\import{theory/schema}{diffusive.pdf_tex}}
            \hfill
            \subfigure[
                Ballistic Regime
                ]{\import{theory/schema}{ballistic.pdf_tex}}
            \caption{
            Schematic transport regimes set by the conductor length $L$, the elastic mean free path $\ell_{\mathrm{qp}}$, and the phase-coherence length $\ell_{\phi}$: (a) classical ($L \gg \ell_{\phi},\, \ell_{\mathrm{qp}}$), (b) phase-coherent diffusive ($\ell_{\mathrm{qp}} \ll L \lesssim \ell_{\phi}$), and (c) ballistic ($L \ll \ell_{\mathrm{qp}}, L \lesssim \ell_{\phi}$). The constriction (\legend{seegrau120}) defines the scattering region of length $L$. Grey dots (\legend{seegrau100}) indicate scattering centers, whose density sets $\ell_{\mathrm{qp}}$. A representative quasiparticle trajectory is shown, phase-coherent segments in blue (\legend{seeblau65}) and phase-randomized segments in grey (\legend{seegrau65}). The phase-coherence length is visualized by a disk of radius $\ell_{\phi}$ (\legend{seeblau20}).
            }
            \label{fig:basics:meso}
        \end{figure}

        %=========================================================
        \subsubsection*{Diffusive Mesoscopic Transport}
        %=========================================================

            In the diffusive mesoscopic regime the elastic mean free path remains much shorter than the system size, $\ell_\mathrm{qp} \ll L$, so that quasiparticles undergo many momentum-randomizing scattering events while traversing the conductor. At the same time, phase coherence can persist over many such events, $L \lesssim \ell_\phi$, such that quantum interference between multiple scattering paths becomes observable.

            For diffusive motion the phase-coherence length is conveniently expressed in terms of the diffusion constant $D$ and a dephasing time $t_\phi$,
            \begin{equation}
                \ell_\phi = \sqrt{D t_\phi}\,.
            \end{equation}
            At finite temperature, interference is additionally reduced by thermal averaging over an energy window $\sim k_\mathrm{B}T$. This introduces the thermal length
            \begin{equation}
                \ell_T = \sqrt{\hbar D / k_\mathrm{B}T}\,,
            \end{equation}
            such that phase-coherent effects are most pronounced when $L \lesssim \min(\ell_\phi,\,\ell_T)$.

            On average, transport still follows the classical drift-diffusion picture with conductance and diffusion constant as introduced in the Drude model above. Quantum mechanically, however, the coherent superposition of scattering amplitudes produces characteristic corrections to the Drude conductance, most prominently weak localization and universal conductance fluctuations. In low-dimensional or strongly disordered systems these interference effects can become strong and may ultimately lead to Anderson localization.
            \cite{bergmann_weak_1984, lee_universal_1985, lee_disordered_1985, anderson_absence_1958, schertel_magnetic-field_2019}

            In the present work we only use the diffusive mesoscopic regime as a reference point for the ballistic limit realized in atomic-scale contacts. 

        %=========================================================
        \subsubsection*{Ballistic Transport and Scattering Fromalism}
        %=========================================================
            
            \begin{wrapfigure}[15]{r}{0.4\textwidth}
                \captionsetup{format=plain}%
                \centering
                \vspace{-1em} % fine-tune vertical position
                \import{theory/schema}{ballistic.pdf_tex}
                \\\textbf{This is a placeholder!}
                \caption{
                    Sketch of a two-terminal scatterer connecting ideal leads. Incoming and outgoing mode amplitudes are related by the scattering matrix $S$.
                }
                \label{fig:basic:scatter}
            \end{wrapfigure}
            In the ballistic regime the length of the constriction is shorter than the elastic mean free path, $L \ll \ell_\mathrm{qp}$, such that quasiparticles traverse the contact without momentum-randomizing scattering. If, in addition, the contact is shorter than the phase-coherence length, $L \lesssim \ell_\phi$, transport is phase coherent across the entire constriction. In the ballistic limit one may express the phase-coherence length in terms of the Fermi velocity and a dephasing time $t_\phi$,
            \begin{equation}
                \ell_\phi = v_\mathrm{F} t_\phi\,.
            \end{equation}

            Ballistic transport is most naturally formulated as an elastic scattering problem. A finite region connects ideal leads in which the electronic modes propagate without dissipation. We denote the complex incoming mode amplitudes in the left and right lead by $A_1$ and $A_2$, and the corresponding outgoing amplitudes by $B_1$ and $B_2$, as shown in Figure~\ref{fig:basic:scatter}. 
            
            Linear scattering then implies a matrix relation which we write explicitly as
            \begin{equation}
                \begin{pmatrix} B_1 \\ B_2 \end{pmatrix} = \mat{S}\,\begin{pmatrix} A_1 \\ A_2 \end{pmatrix}\,.
                \label{eq:basic:scatter-1}
            \end{equation}

            In the general multi-mode case, $A_{1,2}$ and $B_{1,2}$ become vectors collecting all propagating modes in the corresponding lead, and the scattering matrix decomposes into reflection and transmission subblocks. For purely elastic scattering in the normal state, the scattering matrix is unitary, reflecting current conservation. In many situations of interest, additional symmetries reduce the structure of the scattering matrix. For example, for a symmetric two-terminal scatterer one may parameterize it in terms of reflection and transmission blocks $\mat{r}$ and $\mat{t}$, 
            \begin{equation}
                \mat{S}
                = \begin{pmatrix} - \ima \mat{r} & \mat{t} \\ \mat{t} & - \ima \mat{r} \end{pmatrix}\,.
                \label{eq:basic:scatter-2}
            \end{equation}
            
            The central quantities governing transport are then the transmission eigenvalues
            \begin{equation}
                \{\tau_i\} = \operatorname{eig}\!\left(\mat{t}^\dagger\,\mat{t}\right)\,,\quad 
                \tau_i\in[0,1]\,.
                \label{eq:basic:tau_i}
            \end{equation}
            Choosing a basis of transmission eigenchannels diagonalizes $\mat{t}^\dagger\mat{t}$; this basis is unique up to unitary rotations within degenerate subspaces.

            In linear response, the normal-state conductance is then given by the Landauer formula,
            \begin{equation}
                G_\mathrm{N} = G_0 \sum_{i=1}^N \tau_i\,,
                \quad
                G_0 = \frac{2e^2}{h}\,,
                \label{eq:basics:landauer}
            \end{equation}
            where $\tau_i$ are the transmission eigenvalues and $G_0$ is the conductance quantum\footnote{
                Throughout this thesis the conductance quantum is defined as $G_0 \equiv 2e^2/h = 77.48\,\mu\mathrm{S}$. This value includes spin degeneracy and corresponds to the conductance of a fully transmitting normal-state channel. The corresponding resistance quantum is $R_0 = h/(2e^2) = 12.9\,\mathrm{k\Omega}$.
            }.

            The full information about normal-state transport through the contact is thus encoded in the set of transmission eigenvalues $\{\tau_i\}$, often referred to as the mesoscopic pin code of the contact. \cite{landauer_electrical_1970,buttiker_four-terminal_1986, beenakker_random-matrix_1997}

        %=========================================================
        \subsubsection*{Atomic-Scale Contacts}
        %=========================================================

            As a didactic reference for the channel picture, it is useful to recall the textbook case of a split-gate quantum point contact (QPC) in a high-mobility two-dimensional electron gas. There, the width of a short ballistic constriction is tuned electrostatically, and the conductance increases in approximately quantized steps as additional transverse modes become available. In the Landauer picture each newly populated mode contributes roughly one conductance quantum $G_0$ when its transmission approaches unity. The resulting staircase $G(V_\mathrm{G})$ is shown in Fig.~\ref{fig:basics:atomic-contact}(a). \cite{van_wees_quantized_1988, buttiker_four-terminal_1986, beenakker_random-matrix_1997}

            While semiconductor QPCs provide a particularly clean and tunable example of conductance quantization, the focus of this work lies on metallic atomic contacts. In these contacts the number, symmetry, and transparency of the transmission channels are dictated by atomic-scale structure and chemistry rather than by lithographic geometry.

            \begin{figure}
                \centering
                \subfigure[
                    QPC conductance $G$ vs. gate voltage $V_\mathrm{G}$, showing quantized plateaus.
                    \cite{van_wees_quantized_1988}
                    ]{\import{theory/basics}{GV.pgf}}
                \hfill
                \subfigure[
                    Aluminum atomic contact opening trace: conductance $G$ vs. electrode displacement $\Delta x$.
                    \cite{scheer_conduction_1997}
                    ]{\import{theory/basics}{Gx.pgf}}
                \hfill
                \subfigure[
                    Conductance histogram $P(G)$ for aluminum atomic contacts.
                    \cite{yanson_histograms_1997}
                    ]{\import{theory/basics}{PG.pgf}}

                \caption{
                    Representative conductance quantization in mesoscopic constrictions and atomic contacts.
                (a) Split-gate quantum point contact: $G(V_\mathrm{G})$.
                (b) Aluminum atomic contact: opening trace $G(\Delta x)$.
                (c) Conductance histogram $P(G)$ accumulated over many breaking cycles.
                }
                \label{fig:basics:atomic-contact}
            \end{figure}

            Metallic atomic contacts provide a paradigmatic realization of ballistic, phase-coherent transport. The constriction consists of only one or a few atoms, with a characteristic length of order an interatomic spacing. This scale is far shorter than the elastic mean free path and, at low temperature, also much shorter than the phase-coherence length, such that typically $L \ll \ell_\mathrm{qp}$ and $L \ll \ell_\phi$. Experimentally, the formation and rupture of such contacts are tracked by recording the conductance while mechanically separating the electrodes, producing an opening trace as in Fig.~\ref{fig:basics:atomic-contact}(b). \cite{scheer_conduction_1997, agrait_quantum_2025}

            The transmission eigenchannels of an atomic contact are set by the valence orbitals of the atoms forming the narrowest part of the constriction and by their hybridization with the electrodes. For aluminum, whose conduction states have predominantly $sp$ character, this typically results in three dominant transport channels: one mainly $sp_z$-like channel and two transverse $p$-like channels. Depending on atomic geometry and disorder, additional weaker channels may appear, but the transport is usually governed by a small number of highly transparent modes. A complementary statistical view is provided by conductance histograms, Fig.~\ref{fig:basics:atomic-contact}(c), where peaks reflect preferred atomic configurations and their associated typical channel sets. \cite{yanson_histograms_1997,scheer_signature_1998, agrait_quantum_2025}

            Experimentally, the set of transmission probabilities $\{\tau_i\}$ can be extracted from superconducting transport measurements using multiple Andreev reflection and related techniques. Throughout this thesis we therefore characterize each atomic contact by its transmission eigenvalues, the mesoscopic pin code, which serve as input for the microscopic description of superconducting transport phenomena in the following chapters. \cite{scheer_conduction_1997,scheer_signature_1998}

            Together, the opening traces and their corresponding histograms motivate the channel description used throughout this thesis, while the normal-state conductance $G_\mathrm{N}$ reflects $\sum_i \tau_i$, the superconducting subgap transport discussed in later chapters provides access to the full set $\{\tau_i\}$.


    %=========================================================
    \subsection{Aluminum as Material Platform}
    \label{subsec:basics:aluminum}
    %=========================================================

        Aluminum serves as the central material platform of this thesis. It provides both the normal-state metallic electrodes and the weak-coupling BCS superconductor in all transport experiments discussed in the following.

        Aluminum is a simple $sp$ metal with three valence electrons per atom and an fcc crystal structure. In the effective single-band picture introduced above, its low-energy normal-state properties can be summarized by the same quasiparticle parameters $m^\ast$, $v_\mathrm{F}$, $N_0$, and $t_\mathrm{qp}$. Bulk aluminum is nearly free-electron-like, such that $m^\ast \simeq m$ is often a good approximation and the electronic energy scales are set by a large Fermi energy in the eV range. Correspondingly, the Fermi wavelength is on the sub-nanometer scale and the Fermi velocity is of order $10^6\,\mathrm{m/s}$. In thin films, the elastic mean free path $\ell_\mathrm{qp}$ can reach tens of nanometers in high-purity samples, but its precise value depends sensitively on disorder, thickness, and microstructure. \cite{ashcroft_solid_2012,gross_festkorperphysik_2014}

        For mesoscopic aluminum structures the phase-coherence length $\ell_\phi$ is not a fixed material constant but depends on temperature, disorder, and geometry. At sub-kelvin temperatures it is commonly of order micrometers, which justifies treating transport through atomic-scale contacts as fully phase coherent and ballistic, $L \ll \ell_\mathrm{qp},\,\ell_\phi$. At the atomic scale, the conduction channels in an aluminum contact reflect the symmetry of the valence orbitals and their hybridization with the electrodes, typically resulting in one predominantly $sp_z$-like channel and two transverse $p$-like channels. These channel transparencies provide the central microscopic input for the superconducting transport models used throughout this thesis. \cite{steinberg_quasi-particle_2008}

        $E_F=11.7eV$ \cite{chauvin}

        %=========================================================
        \subsubsection*{Aluminum as Superconductor}
        %=========================================================

            As a superconductor, aluminum is a prototypical weak-coupling BCS material with an isotropic $s$-wave order parameter\footnote{
                Here and in the following, ``$s$-wave'' refers to the isotropic angular dependence of the superconducting order parameter on the Fermi surface, classified by spherical harmonics, and should not be confused with atomic $s$ orbitals. In aluminum the conduction band has predominantly $sp$-hybridized character, but the Cooper-pair wave function is $s$-wave in the sense of an isotropic pairing symmetry.
                },
            and a well-understood phonon-mediated pairing mechanism.

            The zero-temperature gap and critical temperature are, to good approximation,
            \begin{equation}
                \Delta_0 \approx 180\,\mu\mathrm{eV}\,,\qquad
                T_\mathrm{C} \approx 1.2\,\mathrm{K}\,.
                \label{eq:basics:aluminum}
            \end{equation}
            The comparatively small gap places typical microwave frequencies well below the pair-breaking threshold\footnote{
                The pair-breaking frequency of aluminum is given by $\nu_\mathrm{pb}=2\Delta_0/h\approx 87\,\mathrm{GHz}$. For further information see Section \ref{subsec:macro:josephson}.
                },
            enabling controlled photon-assisted processes without degrading superconductivity. \cite{tinkham_introduction_2004, murray_material_2021,steinberg_quasi-particle_2008}

            The BCS coherence length in the clean limit is given by, 
            \begin{equation}
                \xi_0 \simeq \hbar v_\mathrm{F}/\pi\Delta_0\,,
            \end{equation}
            yielding a value of order $1$--$2\,\text{\textmu m}$ for aluminum. In disordered (dirty) thin films, where $\ell_\mathrm{qp} \ll \xi_0$, the effective coherence length is reduced and is well approximated by
            \begin{equation}
                \xi \simeq 0.85\,\sqrt{\xi_0\,\ell_\mathrm{qp}}\,\simeq\,\sqrt{\frac{\hbar D}{2\pi k_\mathrm{B}T_\mathrm{C}}}\,.
            \end{equation}
            For typical evaporated aluminum films with $\ell_\mathrm{qp}$ in the range of a few tens of nanometers this gives $\xi$ on the order of $100$--$300\,\mathrm{nm}$.
            Together with its weak spin--orbit scattering and the availability of reproducible tunnel barriers (discussed below), these properties make aluminum ideally suited for tunneling spectroscopy and mesoscopic superconducting transport. \cite{tinkham_introduction_2004, murray_material_2021,steinberg_quasi-particle_2008}

        %=========================================================
        \subsubsection*{Aluminum Oxide as Insulator}
        %=========================================================

            A key practical advantage of aluminum is its native oxide. Upon exposure to oxygen, aluminum rapidly forms a thin insulating AlO$_x$ layer on its surface that passivates the metal and is strongly bound to the underlying film. Under ambient conditions this native oxide is typically only a few nanometers thick and self-limiting. By controlled oxidation in vacuum one can reproducibly tune the barrier properties and thereby the normal-state tunnel resistance over several orders of magnitude. The resulting Al--AlO$_x$--Al junctions are a standard workhorse of superconducting electronics and, when well fabricated, provide stable tunnel barriers with low subgap leakage.

            In the context of this thesis, native and artificially grown aluminum oxide layers define the tunnel barriers in our SIS junctions and contribute to the mechanical stability of atomic contacts formed in aluminum break junctions. \cite{murray_material_2021,steinberg_quasi-particle_2008,saif_effect_2002}
        
    %=========================================================
    \subsection{Microwave Drive and Electromagnetic Phase}
    \label{subsec:basics:micro-wave}
    %=========================================================

            \begin{figure}[ht]
                \centering
                \subfigure[
                    Harmonic drive voltage $V(t)$ (Eq.~\ref{eq:microwave}).
                ]{\import{theory/basics}{mw-V.pgf}}
                \hfill
                \subfigure[
                    Accumulated phase $\phi(t)$ (Eq.~\ref{eq:basics:mw-phase}).
                ]{\import{theory/basics}{mw-phi.pgf}}
                \hfill
                \subfigure[
                    Gauge phase factor $U(t)$ (Eq.~\ref{eq:microwave:phase-factor}).
                ]{\import{theory/basics}{mw-ReU.pgf}}
                \hfill
                \subfigure[
                    Bessel coefficients $J_n(\alpha)$ vs. frequency $\nu$.
                ]{\import{theory/basics}{mw-Jn.pgf}}
                \caption{Illustration of a harmonic microwave drive and the associated electromagnetic phase factors for an aluminum junction (Sec.~\ref{subsec:basics:aluminum}).
                Parameters: $m=1$, $eV_0 = 0.5\,\Delta_0$, $eA = 0.5\,\Delta_0$, and $\nu=10\,\mathrm{GHz}$.}
                \label{fig:basics:microwave}
            \end{figure}
    
        Many of the transport phenomena discussed in this chapter involve the presence of a time-dependent electromagnetic drive. To avoid repeated introductions of the same setup, we summarize here the general assumptions that apply to all subsequent descriptions. The drive is treated as a classical, externally imposed field. No cavity modes or backaction of the junction onto the field are considered. The entire voltage drop is assumed to occur across the junction, a convenient gauge choice. In consequence, the electromagnetic field enters only through the scalar potential in the tunneling Hamiltonian or via the Josephson phase evolution. The drive amplitude is taken to be small enough not to heat the electrodes, allowing both to remain in local thermal equilibrium with Fermi--Dirac distributions. Finally, the drive frequencies are chosen such that photon energy is way smaller than the pair-breaking energy, ensuring that Cooper pairs remain intact while the phase dynamics are driven coherently. In practice, frequencies in the microwave range up to about 20\,GHz are used. \cite{josephson_possible_1962, shapiro_josephson_1963,tien_multiphoton_1963,werthamer_temperature_1966}

        The spatially uniform microwave field is then given by 
        \begin{equation}
            V(t) = V_0 + A \cos (2\pi\nu t)\,,
            \label{eq:microwave}
        \end{equation}
        with a static component $V_0$, drive amplitude $A$, and frequency $\nu$, as shown in Figure~\ref{fig:basics:microwave}(a). 

        %=========================================================
        \subsubsection*{Electromagnetic Phase}
        %=========================================================
        
            A time-dependent voltage does not only change the electrochemical potential of the electrodes but also imprints a time-dependent phase $\phi(t)$ on any charged excitation that tunnels across the junction. In the gauge used here, the entire drive enters through the scalar potential $V(t)$ and appears as a phase factor $U(t)$ multiplying the tunneling amplitude. The relevant quantity is the charge $q$ carried by the process under consideration. For single-quasiparticle tunneling one has $q = e$, for Cooper-pair tunneling and single Andreev reflection processes $q = 2e$, and for higher-order multiple Andreev reflections the effective transferred charge is an integer multiple $q = m e$ with $m>2$.

            In general, the accumulated electromagnetic phase between time $0$ and $t$ is given by
            \begin{equation}
                \phi(t) = \frac{q}{\hbar} \int_0^t V(t')\, dt'\,,
                \label{eq:basics:mw-phase}
            \end{equation}
            as shown in Figure~\ref{fig:basics:microwave}(b).

            The phase enters the tunnel matrix element as a phase factor 
            \begin{equation}
                U(t)\equiv \exp\!\left(-\ima\phi(t)\right)\,.
                \label{eq:microwave:phase-factor}
            \end{equation}

            Inserting the harmonic drive from Eq.~\eqref{eq:microwave}, and performing the time integration separates the phase into a static and an oscillatory contribution,
            \begin{align}
                U(t) = \exp\!\left(-\ima \phi_0(t)\right)\,
                \exp\!\left(\ima \alpha \sin(2\pi\nu t) \right)
            \end{align}
            where
            \begin{equation}
                \phi_0(t) = \frac{q V_0 t}{\hbar}
                \qquad\text{and}\qquad
                \alpha = \frac{q A}{h\nu}\,,
            \end{equation}
            denote the phase accumulated from the static bias and the dimensionless modulation strength, respectively. Both depend linearly on the transferred charge $q$ and therefore distinguish quasiparticle, Cooper-pair, and higher-order Andreev processes within the same formalism. The real part of the phase factor is shown in Figure~\ref{fig:basics:microwave}(c).

            The oscillatory phase factor is conveniently expanded using the Jacobi--Anger identity,
            \begin{equation}
                \exp\!\left(\ima\alpha \sin(2\pi\nu t)\right)
                = \sum_{n=-\infty}^{\infty} J_n(\alpha)\,
                \exp\!\left(\ima n\, 2\pi\nu t\right)\,,
                \label{eq:microwave:jacobi-anger}
            \end{equation}
            where $J_n(\alpha)$ is the $n$-th Bessel function of the first kind. Equivalently, the oscillatory phase factor forms a discrete frequency comb at harmonics $n\nu$, with complex amplitudes given by the Bessel coefficients $J_n(\alpha)$, as illustrated in Fig.~\ref{fig:basics:microwave}(d).
            
            This harmonic decomposition will be used repeatedly in this chapter to describe photon-assisted tunneling processes for different effective charges $q = m e$.
