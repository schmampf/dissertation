% !TEX root = ../thesis.tex

%=========================================================
\section{Macroscopic Description}
\label{sec:macro}
%=========================================================

    \begin{wrapfigure}[13]{r}{0.4\textwidth}
        \captionsetup{format=plain}%
        \centering
        \vspace{-1.5em}
        \import{theory/macro}{delta-r.pgf}
        \caption{
            Spatial profile of $\Delta_1(r)$ (\legend{seeblau100}) and $\Delta_2(r)$ (\legend{seegrau80}) across a tunnel junction (\legend{seegrau65}). Their respective magnitude $|\Delta|$ (shaded) varies only weakly accross the barrier. Coherent coupling is governed by the macroscopic phase difference $\phi$.
            }
        \label{fig:macro:delta-r}
    \end{wrapfigure}
    While the previous section described dissipative transport in terms of single-particle tunneling, the complementary low-energy limit is governed by coherent Cooper-pair tunneling. In this regime, the quasiparticle spectrum plays no direct role. Instead, transport is determined solely by the phase of the superconducting gap introduced in Eq.~\ref{eq:micro:complex-delta}. Because the magnitude $|\Delta|$ varies only weakly across a weak link, the relevant dynamical variable is the phase difference
    \begin{equation}
        \phi = \phi_1 - \phi_2\,,
        \label{eq:macro:phase-difference}
    \end{equation}
    which fully determines the supercurrent. The Josephson effect discussed here is treated in the weak-coupling (tunneling) limit, analogous to the quasiparticle tunneling regime introduced in the previous section.
    Figure~\ref{fig:macro:delta-r} shows a schematic illustration of this situation.
 
    A weak link, such as an insulating barrier, a metallic constriction, or a short normal region, allows the pairing potentials of the two superconductors to overlap. Since their amplitudes remain essentially constant on the junction scale, the coupling depends only on the relative phase. This phase difference is the essential quantity governing coherent Cooper-pair tunneling and forms the basis of the Josephson effect.

    The following subsections introduce the two Josephson relations, discuss their physical implications, and develop the framework required to describe microwave-driven junctions and the RCSJ model.

    %=========================================================
    \subsection{Josephson Effect}
    \label{subsec:macro:josephson}
    %=========================================================
        
        \begin{figure}[t]
            \captionsetup{format=plain}%
            \centering
            \subfigure[
                Current-phase relation of a weakly coupled Josephson junction (Eq.~\ref{eq:macro:dc}).
            ]{\import{theory/macro}{josephson-iphi.pgf}}
            \hspace{3mm}
            \subfigure[
                Josephson current over time for a constant voltage $V_0 = \Delta_0/e$ (Eq.~\ref{eq:macro:dc} \& \ref{eq:macro:phi-t}).
            ]{\import{theory/macro}{josephson-it.pgf}}
            \subfigure[
                Combined supercurrent and quasiparticle contribution to the \textit{I--V} characteristic.
            ]{\import{theory/macro}{josephson-iv.pgf}}
            \hspace{3mm}
            \subfigure[
                Temperature dependence of the critical current $I_\mathrm{C}(T)$ (Eq.~\ref{eq:macro:critical-current}).
            ]{\import{theory/macro}{critical-current.pgf}}
            \caption{
                Josephson current over phase (a) and time (b). \textit{I--V} characteristic (c) and temperature dependence of the critical current (d).
            }
            \label{fig:macro:josephson}
        \end{figure}

        When two superconductors are weakly coupled through a thin insulating barrier, constriction, or short metallic link, the amplitudes of their order parameters vary only minimally across the junction. The relevant degree of freedom is therefore the phase, which changes from $\phi_1$ to $\phi_2$ across the weak link. This phase difference $\phi$ governs the supercurrent flowing through the junction, and its maximum magnitude is set by the critical current $I_\mathrm{C}$.

        This perspective emphasizes that the Josephson effect arises from coherent phase coupling between the two superconductors, not from changes in the order-parameter amplitude. The resulting phase-driven transport is captured by two fundamental relations, known as the Josephson equations.

        % DC Josephson
        The DC Josephson effect is described by the current-phase relation (CPR)
        \begin{equation}
            I_\mathrm{J} = I_\mathrm{C}\sin\phi\,,
            \label{eq:macro:dc}
        \end{equation}
        where $I_\mathrm{C}$ denotes the critical current of the weak link, as shown in Figure~\ref{fig:macro:josephson} (a). This CPR states that a dissipationless supercurrent can flow at zero voltage, driven solely by the phase difference between the two superconductors and is a direct consequence of coherent Cooper-pair tunneling. 

        % AC Josephson
        The AC~Josephson relation links the temporal evolution of the phase to the voltage across the junction,
        \begin{equation}
            \frac{\mathrm{d}\phi}{\mathrm{d}t} = \frac{2e}{\hbar}\,V_0\,,
            \label{eq:macro:ac}
        \end{equation}
        implying that a constant voltage causes the phase difference to increase uniformly in time.

        Consequently, the phase evolves linearly,
        \begin{equation}
            \phi(t) = \phi_0 + 2\pi \nu_0t\,,\quad
            \nu_0 = \frac{2e}{h}\,V_0\,,
            \label{eq:macro:phi-t}
        \end{equation}
        which results in an oscillating supercurrent with frequency $\nu_0$. This is illustrated in Fig.~\ref{fig:macro:josephson} (b), where the time-dependent current $I(t)$ is shown together with the corresponding time scale $\Delta t = 1/\nu_0$ set by the Josephson frequency.
        
        % IV Josephson
        In a real junction, quasiparticle tunneling appears in parallel to the phase-driven supercurrent. The resulting \textit{I--V} characteristic, shown in Fig.~\ref{fig:macro:josephson} (c), features a dissipationless supercurrent branch at zero voltage and a dissipative quasiparticle branch that onsets above the gap. This combined response forms the characteristic transport signature of a weakly coupled Josephson junction.
        
        % Ambegaokar–Baratoff
        The critical current is set microscopically by the Ambegaokar--Baratoff (AB) relation,
        \begin{equation}
            I_\mathrm{C}(T) = \frac{\pi}{2}\,\frac{G_\mathrm{N}\Delta(T)}{e}\,
            \tanh\!\left(\frac{\Delta(T)}{2k_\mathrm{B}T}\right),
            \label{eq:macro:critical-current}
        \end{equation}
        shown in Fig.~\ref{fig:macro:josephson} (d). It reflects the BCS temperature dependence of the superconducting gap (Eq.~\ref{eq:micro:DeltaT}) together with the thermal occupation of quasiparticle states. At zero temperature, the expression simplifies to the well known result $I_\mathrm{C} = (\pi/2)\,G_\mathrm{N}\Delta_0/e$.

        % Josephson Coupling Energy
        The strength of phase coupling in a Josephson junction is quantified by the Josephson energy
        \begin{equation}
            E_\mathrm{J}=\frac{\hbar I_\mathrm{C}}{2e}\,,
            \label{eq:macro:josephson-energy}
        \end{equation}
        which sets the energetic stiffness of the phase and thus the robustness of coherent Cooper-pair tunneling. A large $E_\mathrm{J}$ corresponds to a well-defined phase, while a small $E_\mathrm{J}$ makes the junction susceptible to fluctuations.

        Together with the thermal energy $E_\mathrm{T}=k_\mathrm{B}T$ and the charging energy $E_\mathrm{C}=e^2/(2C)$, the Josephson energy sets the scale on which the phase behaves either classically or becomes susceptible to fluctuations. In the regime $E_\mathrm{J}\gg\{E_\mathrm{T},E_\mathrm{C}\}$, the phase is stiff and the Josephson relations hold in their ideal form, yielding a stable supercurrent. At finite temperature, thermally activated phase slips lead to a rounding of the sharp switching expected in an ideal \textit{I--V} curve. This behavior is naturally described within the classical RCSJ framework, which incorporates thermal noise through the resistive branch. In the charge-dominated regime $E_\mathrm{C} \gtrsim E_\mathrm{J}$, quantum and fully stochastic descriptions of phase dynamics go beyond the classical RCSJ model and are discussed in Section~\ref{sec:stochastic}.

        Having established the phase dynamics in static conditions, we now turn to the interplay between the intrinsic Josephson oscillation and external microwave fields.


    %=========================================================
    \subsection{Shapiro Steps}
    \label{subsec:macro:shapiro}
    %=========================================================
        \begin{figure}[t]
            \centering
            \import{theory/macro}{shapiro-ideal.pgf}
            \caption{
                Microwave-driven transport in a weakly coupled Josephson junction. Parameters correspond to aluminum (Sec.~\ref{subsec:basics:aluminum}), with $T=0$, $\gamma=0$, and $\nu=10.0\,\mathrm{GHz}$.
                }
            \label{fig:macro:shaprio-iv}
        \end{figure}

        When a Josephson junction is exposed to an external microwave field, the phase dynamics become modulated by the drive introduced in Section~\ref{subsec:basics:micro-wave}. The Josephson frequency $\nu_0$ mixes with the external frequency $\nu$, and the phase takes the form
        \begin{equation}
            \phi(t)=\phi_0 + 2\pi\nu_0 t + \alpha\sin(2\pi\nu t)\,,
            \label{eq:macro:shapiro-phase}
        \end{equation}
        with $\alpha = 2eA/h\nu$ the dimensionless drive strength.

        Inserting this expression into the Josephson current-phase relation (Eq.~\ref{eq:macro:dc}) and expanding the phase modulation using the Jacobi-Anger identity (Eq.~\ref{eq:microwave:jacobi-anger}) yields harmonics at frequencies $\nu_0 - n\nu$. Whenever the resonance condition $\nu_0 = n\nu$ is satisfied, the $n$-th harmonic becomes stationary and contributes a finite dc component to the current. This produces quantized voltage plateaus at
        \begin{equation}
            V_n = \frac{nh\nu}{2e}\,,
            \label{eq:macro:shapiro-step}
        \end{equation}
        known as Shapiro steps.

        The amplitude of each step is governed by the Bessel weight $J_n(\alpha)$, which reflects the strength of phase modulation by the microwave field. In contrast to the Tien--Gordon description of quasiparticle tunneling, the Bessel functions appear here without being squared because the Josephson current depends linearly on the phase modulation rather than on squared transition probabilities. Consequently, the step amplitudes scale as $J_n(\alpha)$ instead of $J_n^2(\alpha)$. The time-averaged \textit{I--V} characteristic can therefore be written in direct analogy to the Tien--Gordon description,
        \begin{equation}
            I(V_0)
                = \sum_{n=-\infty}^{\infty}
                    J_n(\alpha)\,
                    I_{0}\!\left(V_0 - \frac{nh\nu}{2e}\right),
            \label{eq:macro:shapiro-iv}
        \end{equation}
        where $I_0(V_0)$ denotes the static Josephson \textit{I--V} curve in the absence of microwaves.

        In a real junction, quasiparticle tunneling occurs in parallel with the phase-driven supercurrent. Under microwave irradiation, both Shapiro steps and PAT appear simultaneously. This combined response constitutes the characteristic microwave-driven transport signature of a weakly coupled Josephson junction.
        As illustrated in Fig.~\ref{fig:macro:shaprio-iv}, the interplay of phase locking and photon-assisted tunneling produces a rich structure in both the \textit{I--V} and differential conductance characteristics.


    %=========================================================
    \subsection{RCSJ Model}
    \label{subsec:macro:rcsj}
    %=========================================================    

        The ideal Josephson relations describe how the supercurrent depends on the phase and how the phase evolves under a constant voltage. However, real junctions exhibit dissipation, capacitance, thermal fluctuations, and microwave-driven phase dynamics that cannot be    captured by the ideal equations alone.

        A convenient way to incorporate these additional contributions is to represent the junction as an effective circuit. To describe these effects, the Josephson element    must be embedded into an effective circuit model.

        Throughout this section, the junction is assumed to operate in the tunneling limit and in the classical regime $E_\mathrm{J} \gg {E_\mathrm{C}, E_\mathrm{T}}$, such that phase dynamics are well described by the RCSJ equation without quantum corrections.

        %========================================================= 
        % \subsubsection*{Schaltplan}
        %=========================================================

            \begin{wrapfigure}[12]{r}{35mm}
                \captionsetup{format=plain}%
                \centering
                \vspace{-.5em}
                \import{theory/macro}{rcsj-model.pdf_tex}
                \caption{
                    RCSJ model of a real Josephson junction.
                }
                \label{fig:macro:rcsj}
            \end{wrapfigure}
            The resistively and capacitively shunted junction (RCSJ) model provides the minimal dynamical description of a real Josephson junction. It augments the ideal Josephson element by a normal resistance representing quasiparticle tunneling and a capacitance associated with the junction electrodes, as sketched in Fig.~\ref{fig:macro:rcsj}. These additions capture the essential dissipative and inertial mechanisms that govern the phase dynamics.

            The damping regime encoded in the quality factor $Q$ strongly affects the visibility and stability of microwave-driven features such as Shapiro steps. Overdamped junctions exhibit clean, well-defined plateaus, whereas underdamped junctions show hysteresis and residual phase oscillations that distort the step structure.

            In the following, we derive the RCSJ equation of motion for the phase and introduce the dynamical regimes that arise from the interplay of Josephson nonlinearity, dissipation, and inertia.

        %========================================================= 
        \subsubsection*{Current Bias}
        %=========================================================

            To obtain the phase dynamics of the junction, the currents through the three parallel elements are summed according to Kirchhoff's law,
            \begin{equation}
                I_\mathrm{bias} = I_\mathrm{C}\sin(\phi)
                    + \frac{V}{R}
                    + C\frac{\mathrm{d}V}{\mathrm{d}t}\,.
                \label{eq:macro:rcsj-ibias}
            \end{equation}
            We use $I_\mathrm{bias}$ to denote the externally applied current, distinguishing it from the Josephson supercurrent $I_\mathrm{J}$, the resistive current, and the capacitive displacement current that appear in parallel in the RCSJ model.

            Using the AC~Josephson voltage-phase relation, this expression can be written entirely in terms of the phase,
            \begin{equation}
                I_\mathrm{bias} = I_\mathrm{C} \sin(\phi)
                + \frac{\hbar}{2eR}\,\frac{\mathrm{d}\phi}{\mathrm{d}t}
                + \frac{\hbar C}{2e}\,\frac{\mathrm{d}^2\phi}{\mathrm{d}t^2}
                \,.
                \label{eq:macro:rcsj-cpr}
            \end{equation}
            This is the RCSJ equation of motion and forms the basis for all classical descriptions of Josephson phase dynamics. It contains an inertial term, a damping term, and the nonlinear Josephson restoring force, producing the characteristic tilted-washboard potential and the associated dynamical regimes discussed below.

        %========================================================= 
        \subsubsection*{Dimensionless RCSJ Model}
        %=========================================================

            To analyze the dynamics implied by Eq.~\ref{eq:macro:rcsj-cpr}, it is useful to cast the equation into a dimensionless form. This separates the contributions of inertia, damping, and the nonlinear Josephson term.

            The natural time scale of the junction is set by the Josephson plasma frequency,
            \begin{equation}
                \omega_\mathrm{p}
                    = \sqrt{\frac{2eI_\mathrm{C}}{\hbar C}}\,,
                \label{eq:macro:rcsj-plasma-frequency}
            \end{equation}
            which determines the small-oscillation frequency of the phase in a potential minimum. It is distinct from the AC Josephson frequency, which reflects voltage-driven phase evolution, and instead characterizes the intrinsic resonance of the phase in the absence of bias.

            Introducing the rescaled time $t'=\omega_\mathrm{p} t$ and the normalized current $i = I_\mathrm{bias}/I_\mathrm{C}$ gives the compact dimensionless form
            \begin{equation}
                i
                = \sin\phi
                + \frac{1}{Q}\,\frac{\mathrm{d}\phi}{\mathrm{d}t'}
                + \frac{\mathrm{d}^2\phi}{\mathrm{d}t'^2}\,,
                \label{eq:macro:rcsj-cpr-norm}
            \end{equation}
            where the quality factor
            \begin{equation}
                Q = \sqrt{\frac{2eI_\mathrm{C}R^2C}{\hbar}}
                \label{eq:macro:rcsj-damping}
            \end{equation}
            quantifies the damping of the phase dynamics.

            In this normalized form, the physical regimes become transparent. The second derivative describes inertial motion, $(1/Q)\mathrm{d}\phi/\mathrm{d}t'$ represents viscous damping due to the shunt resistor, and $\sin\phi$ provides the nonlinear restoring force from the Josephson coupling. The single parameter $Q$ therefore distinguishes underdamped ($Q\gg 1$) from overdamped ($Q\ll 1$) junctions and sets the stage for the dynamical behavior discussed next.

        %========================================================= 
        \subsubsection*{Tilted Washboard Potential}
        %=========================================================

            The normalized RCSJ equation~\ref{eq:macro:rcsj-cpr-norm}, admits a mechanical interpretation that provides an intuitive picture of the phase dynamics. The equation is mathematically equivalent to the motion of a particle of unit mass in a tilted, periodic potential subject to viscous damping. 

            The normalized potential is obtained by identifying the restoring force with
            \begin{equation}
                -\partial u/\partial\phi = \sin\phi - i\,,
                \quad
                u = U/ E_\mathrm{J}\,,
            \end{equation}
            which yields
            \begin{equation}
                u(\phi)
                = -\cos\phi \;-\; i\,\phi\,.
                \label{eq:macro:washboard-potential}
            \end{equation}
            In the untilted case ($i=0$) the barrier height is given by $\Delta U = 2E_\mathrm{J}$. This ''tilted washboard potential'' consists of a cosine landscape whose overall slope is controlled by the normalized bias current $i$, as shown in Figure~\ref{fig:macro:rcsj-u-phi}.
            
            \begin{figure}[t]
                \centering
                \import{theory/macro}{u-phi.pgf}
                \caption{
                    Tilted washboard potential of the RCSJ model (Eq.~\ref{eq:macro:washboard-potential}) for increasing normalized bias current $i$. For $i=0$ (\legend{seeblau100}) the phase is trapped in a stable minimum. At intermediate tilt, $0<i<1$ (\legend{seeblau65}), the reduced barrier enables thermally activated or noise-driven phase slips (\legend{seegrau35}) between adjacent minima. At $i=1$ (\legend{seeblau35}) the barriers vanish and the phase runs downhill, corresponding to the finite-voltage state.
                    }
                \label{fig:macro:rcsj-u-phi}
            \end{figure}

            For small bias, $i<1$, the potential exhibits a series of metastable minima. In this regime, the phase can remain localized in one of these wells, corresponding to the zero-voltage superconducting branch of the \textit{I--V} characteristic. Small oscillations of the phase around the minimum occur at the plasma frequency defined in Eq.~\ref{eq:macro:rcsj-plasma-frequency}, but the average voltage remains zero.

            As the bias is increased, the tilt of the potential grows and the barriers separating adjacent minima are reduced. At $i=1$, the barriers vanish and the phase becomes free to run down the potential without encountering any local minima. This running solution corresponds to the finite-voltage, resistive state of the junction. In the intermediate regime, fluctuations, either thermal or current-induced, can cause the phase to escape from a metastable minimum even for $i<1$, giving rise to switching from the superconducting to the resistive branch.
        
            A finite phase momentum acquired during escape might, once the bias is reduced again, retrap the phase only at a smaller tilt, leading to a retrapping back to the superconducting branch.

            The washboard picture thus provides a unified interpretation of phase localization, escape, and running dynamics. It naturally explains the origin of the switching current, the emergence of retrapping current, and the role of thermal activation and noise, which are explored in the following subsections.

        %========================================================= 
        \subsubsection*{Switching Dynamics and Phase Diffusion}
        %=========================================================

            \begin{wrapfigure}[19]{r}{0.4\textwidth}
                \captionsetup{format=plain}%
                \centering
                \vspace{-1em} % fine-tune vertical position
                \import{theory/macro}{rcsj-iv.pgf}
                \caption{
                    \textit{I--V} characteristic of a Josephson junction in the intermediate damping regime ($Q\sim 1$), illustrating the combined effects of phase diffusion and residual inertia. Thermal and current noise produce a finite slope in the zero-voltage branch, while moderate damping leads to a hysteretic transition into the running state.
                }
                \label{fig:macro:rcsj-iv}
            \end{wrapfigure}
            The switching current $I_\mathrm{SW}$ is the experimentally observed current at which a Josephson junction leaves the zero-voltage state and enters the resistive, running-phase regime. Unlike the intrinsic critical current $I_\mathrm{C}$, which is a microscopic parameter set by the Josephson coupling energy, the switching current is a stochastic quantity. Its value is determined by the competition between the decreasing barrier height of the tilted washboard potential and fluctuations that drive premature escape. Thermal activation, current noise from the electromagnetic environment, and the rate at which the bias current is ramped all increase the likelihood of early escape and thus reduce ths switching current below the critical current. Only in the ideal, noise-free limit does the switching current approach the intrinsic critical current.

            In real Josephson junctions the supercurrent branch acquires a finite slope due to thermal and noise-induced fluctuations of the phase. Even for currents below the intrinsic critical value, stochastic forces drive small excursions of the phase within the tilted washboard potential, leading to a finite average phase velocity and hence a small, non-zero voltage. In the RCSJ model this effect is enhanced by the resistive branch in parallel with the Josephson element, which provides a dissipative path whenever the phase is not perfectly static. As a result, the ideal vertical supercurrent step is replaced by a broadened zero-voltage region whose slope grows with temperature, current noise and environmental damping.

        %========================================================= 
        \subsubsection*{Retrapping Dynamics}
        %=========================================================

            After the phase has escaped into the running state, the junction develops a finite average voltage and the phase accelerates down the tilted washboard potential. When the bias current is reduced again, the junction does not immediately return to the zero-voltage state. Instead, the phase continues to move until sufficient kinetic energy has been dissipated for it to fall back into a metastable minimum. The current at which this return to the superconducting branch occurs is the retrapping current $I_\mathrm{R}$.

            In contrast to the switching current, set by escape from a minimum, the retrapping current reflects how efficiently damping removes the kinetic energy accumulated during the running state. Thermal and current noise smooth this transition, producing a broadened return to the superconducting branch rather than a sharp jump.

            A representative \textit{I--V} trace for this intermediate damping regime is shown in Fig.~\ref{fig:macro:rcsj-iv}. It highlights the finite slope of the supercurrent branch due to phase diffusion, the broadened switching transition, and the smooth return at $I_\mathrm{R}$ characteristic of junctions with $Q\!\sim\!1$.

            In the overdamped limit ($Q\ll 1$), the phase has essentially no inertia. As soon as the bias is lowered below the point where the washboard potential recreates minima, the phase immediately relocks. The retrapping current equals the switching current.
            
            In the underdamped limit ($Q\gg 1$), the phase retains significant kinetic energy after escape and cannot relock until the tilt is nearly removed. The retrapping current therefore approaches zero, giving rise to strong hysteresis in the \textit{I--V} characteristic.

        %========================================================= 
        \subsubsection*{Shapiro Steps in the RCSJ Model}
        %=========================================================

            In the full RCSJ model, Shapiro steps arise through frequency locking between the intrinsic Josephson oscillation and an external microwave drive, as described in Section~\ref{subsec:basics:micro-wave}. The normalized RCSJ equation becomes
            \begin{equation}
                i + a\sin(2\pi\nu' t')
                = \sin\phi
                + \frac{1}{Q}\,\frac{\mathrm{d}\phi}{\mathrm{d}t'}
                + \frac{\mathrm{d}^2\phi}{\mathrm{d}t'^2}\,,
                \label{eq:macro:rcsj-driven}
            \end{equation}
            where $i$ is the normalized bias current. The normalized amplitude $a$ and frequency $\nu$ are given by
            \begin{equation}
                a = A/I_\mathrm{C} \sqrt{R^{-2} + \left(2\pi\nu C\right)^2}\,,\qquad
                \nu' = \nu/\nu_\mathrm{p}\,.
            \end{equation}

            The washboard potential then becomes time dependent
            \begin{equation}
                u(\phi)
                = -\cos\phi \;-\; i\,\phi - a\,\phi\sin(2\pi\nu' t')\,.
                \label{eq:macro:rcsj-washboard-potential}
            \end{equation}
            
            Weak damping makes the junction sensitive to residual oscillations, which can distort Shapiro steps and reduce their visibility under microwave irradiation.
            
            In the overdamped regime, damping suppresses inertial phase oscillations and stabilizes the frequency-locked state, yielding clean, well-defined Shapiro steps. Thermal fluctuations broaden the step edges but do not compromise their visibility. This regime most closely reflects the behavior of the atomic aluminum contact studied in this work.

            Figure~\ref{fig:macro:rcsj-shapiro-iv} shows the resulting \textit{I--V} characteristics in the under- and overdamped limit. The phase locks robustly to the external drive, producing a sequence of Shapiro plateaus.
            \begin{figure}
                \centering
                \subfigure[
                    Overdamped junction ($Q\ll 1$), switching current parameter $I_\mathrm{SW}=0.5\,I_\mathrm{C}$.
                    ]{\import{theory/macro/}{shapiro-under.pgf}}
                \subfigure[
                    Underdamped junction ($Q\gg 1$), switching current parameter $I_\mathrm{SW}=0.2\,I_\mathrm{C}$.
                    ]{\import{theory/macro/}{shapiro-over.pgf}}
                \caption{
                    Microwave-driven dc transport of a Josephson junction within the RCSJ model for (a) overdamped and (b) underdamped phase dynamics. Shown is the normalized bias current $I_\mathrm{bias}/(G_\mathrm{N}\Delta_0/e)$ versus the normalized voltage $eV/\Delta_0$ for drive amplitudes $eA/\Delta_0\in\{0,0.3,0.6\}$ at $\nu=10\,\mathrm{GHz}$. Parameters correspond to aluminum (Sec.~\ref{subsec:basics:aluminum}) with $T=0$ and $\gamma=0$.
                    }
                \label{fig:macro:rcsj-shapiro-iv}
            \end{figure}

            The RCSJ framework bridges the gap between the ideal Josephson prediction of perfectly sharp Shapiro plateaus and the experimentally observed \textit{I--V} characteristics, where damping, noise, and capacitive dynamics shape the visibility and stability of the steps.