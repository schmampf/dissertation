% !TEX root = ../thesis.tex


\section{Macroscopic Description}
\label{sec:macro}

    The Josephson effect arises naturally from the macroscopic phase coherence of the superconducting condensate, a concept formalized in the phenomenological Ginzburg–Landau (GL) theory. In this framework, the superconducting state is described by a complex order parameter,
    \begin{equation}
        \Psi(\vec{r}) = |\Psi(\vec{r})| e^{\ima\phi(\vec{r})}\,,
        \label{eq:macro:GL-wavefunction}
    \end{equation}
    which represents the collective quantum state of the Cooper-pair condensate. The amplitude $|\Psi|^2$ corresponds to the local density of Cooper pairs, while the phase $\phi(\vec{r})$ encodes the long-range coherence of the superconducting state. A spatial variation of this phase gives rise to a flow of supercurrent according to
    \begin{equation}
        \vec{j}_\mathrm{S} = \frac{2e\hbar}{m^*}|\Psi|^2\left(\nabla\phi - \frac{2e}{\hbar}\vec{A}\right)\,,
        \label{eq:macro:GL-current-density}
    \end{equation}
    where $\vec{A}$ is the electromagnetic vector potential, related to the magnetic field by $\vec{B} = \nabla \times \vec{A}$, and $m^* = 2m_\mathrm{e}$ denotes the effective mass of a Cooper pair. The term in parentheses represents the effective phase gradient of the condensate, which determines the supercurrent and incorporates the influence of the electromagnetic field. Although the GL theory is phenomenological, it accurately captures the essential electrodynamics of superconductivity and provides the conceptual foundation for the Josephson effect, where the phase difference between two superconductors directly determines the supercurrent through a weak link.

    The following subsections elaborate on the different manifestations of these macroscopic phase dynamics in superconducting weak links. Section~\ref{subsec:macro:josephson} introduces the fundamental Josephson relations that describe the static and dynamic coupling of superconducting condensates. Section~\ref{subsec:macro:shapiro} extends this framework to include the interaction with oscillating electromagnetic fields, giving rise to quantized voltage plateaus known as Shapiro steps. The incoherent regime, where phase coherence is destroyed by environmental fluctuations, is discussed in Section~\ref{subsec:macro:incoherent}, along with its relation to photon-assisted tunneling through the Tien-Gordon model. Finally, Section~\ref{subsec:macro:rcsj} presents the resistively and capacitively shunted junction (RCSJ) model, which provides a classical circuit description that unifies coherent and dissipative effects in Josephson junction dynamics.

    \subsection{Josephson Effect}
    \label{subsec:macro:josephson}

        When two superconductors are weakly coupled through a thin insulating barrier, constriction, or metallic link, their macroscopic wavefunctions overlap across the junction. The phase difference $\phi = \phi_1 - \phi_2$ between the two condensates then governs the supercurrent flowing through the weak link. The phenomenology of this coupling is captured by two fundamental relations, known as the Josephson equations. 

        The first, describing the DC Josephson effect, is given by
        \begin{equation}
            I_\mathrm{S} = I_\mathrm{C}\sin\phi\,,
            \label{eq:macro:dc}
        \end{equation}
        where $I_\mathrm{C}$ denotes the critical current. It states that a stationary supercurrent, with a maximum value $I_\mathrm{C}$, can flow through the weak link even in the absence of an applied voltage, driven solely by the phase difference between the two superconductors. This purely phase-dependent current is a direct manifestation of macroscopic quantum coherence and forms the basis for most applications of Josephson junctions.

        The second, describing the AC Josephson effect, links the temporal evolution of the phase to the voltage $V_0$ across the junction,
        \begin{equation}
            \frac{\mathrm{d}\phi}{\mathrm{d}t} = \frac{2e}{\hbar}\,V_0\,,
            \label{eq:macro:ac}
        \end{equation}
        implying that a constant voltage induces a continuously evolving phase difference. Consequently, the phase evolves linearly in time,
        \begin{equation}
            \phi(t) = \phi_0 + \frac{2e}{\hbar}\,V_0t\,,
            \label{eq:macro:phi-t}
        \end{equation}
        resulting in an oscillating supercurrent with frequency
        \begin{equation}
            \nu_0 = \frac{2e}{h}\,V_0\,.
            \label{eq:macro:nu}
        \end{equation}
        This relation establishes a direct connection between voltage and frequency. For typical voltages in the microvolt range, the corresponding Josephson frequency lies in the microwave domain (tens to hundreds of gigahertz), which enables direct coupling between the junction dynamics and external microwave fields.

        These relations are obtained within the tunneling limit, where the coupling between the superconductors is weak. In this regime, the amplitude of the order parameter can be regarded as constant across the junction, and only its phase varies. This simplification highlights that the Josephson effect fundamentally arises from the coherent phase dynamics of the condensate, rather than from changes in its magnitude.

        The behavior of a Josephson junction can also be discussed in terms of characteristic energy scales. The Josephson coupling energy, $E_\mathrm{J} = \hbar I_\mathrm{C}/2e$, quantifies the strength of coherent Cooper-pair tunneling and thus the stiffness of the superconducting phase across the junction. Thermal fluctuations are characterized by the thermal energy $E_\mathrm{T} = k_\mathrm{B}T$, which tends to randomize the phase. For $E_\mathrm{J} \gg E_\mathrm{T}$, the Josephson relations hold in their coherent form. For $E_\mathrm{J} \lessapprox E_\mathrm{T}$ the junction becomes dominated by thermal noise, therfore the phase fluctuates, leading to a suppression of the supercurrent. 

        Together, these two relations constitute the fundamental Josephson equations, describing the static and dynamic behavior of phase-coherent weak links. The DC Josephson effect reflects the existence of a stationary supercurrent determined solely by the phase difference, while the AC effect reveals the time evolution of this phase in response to a finite voltage. Their interplay underlies many of the phenomena observed in superconducting junctions, such as Shapiro steps and photon-assisted tunneling, which arise when an additional oscillating field interacts with the Josephson dynamics.

    \subsection{Shapiro Steps}
    \label{subsec:macro:shapiro}

        When a Josephson junction is subjected to an external oscillating voltage or current, the interplay between the intrinsic Josephson oscillation and the external drive leads to quantized voltage plateaus in the $I(V)$ characteristics, known as Shapiro Steps. First observed by S.~Shapiro in 1963, this phenomenon provides a direct and precise verification of the AC Josephson relation.

        Assume an oscillating voltage across the junction    
        \begin{equation}
            V(t) = V_0 + A \cos (2\pi\nu t)\,,
            \label{eq:josepson:shapiro-drive}
        \end{equation}
        similar to Equation~\ref{eq:pat-sinusoidal} and insert in the AC Josephson Equation~\ref{eq:macro:ac}. By Integration this relation gives the time-dependent phase difference
        \begin{equation}
            \phi(t) = \phi_0 + \nu_0t + (2eA/h\nu) \sin(2\pi\nu t)
            \label{eq:josepson:shapiro-phase}
        \end{equation}
        where $\nu_0$ is the Josephson frequency, $\nu$ and $A$ are the frequency and amplitude of the applied voltage. Inserting in the DC Josephson Equation~\ref{eq:macro:dc} and usage of the Jacobi-Anger identity, leads to 
        \begin{equation}
            I_\mathrm{S}(V_0) = \sum_{n=-\infty}^{\infty} (-1)^n J_n\!\left( \frac{2eA}{h\nu}\right) \cdot I_\mathrm{C} \sin(\phi_0 + 2\pi(\nu_0t-n\nu t))\,.
            \label{eq:josepson:shapiro-current}
        \end{equation}
        where the factor $(-1)^n$ follows from the symmetry $J_{-n}(a) = (-1)^nJ_n(a)$. Each term in the sum corresponds to a tunneling process in which $n$ photons are absorbed or emitted. Whenever the resonance condition $\nu_0 = n\nu$ is fulfilled, the corresponding term becomes stationary and yields a DC contribution to the time-averaged current. The resulting voltage plateaus occur at discrete voltages
        \begin{equation}
            V_n = n \, \frac{h\nu}{2e}\,.
            \label{eq:macro:shapiro-steps}
        \end{equation}
        known as Shapiro steps. Their amplitudes are governed by the Bessel functions $J_n(a)$, which determine the strength of coupling between the Josephson oscillation and the external microwave field.

        The time-averaged $I(V)$ relation can be expressed in a compact form analogous to the Tien-Gordon model of photon-assisted tunneling,
        \begin{equation}
            I_\mathrm{S}(V_0) = \sum_{n=-\infty}^{\infty} J_n\!\left( \frac{2eA}{h\nu}\right) \cdot I_0\!\left(V_0 - \frac{n h\nu}{2e}\right)
            \label{eq:josepson:shapiro-iv}
        \end{equation}
        where $I_0(V_0)$ denotes the static $I(V)$ characteristic in the absence of microwave radiation. In difference to the Tien-Gordon Model described in Section~\ref{subsec:bcs:pat}, the Bessel function is not square and the charge $e$ is substituted by $2e$, according to the charge of a Cooper pair.

    \subsection{Incoherent Tunneling of Cooper Pairs}
    \label{subsec:macro:incoherent-cooper-pair-tunnneling}
    
        When the Josephson coupling between two superconductors becomes weak or the phase coherence is strongly disturbed by thermal or environmental fluctuations, the tunneling of Cooper pairs can no longer be treated as a coherent, collective process. Instead, tunneling events occur stochastically and are referred to as \textit{incoherent Cooper-pair tunneling}. In this regime, the phase difference across the junction fluctuates rapidly, and successive tunneling events are uncorrelated in phase and time. The Josephson relations no longer describe a deterministic oscillation but an ensemble-averaged, diffusive process.

        Microscopically, the tunneling still involves pairs of electrons with charge $2e$, but the phase correlation between the superconductors decays due to coupling to an electromagnetic environment or thermal noise. The loss of coherence can be described in terms of the correlation function of the phase,
        \begin{equation}
            \left\langle e^{i[\phi(t)-\phi(0)]} \right\rangle = e^{-J(t)}\,,
        \end{equation}
        where $J(t)$ quantifies the influence of environmental fluctuations. The probability that the environment exchanges an energy $E$ with the tunneling Cooper pair is given by the function $P(E)$, which satisfies
        \begin{equation}
            P(E) = \frac{1}{2\pi\hbar} \int_{-\infty}^{\infty} \mathrm{d}t \, e^{J(t) + iEt/\hbar}\,.
        \end{equation}
        The resulting current can then be expressed as
        \begin{equation}
            I(V) = \frac{\pi e E_\mathrm{J}^2}{\hbar} [P(2eV) - P(-2eV)]\,,
        \end{equation}
        which describes sequential pair tunneling events assisted by energy exchange with the environment. This framework is known as the \textit{P(E)-theory} of incoherent Cooper-pair tunneling.

        The incoherent regime is realized when the Josephson coupling energy is small compared to either the charging or thermal energy,
        \begin{equation}
            E_\mathrm{J} \ll \mathrm{max}(E_\mathrm{C}, E_\mathrm{T})\,,
        \end{equation}
        where $E_\mathrm{C} = e^2/2C$ denotes the charging energy and $E_\mathrm{T} = k_\mathrm{B}T$ the thermal energy. Under these conditions, the phase is no longer a well-defined variable but a fluctuating quantity governed by the impedance $Z(\omega)$ of the environment. The tunneling process thus becomes probabilistic and dissipative, dominated by energy exchange with the surroundings rather than coherent phase evolution.

        Incoherent Cooper-pair tunneling differs fundamentally from the coherent phenomena described by the Shapiro effect. In the coherent case, a well-defined phase relationship allows phase locking between the intrinsic Josephson oscillation and an external microwave field, giving rise to quantized voltage plateaus. In contrast, incoherent tunneling lacks such phase locking, leading instead to smooth, non-quantized $I(V)$ characteristics that reflect the spectral properties of the electromagnetic environment. While the Shapiro effect requires $E_\mathrm{J} \gg E_\mathrm{T}$ and a stable phase, incoherent tunneling dominates for $E_\mathrm{J} \ll E_\mathrm{T}$ or in high-impedance environments where phase diffusion destroys coherence.

        The crossover between coherent and incoherent regimes represents a transition from collective to stochastic dynamics of the superconducting phase. It is particularly relevant in small-capacitance or highly resistive junctions, where environmental fluctuations strongly influence the tunneling process. In such systems, the current–voltage characteristics provide valuable information about the interaction between the junction and its electromagnetic environment.

        The stochastic picture of incoherent Cooper-pair tunneling, in which the environment randomly provides or absorbs discrete quanta of energy, connects naturally to the coherent case of photon-assisted tunneling. When the energy exchange is no longer driven by thermal or environmental fluctuations but by a controlled oscillating potential, the corresponding current-voltage relation can be described by the Tien–Gordon model. In this limit, the exchange of photons with a monochromatic field replaces the probabilistic $P(E)$ spectrum with a set of discrete sidebands weighted by Bessel functions, analogous to the formulation used for Shapiro steps in Section~\ref{subsec:macro:shapiro}. The Tien-Gordon model thus represents the coherent extension of the $P(E)$-framework and provides the quantitative basis for describing photon-assisted tunneling in superconducting junctions.

        The corresponding $I(V)$ relation can be expressed in analogy to the Tien–Gordon model, with the elementary charge $e$ replaced by $2e$ to account for the tunneling of Cooper pairs:
        \begin{equation}
            I(V) = \sum_{n=-\infty}^{\infty} J_n^2\!\left(\frac{2eA}{h\nu}\right) \cdot I_0\!\left(V - \frac{n h\nu}{2e}\right)\,,
            \label{eq:macro:tien-gordon}
        \end{equation}
        where $I_0(V)$ denotes the static $I(V)$ characteristic in the absence of oscillating drive. Each term in the sum represents a tunneling process accompanied by the absorption or emission of $n$ photons of energy $h\nu$, and the corresponding weights $J_n^2$ describe the probability of these photon-assisted transitions.


    \subsection{RCSJ Model}
    \label{subsec:macro:rcsj}

        The resistively and capacitively shunted junction (RCSJ) model provides a classical description of the dynamics of a Josephson junction biased by an external current. It extends the ideal Josephson relations by incorporating both dissipative and inertial effects through a shunt resistance and capacitance. This model is of central relevance because it captures the transition between purely superconducting and resistive behavior and correctly describes experimentally observed phenomena such as hysteresis and Shapiro steps.

        The equivalent circuit of the RCSJ model consists of three parallel branches: the ideal Josephson element carrying the supercurrent $I_\mathrm{S} = I_\mathrm{C}\sin\phi$, a normal resistance $R$ that accounts for quasiparticle tunneling and dissipation, and a capacitance $C$ representing the electrostatic energy stored across the junction barrier. 
        
        The resistor $R$ thereby models a parallel quasiparticle channel that allows dissipative single-particle tunneling in addition to the coherent Cooper-pair supercurrent, ensuring that both coherent and incoherent transport processes are accounted for in the model. 

        The capacitance $C$ can be understood as the geometric capacitance of the junction electrodes separated by the insulating barrier; it stores electrostatic energy and acts as an effective inertia for the phase, smoothing or attenuating fast oscillations of the voltage such that a large $C$ effectively reduces the oscillating component of the voltage seen by the Josephson element.
        
        The total current through the junction is then given by
        \begin{equation}
            I = I_\mathrm{C}\sin\phi + \frac{V}{R} + C\frac{\mathrm{d}V}{\mathrm{d}t}\,,
            \label{eq:macro:rcsj-current}
        \end{equation}
        where $V$ is the instantaneous voltage across the junction. Using the AC Josephson relation,
        \begin{equation}
            \frac{\mathrm{d}\phi}{\mathrm{d}t} = \frac{2e}{\hbar}V\,,
            \label{eq:macro:rcsj-ac}
        \end{equation}
        this equation can be rewritten as a second-order differential equation for the phase,
        \begin{equation}
            \frac{\hbar C}{2e}\frac{\mathrm{d}^2\phi}{\mathrm{d}t^2} + \frac{\hbar}{2eR}\frac{\mathrm{d}\phi}{\mathrm{d}t} + I_\mathrm{C}\sin\phi = I\,.
            \label{eq:macro:rcsj-phase}
        \end{equation}

        This expression can be interpreted as the equation of motion of a fictitious particle with an effective mass $m_\mathrm{eff} = \hbar C / 2e$, moving in a tilted washboard potential
        \begin{equation}
            U(\phi) = -E_\mathrm{J}\left(\cos\phi + \frac{I}{I_\mathrm{C}}\phi\right)\,,
            \label{eq:macro:washboard}
        \end{equation}
        where $E_\mathrm{J} = \hbar I_\mathrm{C} / 2e$ is the Josephson coupling energy. In this analogy:
        \begin{itemize}
            \item the particle position corresponds to the phase $\phi$,
            \item its velocity $\dot{\phi}$ corresponds to the voltage across the junction,
            \item the damping term $\hbar/2eR$ corresponds to viscous friction,
            \item and the tilt of the potential is controlled by the bias current $I$.
        \end{itemize}
        For $I < I_\mathrm{C}$, the particle is trapped in one of the potential wells, corresponding to the zero-voltage (superconducting) state. When $I > I_\mathrm{C}$, the potential minima disappear, the phase starts to run, and a finite voltage develops.

        The strength of damping in this system is characterized by the dimensionless Stewart–McCumber parameter
        \begin{equation}
            \beta_\mathrm{C} = \frac{2e I_\mathrm{C} R^2 C}{\hbar}\,.
            \label{eq:macro:stewart-mccumber}
        \end{equation}
        This parameter determines the dynamical regime of the junction:
        \begin{itemize}
            \item For $\beta_\mathrm{C} \ll 1$, the junction is \textbf{overdamped}: the damping term dominates, inertia is negligible, and the phase follows the tilt of the potential smoothly. The resulting $I(V)$ characteristic is non-hysteretic and stable. This regime corresponds to a low-$Q$ system.
            \item For $\beta_\mathrm{C} \gg 1$, the junction is \textbf{underdamped}: inertia dominates over damping, the phase can oscillate within the potential wells, and the $I(V)$ curve becomes hysteretic. This corresponds to a high-$Q$ system with resonant behavior.
            \item For $\beta_\mathrm{C} \approx 1$, the junction is in the \textbf{intermediate (normal)} regime, where both capacitance and resistance influence the dynamics, leading to weak or partial hysteresis.
        \end{itemize}

        In the overdamped limit, Equation~\ref{eq:macro:rcsj-phase} simplifies to
        \begin{equation}
            \frac{\hbar}{2eR}\frac{\mathrm{d}\phi}{\mathrm{d}t} + I_\mathrm{C}\sin\phi = I\,,
            \label{eq:macro:rsj-phase}
        \end{equation}
        which defines the resistively shunted junction (RSJ) model. This limit is particularly relevant for the junctions investigated in this work, where strong damping suppresses oscillatory behavior and ensures smooth, non-hysteretic $I(V)$ curves.

        The damping regime has a direct impact on the visibility and stability of Shapiro steps. In underdamped junctions, residual oscillations of the phase can lead to hysteresis and distort the step structure. In contrast, in overdamped junctions, the strong damping stabilizes the phase-locking process, resulting in clean and well-defined Shapiro steps. This makes overdamped Josephson junctions particularly suitable for quantitative analysis of microwave-driven transport phenomena.

        % The Josephson framework thus provides a macroscopic description of superconducting transport, emphasizing the role of phase coherence and its coupling to external fields and dissipative environments. 

        % While this picture successfully describes weak links dominated by coherent Cooper-pair tunneling, real superconducting contacts can also exhibit strong quasiparticle contributions. 
        
        % To capture such processes, particularly at higher transmission where multiple charge conversions occur within the superconducting gap, we turn to the microscopic mechanism of Andreev reflection and its extension to multiple Andreev reflection (MAR).