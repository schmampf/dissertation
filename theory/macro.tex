% !TEX root = ../thesis.tex

%=========================================================
\section{Macroscopic Description}
\label{sec:macro}
%=========================================================

        \begin{wrapfigure}[20]{r}{0.4\textwidth}
            \captionsetup{format=plain}%
            \centering
            \vspace{-1em} % fine-tune vertical position
            \includegraphics[width=.35\textwidth]{theory/macro/wavefunction.png}
            \caption{\textbf{in Progress... (die figure ist auf alle fälle falsch und macht was mit mir)\\}
            Schematic illustration of two superconductors weakly coupled by a tunnel barrier. The order parameter amplitude remains essentially constant across the junction, while the phase changes from $\phi_1$ to $\phi_2$, giving a well-defined phase difference $\phi = \phi_1 - \phi_2$ that controls the Josephson supercurrent.
            }
            \label{fig:macro:wave-function}
        \end{wrapfigure}
    While the previous section described dissipative transport in terms of single-particle tunneling, the complementary low-energy limit is governed by coherent Cooper-pair tunneling.  In this regime, the quasiparticle spectrum plays no direct role. Instead, transport is determined solely by the phase of the superconducting gap introduced in Eq.~\ref{eq:micro:complex-delta}. Because the magnitude $|\Delta|$ varies only weakly across a weak link, the relevant dynamical variable is the phase difference
    \begin{equation}
        \phi = \phi_1 - \phi_2\,,
        \label{eq:macro:phase-difference}
    \end{equation}
    which fully determines the supercurrent.
    The Josephson effect discussed here is treated in the weak-coupling (tunneling) limit, analogous to the quasi-particle tunneling regime introduced in the previous section.
    Figure~\ref{fig:macro:wave-function} shows a schematic illustration of this situation.
 
    A weak link, such as an insulating barrier, a metallic constriction, or a short normal region, allows the pairing potentials of the two superconductors to overlap. Since their amplitudes remain essentially constant on the junction scale, the coupling depends only on the relative phase. This phase difference is the essential quantity governing coherent Cooper-pair tunneling and forms the basis of the Josephson effect.

    The following subsections introduce the two Josephson relations, discuss their physical implications, and develop the framework required to describe microwave-driven junctions and the RCSJ model.

    %=========================================================
    \subsection{Josephson Effect}
    \label{subsec:macro:josephson}
    %=========================================================

        When two superconductors are weakly coupled through a thin insulating barrier, constriction, or short metallic link, the amplitudes of their order parameters vary only minimally across the junction. The relevant degree of freedom is therefore the phase, which changes from $\phi_1$ to $\phi_2$ across the weak link. This phase difference $\phi$ governs the supercurrent flowing through the junction, and its maximum magnitude is set by the critical current $I_\mathrm{C}$.

        This perspective emphasizes that the Josephson effect arises from coherent phase coupling between the two superconductors, not from changes in the order-parameter amplitude. The resulting phase-driven transport is captured by two fundamental relations, known as the Josephson equations.

        % DC Josephson
        \begin{wrapfigure}[12]{r}{0.4\textwidth}
            \captionsetup{format=plain}%
            \centering
            \import{theory/macro}{josephson-iphi.pgf}
            \caption{CPR of a weakly coupled Josephson junction (Eq.~\ref{eq:macro:dc}).}
            \label{fig:macro:josephson-iphi}
        \end{wrapfigure}
        The DC Josephson effect is described by the current-phase relation (CPR)
        \begin{equation}
            I_\mathrm{J} = I_\mathrm{C}\sin\phi\,,
            \label{eq:macro:dc}
        \end{equation}
        where $I_\mathrm{C}$ denotes the critical current of the weak link, as shown in Figure~\ref{fig:macro:josephson-iphi}. This CPR states that a dissipationless supercurrent can flow at zero voltage, driven solely by the phase difference between the two superconductors and is a direct consequence of coherent Cooper-pair tunneling. 

        % AC Josephson
        The AC~Josephson relation links the temporal evolution of the phase to the voltage across the junction,
        \begin{equation}
            \frac{\mathrm{d}\phi}{\mathrm{d}t} = \frac{2e}{\hbar}\,V_0\,,
            \label{eq:macro:ac}
        \end{equation}
        implying that a constant voltage causes the phase difference to increase uniformly in time.

        % I(t) josephson
        \begin{wrapfigure}[15]{r}{0.4\textwidth}
            \captionsetup{format=plain}%
            \centering
            \import{theory/macro}{josephson-it.pgf}
            \caption{Time-dependent Josephson current for a constant voltage $V_0$. The period $\Delta t$ of the oscillation is set by the Josephson frequency $\nu_0$.}
            \label{fig:macro:josephson-it}
        \end{wrapfigure}
        Consequently, the phase evolves linearly,
        \begin{equation}
            \phi(t) = \phi_0 + 2\pi \nu_0t\,,\quad
            \nu_0 = \frac{2e}{h}\,V_0\,,
            \label{eq:macro:phi-t}
        \end{equation}
        which results in an oscillating supercurrent with frequency $/nu_0$. This is illustrated in Fig.~\ref{fig:macro:josephson-it}, where the time-dependent current $I(t)$ is shown together with the corresponding time scale $\Delta t = 1/\nu_0$ set by the Josephson frequency.
        
        % IV Josephson
        \begin{wrapfigure}[18]{r}{0.4\textwidth}
            \captionsetup{format=plain}%
            \centering
            \import{theory/macro}{josephson-iv.pgf}
            \caption{Combined supercurrent and quasiparticle contribution to the \textit{I--V} characteristic for a fixed phase $\phi=\pi/9$.}
            \label{fig:macro:josephson-iv}
        \end{wrapfigure}
        In a real junction, quasiparticle tunneling appears in parallel to the phase-driven supercurrent. The resulting \textit{I--V} characteristic, shown in Fig.~\ref{fig:macro:josephson-iv}, features a dissipationless supercurrent branch at zero voltage and a dissipative quasi-particle branch that onsets above the gap. This combined response forms the characteristic transport signature of a weakly coupled Josephson junction.
        
        % Ambegaokar–Baratoff I_\mathrm{C}
        \begin{wrapfigure}[13]{r}{0.4\textwidth}
            \captionsetup{format=plain}%
            \centering
            \import{theory/macro}{I_C-suppression.pgf}
            \caption{Temperature dependence of the critical current $I_\mathrm{C}(T)$ according to Ambegaokar--Baratoff.}
            \label{fig:macro:I_C-suppression}
        \end{wrapfigure}
        For a tunnel junction, the critical current is set microscopically by the Ambegaokar--Baratoff (AB) relation,
        \begin{equation}
            I_\mathrm{C}(T) = \frac{\pi}{2}\,\frac{G_\mathrm{N}\Delta(T)}{e}\,
            \tanh\!\left(\frac{\Delta(T)}{2k_\mathrm{B}T}\right),
            \label{eq:macro:critical-current}
        \end{equation}
        shown in Fig.~\ref{fig:macro:I_C-suppression}. It reflects the BCS temperature dependence of the superconducting gap (Eq.~\ref{eq:micro:DeltaT}) together with the thermal occupation of quasiparticle states. At zero temperature, the expression simplifies to the well known result $I_\mathrm{C} = (\pi/2)\,G_\mathrm{N}\Delta_0/e$.

        % Josephson Coupling Energy
        The strength of phase coupling in a Josephson junction is quantified by the Josephson energy
        \begin{equation}
            E_\mathrm{J}=\frac{\hbar I_\mathrm{C}}{2e}\,,
            \label{eq:macro:josephson-energy}
        \end{equation}
        which sets the energetic stiffness of the phase and thus the robustness of coherent Cooper-pair tunneling. A large $E_\mathrm{J}$ corresponds to a well-defined phase, while a small $E_\mathrm{J}$ makes the junction susceptible to fluctuations.

        Together with the thermal energy $E_\mathrm{T}=k_\mathrm{B}T$ and the charging energy $E_\mathrm{C}=e^2/(2C)$, the Josephson energy sets the scale on which the phase behaves either classically or becomes susceptible to fluctuations. In the regime $E_\mathrm{J}\gg\{E_\mathrm{T},E_\mathrm{C}\}$, the phase is stiff and the Josephson relations hold in their ideal form, yielding a stable supercurrent. At finite temperature, thermally activated phase slips lead to a rounding of the sharp switching expected in an ideal \textit{I--V} curve. This behavior is naturally described within the classical RCSJ framework, which incorporates thermal noise through the resistive branch. However, quantum and fully stochastic descriptions of phase dynamics lie beyond this classical model and do not alter the fundamental Josephson relations introduced above.

        Having established the phase dynamics in static conditions, we now turn to the interplay between the intrinsic Josephson oscillation and external microwave fields.


    %=========================================================
    \subsection{Shapiro Steps}
    \label{subsec:macro:shapiro}
    %=========================================================

        When a Josephson junction is exposed to an external microwave field, the phase dynamics become modulated by the drive introduced in the introduction of this chapter. The Josephson frequency $\nu_0$ mixes with the external frequency $\nu$, and the phase takes the form
        \begin{equation}
            \phi(t)=\phi_0 + 2\pi\nu_0 t + \alpha\sin(2\pi\nu t)\,,
            \label{eq:macro:shapiro-phase}
        \end{equation}
        with $\alpha = 2eA/h\nu$ the dimensionless drive strength.

        Inserting this expression into the Josephson current-phase relation and expanding the phase modulation using the Jacobi-Anger identity (Eq.~\ref{eq:microwave:jacobi-anger}) yields harmonics at frequencies $\nu_0 - n\nu$. Whenever the resonance condition $\nu_0 = n\nu$ is satisfied, the $n$-th harmonic becomes stationary and contributes a finite dc component to the current. This produces quantized voltage plateaus at
        \begin{equation}
            V_n = \frac{nh\nu}{2e}\,,
            \label{eq:macro:shapiro-step}
        \end{equation}
        known as Shapiro steps.

        The amplitude of each step is governed by the Bessel weight $J_n(\alpha)$, which reflects the strength of phase modulation by the microwave field. In contrast to the Tien--Gordon description of quasi-particle tunneling, the Bessel functions appear here without being squared because the Josephson current depends linearly on the phase modulation rather than on transition probabilities. Consequently, the step amplitudes scale as $J_n(\alpha)$ instead of $J_n^2(\alpha)$. The time-averaged \textit{I--V} characteristic can therefore be written in direct analogy to the Tien--Gordon description,
        \begin{equation}
            I(V_0)
                = \sum_{n=-\infty}^{\infty}
                    J_n(\alpha)\,
                    I_{0}\!\left(V_0 - \frac{nh\nu}{2e}\right),
            \label{eq:macro:shapiro-iv}
        \end{equation}
        where $I_0(V_0)$ denotes the static Josephson \textit{I--V} curve in the absence of microwaves.

        In a real junction, quasiparticle tunneling occurs in parallel with the phase-driven supercurrent. Under microwave irradiation, both Shapiro steps and PAT appear simultaneously. This combined response constitutes the characteristic microwave-driven transport signature of a weakly coupled Josephson junction.
        As illustrated in Fig.~\ref{fig:macro:shaprio-iv}, the interplay of phase locking and photon-assisted tunneling produces a rich structure in both the \textit{I--V} and differential conductance characteristics.
        \begin{figure}[t]
            \centering
            \subfigure[
                Calculated \textit{I--V} characteristics for different drive amplitudes $eA/\Delta_0$, showing the emergence of Shapiro steps at voltages $V_n = nh\nu/2e$ together with photon-assisted quasiparticle features.
                ]{\import{theory/macro}{shapiro-iv.pgf}}
            \subfigure[
                Corresponding differential conductances $\mathrm{d}I/\mathrm{d}V$, highlighting sideband replicas of the coherence peaks weighted by $J_n^2(eA/h\nu)$.
                ]{\import{theory/macro}{shapiro-didv.pgf}}
            \caption{
                Microwave-driven transport in a weakly coupled Josephson junction. Parameters correspond to aluminum (Eq.~\ref{eq:aluminum}), with $T=0$, $\gamma=0$, and $\nu=5.0\,\mathrm{GHz}$.
                }
            \label{fig:macro:shaprio-iv}
        \end{figure}


    %=========================================================
    \subsection{RCSJ Model}
    \label{subsec:macro:rcsj}
    %=========================================================    

        The ideal Josephson relations describe how the supercurrent depends on the phase and how the phase evolves under a constant voltage. However, real junctions exhibit dissipation, capacitance, thermal fluctuations, and microwave-driven phase dynamics that cannot be    captured by the ideal equations alone.

        A convenient way to incorporate these additional contributions is to represent the junction as an effective circuit. To describe these effects, the Josephson element    must be embedded into an effective circuit model.

        \begin{wrapfigure}[11]{r}{0.3\textwidth}
            \captionsetup{format=plain}%
            \centering
            \vspace{-1em} % fine-tune vertical position
            \includegraphics[width=.2\textwidth]{theory/macro/RCSJ.png}
            \caption{
            RCSJ model of a real Josephson Junction.\\\textbf{in progress..}
            }
            \label{fig:macro:rcsj}
        \end{wrapfigure}
        The resistively and capacitively shunted junction (RCSJ) model provides the minimal    dynamical description of a real Josephson junction. It augments the ideal Josephson    element with a normal resistance accounting for quasiparticle tunneling and a    capacitance associated with the junction electrodes, as sketched in Figure~\ref{fig:macro:rcsj}. The resulting dynamics of the phase capture key experimental features such as switching between the superconducting and resistive branches, hysteresis in the \textit{I--V} characteristic, and thermally activated phase diffusion.

        The RCSJ model is also essential for understanding microwave-driven phenomena. In particular, the damping regime, strongly influences the visibility and stability of Shapiro steps. Overdamped junctions exhibit clean and well-defined steps, while underdamped junctions show hysteresis and residual phase oscillations that distort the step structure.

        In the following, we derive the RCSJ equation of motion for the phase and discuss the different dynamical regimes relevant for the interpretation of experimentally measured \textit{I--V} curves.

        To obtain the phase dynamics of the junction, the currents through the three parallel elements are summed according to Kirchhoff's law,
        \begin{equation}
            I_\mathrm{bias} = I_\mathrm{C}\sin(\phi)
                + \frac{V}{R}
                + C\frac{\mathrm{d}V}{\mathrm{d}t}\,.
            \label{eq:macro:rcsj-Ibias}
        \end{equation}
        We use $I_\mathrm{bias}$ to denote the externally applied current, distinguishing it from the Josephson supercurrent $I_\mathrm{J}$, the resistive current, and the capacitive displacement current that appear in parallel in the RCSJ model.

        Using the AC~Josephson voltage-phase relation, this expression can be written entirely in terms of the phase,
        \begin{equation}
            I_\mathrm{bias} = I_\mathrm{C} \sin(\phi)
            + \frac{\hbar}{2eR}\,\frac{\mathrm{d}\phi}{\mathrm{d}t}
            + \frac{\hbar C}{2e}\,\frac{\mathrm{d}^2\phi}{\mathrm{d}t^2}
            \,.
            \label{eq:macro:rcsj-cpr}
        \end{equation}
        This is the RCSJ equation of motion and forms the basis for all classical descriptions of Josephson phase dynamics. It contains an inertial term, a damping term, and the nonlinear Josephson restoring force, producing the characteristic tilted-washboard potential and the associated dynamical regimes discussed below.

        To analyze the dynamics implied by Eq.~\ref{eq:macro:rcsj-cpr}, it is convenient to express the equation in a dimensionless form. This highlights the relative importance of inertia, damping, and the nonlinear Josephson term, and introduces the characteristic time and energy scales of the junction.

        We first define the Josephson plasma frequency,
        \begin{equation}
            \omega_\mathrm{p}
            = \sqrt{\frac{2eI_\mathrm{C}}{\hbar C}}\,,
            \label{eq:macro:rcsj-plasma-frequency}
        \end{equation}
        which sets the natural oscillation frequency of the phase in the absence of damping. 
        The plasma frequency is distinct from the AC Josephson frequency $\nu_0$. While $\nu_0$ reflects voltage-driven phase evolution, $\omega_p$ sets the natural resonance frequency of the phase in the absence of bias and therefore provides the appropriate time scale for normalizing the RCSJ equation.

        Using the rescaled time variable $t' = \omega_\mathrm{p} t$ and normalizing the current by the critical current, $i = I_\mathrm{bias}/I_\mathrm{C}$, the RCSJ equation takes the compact dimensionless form
        \begin{equation}
            i
            = \sin\phi
            + \frac{1}{Q}\,\frac{\mathrm{d}}{\mathrm{d}t'}\,\phi
            + \frac{\mathrm{d}^2}{\mathrm{d}t'^2}\,\phi\,.
            \label{eq:macro:rcsj-cpr-norm}
        \end{equation}
        
        The quality factor that characterizes the damping of the phase dynamics, is given by
        \begin{equation}
            Q = \sqrt{\frac{2eI_\mathrm{C}R^2C}{\hbar}}\,.
            \label{eq:macro:rcsj-damping}
        \end{equation}

        This normalized equation makes the physical regimes transparent. The term $\mathrm{d}^2\phi/\mathrm{d}t'^2$ describes the inertial motion of the phase, $(1/Q)(\mathrm{d}\phi/\mathrm{d}t')$ represents viscous damping due to the normal resistance, and $\sin(\phi)$ provides the nonlinear restoring force from the Josephson coupling. The single dimensionless parameter $Q$ therefore determines whether the junction is underdamped ($Q\gg 1$) or overdamped ($Q\ll 1$), anticipating the dynamical behavior discussed in the following subsections.


        \subsubsection*{Washboard Potential Interpretation}

            The normalized RCSJ equation~\ref{eq:macro:rcsj-cpr-norm}, admits a mechanical interpretation that provides an intuitive picture of the phase dynamics. The equation is mathematically equivalent to the motion of a particle of unit mass in a tilted, periodic potential subject to viscous damping. The potential is obtained by identifying the restoring force with $-\partial U/\partial\phi = I_\mathrm{C}\sin\phi$, which yields
            \begin{equation}
                U(\phi)
                = -\,\cos\phi \;-\; i\,\phi.
                \label{eq:macro:washboard-potential}
            \end{equation}

            \begin{wrapfigure}[13]{r}{0.4\textwidth}
                \captionsetup{format=plain}%
                \centering
                \vspace{-1em} % fine-tune vertical position
                \import{theory/macro}{washboard-potential.pgf}
                \caption{
                U(phi) for i = 0, 1/2, 1.\\\textbf{in progress.. incl. ball?}
                }
                \label{fig:macro:rcsj-washboard}
            \end{wrapfigure}
            This ''tilted washboard potential'' consists of a cosine landscape whose overall slope is controlled by the normalized bias current $i$, as shown in Figure~\ref{fig:macro:rcsj-washboard}.

            For small bias, $i<1$, the potential exhibits a series of metastable minima. In this regime, the phase can remain localized in one of these wells, corresponding to the zero-voltage superconducting branch of the \textit{I--V} characteristic. Small oscillations of the phase around the minimum occur at the plasma frequency defined in Eq.~\ref{eq:macro:rcsj-plasma-frequency}, but the average voltage remains zero.

            As the bias is increased, the tilt of the potential grows and the barriers separating adjacent minima are reduced. At $i=1$, the barriers vanish and the phase becomes free to run down the potential without encountering any local minima. This running solution corresponds to the finite-voltage, resistive state of the junction. In the intermediate regime, fluctuations, either thermal or current-induced, can cause the phase to escape from a metastable minimum even for $i<1$, giving rise to switching from the superconducting to the resistive branch.
        
            A finite phase momentum acquired during escape might, once the bias is reduced again, retraps the phase only at a smaller tilt, leading to a retrapping back to the superconducting branch.

            The washboard picture thus provides a unified interpretation of phase localization, escape, and running dynamics. It naturally explains the origin of the critical current, the emergence of hysteresis in underdamped junctions, and the role of thermal activation and noise, which are explored in the following subsections.


        \subsubsection*{Switching Dynamics and Phase Diffusion}

            \begin{wrapfigure}[22]{r}{0.4\textwidth}
                \captionsetup{format=plain}%
                \centering
                \vspace{-1em} % fine-tune vertical position
                \import{theory/macro}{rcsj-iv.pgf}
                \caption{
                    Qualitative \textit{I--V} characteristic of a Josephson junction in the intermediate-damping regime ($Q\sim 1$), illustrating the combined effects of phase diffusion and residual inertia. Thermal and current noise produce a finite slope in the zero-voltage branch, while moderate damping leads to a smooth but non-hysteretic transition into the running state. At finite voltage, the current approaches the resistive branch set by the shunt resistance. 
                }
                \label{fig:macro:rcsj-iv}
            \end{wrapfigure}
            The switching current $I_\mathrm{sw}$ is the experimentally observed current at which a Josephson junction leaves the zero-voltage state and enters the resistive, running-phase regime. Unlike the intrinsic critical current $I_\mathrm{C}$, which is a microscopic parameter set by the Josephson coupling energy, the switching current is a stochastic quantity. Its value is determined by the competition between the decreasing barrier height of the tilted washboard potential and fluctuations that drive premature escape. Thermal activation, current noise from the electromagnetic environment, and the rate at which the bias current is ramped all increase the likelihood of early escape and thus reduce $I_\mathrm{sw}$ below $I_\mathrm{C}$. Only in the ideal, noise-free limit does the switching current approach the intrinsic critical current, $I_\mathrm{sw} \to I_\mathrm{C}$.

            In real Josephson junctions the supercurrent branch acquires a finite slope due to thermal and noise-induced fluctuations of the phase. Even for currents below the intrinsic critical value, stochastic forces drive small excursions of the phase within the tilted washboard potential, leading to a finite average $\langle\mathrm{d}\phi/\mathrm{d}t\rangle$ and hence a small, non-zero voltage. In the RCSJ model this effect is enhanced by the resistive branch in parallel with the Josephson element, which provides a dissipative path whenever the phase is not perfectly static. As a result, the ideal vertical supercurrent step is replaced by a broadened zero-voltage region whose slope grows with temperature, current noise and environmental damping.Consequently, both the switching and retrapping branches show a finite slope rather than the ideal vertical transitions expected from the noise-free Josephson relations.
            
            A qualitative \textit{I--V} trace for this intermediate-damping regime is shown in Fig.~\ref{fig:macro:rcsj-iv}, highlighting the finite slope of the supercurrent branch due to phase diffusion and the smooth transition into the running state characteristic of $Q\sim 1$ junctions.


        \subsubsection*{Over- and Underdamped Regime}

            In the RCSJ framework, the single dimensionless parameter $Q$ introduced in Eq.~\ref{eq:macro:rcsj-damping} distinguishes two qualitatively different dynamical regimes. These regimes determine whether the phase behaves inertially or is dominated by damping, and they govern the appearance or absence of hysteresis, plasma oscillations, and the overall shape of the \textit{I--V} characteristic.

            In the underdamped regime ($Q \gg 1$), the inertial term in Eq.~\ref{eq:macro:rcsj-cpr-norm} dominates over the damping term. The phase therefore retains sufficient kinetic energy to oscillate within the minima of the washboard potential, giving rise to small-amplitude plasma oscillations at the frequency $\omega_\mathrm{p}$. When the bias approaches the critical value, the tilt of the potential becomes large enough that even modest fluctuations allow the phase to escape a metastable minimum and enter the running state. Because the phase carries kinetic energy after escape, it does not immediately relock when the bias is reduced, leading to a retrapping current that is significantly smaller than the switching current. This difference between switching and retrapping currents produces the characteristic hysteresis observed in underdamped \textit{I--V} curves. Weak damping also makes the junction sensitive to residual oscillations, which can distort Shapiro steps and reduce their visibility under microwave irradiation.

            In the overdamped regime ($Q \ll 1$), the phase possesses essentially no inertia and cannot oscillate within the minima of the washboard potential. Instead, it follows the local slope of the potential without overshoot, causing the transition from the superconducting to the resistive state to occur smoothly and without hysteresis. Thermal and current noise then induce a diffusive wandering of the phase, producing a finite average voltage even for bias currents below $I_\mathrm{C}$. This phase-diffusion regime replaces the sharp supercurrent branch expected in the ideal model by a broadened but continuous zero-voltage region whose slope grows with temperature and environmental noise. Because the phase does not acquire sufficient kinetic energy to overshoot potential minima, switching and retrapping currents coincide in this regime. Overdamped junctions also exhibit clean and well-defined Shapiro steps, since strong damping suppresses residual oscillations and stabilizes the frequency-locked phase dynamics. This behavior is characteristic of small-capacitance tunnel junctions such as the aluminum atomic contacts investigated in this work.

            In practice, the experimentally observed switching current $I_\mathrm{sw}$ and retrapping current $I_\mathrm{r}$ satisfy $I_\mathrm{r} \le I_\mathrm{sw} < I_\mathrm{C}$. In the underdamped regime ($Q \gg 1$), inertial phase dynamics lead to a pronounced hysteresis with $I_\mathrm{r} \ll I_\mathrm{sw}$. In the overdamped regime ($Q \ll 1$), switching and retrapping currents coincide and no hysteresis is observed. Only in the ideal, noise-free limit does the switching current approach the intrinsic critical current, $I_\mathrm{sw} \to I_\mathrm{C}$. Any finite thermal or current noise reduces $I_\mathrm{sw}$ below $I_\mathrm{C}$ via thermally activated phase slips.


        \subsubsection*{Shapiro Steps in the RCSJ Model}

            In the full RCSJ model, the appearance and visibility of Shapiro steps follow from the driven nonlinear dynamics of the Josephson phase. Including a microwave drive of amplitude $a$ and angular frequency $\omega$, the normalized RCSJ equation becomes
            \begin{equation}
                \ddot{\phi} + \frac{1}{Q}\dot{\phi} + \sin\phi
                = i + a\sin(\omega t).
                \label{eq:macro:rcsj-driven}
            \end{equation}
            Shapiro steps arise when the average phase velocity locks to an integer multiple of the drive frequency, $\langle\dot{\phi}\rangle = n\omega$, producing voltage plateaus at $V_n = nh\nu/2e$. In contrast to the ideal Josephson model, the existence and sharpness of these steps depend sensitively on the damping parameter $Q$ and on thermal fluctuations.

            Overdamped junctions ($Q\ll 1$) exhibit clean and well-defined Shapiro steps, since the strong damping suppresses inertial oscillations and stabilizes the frequency locked state. In underdamped junctions ($Q\gg 1$), inertia leads to hysteresis and residual oscillations of the phase, which distort or even obscure the step structure. Thermal noise further rounds the step edges and reduces the visibility of higher-order steps, linking the appearance of Shapiro features directly to the phase-diffusion mechanisms discussed in the previous subsection.

            A direct comparison of the resulting \textit{I--V} characteristics in the underdamped and overdamped regimes is shown in Fig.~\ref{fig:macro:rcsj-shapiro-iv}, highlighting the distorted steps in the underdamped case and the clean phase-locked plateaus characteristic of strongly overdamped junctions.
            \begin{figure}[h]
                \centering
                \includegraphics[width=.6\textwidth]{theory/macro/rcsj-shapiro-iv.png}
                \caption{
                    Comparison of Shapiro steps in underdamped and overdamped Josephson junctions. \textbf{Top:} Static \textit{I--V} characteristics at zero microwave drive ($f=0$). An underdamped junction ($Q\gg 1$) exhibits hysteresis with distinct switching and retrapping currents, whereas an overdamped junction ($Q\ll 1$) shows a smooth, non-hysteretic transition into the resistive branch. \textbf{Bottom:} Under microwave irradiation, phase locking produces Shapiro steps. In the underdamped case the steps are distorted and incomplete due to inertia and residual oscillations, while in the overdamped case clean and well-defined steps appear at voltages $V_n = nh\nu/2e$. The index $n$ labels the locking condition $\langle\dot{\phi}\rangle = n\omega$.
                    }
                \label{fig:macro:rcsj-shapiro-iv}
            \end{figure}

            The RCSJ model therefore provides the natural bridge between the microscopic Tien--Gordon description of photon-assisted tunneling and the experimentally observed microwave-driven \textit{I--V} characteristics of weakly coupled Josephson junctions.
