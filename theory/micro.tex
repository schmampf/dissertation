% !TEX root = ../thesis.tex

%=========================================================
\section{Microscopic Description}
\label{sec:micro}
%=========================================================

    The microscopic description of superconductivity used here begins with the Bardeen--Cooper--Schrieffer (BCS) theory. In this framework, an effective phonon mediated attraction allows electrons near the Fermi surface to form bound pairs of opposite momentum and spin, so called Cooper pairs. The resulting condensate is characterized by a complex pairing order parameter whose magnitude determines the gap in the quasi-particle spectrum. In the microscopic tunneling regime considered here, only the quasi-particle spectrum and the magnitude of the gap enter. Phase coherence becomes relevant for transport only when the phase difference between two superconductors matters, as discussed in later sections. \cite{cooper_bound_1956,bardeen_microscopic_1957}

    A key assumption of the BCS framework is that this effective interaction is weak compared to the electronic energy scales of the metal. This situation is referred to as the weak-coupling limit. In this regime, the attractive interaction acts only within a narrow energy shell around the Fermi surface, typically a few meV compared to Fermi energies of several eV. As a result, only a small fraction of electrons near the Fermi level participate in pairing, while the rest of the electronic structure remains essentially unaffected. Because the coupling is weak, the resulting energy gap and the critical temperature are small but can be predicted with high accuracy from the microscopic parameters. For all numerical examples and illustrations in this chapter, aluminum is used as a representative weak-coupling BCS superconductor (Eq.~\ref{eq:aluminum}).

    The superconducting order parameter is the complex pairing amplitude 
    \begin{equation}
        \Delta(\vec{r}) = \left|\Delta\right| \exp\!\left(\ima\phi(\vec{r})\right)\,.
        \label{eq:micro:complex-delta}
    \end{equation}
    Its magnitude represents the pairing strength and whose phase encodes macroscopic phase coherence. For the purposes of tunneling spectroscopy, only the magnitude $|\Delta|$ is experimentally relevant. In consequence, we treat $\Delta$ as a scalar quantity, i.e., we consider only its magnitude and assume a uniform, spatially constant phase, since tunneling spectroscopy in the weak-coupling limit is sensitive solely to the energy gap $|\Delta|$ and not to spatial or vectorial structure of the order parameter.

    The following subsections introduce the key elements of this microscopic picture. First, the temperature dependence of the superconducting gap is discussed. Second, the superconducting DOS and introduce realistic broadening are derived. Third, these quantities are connected to tunneling spectroscopy through the microscopic expressions for the tunnel current. Finally, photon-assisted tunneling extends this static picture to driven transport in the presence of an oscillating electromagnetic field.

    %=========================================================
    \subsection{Superconducting Gap}
    \label{subsec:micro:sc-gap}
    %=========================================================

        In the weak-coupling limit, the theoretical treatment of superconductivity simplifies considerably. The normal-state density of states can be taken as constant in the narrow energy shell relevant for pairing, and the attractive interaction may be approximated as uniform across this shell. As a consequence, many observable quantities assume universal forms, most notably the ratio between the zero-temperature gap and the critical temperature,
        \begin{equation}
            \Delta_0 \approx 1.764\, k_\mathrm{B} T_\mathrm{C}\,,
            \label{eq:micro:Delta0}
        \end{equation}
        which holds for all weak-coupling BCS superconductors.

        The superconducting gap, however, is not constant with temperature. At $T=0$, electrons within the pairing shell around the Fermi energy form a condensate characterized by a well-defined order parameter. With increasing temperature, thermal excitations break an increasing fraction of Cooper pairs, reducing the number of electrons bound in the superconducting state. As fewer pairs contribute to the condensate, the magnitude of the order parameter decreases, and the energy required to break a pair, $\Delta(T)$, is gradually reduced.

        This reduction proceeds smoothly until the critical temperature $T_\mathrm{C}$ is reached. At $T=T_\mathrm{C}$, thermal fluctuations fully disrupt the pairing correlations, the order parameter vanishes continuously, and superconductivity is lost. The temperature dependence of the gap reflects the balance between the condensation energy gained by pairing and the entropy associated with thermal excitations. The universal weak-coupling curve is shown in Fig.~\ref{fig:micro:gap}.

        \begin{wrapfigure}[13]{r}{0.4\textwidth}
            \captionsetup{format=plain}%
            \centering
            % \vspace{-1em}
            \import{theory/micro}{gap.pgf}
            \caption{Temperature dependence of the superconducting gap $\Delta(T)$ in the weak-coupling limit.}
            \label{fig:micro:gap}
        \end{wrapfigure}
        In the BCS framework, the functional form of $\Delta(T)$ is obtained from the self-consistent gap equation. Its numerical solution yields a universal curve valid for all weak-coupling superconductors. A commonly used approximation to this solution is
        \begin{equation}
            \frac{\Delta(T)}{\Delta_0}
                \approx \tanh\!\left(
                    1.74\,\sqrt{\frac{T_\mathrm{C}}{T} - 1}
                \right),
            \label{eq:micro:DeltaT}
        \end{equation}
        which reproduces the numerical result with high accuracy throughout the full temperature range. \cite{muhlschlegel_thermodynamischen_1959}


    %=========================================================
    \subsection{Density of States}
    \label{subsec:micro:dos}
    %=========================================================

        Throughout this section, energies are measured relative to the chemical potential, which for conventional metals coincides with the Fermi energy to very good approximation. Thus $E=0$ corresponds to the Fermi level around which superconducting correlations form.

        In the normal state, the electronic density of states (DOS) varies only weakly with energy near the Fermi level. Over the narrow pairing shell relevant to weak-coupling superconductivity, this variation can be neglected and the DOS may be taken as constant. The corresponding value at the Fermi level is
        \begin{equation}
            N_0 \equiv
                 \frac{1}{2\pi^2}\!
                \left(\frac{2m}{\hbar^2}\right)^{3/2}
                \sqrt{E_\mathrm{F}}\,,
            \label{eq:micro:DOS-N0}
        \end{equation}
        which follows from a free-electron parabolic dispersion and is an excellent approximation for aluminum.

        When superconductivity sets in, pairing correlations reorganize this otherwise flat spectrum. A gap of width $2\Delta$ opens around the Fermi level where single-particle excitations are absent in the ideal BCS limit. The spectral weight removed from the gap interior is redistributed to the gap edges, producing the characteristic coherence peaks. The resulting normalized DOS is
        \begin{equation}
            \frac{N_\mathrm{S}(E)}{N_0} = 
                \left\{
                \begin{array}{@{}r@{\quad}l@{}}
                    0 & (|E| < \Delta)\\
                    \dfrac{|E|}{\sqrt{E^2-\Delta^2}} & (|E| \ge \Delta)
                \end{array}
                \right.\,.
            \label{eq:micro:dos-bcs}
        \end{equation}       
        which diverges at the gap edges due to the flattening of the Bogoliubov quasi-particle dispersion. Within mean-field BCS theory this DOS is strictly sharp. Temperature affects only the occupation of the states through the Fermi--Dirac distribution and does not broaden the DOS itself.

        Real superconductors, however, never exhibit perfectly sharp coherence peaks. A widely used and remarkably successful phenomenology is the Dynes broadening, which models finite quasi-particle lifetimes by the substitution $E \rightarrow E + \ima\gamma$
        \begin{equation}
            \frac{N_\mathrm{S}(E)}{N_0}
                = \Re\!\left(
                    \frac{E+\ima\gamma}{
                    \sqrt{(E+\ima\gamma)^{2}-\Delta^{2}}}
                \right).
            \label{eq:micro:dos-dynes}
        \end{equation}
        The Dynes parameter\footnote{Throughout this work we distinguish between the Dynes broadening parameter $\gamma$, which enters the quasi-particle DOS in the microscopic tunneling description, and the tunneling rates $\Gamma$ that appear in the stochastic description of incoherent charge transfer. These quantities refer to different physical mechanisms and should not be confused.} $\gamma$ accounts for inelastic scattering, spatial inhomogeneity, pair breaking by magnetic impurities, or non-equilibrium effects, all of which smear the ideal BCS singularities. \cite{dynes_direct_1978}

        The DOS specifies the available single-particle states, whereas the Fermi--Dirac distribution determines their occupation at a given temperature. For conventional metals $\mu \approx E_\mathrm{F}$, so the energy scale remains consistent with the above convention. The equilibrium occupation is
        \begin{equation}
            f(E) = \frac{1}{1+\exp\!\left(\frac{E}{k_\mathrm{B}T}\right)},
            \label{eq:micro:fermi-dirac}
        \end{equation}
        which reduces to the step function $\theta(E)$ as $T\to 0$.

        For future application, we can define the thermal broadening kernel
        \begin{equation}
            K_\mathrm{T}(E) := - \frac{\partial f(E)}{\partial E}\,,
            \label{eq:micro:fermi-kernel}
        \end{equation}
        which reduces to a delta distribution $\delta(E)$ as $T\to 0$.

        Figure~\ref{fig:micro:dos-fermi}(a) visualizes how Dynes broadening smooths the coherence peaks of $N_\mathrm{S}(E)$, while Fig.~\ref{fig:micro:dos-fermi}(b) and (c) shows how finite temperature smears the Fermi edge and the thermal broadening kernel. 
        \begin{figure}
            \centering
            \subfigure[
                Dynes-broadened superconducting DOS $N_\mathrm{S}(E)/N_0$ for several values of $\gamma/\Delta_0$, showing how finite quasi-particle lifetimes smooth the coherence peaks and introduce subgap spectral weight. 
                ]{\import{theory/micro}{dos.pgf}}
            \subfigure[
                Fermi--Dirac distributions at different temperatures $T/T_\mathrm{C}$ illustrating thermal occupation smearing.
                ]{\import{theory/micro}{fermi.pgf}}
            \hspace{.2in}
            \subfigure[
                Thermal broadening kernel at different temperatures $T/T_\mathrm{C}$.
                ]{\import{theory/micro}{fermi-kernel.pgf}}
            \caption{
                Together, these three ingredients determine the qualitative shape and resolution of the microscopic \textit{I--V} and \textit{dI--dV} characteristics. Parameters correspond to aluminum (Eq.~\ref{eq:aluminum}).
                }
            \label{fig:micro:dos-fermi}
        \end{figure}

    
    %=========================================================
    \subsection{Tunnel Current}
    \label{subsec:micro:tunnel-current}
    %=========================================================

        Tunneling spectroscopy provides a direct and conceptually transparent probe of the quasi-particle excitation spectrum. When two electrodes are separated by a sufficiently thin insulating barrier, electrons may tunnel quantum-mechanically between them even though the process is classically forbidden. In the tunneling limit, the barrier is high and wide enough that phase coherence between the electrodes is lost and momentum conservation across the junction no longer applies. In this regime, tunneling reduces to an incoherent single-particle transfer process governed solely by the available electronic states and their occupation. \cite{bardeen_tunnelling_1961,giaever_electron_1960,giaever_study_1961}
        
        % tunneling I/V
        Under these conditions, the microscopic current flowing from electrode $1$ to electrode $2$ is proportional to the number of occupied states at energy $E$ in electrode $1$ and the number of empty states at energy $E+eV$ in electrode $2$
        \begin{equation}
            I_{1\to2}(V) \propto \int_{-\infty}^\infty \left(N_1(E) f_1(E) \right)\cdot \left(N_2(E+eV)\left(1-f_2(E+eV)\right)\right)\mathrm{d}E\,,
            \label{eq:micro:tunnel-1to2}
        \end{equation}        
        where $V$ is the externally applied bias voltage. The reverse process $2\to1$ is described by an analogous expression. Subtracting the two contributions yields the measurable tunnel current
        \begin{equation}
            I(V) \propto \int_{-\infty}^\infty N_1(E)\, N_2(E+eV)\, \left(f_1(E)-f_2(E+eV)\right)\mathrm{d}E\,.
            \label{eq:micro:tunnel}
        \end{equation}
        
        This equation highlights the two microscopic ingredients of tunneling. The densities of states $N_1(E)$ and $N_2(E)$, specifying where electrons may tunnel, and the Fermi--Dirac distributions, specifying which of those states are occupied. The factor $f_1(E) - f_2(E+eV)$ encodes the occupation imbalance created by the applied bias and ensures that current flows only when filled states in one electrode overlap with empty states in the other. The tunnel current therefore represents a convolution of the electronic structure of the two electrodes with their occupation functions, directly linking microscopic quasi-particle physics to the measurable \textit{I--V} characteristics.

        Throughout this section we adopt the asymmetric voltage convention, in which the applied bias shifts only the quasi-particle energies of electrode 2, $E \to E+eV$. This choice is equivalent to a gauge transformation and leaves the physics unchanged, but keeps the microscopic expressions particularly transparent.

        Depending on whether either electrode is normal or superconducting, Eq. \ref{eq:micro:tunnel} reduces to three experimentally relevant cases.
        
        % NN case
        \subsubsection*{N-N tunnel junction}
        In the case of a fully normal-conducting tunnel junction, both electrodes provide a flat density of states near the Fermi level. The microscopic expression for the current therefore reduces to
        \begin{equation}
            I_\mathrm{NN}(V) = \frac{G_\mathrm{N}}{e} \int_{-\infty}^{\infty}f(E) - f(E + eV)\mathrm{d}E\,,
            \label{eq:micro:tunnel-nn}
        \end{equation}
        where all geometric and matrix-element factors, as well as the constant DOS $N_0$, are absorbed into the normal-state conductance $G_\mathrm{N}$. 
        If both electrodes share the same temperature, the difference of Fermi functions integrates to the applied energy shift $eV$. The microscopic expression thus reproduces Ohm's law,
        \begin{equation}
            I_\mathrm{NN}(V) =  G_\mathrm{N}V\,,
            \label{eq:micro:ohms-law}
        \end{equation}
        showing that an NN junction exhibits purely ohmic behaviour in the tunneling limit.

        % NS case
        \subsubsection*{N-S tunnel junction}
        In a junction between a normal metal and a superconductor, the tunneling current becomes sensitive to the superconducting quasi-particle spectrum. Since the normal electrode provides a flat DOS, only the superconducting DOS enters the microscopic expression,
        \begin{equation}
            I_\mathrm{NS}(V) = \frac{G_\mathrm{N}}{e} \int_{-\infty}^{\infty} N_\mathrm{S}(E) \left(f(E) - f(E + eV)\right) \mathrm{d}E\,.
            \label{eq:micro:tunnel-ns}
        \end{equation}

        % dI/dV NS case
        In the N-S case, one can probe the superconducting DOS, by measuring the \textit{dI--dV} characteristic. It is proportional to the convolution, of the superconducting DOS with the thermal broadening kernel (Eq.~\ref{eq:micro:fermi-kernel}). 
        The thermal broadening kernel sets the fundamental energy resolution\footnote{
            A commonly used estimate is $\delta E \approx 3.5\,k_\mathrm{B}T$. At 30\,mK (typical for a dilution refrigerator), this corresponds to about 10\,\textmu eV, whereas at 300\,mK (He$_\text{3}$ cryostat) the resolution is limited to about 100\,\textmu eV. The relevant temperature is the electronic temperature, which can exceed the cryostat base temperature due to imperfect filtering, finite electron-phonon coupling, or microwave leakage.
        } in N-S tunnel spectroscopy. However, it does not change the superconducting DOS, it alters the measured \textit{dI--dV}. As temperature approaches zero, the \textit{dI--dV} characteristic becomes directly proportional to the superconducting DOS.
        
        Differentiating Equation~\ref{eq:micro:tunnel-ns} with respect to $V$ acts only on the term $f(E+eV)$, since both $N_{\mathrm{S}}(E)$ and $f(E)$ are independent of the applied bias. Using the chain rule, reveals the thermal broadening kernel at shifted energy $E+eV$.
        This results precisely in the convolution\footnote{
            Convolution is mathematically defined as $\left( f \otimes g \right)(x) = \int f(y)g(x-y)\mathrm{d}y$. A convolution weights all shifted copies of one function by another, expressing how strongly their features overlap as one is slid across the other.
            } 
        of the superconducting DOS with the thermal broadening kernel evaluated at the energy shift $eV$,
        \begin{equation}
            \frac{\mathrm{d}I_\mathrm{NS}(V)}{\mathrm{d}V} = G_\mathrm{N} \left( N_\mathrm{S} \otimes K_\mathrm{T}\right)(eV)\,.
            \label{eq:micro:tunnel-ns-didv}
        \end{equation}

        These considerations directly underlie scanning tunneling microscopy, where a normal-metal tip above a superconducting surface realizes an N-S junction in the tunneling limit. Measuring the resulting \textit{I--V} or \textit{dI--dV} characteristics, at sufficiently low electronic temperatures, therefore provides a direct and spatially localized probe of the superconductor's quasi-particle DOS. In addition, the thermal broadening of the spectrum enables a reliable determination of the system's electronic temperature.

        % SS case
        \subsubsection*{S-S tunnel junction}
        In a junction where both electrodes are superconducting, the tunnel current depends on the multiplicative overlap of the two gapped densities of states, evaluated at energies $E$ and $E+eV$. The microscopic expression now reads
        \begin{equation}
            I_\mathrm{SS}(V)
                = \frac{G_\mathrm{N}}{e}
                  \int_{-\infty}^{\infty}
                      N_\mathrm{S}(E)\,
                      N_\mathrm{S}(E+eV)\,
                      \left(f(E) - f(E+eV)\right)
                  \mathrm{d}E\,.
            \label{eq:micro:tunnel-ss}
        \end{equation} 

        For S-S junctions, the differential conductance cannot be written as a convolution of the superconducting DOS with the thermal kernel. The shifted DOS $N_{\mathrm{S}}(E+eV)$ carries an explicit voltage dependence, so differentiating the microscopic current generates additional contributions. Although the convolution form is sometimes quoted in the literature by analogy to the N-S case, this represents a misinterpretation and an unjustified generalization of the N-S result.

        In contrast to the N-S case, the S-S differential conductance reflects the full multiplicative structure of the two gapped DOS. As a result, the onset of tunneling occurs only once $|eV|$ exceeds $2\Delta$, producing a pronounced threshold in both \textit{I--V} and \textit{dI--dV}.

        % ref figures
        Figure~\ref{fig:micro:tunnel-iv-didv} summarizes these effects and makes the contrasting roles of temperature and Dynes broadening in NS and SS junctions explicit.  
        \begin{figure}
            \centering
            \subfigure[
                Calculated \textit{I--V} characteristics for different temperatures $T/T_\mathrm{C}$ at vanishing Dynes broadening ($\gamma = 0$). Increasing temperature rounds the coherence peaks and gradually fills the subgap region. 
                ]{\import{theory/micro}{tunnel-iv-T.pgf}}
            \subfigure[
                Corresponding differential conductances $\mathrm{d}I/\mathrm{d}V$, showing the thermally induced suppression and broadening of the coherence peaks.
                ]{\import{theory/micro}{tunnel-didv-T.pgf}}
            \subfigure[
                Calculated \textit{I--V} characteristics for different Dynes parameters $\gamma/\Delta_0$ at $T=0$. Lifetime broadening smooths the onset of quasi-particle tunneling and produces finite subgap conductance.
                ]{\import{theory/micro}{tunnel-iv-gamma.pgf}}
            \subfigure[
                Corresponding differential conductances $\mathrm{d}I/\mathrm{d}V$, illustrating the strong suppression and smearing of the coherence peaks due to lifetime broadening. 
                ]{\import{theory/micro}{tunnel-didv-gamma.pgf}}
            \caption{
                Temperature and Dynes broadening effects on NS (grey) and SS (blue) tunnel junctions within the microscopic tunneling model. Includes NN (lightgrey dashed) junction for reference. Parameters correspond to aluminum (Eq.~\ref{eq:aluminum}).
                }
            \label{fig:micro:tunnel-iv-didv}
        \end{figure}

        In NS junctions, temperature primarily broadens the occupation factor and therefore rounds the coherence peak and smooths the gap edge, while Dynes broadening directly reduces the sharpness of the superconducting DOS itself.  

        In SS junctions, the impact of both mechanisms is amplified by the multiplicative appearance of the DOS from each electrode. Temperature generates a noticeable subgap conductance and diminishes the peak height as it approaches the critical temperature. Whereas Dynes broadening suppresses the spectral features far more strongly than in the NS case due to its action on both electrodes simultaneously.
        
        By fitting measured \textit{I-V} or \textit{dI-dV} data, with $\Delta_0$, $T$, and $\gamma$ as variables, one can disentangle thermal effects from genuine lifetime or inelastic processes. These interpretations form the foundation of tunneling spectroscopy as a quantitative probe of superconductivity, allowing the extraction of these parameter from experimental data with high accuracy.

    %=========================================================
    \subsection{Photon-Assisted Tunneling}
    \label{subsec:micro:pat}
    %=========================================================

        In the static tunneling expressions above, the Hamiltonian is time-independent and the Bardeen tunneling matrix element carries only its intrinsic phase. Once a microwave drive is applied, as introduced in the introduction of this chapter, this matrix element acquires the time-dependent phase factor of Eq.~\ref{eq:microwave:phase-factor}. Here the relevant charge for quasi-particle tunneling is $q=e$. The drive thus modulates the single-electron tunneling phase, not the macroscopic superconducting phase, and this modulation is the microscopic origin of photon-assisted tunneling (PAT).

        PAT remains an incoherent quasi-particle process. The AC drive mixes electronic energies separated by integer multiples of $h\nu$, but it does not induce any phase-coherent Cooper-pair dynamics. This sharply distinguishes PAT from the Josephson phase dynamics responsible for Shapiro steps (Sec.~\ref{subsec:macro:shapiro}).

        The periodic phase modulation shifts the quasi-particle energy by discrete amounts. Let $E_0$ denote the unmodulated quasi-particle energy appearing in the static tunneling integral. Absorption or emission of $|n|$ photons then produces photon-shifted energies
        \begin{equation}
            E_n = E_0 + n h\nu \qquad (n\in\mathbb{Z})\,,
            \label{eq:micro:pat-En}
        \end{equation}
        where $n>0$ corresponds to absorption and $n<0$ to emission. Using the Jacobi--Anger expansion of the global phase factor, the static quasi-particle DOS is replaced by the photon-dressed DOS
        \begin{equation}
            N(E_0)
                = \sum_{n=-\infty}^{\infty}
                    J_n^2(\alpha)\, N(E_0 + n h\nu)\,,
            \label{eq:micro:pat-dos}
        \end{equation}
        where $\alpha = eA/h\nu$ is the single-electron drive strength. Here, the photon-dressed DOS refers to the electrode whose quasi-particle energies are shifted by the applied bias, consistent with the asymmetric voltage convention used throughout this section.

        Analogously, the tunnel current becomes a weighted sum of shifted copies of the static characteristic,
        \begin{equation}
            I(V_0)
                = \sum_{n=-\infty}^{\infty}
                    J_n^2\!\left(\frac{eA}{h\nu}\right)
                    I_0\!\left(V_0 - \frac{n h\nu}{e}\right),
            \label{eq:micro:pat-iv}
        \end{equation}
        which is the Tien--Gordon result. Each term contributes a replica of $I_0(V_0)$ shifted by $n h\nu/e$ and weighted by the probability $J_n^2(\alpha)$ of absorbing or emitting $|n|$ photons. \cite{tien_multiphoton_1963}

        PAT therefore provides a universal semiclassical modification of quasi-particle tunneling. All differences between NS and SS junctions under microwave irradiation arise solely from their static DOS and not from the PAT mechanism itself. Figure~\ref{fig:micro:pat} illustrates this behaviour for NS (grey) and SS (blue) junctions.
        \begin{figure}
            \centering
            \subfigure[
                Calculated \textit{I--V} characteristics for NS (grey) and SS (blue) junctions under microwave irradiation for different drive amplitudes $eA/\Delta_0$. Increasing amplitude generates additional photon-assisted tunneling channels at voltages shifted by $\pm n h\nu/e$.
                ]{\import{theory/micro}{pat-iv.pgf}}
            \subfigure[
                Corresponding differential conductances $\mathrm{d}I/\mathrm{d}V$, showing photon-assisted replicas of the coherence peaks, weighted by $J_n^2(eA/h\nu)$. 
                ]{\import{theory/micro}{pat-didv.pgf}}
            \caption{
                Photon-assisted tunneling in NS (grey) and SS (blue) junctions. The microwave drive couples electronic states differing in energy by $h\nu$, producing sideband replicas of the static \textit{I--V} and \textit{dI--dV} characteristics. Parameters correspond to aluminum (Eq.~\ref{eq:aluminum}), with $T = 0$, $\gamma = 0$, and $\nu = 5.0\,\mathrm{GHz}$.
                }
            \label{fig:micro:pat}
        \end{figure}

        Experimentally, the drive amplitude $A$ can be extracted from the relative sideband heights through the Bessel weights, while the spacing $h\nu$ between replicas provides a robust calibration of the microwave frequency. Photon-assisted tunneling thereby bridges static tunneling spectroscopy and driven superconducting transport in a fully microscopic and conceptually transparent way.