% !TEX root = ../thesis.tex

\section{Microscopic Description}
\label{sec:micro}

    The microscopic\footnote{quasi-particle tunneling (no phase coherence)} description of superconductivity is provided by the Bardeen--Cooper--Schrieffer (BCS) theory, which assumes that electrons near the Fermi surface form bound pairs of opposite momentum and spin, so called Cooper-pairs.
    
    At its heart, this pairing mechanism reflects a subtle interplay between electrons and phonons. When an electron moves through the metal, it slightly displaces the positively charged ions, creating a momentary region of excess positive charge. A second electron passing nearby can be attracted to this distortion, leading to an effective, though very weak attraction between the two. 
    
    This describes how electrons in a metal, which normally repel each other due to their negative charge, can nevertheless form bound pairs, known as Cooper-pairs, through an effective attraction mediated by vibrations of the crystal lattice. When many such pairs form simultaneously, they condense into a collective quantum state that carries electrical current without resistance.

    A key assumption of the BCS framework is that this effective interaction is weak compared to the electronic energy scales of the metal. This situation is referred to as the weak-coupling limit. In this regime, the attractive interaction acts only within a narrow energy shell around the Fermi surface, typically a few meV compared to Fermi energies of several eV. As a result, only a small fraction of electrons near the Fermi level participate in pairing, while the rest of the electronic structure remains essentially unaffected. Because the coupling is weak, the resulting energy gap and the critical temperature are small but can be predicted with high accuracy from the microscopic parameters.

    Among all known superconductors, aluminum is one of the best realizations of this weak-coupling scenario. Its superconducting transition temperature and energy gap are small, the electron-phonon interaction is weak and well understood, and its electronic structure is simple and free from strong correlations or magnetic effects. These features make aluminum behave almost exactly as predicted by the original BCS theory, allowing quantitative agreement between experiment and theory without the need for corrections or more advanced models.

    The superconducting order parameter is the complex pairing amplitude 
    \begin{equation}
        \Delta = |\Delta| e^{\ima\phi}\,,
        \label{eq:micro:complex-delta}
    \end{equation}
    whose magnitude represents the pairing strength and whose phase encodes global coherence. In the following, we treat $\Delta$ as a scalar quantity, i.e., we consider only its magnitude and assume a uniform, spatially constant phase, since tunneling spectroscopy in the weak-coupling limit is sensitive solely to the energy gap $|\Delta|$ and not to spatial or vectorial structure of the order parameter.

    In summary, the weak-coupling limit describes a situation where the effective electron-phonon attraction is small compared to the Fermi energy, leading to a narrow pairing region and universal relationships between microscopic and macroscopic quantities. Aluminum exemplifies this regime and therefore serves as an archetype of a conventional BCS superconductor.

    The following subsections outline the key aspects of this microscopic picture. 
    First, the superconducting gap and its temperature dependency is introduced. This shows how the collective pairing strength evolves with thermal excitation. 
    Second, the electronic density of states is derived. The phenomenological Dynes parameter as measure of quasi-particle lifetime is introduced. The Fermi-Dirac distribution function and its temperature dependency is introduced as well.
    Third, the microscopic expressions for tunneling current are discussed, establishing the direct connection between these theoretical quantities and experimentally measurable \textit{I-V} and \textit{dI-dV} characteristics. 

    Finally, the extension to photon-assisted tunneling is presented, where an oscillating electromagnetic field enables quasi-particles to exchange discrete energy quanta during tunneling, thereby extending the static tunneling picture to the regime of externally driven superconducting transport.

    \cite{cooper_bound_1956, bardeen_microscopic_1957}

    \subsection{Superconducting Gap}
    \label{subsec:micro:sc-gap}

        The weak-coupling approximation simplifies the theoretical treatment significantly. The normal-state density of states can be treated as constant, and the pairing interaction can be approximated as uniform within the relevant energy range. Many observable quantities therefore become universal, such as the ratio between the zero-temperature gap and the critical temperature, 
        \begin{equation}
            \Delta_0 \approx 1.764\, k_\mathrm{B} T_\mathrm{C}\,.
            \label{eq:micro:Delta0}
        \end{equation}

        The superconducting gap does not remain constant with temperature. At absolute zero, all available electrons near the Fermi surface form Cooper-pairs, and the condensate is characterized by a well-defined order parameter. As the temperature rises, thermal excitations begin to break some of these pairs, leaving fewer electrons bound in the superconducting state. With fewer pairs contributing to the collective order, the overall pairing strength weakens, and the energy required to break a pair the gap $\Delta(T)$ gradually decreases. 

        \begin{wrapfigure}[11]{r}{0.4\textwidth}
            \captionsetup{format=plain}%
            \centering
            \vspace{-1em} % fine-tune vertical position
            \import{theory/micro}{gap-suppression.pgf}
            \caption{Temperature dependence of the superconducting gap $\Delta(T)$.}
            \label{fig:micro:gap_suppression}
        \end{wrapfigure} 
        This reduction continues smoothly until the critical temperature $T_\mathrm{C}$ is reached. At that point, thermal energy becomes strong enough to completely disrupt the pairing correlations, and the superconducting state collapses, leading to $\Delta(T_\mathrm{C}) = 0$. The temperature dependence of the gap is a direct reflection of this balance between thermal disorder and the pairing interaction. The universal temperature dependence of the gap is shown in Figure~\ref{fig:micro:gap_suppression}.

        In the BCS framework, $\Delta(T)$ follows a universal curve that results from solving the self-consistent gap equation, meaning the same functional form applies to all weak-coupling superconductors. However, solving the underlying integrals over the Fermi distribution in the microscopic theory numerically, results in
        \begin{equation}
            \frac{\Delta(T)}{\Delta_0} \approx \tanh\left(1.74\,\sqrt{\frac{T_\mathrm{C}}{T}-1}\right)\,.
            \label{eq:micro:DeltaT}
        \end{equation}

        \cite{bardeen_microscopic_1957, muhlschlegel_thermodynamischen_1959}


    \subsection{Density of States}
    \label{subsec:micro:dos}

        In the following, all energies are expressed relative to the Fermi energy $E_\mathrm{F}$, such that $E=0$ corresponds to the Fermi level around which superconducting correlations develop.
        
        In the normal state, the density of states (DOS) near the Fermi level varies only weakly with energy. Over the narrow range relevant to superconductivity this variation can be neglected, and the DOS can be treated as constant. The corresponding value at the Fermi energy is denoted
        \begin{equation}
            N_0 \equiv N_\mathrm{N}(E_\mathrm{F}) = \frac{1}{2\pi^2} \left(\frac{2m}{\hbar^2}\right)^{3/2} \sqrt{E_\mathrm{F}}\,,
            \label{eq:micro:DOS-N0}
        \end{equation}
        representing the normal state DOS per spin at the Fermi level. This expression assumes a free-electron parabolic dispersion. For aluminum, whose Fermi energy is large and whose band structure is close to a free-electron gas, this approximation is highly accurate.

        When superconductivity sets in, pairing correlations reorganize this otherwise flat spectrum. A gap of width $2\Delta$ opens around $E_\mathrm{F}$ where single-particle excitations are absent in the ideal BCS limit, and the missing spectral weight is redistributed to the gap edges. These edges appear as sharp coherence peaks in the superconducting DOS, reflecting the high density of available quasi-particle states at the threshold for pair breaking. The resulting expression reads
        \begin{equation}
            \frac{N_\mathrm{S}(E)}{N_0} = 
                \left\{
                \begin{array}{@{}r@{\quad}l@{}}
                    0 & (|E| < \Delta)\\
                    \dfrac{|E|}{\sqrt{E^2-\Delta^2}} & (|E| \ge \Delta)
                \end{array}
                \right.\,.
            \label{eq:micro:DOS-BCS}
        \end{equation}        
        Thus, in the ideal BCS limit the quasi-particle DOS is strictly zero within the energy gap and diverges at its edges. The divergence at the gap edges originates from the flattening of the Bogoliubov quasiparticle dispersion, which enhances the density of available states. Temperature affects only the occupation factor, i.e., the Fermi--Dirac distribution, and does not smear the DOS itself.

        However, real spectra are never perfectly sharp. A simple and very effective phenomenology is the broadening by Dynes parameter. It is implemented by the substitution $E\to E+\ima\Gamma$, while just considering the real part
        \begin{equation}
            \frac{N_\mathrm{S}(E)}{N_0} = \Re\!\left(\frac{E+\ima\Gamma}{\sqrt{(E+\ima\Gamma)^2-\Delta^2}}\right) \quad (|E| \ge \Delta)\,.
            \label{eq:micro:DOS-Dynes}
        \end{equation}
        Such broadening arises from finite quasi-particle lifetimes due to inelastic scattering, spatial inhomogeneity, pair breaking by magnetic impurities, or non-equilibrium effects, all of which smear the ideal BCS singularities. 

        Whereas the DOS specifies where states exist, the Fermi--Dirac distribution encodes how single-particle states are occupied at a given temperature and chemical potential. In the context of metals and conventional superconductors treated in this thesis, the chemical potential can be identified with the Fermi energy to very good approximation, $\mu\approx E_\mathrm{F}$, because thermal corrections are small on the scale of eV. So the energy scale is still relative to the Fermi energy, as before.

        In equilibrium the occupation probability of a state at energy $E$ is given by        
        \begin{equation}
            f(E) = \frac{1}{1+\exp\left(\frac{E}{k_\mathrm{B}T}\right)}\,.
            \label{eq:micro:fermidirac}
        \end{equation}        
        At zero temperature the distribution reduces to a sharp step $ \theta(E)$.
        
        Figure~\ref{fig:micro:dos-fermi} illustrates the broadening of DOS by quasi-particle lifetimes and the temperature-broadened Fermi function.
        \begin{figure}[t]
            \centering
            \import{theory/micro}{dos-fermi.pgf}
            \caption{Superconducting quasi-particle density of states $N_\mathrm{S}(E)$ and Fermi--Dirac distribution $f(E)$ for different Dynes parameters $\Gamma/\Delta_0$ and temperatures $T/T_\mathrm{C}$. Increasing $\Gamma$ broadens the coherence peaks of $N_\mathrm{S}(E)$, while increasing temperature smooths the Fermi edge. Parameters correspond to aluminum with $\Delta_0 = 180\,$\textmu eV and $T_\mathrm{C} = 1.18\,\mathrm{K}$, representative of a weak-coupling BCS superconductor.}
            \label{fig:micro:dos-fermi}
        \end{figure}

        At finite temperature the step is thermally broadened over an energy scale of order $k_\mathrm{B}T$, as shown in Figure~\ref{fig:micro:dos-fermi}. 

        \cite{dynes_direct_1978}
    
    \subsection{Tunnel Current}
    \label{subsec:micro:tunnel-current}

        Tunneling spectroscopy offers a powerful means to probe the quasi-particle spectrum of superconductors in a controlled and conceptually simple way. The key idea is that if two electrodes are separated by a sufficiently thin insulating barrier, quasi-particles can quantum-mechanically tunnel between them, even though classically forbidden. 
        
        In the tunneling limit, the barrier is high and wide enough that the process is incoherent and each quasi-particle tunnels independently, so momentum conservation is effectively relaxed. 
        
        Under these conditions, the tunnel current from material $1$ to $2$ is given by
        \begin{equation}
            I_{1\to2}(V) \propto \int_{-\infty}^\infty \left(\frac{N_1(E)}{N_0} f_1(E) \right)\cdot \left(\frac{N_2(E+eV)}{N_0} \left(1-f_2(E+eV)\right)\right)\mathrm{d}E\,,
            \label{eq:micro:tunnel-1to2}
        \end{equation}
        where $eV$ is an externally applied voltage bias. The first part in Equation~\ref{eq:micro:tunnel-1to2}, is given by the occupied states in material 1, the second part is given by the unoccupied states in material 2. However, in order to get the total tunnel current, one has to substract the tunnel current from material 2 to 1, resulting in
        \begin{equation}
            I(V) \propto \int_{-\infty}^\infty \frac{N_1(E)}{N_0} \frac{N_2(E+eV)}{N_0} \left(f_1(E)-f_2(E+eV)\right)\mathrm{d}E\,.
            \label{eq:micro:tunnel}
        \end{equation}
        
        The difference $f_1(E)-f_2(E+eV)$ accounts for the imbalance in occupation between the two electrodes induced by the applied bias voltage $eV$, ensuring that current flows only when filled states on one side overlap with empty states on the other. In this picture, the densities of states $N_1(E)$ and $N_2(E)$ define where electrons can tunnel, while the Fermi--Dirac distributions define which of those states are populated. The convolution of these terms thus directly connects the microscopic electronic structure to the measurable \textit{I-V} characteristics.

        In the case of an all normal conducting tunnel barrier, the current is given by
        \begin{equation}
            I_\mathrm{NN}(V) = \frac{G_\mathrm{N}}{e} \int_{-\infty}^{\infty}f(E) - f(E + eV)\mathrm{d}E\,.
            \label{eq:micro:tunnel-nn}
        \end{equation}
        All geometric and matrix-element factors, along with $N_0$ are collapsing into the normal conductance $G_\mathrm{N}$. 
        Given the two electrodes are in thermal equilibrium, effectively collapses the equation to Ohm's law
        \begin{equation}
            I_\mathrm{NN}(V) =  G_\mathrm{N}V\,.
            \label{eq:micro:ohms-law}
        \end{equation}

        In the case of a junction between a normal metal and a superconductor, the tunneling current is given by
        \begin{equation}
            I_\mathrm{NS}(V) = \frac{G_\mathrm{N}}{e} \int_{-\infty}^{\infty} \frac{N_\mathrm{S}(E)}{N_0} \left[f(E) - f(E + eV)\right] \mathrm{d}E\,,
            \label{eq:micro:tunnel-ns}
        \end{equation}
        where $N_\mathrm{S}(E)$ denotes the superconducting quasi-particle density of states and $f(E)$ the Fermi--Dirac distribution. The resulting \textit{I-V} and \textit{dI-dV} curves are shown in Figure~\ref{fig:micro:tunnel-current}. 
        
        The differential conductance then follows as
        \begin{equation}
            \frac{\mathrm{d}I_\mathrm{NS}(V)}{\mathrm{d}V} = G_\mathrm{N} \left[ \frac{N_\mathrm{S}(E)}{N_0} \otimes -\frac{\partial f(E)}{\partial E}\right]_{E=eV}\,,
            \label{eq:micro:tunnel-ns-didv}
        \end{equation}
        showing that the measured \textit{dI-dV} corresponds to the superconducting density of states thermally broadened by the derivative of the Fermi function. This derivative is a symmetric, bell-shaped function whose width scales with temperature. As $T \to 0$, it approaches a delta function, and the convolution becomes negligible. At finite temperature, however, the \textit{dI-dV} smears out, by the convolution. The energy resolution is then limited by thermal broadening, which sets the smallest energy scale over which spectral features can be resolved\footnote{This energy resolution can be approximated as $\mathrm{d} E \approx 3.5\,k_\mathrm{B}T$. In a dilution fridge, one can reach 30\,mK which results in a energy resolution of about 10\,\textmu eV. In a He$_\text{3}$ stick, 300\,mK can be reached, what limits the resolution to about 100\,\textmu eV. Here the temperature refers to the electronic temperature, which may differ from the cryostat base temperature depending on filtering, electron-phonon coupling, and microwave leakage.}.

        Figure~\ref{fig:micro:tunnel-current} displays the effect of temperature and quasi-particle lifetimes on the tunnel current. One can easily observe, that even really small temperatures smears out the coherence peaks drastically. Same holds true for smearing due to quasi-particle lifetimes. However, both parameter affect the tunnel current in distinct ways.
        \begin{figure}
            \centering
            \import{theory/micro}{tunnel-current.pgf}
            \caption{
                Calculated tunnel currents and differential conductances for NS (grey) and SS (blue) junctions within the microscopic tunneling model. Parameters correspond to aluminum with $\Delta_0 = 180\,$\textmu eV and $T_\mathrm{C} = 1.18\,\mathrm{K}$.
                The upper two panels show the effect of temperature $T/T_\mathrm{C}$ at fixed Dynes parameter $\Gamma = 0$. 
                The lower two panels show the influence of Dynes broadening $\Gamma/\Delta_0$ at $T=0$. 
                }
            \label{fig:micro:tunnel-current}
        \end{figure}

        The expressions derived above not only describe the origin of tunneling currents but also provide a direct link to experimental observables. In scanning tunneling microscopy (STM), for example, a normal-metal tip above a superconducting surface forms an NIS junction. Measuring the \textit{I-V} or \textit{dI-dV} characteristics at sufficiently low temperature thus enables a direct mapping of the energy-resolved quasi-particle spectrum of the superconductor. 

        In case of all superconducting tunnel barrier, both electrodes contribute gapped densities of states. Their convolution, together with thermal and lifetime broadening, determines the observed shape of the \textit{I-V} and \textit{dI-dV} curves. 
        \begin{equation}
            I_\mathrm{SS}(V) = \frac{G_\mathrm{N}}{e}\int_{-\infty}^{\infty} \frac{N_\mathrm{S}(E)}{N_0} \cdot \frac{N_\mathrm{S}(E+eV)}{N_0} \cdot \left[f(E) - f(E + eV)\right] \mathrm{d}E
            \label{eq:micro:tunnel-ss}
        \end{equation}
        Importantly, temperature $T$ and Dynes broadening $\Gamma$ influence the spectra in distinct ways. Finite temperature broadens the Fermi edges via the Fermi--Dirac distribution, while $\Gamma$ introduces intrinsic smearing of the quasi-particle DOS itself. 

        In contrast to an NS junction, the SS junction is way more robust to temperature rounding. Instead of smearing, a finite conductance at zero voltage appears, while the height of the coherence peaks decrease as the gap closes. However, the Dynes broadening affects the tunnel current way more than in the NS case.
        
        By fitting measured \textit{I-V} or \textit{dI-dV} data, with both parameters as variables, one can disentangle thermal effects from genuine lifetime or inelastic processes. These interpretations form the foundation of tunneling spectroscopy as a quantitative probe of superconductivity, allowing the extraction of $\Delta_0$, $T$, and $\Gamma$ from experimental data with high accuracy.

        \cite{bardeen_tunnelling_1961,giaever_electron_1960,giaever_study_1961}

    \subsection{Photon-Assisted Tunneling}
    \label{subsec:micro:pat}

        While the previous expressions describe static tunneling, they implicitly assume a time-independent Hamiltonian. Introducing a time-dependent voltage adds a periodic phase to the tunneling matrix element, which is the microscopic origin of photon-assisted tunneling (PAT).

        In contrast to the macroscopic description of phase-coherent Cooper-pair dynamics discussed later in Section~\ref{subsec:macro:shapiro}, the microscopic PAT picture describes incoherent quasi-particle tunneling modified by an external time-dependent voltage. Even in the SS case, PAT remains an incoherent single-particle process.

        In practice, this effect is typically studied using electromagnetic radiation in the microwave range, since photon energies $h\nu$ in this regime are comparable to the superconducting energy gap $\Delta_0$ and can therefore induce measurable sidebands without breaking the Cooper-pairs.

        Possible energies are given by
        \begin{equation}
            E_n = n h\nu\quad (n \in \mathbb{Z})\,,
            \label{eq:micro:pat-En}
        \end{equation}
        where each photon carries an energy $h\nu$ and $n$ is the number of photons absorbed ($n>0$) or emitted ($n<0$). As a result, additional tunneling channels open at these energies, producing characteristic replicas of the coherence peaks in \textit{dI-dV}.

        The Tien--Gordon model provides a simple and intuitive framework to describe this effect. It applies to incoherent single-particle tunneling and does not describe coherent Cooper-pair dynamics such as Josephson (Section~\ref{sec:macro})or MAR (Section~\ref{sec:meso}) processes. It assumes that the junction operates in the tunneling limit, as described before in Section~\ref{subsec:micro:tunnel-current}. The electromagnetic field is treated classically and is represented by a spatially uniform, time-dependent voltage across the junction,
        \begin{equation}
            V(t) = V_0 + A \cos (2\pi\,\nu t)\,.
            \label{eq:micro:pat-V(t)}
        \end{equation}
        Here $V_0$ is the applied voltage bias, $A$ the amplitude, and $\nu$ the frequency of the microwave field across the junction. The field is assumed to remain unaltered by the tunneling current, meaning no cavity or backaction effects, and the drive is weak enough not to heat the electrodes, allowing both to stay in thermal equilibrium.

        In a gauge where the entire voltage drop appears as a scalar potential, a tunneling electron acquires a time-dependent factor in its wavefunction,
        \begin{equation}
            \psi(t) \propto \exp\!\left(-\frac{\ima}{\hbar}\int_0^t e V(t')\, dt' \right)\,.
            \label{eq:micro:pat-wf0}
        \end{equation}

        For the harmonic drive of Eq.~\eqref{eq:micro:pat-V(t)}, the integral can be solved by
        \begin{equation}
            \int_0^t e V(t')\, dt' = eV_0 t + \frac{eA}{2\pi\nu}\sin(2\pi\nu t)\,.
            \label{eq:micro:pat-int}
        \end{equation}
        This equation is further solved by the Jacobi--Anger identity,
        \begin{equation}
            \exp\left(\ima\alpha \sin(2\pi\nu t)\right) = \sum_{n=-\infty}^{\infty} J_n(\alpha)\, \exp\left(\ima n 2\pi\nu t\right)\,,
            \label{eq:micro:pat-jacobi}
        \end{equation}
        where $\alpha = eA/h\nu$ denotes the dimensionless modulation strength and $J_n(\alpha)$ is the $n$-th Bessel function of first kind.

        The wavefunction becomes then
        \begin{equation}
            \psi(t) \propto \sum_{n=-\infty}^{\infty} J_n(\alpha)\, \exp\!\left( -\frac{\ima}{\hbar}(eV_0 - n h\nu)\, t  \right)\,,
            \label{eq:micro:pat-wf}
        \end{equation}
        revealing that an electronic state subjected to an AC voltage becomes a superposition of components at energies shifted by $n h\nu$. Because tunneling probabilities are proportional to the squared amplitude, these channels carry weights $J_n^2(\alpha)$.

        This perspective permits a particularly transparent interpretation in terms of an effective, photon-dressed density of states
        \begin{equation}
            N_\mathrm{PAT}(E) = \sum_{n=-\infty}^{\infty} J_n^2(\alpha)\, N(E + n h\nu)\,.
            \label{eq:micro:pat-dos}
        \end{equation}
        This DOS replaces $N_2(E)$ in the tunnel current expression. The PAT-modified current therefore becomes a weighted sum of shifted copies of the static \textit{I-V} curve.

        Putting these contributions together leads to the Tien--Gordon formula,
        \begin{equation}
            I(V_0) = \sum_{n=-\infty}^{\infty} J_n^2\!\left( \frac{eA}{h\nu}\right) \cdot I_0\!\left(V_0 - \frac{n h\nu}{e}\right)\,.
            \label{eq:micro:pat-tien-gordon}
        \end{equation}
        which expresses the total current as the incoherent sum of all photon-assisted tunneling channels.  Each channel contributes a copy of the static \textit{I-V} curve shifted by $n h\nu/e$ and weighted by the probability $J_n^2(\alpha)$ for absorbing or emitting $n$ photons, producing the characteristic sideband structure observed in experiment.

        It is important to emphasize that the Tien--Gordon mechanism itself is universal and does not distinguish between NS and SS junctions. The AC drive simply produces photon-shifted replicas of the underlying static $I_0(V_0)$ curve, weighted by the Bessel probabilities $J_n^2(\alpha)$. The qualitative differences observed between NS and SS under microwave irradiation arise solely from the different static spectra: in NS junctions only a single gapped DOS contributes, whereas SS junctions involve the convolution of two superconducting DOS. Thus, PAT provides a universal modification of quasiparticle tunneling, independent of the detailed structure of the electrodes. The differences between NS and SS junctions reflect only the underlying static \textit{I-V} curves, not the photon-dressing mechanism itself.

        Figure~\ref{fig:micro:pat} shows \textit{I-V} and \textit{dI-dV} characteristic of a NS and SS junction which is irradiated. 
        \begin{figure}[t]
            \centering
            \import{theory/micro}{pat.pgf}
            \caption{
                Calculated tunnel currents and differential conductances for NS (grey) and SS (blue) junctions under microwave irradiation with different amplitudes $eA/\Delta_0$. 
                Photon absorption and emission create replicas of the coherence peaks spaced by $h\nu$, weighted by the Bessel factors $J_n^2(eA/h\nu)$ according to the Tien--Gordon model. 
                Parameters correspond to aluminum with $\Delta_0 = 180\,$\textmu eV, $T = 0\,$K, $\Gamma = 0\,$\textmu eV and $\nu = 5.0\,$GHz.  
                }
            \label{fig:micro:pat}
        \end{figure}
        In both NS and SS junctions, microwave irradiation leads to photon-assisted replicas of the superconducting coherence peaks.  
        In NS junctions, these sidebands appear as shifted copies of the superconducting DOS, because only a single gapped electrode contributes to the spectrum.  
        In contrast, SS junctions produce sharper and more symmetric replicas, since both electrodes provide gapped DOS that are jointly shifted by the absorbed or emitted photon energy.

        In practice, the microwave amplitude $eA$ can be determined experimentally by comparing the relative heights of the sidebands with the expected Bessel function weights. The spacing $h\nu$ between sidebands provides a direct and robust calibration of the frequency. Photon-assisted tunneling thus offers a straightforward semiclassical description of how an oscillating field modifies quasi-particle transport, serving as a bridge between static tunneling spectroscopy and driven quantum dynamics.

        The microscopic picture developed here describes incoherent single-particle tunneling.In the next sections, we move beyond this regime: first to macroscopic phase-coherent Cooper-pair dynamics (Josephson effect), and then to mesoscopic coherent quasi-particle processes such as Andreev reflection, ABS formation, and MAR.

        \cite{tien_multiphoton_1963}