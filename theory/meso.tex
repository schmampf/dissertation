% !TEX root = ../thesis.tex

%=========================================================
\section{Mesoscopic Description}
\label{sec:meso}
%=========================================================

    In the mesoscopic regime, the electrodes are connected by a small number of quantum channels whose transmission probabilities $\tau_i$ determine the transport properties. This picture departs from both the microscopic description of quasi-particle tunneling and the macroscopic Josephson dynamics of weakly coupled junctions. Conduction is no longer a bulk property, but is governed by the discrete transmission eigenvalues of the available channels.

    The natural starting point is the Landauer approach, originally formulated for phase coherent normal conductors. Each channel is characterized by a transmission probability $\tau_i$, and the normal-state conductance reads
    \begin{equation}
        G_\mathrm{N} = G_0 \sum_{i=1}^N \tau_i\,, \qquad \tau_i \in [0,1]\,,
        \label{eq:meso:landauer}
    \end{equation}
    where $G_0$ is the conductance quantum accounting for spin degeneracy\footnote{Throughout this thesis the conductance quantum is defined as $G_0 = 2e^2/h = 77.48\,\mu\mathrm{S}$. This value includes spin degeneracy and corresponds to the conductance of a fully transmitting normal-state channel. The corresponding resistance quantum is $R_0 = h/(2e^2) = 12.9\,\mathrm{k\Omega}$.}.

    Although the Landauer formula applies strictly to the normal state, the transmission eigenvalues remain well defined in superconducting contacts. For short, coherent junctions, the $\tau_i$ are set by the underlying scattering matrix and are essentially temperature independent.

    These same transmission probabilities govern the amplitude of Andreev reflection, the energies of Andreev bound states, and the entire hierarchy of multiple Andreev reflection processes. The Landauer picture therefore provides the microscopic foundation of mesoscopic superconducting transport.

    %=========================================================
    \subsection{Andreev Reflection}
    \label{subsec:meso:ar}
    %=========================================================

        At a normal-superconductor (N-S) interface, quasi-particles with energies $|E| < \Delta$ cannot enter the superconductor as single electrons because no states exist within the superconducting gap. Instead, an incident electron is retro reflected as a hole, while a Cooper pair carrying charge $2e$ is added to the condensate. This process, known as Andreev reflection (AR), preserves both charge and parallel momentum at the interface and provides the fundamental link between normal-state transport and superconducting correlations.

        Andreev reflection can equivalently be viewed as the coherent transfer of two electrons into the superconductor, where they combine to form a Cooper pair. The probability of AR depends on the interface transparency. It reaches unity for a perfectly transparent contact and decreases when a potential barrier introduces normal reflection. The Andreev approximation applies when the Fermi energies of both materials far exceed the relevant excitation energies ($E, \Delta \ll E_\mathrm{F}$), ensuring that the electron and reflected hole have nearly equal momenta.

        AR is responsible for the finite subgap conductance in N-S junctions and forms the microscopic basis for all superconducting transport in mesoscopic contacts. At a single interface, it transfers one Cooper pair per reflection. When two superconductors are connected, successive Andreev processes give rise to discrete Andreev bound states in equilibrium and to multiple Andreev reflection under finite bias. The following sections build directly on this Andreev-based picture, developing ABS and MAR from the same underlying mechanism.

        To make this qualitative picture quantitative and to calculate the corresponding \textit{I--V} characteristics, we now turn to the Blonder--Tinkham--Klapwijk (BTK) model.

        %=========================================================
        \subsubsection*{Blonder--Tinkham--Klapwijk Model}

        Building on the qualitative picture of Andreev reflection introduced above, the Blonder--Tinkham--Klapwijk (BTK) model provides a quantitative scattering description of an N--S interface with arbitrary transparency. It yields the energy-resolved  probabilities for Andreev reflection ($A$), normal reflection ($B$), and transmission into the superconductor ($C$). The interface is represented by the dimensionless barrier strength $Z$, related to the normal-state transmission by
        \begin{equation}
            Z = \sqrt{1/\tau - 1}\,.
        \end{equation}
        To incorporate finite quasiparticle lifetimes, we use the phenomenological broadening introduced by Pleceník \textit{et al.}, replacing the quasiparticle energy by $E \rightarrow |E| + \ima\gamma$. This yields the modified coherence factors
        \begin{equation}
            u_0^2 = \frac{1}{2} \left( 1 + \frac{\sqrt{(|E|+\ima\gamma)^2-\Delta^2}}{|E|+\ima\gamma}\right)\,,\quad v_0^2 = 1 - u_0^2\,.
        \end{equation}

        Using the normalization factor $d$, the Andreev and normal reflection amplitudes ($a$, $b$) acquire the compact form,
        \begin{equation}
            \begin{array}{ll}
                A(E) = aa^*\,,&
                a= u_0v_0/d\,,\quad
                d = u^2_0+(u^2_0-v^2_0)Z^2\,,\\
                B(E) = bb^*\,,&
                b= -(u_0^2-v_0^2)(Z^2+\ima Z)/d\,.
            \end{array}
            \label{eq:meso:btk-paramter}
        \end{equation}

        For subgap energies, transmission into the quasi-particle continuum vanishes ($C\to0$), so $A(E) + B(E) = 1$. It is therefore convenient to define the energy-resolved spectral weights $\rho$ of the single-particle (1e) and two-particle (2e) processes,
        \begin{align}
            \begin{array}{ll}
            &\rho_\mathrm{1e}(E) = 1 - A(E) - B(E)\,,\\
            &\rho_\mathrm{2e}(E) = 2 \cdot A(E) \,,
            \end{array}
            \label{eq:meso:btk-dos}
        \end{align}
        which quantify the relative importance of normal and Andreev processes. Although these functions resemble densities of states, they should not be interpreted as the physical BCS DOS. Instead, they represent the BTK spectral kernels entering the current integral, shown in Fig.~\ref{fig:meso:btk-dos}.
        \begin{figure}[t]
            \centering
            \import{theory/meso}{btk-dos.pgf}
            \caption{
                BTK spectral weights for the single-particle (1e) and two-particle (2e) channels as a function of energy. The curves illustrate how the interface transparency $\tau$ redistributes spectral weight from normal reflection into Andreev reflection: for low $\tau$, the 1e contribution dominates and the subgap region remains nearly gapped, whereas increasing $\tau$ enhances the 2e channel and fills the subgap region through Andreev processes. These spectral kernels enter directly into the BTK current integral and determine the shape of the resulting \textit{I--V} characteristics.
                \\\textbf{figure in progress...}
                }
            \label{fig:meso:btk-dos}
        \end{figure}

        The current through an N-S junction then follows from the BTK kernel,
        \begin{equation}
            I_\mathrm{NS}(V) = \frac{G_0}{e} \int_{-\infty}^{\infty} \left( \rho_\mathrm{1e}(E) + \rho_\mathrm{2e}(E) \right) \left(f(E) - f(E + eV)\right) \mathrm{d}E\,.
            \label{eq:meso:btk-iv}
        \end{equation}
        yielding the \textit{I--V} and \textit{dI--dV} characteristics in Fig.~\ref{fig:meso:btk-iv}. Increasing the transparency enhances the weight of Andreev processes, leading to a pronounced subgap conductance and a gradual reduction of the coherence-peak height. In the tunneling limit ($\tau \ll 1$), Eq.~\eqref{eq:meso:btk-iv} reduces to the conventional quasiparticle-tunneling expression, while for $\tau \approx 1$ the transport approaches the Andreev limit, where charge is transferred predominantly in units of $2e$.
        \begin{figure}[t]
            \centering
            \import{theory/meso}{btk-current.pgf}
            \caption{
                Analytical \textit{I--V} and \textit{dI--dV} characteristics obtained from the BTK model for various channel transparencies $\tau$. The curves show the continuous evolution from the tunneling regime ($\tau\!\ll\!1$), characterized by sharp coherence peaks and strongly suppressed subgap current, to the high-transparency Andreev limit ($\tau\!\approx\!1$), where pronounced subgap conductance and reduced coherence-peak contrast emerge. Parameters correspond to aluminum (Eq.~\ref{eq:aluminum}), with $T = 0$, and $\gamma = 0$.
                \\\textbf{figure in progress...}
                }
            \label{fig:meso:btk-iv}
        \end{figure}

        
    %=========================================================
    \subsection{Andreev Bound States}
    \label{subsec:meso:abs}
    %=========================================================
    

    \textbf{Ambegaokar--Baratoff, Kulik--Omelyanchuk}

    \begin{align}        
    I_\mathrm{C} &= \frac{\pi \Delta}{2e} \, G_\mathrm{N}.\\
    I_\mathrm{C} R_\mathrm{N} &= \frac{\pi \Delta}{2e}.\\
    I_\mathrm{C}(T) R_\mathrm{N}
&= \frac{\pi \Delta(T)}{2e} \tanh\!\left(\frac{\Delta(T)}{2 k_\mathrm{B} T}\right).\\
I(\varphi) &= \frac{e\Delta}{\hbar}
\frac{\tau \sin\varphi}{\sqrt{1 - \tau \sin^2(\varphi/2)}}
\tanh\!\left(\frac{\Delta}{2k_\mathrm{B}T}
\sqrt{1 - \tau \sin^2(\varphi/2)}\right). \quad \tau \approx 1
    \end{align}

        When two superconductors are connected through a short, phase-coherent constriction, the subgap spectrum is no longer continuous. Instead, coherent electron-hole trajectories become confined between successive Andreev reflections at the two superconducting interfaces. The resulting standing waves form discrete energy levels within the superconducting gap, known as Andreev bound states (ABS). These states constitute the fundamental microscopic origin of both the DC Josephson effect and the non-linear subgap transport observed at finite bias.

        The physical mechanism underlying ABS formation can be understood from the Andreev reflection process introduced in Section~\ref{subsec:meso:ar}. An electron entering the weak link from the left superconductor is retroreflected as a hole at the right interface, transferring a Cooper-pair into the right condensate. The hole then propagates back toward the left interface, where it undergoes a second Andreev reflection, converting back into an electron and transferring a second Cooper-pair to the left condensate. After these two reflections the system has returned to its original quasiparticle character, but the wavefunction has accumulated a phase. If the total phase gained over this round-trip is an integer multiple of $2\pi$, the cycle interferes constructively, yielding a stationary, bound quasiparticle mode.

        This quantization condition contains two contributions: (i) the Andreev reflection phase $\arccos(E/\Delta)$ acquired at each interface and (ii) the superconducting phase difference $\phi = \varphi_\mathrm{L} - \varphi_\mathrm{R}$ between the two electrodes. Solving the resulting phase-quantization condition leads to the characteristic energy--phase relation of a single transport channel with transmission $\tau$,
        \begin{equation}
            E_\pm(\phi)
            = \pm \Delta \sqrt{\,1 - \tau \sin^2\!\left(\frac{\phi}{2}\right)}\,.
            \label{eq:meso:abs-dispersion}
        \end{equation}
        Each channel supports a pair of particle-hole symmetric ABS, labeled by the signs $\pm$. The transmission $\tau$ governs the curvature of these levels: in the tunneling limit ($\tau \ll 1$), the energies remain close to the gap edges ($E \approx \pm \Delta$), whereas for highly transparent channels ($\tau \approx 1$), the lower branch approaches zero energy at $\phi = \pi$. This dispersion encapsulates the essential physics of ballistic superconducting point contacts.

        The bound states directly determine the phase-dependent supercurrent. At zero temperature, the negative branch $E_-(\phi)$ is occupied, and differentiation with respect to the phase yields
        \begin{equation}
            I(\phi)
            = \frac{2e}{\hbar}\,
            \frac{\partial E_-(\phi)}{\partial \phi}
            = \frac{e\Delta}{\hbar}\,
            \frac{\tau \sin\phi}{\sqrt{1 - \tau \sin^2(\phi/2)}}\,,
            \label{eq:meso:abs-current}
        \end{equation}
        which is the current-phase relation (CPR) of a short Josephson junction with a single transport channel. For $\tau \ll 1$, this expression reduces to the sinusoidal CPR known from tunnel junctions. For $\tau \approx 1$, the CPR becomes strongly forward-skewed, reflecting the enhanced susceptibility of the ABS to changes in the superconducting phase.

        Andreev bound states are not only responsible for the equilibrium supercurrent but also govern non-equilibrium transport at finite bias. When a DC voltage is applied across the junction, the superconducting phase evolves according to the Josephson relation,
        \begin{equation}
            \dot{\phi}(t) = \frac{2eV}{\hbar},
        \end{equation}
        rendering the ABS energies time-dependent. The bound states periodically traverse the superconducting gap, repeatedly crossing the continuum edges and allowing quasiparticles to enter or leave the ABS branches. This periodic population transfer gives rise to the highly non-linear subgap current known as multiple Andreev reflection (MAR), which is discussed in Section~\ref{subsec:meso:mar}. In this sense, MAR can be interpreted as the dynamical evolution of the ABS spectrum under a linearly time-varying superconducting phase.

        Finally, ABS play an increasingly important role in spectroscopic experiments on atomic contacts, proximitized nanowires, and Josephson quantum circuits. In these systems, ABS can be probed through microwave spectroscopy, quasiparticle injection, or gate-tunable transparency, and may serve as building blocks for novel qubit architectures. Their sensitivity to both phase and transmission makes them a uniquely versatile probe of mesoscopic superconductivity, bridging microscopic Bogoliubov quasiparticles and macroscopic Josephson dynamics within a single, unified framework.

        \subsubsection*{fractional Shapiro Steps}


    \subsection{Multiple Andreev Reflection}
    \label{subsec:meso:mar}

        When two superconductors are connected through a constriction of atomic dimensions, the quasi-particle transport at subgap voltages is governed by multiple Andreev reflection (MAR). In this regime, a quasi-particle incident on the interface cannot tunnel directly through the gap but undergoes successive Andreev reflections between the two superconducting electrodes. Each reflection converts an electron into a hole (or vice versa) while transferring a Cooper-pair to the condensate, effectively advancing the quasi-particle energy by $eV$ with every traversal of the junction.

        After $m$ such reflections, the quasi-particle gains an energy of $meV$ and can finally escape into the continuum when $meV = 2\Delta$, thus defining the characteristic subharmonic structure in the \textit{I-V} curve. Distinct features appearing at
        \begin{equation}
            eV_m=\frac{2\Delta}{m}\,,\quad(m\in\mathbb{N}^+)\,.
            \label{eq:meso:mar-voltage-onset}
        \end{equation}
        The process probability is given by
        \begin{equation}
            P_m \propto \tau^m\,,
        \end{equation}
        what implies a unique \textit{I-V} characteristics for each transmission. In the tunneling limit ($\tau \ll 1$), the subgap current is weak and dominated by single quasi-particle tunneling ($m=1$). As the transmission increases, higher-order MAR processes become more pronounced, producing a series of peaks in the differential conductance. In the fully transparent limit ($\tau \approx 1$), these discrete features merge into a smooth subgap current approaching the Andreev limit, where transport becomes dominated by successive pair transfers rather than discrete tunneling events.

        In contrast, multiple Andreev reflection cannot occur at a single N-S interface, since the normal electrode provides no second superconducting condensate to sustain repeated electron-hole conversions. After a single Andreev reflection, the reflected hole simply escapes into the normal reservoir instead of being reflected back toward the interface, limiting the process to one conversion event per incident quasi-particle.

        Obtaining the \textit{I-V} characteristics of multiple Andreev reflection (MAR) for arbitrary transmission represents a nontrivial problem, since the transport involves an infinite hierarchy of correlated two-particle processes occurring under nonequilibrium conditions. The first phenomenological descriptions were provided by BTK in 1982 and subsequently by Octavio et al. in 1983. These approaches treated MAR as a sequence of independent Andreev reflections within a semiclassical framework, successfully explaining the appearance of the subharmonic gap structure in the tunneling and weak-coupling limits. However, they relied on rate-equation or transmission-probability arguments and could not describe the full quantum coherence between successive reflections, nor the smooth crossover to the ballistic regime.

        A major theoretical step forward was achieved by Cuevas, Martín-Rodero, and Levy Yeyati (1996, 1998), who developed a fully microscopic theory of MAR based on the nonequilibrium Keldysh Green's-function formalism. Their Hamiltonian approach (HA) treated the applied voltage self-consistently as a time-dependent phase $\phi(t) = \phi_0 + 2eVt/\hbar$, rendering the system periodic in time. By solving this Floquet problem recursively, they obtained stationary solutions for the dc current that naturally include all orders of multiple Andreev reflections and remain valid for any channel transmission. 
        \begin{figure}[t]
            \centering
            \import{theory/meso}{ha-current.pgf}
            \caption{
                Numerical \textit{I-V} and \textit{dI-dV} characteristic calculated with the HA model by Cuevas ($\Delta_0 = 180\,\mu e\mathrm{V}$, $T=0\,\mathrm{K}$, $\gamma = 0$). The curves illustrate the smooth crossover between the tunneling regime ($\tau\ll 1$) and the Andreev limit ($\tau\approx 1$).}
            \label{fig:meso:mar-ha}
        \end{figure}
        
        This formulation provides a continuous description linking the tunneling limit, where transport reduces to single-particle tunneling and reproduces the BCS density of states, with the fully transparent case, where coherent two-particle Andreev reflection dominates and yields a nearly linear subgap current. Intermediate transparencies show the gradual redistribution of spectral weight from the coherence peaks at the gap edge into the subgap region as successive Andreev processes become increasingly likely. This microscopic framework thus unifies the different transport regimes of superconducting point contacts within a single, quantitative model.

        A further conceptual development was introduced through the framework of full counting statistics (FCS), which extends the microscopic MAR theory to include the entire probability distribution of transmitted charge. Instead of describing only the mean current, FCS characterizes the stochastic sequence of charge transfer events by introducing a counting field that tracks the passage of discrete charge quanta during a measurement interval. The resulting cumulant generating function allows the evaluation of all current moments and cumulants, providing access to both the noise spectrum and higher-order correlations. 
        \begin{figure}
            \centering
            \subfigure[$\tau = 0.8$]{\import{theory/meso}{fcs-08.pgf}}
            \subfigure[$\tau = 0.5$]{\import{theory/meso}{fcs-05.pgf}}
            \subfigure[$\tau = 0.3$]{\import{theory/meso}{fcs-03.pgf}}
            \caption{Gesamtbeschriftung der Figur.}
            \label{fig:meso:fcs}
        \end{figure}

        
        Within this picture, each MAR trajectory corresponds to the coherent transfer of a well-defined multiple of the electron charge, and the weight of each process is determined by its transmission dependent amplitude. This approach, pioneered by Belzig, Nazarov, and others, reveals that the subgap current in superconducting contacts is not continuous but built from discrete charge-transfer events whose effective charge increases as the bias is reduced. It thereby extends the Cuevas theory beyond the average current, offering a comprehensive, charge-resolved description of Andreev transport.


    \subsection{Photon-Assisted Multiple Andreev Reflection}
    \label{subsec:meso:pamar}

        Before discussing the effect of microwaves on multiple Andreev reflection (MAR), it is helpful to restate the two key mechanisms on which PAMAR is built. Photon-assisted tunneling (PAT), introduced in Section~\ref{subsec:micro:pat}, arises whenever the applied voltage contains an AC component,
        \begin{equation}
            V(t) = V_{\mathrm{dc}} + A \sin(2\pi\nu t),
            \label{eq:meso:pamar-driving}
        \end{equation}
        which modulates the tunneling phase and generates a ladder of sidebands spaced by the photon energy $h\nu$. \cite{tien_multiphoton_1963}   
        In contrast, MAR (Section~\ref{subsec:meso:mar}) describes the coherent motion of
        a quasiparticle that undergoes $m$ Andreev reflections between two superconductors,
        gaining energy $eV$ on each traversal until $meV = 2\Delta$.\cite{Cuevas1996,Cuevas1998}
        Each MAR trajectory corresponds to the transfer of an effective charge
        \begin{equation}
            q_m = me,
        \end{equation}
        and the subgap current is a weighted sum of these elementary charge-transfer
        processes.

        When an AC voltage is applied, the coherent MAR trajectories remain operative,
        but the time dependence of Eq.~\eqref{eq:meso:pamar-driving} introduces an
        oscillatory phase into each order-$m$ process.  
        For a MAR trajectory transferring the charge $q_m$, the superconducting phase
        difference becomes
        \begin{equation}
            \phi_m(t)
            = \phi_0 + \frac{q_m V_{\mathrm{dc}}}{\hbar}\, t
            - \alpha_m \cos(2\pi\nu t),
            \qquad
            \alpha_m = \frac{q_m A}{h\nu}.
            \label{eq:meso:pamar-phase}
        \end{equation}
        The modulation amplitude $\alpha_m$ increases linearly with the effective
        charge $q_m$, implying that higher-order MAR processes couple more strongly to
        microwave irradiation.  
        This reflects that MAR transfers multiple electron charges in a single coherent
        sequence; the entire string of reflections is phase-modulated as a whole.

        The time-periodic phase renders the Hamiltonian a Floquet problem, as already
        discussed in the microscopic MAR theory of
        Cuevas~\textit{et al.}\cite{Cuevas1996,Cuevas1998}
        Expanding $e^{i\phi_m(t)}$ into its Fourier components gives
        \begin{equation}
            e^{i\phi_m(t)}
            = \sum_{n=-\infty}^{\infty}
                J_n(\alpha_m)
                \exp\!\left[
                    i\left(
                        \frac{q_m V_{\mathrm{dc}}}{\hbar}
                        + n 2\pi\nu
                    \right)t
                \right],
            \label{eq:meso:pamar-floquet}
        \end{equation}
        revealing that each MAR trajectory generates a ladder of photon-dressed sidebands.
        The Bessel function $J_n(\alpha_m)$ is the amplitude to absorb or emit $n$
        photons, and the observable current depends on the probability $J_n^2(\alpha_m)$
        of occupying the corresponding Floquet mode. \cite{tien_multiphoton_1963, kot}  
        This mechanism is identical to PAT but extended to charge-transfer processes of
        arbitrary order $m$.

        Averaging the microscopic current operator over one period of the drive yields
        a charge-resolved generalization of the Tien--Gordon expression,
        \begin{equation}
            I(V_{\mathrm{dc}},A)
            = \sum_{m=1}^{\infty}
                \sum_{n=-\infty}^{\infty}
                    J_n^2(\alpha_m)\,
                    I_m\!\left(
                        V_{\mathrm{dc}}
                        - \frac{n h\nu}{q_m}
                    \right),
            \label{eq:meso:pamar-unified}
        \end{equation}
        where $I_m(V)$ denotes the DC MAR contribution of order $m$ in the absence of
        irradiation.
        Equation~\eqref{eq:meso:pamar-unified} is the \emph{unified, charge-resolved
        Tien--Gordon formula} for superconducting point contacts.  
        It shows that microwaves do not alter the internal structure of MAR---as
        captured by $I_m(V)$---but merely generate shifted replicas of each MAR
        trajectory spaced by $\pm n h\nu/q_m$.  
        The strength of each replica is determined by $\alpha_m$ and thus grows with the
        MAR order.

        The resulting \textit{I--V} characteristics display a hierarchy of
        photon-assisted features at
        \begin{equation}
            eV_{m,n}
            = \frac{2\Delta}{m}
                \pm \frac{n h\nu}{m},
        \end{equation}
        corresponding to the intersections of the MAR thresholds with the photon
        sidebands.
        Changing the microwave frequency controls the spacing of these replicas,
        while changing the amplitude redistributes spectral weight among them.
        All structures occur at finite bias and arise from quasiparticle dynamics,
        distinguishing PAMAR from the Shapiro effect (Section~\ref{subsec:macro:shapiro}),
        which originates from phase locking in the zero-bias Josephson regime and
        follows a different (non-Tien--Gordon) mechanism.

        This framework naturally unifies photon-assisted transport across normal,
        N--S, and S--S junctions.
        PAT corresponds to the $m=1$ single-electron process ($q_1=e$).
        PAAR at an N--S interface arises from the $m=1$ and $m=2$ channels  
        ($q_1=e$, $q_2=2e$).
        PAMAR includes the entire hierarchy $q_m=me$ associated with MAR.
        Thus, the charge-resolved Tien--Gordon formalism provides a single,
        conceptually transparent mechanism describing PAT, PAAR, and PAMAR within the
        same theoretical structure, differing only by the effective charge transferred
        in the underlying microscopic process.