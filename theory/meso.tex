% !TEX root = ../thesis.tex

\section{Mesoscopic Description}
\label{sec:meso}

    The mesoscopic description of superconducting transport bridges the gap between microscopic quasi-particle tunneling and macroscopic Josephson dynamics. In this regime, electrical conduction is no longer determined by bulk properties but by a finite set of quantum channels that connect the two electrodes. According to the Landauer picture, each channel is characterized by a transmission probability $\tau_i$, which quantifies the likelihood that an electron incident from one electrode is transmitted to the other. The total conductance of a contact is then the sum over all available channels, 
    \begin{equation}
        G_\mathrm{N} = G_0 \sum_{i=1}^N \tau_i\,, \quad (\tau_i \in [0,1])\,,
        \label{eq:meso:landauer}
    \end{equation}
    where $G_0 = 2e^2/h$ is the conductance quantum accounting for spin degeneracy. This framework captures the essence of quantum transport in nanostructures and naturally explains phenomena such as conductance quantization in atomic contacts.

    When such mesoscopic conductors are made superconducting, the same transmission channels govern not only single quasi-particle motion but also correlated two quasi-particle transfer through Andreev reflection. The transparency $\tau_i$ of each channel determines the probability of this process and thus directly shapes the subgap current. In junctions where both electrodes are superconducting, successive Andreev reflections on both sides lead to multiple Andreev reflection, producing the characteristic subharmonic structure in the \textit{I-V} characteristics.

    The following sections develop this mesoscopic picture in detail. We begin with Andreev reflection at an N-S interface and extend it to multiple Andreev reflection in S-S contacts, linking these coherent processes to the microscopic transmission probabilities that define the contact's configuration. Finally, we discuss how environmental coupling and charging effects lead to the breakdown of phase coherence, resulting in incoherent Cooper-pair tunneling, thus completing the connection between the macroscopic Josephson regime and the fully mesoscopic quantum limit.


    \subsection{Andreev Reflection}
    \label{subsec:meso:ar}

        At the interface between a normal metal (N) and a superconductor (S), the continuity of the electron wavefunction imposes a unique boundary condition on charge transport. When an electron in a normal metal approaches the interface with an energy smaller than the superconducting energy gap ($|E| < \Delta$), it cannot enter the superconductor as a single quasi-particle because no available states exist within the gap. Instead, the electron is reflected as a hole with opposite momentum and spin, while a Cooper-pair carrying charge $2e$ is transmitted into the superconductor. This process, known as Andreev reflection (AR), ensures charge and momentum conservation across the interface.
        
        Equivalently, Andreev reflection can be viewed as the transmission of two electrons from the normal metal into the superconducting condensate, where they form a Cooper-pair. Historically, AR is often described in terms of hole reflection. However, I will use the convention of two electrons are transmitted, since it is more intuitive.

        In the simplest picture, the two incident electrons are phase-correlated; they form the two components of a time-reversed pair. The amplitude of Andreev reflection depends on the transparency of the interface and is maximal for a perfectly clean contact. For a barrier of finite strength, normal reflection competes with Andreev reflection, reducing the probability of Cooper-pair transfer.

        The Andreev reflection picture is valid under several key assumptions regarding the materials and interface. This description assumes that both materials are well described by the mean-field BCS theory, that the Fermi energies in N and S are much larger than the excitation energies ($E, \Delta \ll E_\mathrm{F}$), and that the interface can be treated within the Andreev approximation, where the momenta of the two electrons forming a Cooper-pair differ only slightly from the Fermi momentum.

        This gives rise to a finite conductance at subgap voltages, even when quasi-particle tunneling is forbidden. The probability of Andreev reflection can be derived from the Bogoliubov--de Gennes equations and depends on both the voltage bias $eV$ and the interface transparency $\tau$. Andreev reflection is thus the fundamental process linking normal-metal transport to superconducting correlations.

        Analytical descriptions of the \textit{I-V} characteristics based on the Blonder--Tinkham--Klapwijk (BTK) model provide a quantitative framework for calculating the Andreev reflection probabilities. In its modified form proposed by Pleceník et al., finite quasi-particle lifetimes are included phenomenologically by introducing a complex quasi-particle energy $E \rightarrow |E| + \mathrm{i}\Gamma$. The dimensionless barrier strength $Z$ related to the transmission $\tau$ is used. The resulting coherence factors $u_0^2$ and $v_0^2$ are given in the following,
        \begin{equation}
            \begin{array}{ll}
                u_0^2 = \frac{1}{2} \left( 1 + \frac{\sqrt{(|E|+\ima\Gamma)^2-\Delta^2}}{|E|+\ima\Gamma}\right)\,,&
                v_0^2 = 1 - u_0^2\,\\
                \gamma = u^2_0+(u^2_0-v^2_0)Z^2\,,& 
                Z = \sqrt{1/\tau - 1}\,,\\
                a= u_0v_0/\gamma\,,&
                A(E) = aa^*\,,\\
                b= -(u_0^2-v_0^2)(Z^2+\ima Z)/\gamma\,,&
                B(E) = bb^*\,.
            \end{array}
            \label{eq:meso:btk-paramter}
        \end{equation}

        The normalized superconducting density of states can be expressed in terms of these coherence factors as
        \begin{equation}
            N_\mathrm{S}(E) = \Re\!\left((u_0^2-v_0^2)^{-1}\right)\,.
            \label{eq:meso:ar-dos}
        \end{equation}
        
        With these quantities, the total current through an NS contact decomposes into single-particle and Andreev contributions,
        \begin{align}
            I_\mathrm{1e}(V) &= \frac{G_0}{e} \int_{-\infty}^{\infty} \left( 1 - A(E) - B(E) \right) \left[f(E) - f(E + eV)\right] \mathrm{d}E\,,
            \label{eq:meso:ar-1e}\\
            I_\mathrm{2e}(V) &= \frac{G_0}{e} \int_{-\infty}^{\infty} 2 A(E) \left[f(E) - f(E + eV)\right] \mathrm{d}E\,,
            \label{eq:meso:ar-2e}\\
            I_\mathrm{NS}(V) &= I_\mathrm{1e}(V) + I_\mathrm{2e}(V)\,.
            \label{eq:meso:ar-IV}
        \end{align}
        This formulation smoothly connects the tunneling and Andreev limits and reproduces the Dynes-broadened density of states for small transmissions ($\tau \ll 1$). 

        Figure~\ref{fig:meso:ar-iv} illustrates the resulting \textit{I-V} and \textit{dI-dV} characteristics obtained from the analytical BTK formulation, showing how the subgap conductance and coherence peaks evolves with interface transparency.
        \begin{figure}
            \centering
            \import{theory/meso}{ar-iv.pgf}
            \caption{
                Analytical \textit{I-V} and \textit{dI-dV} characteristic calculated within the BTK model ($\Delta_0 = 180\,\mu e\mathrm{V}$, $T=0\,\mathrm{K}$, $\Gamma = 0$). The curves illustrate the smooth crossover between the tunneling regime ($\tau\ll 1$) and the Andreev limit ($\tau\approx 1$).  Subgap conductance increases with transparency due to the growing probability of two-particle Andreev processes, while the height of the coherence peaks at $eV = \pm \Delta_0$ decreases.}
            \label{fig:meso:ar-iv}
        \end{figure}

        While Andreev reflection is most easily visualized at an N-S interface, it also governs charge transfer between two superconductors. In this case, repeated Andreev processes on both sides of the junction lead to multiple Andreev reflection, discribed in the following.


    \subsection{Andreev Bound States}
    \label{subsec:meso:abs}

        When two superconductors are connected through a short, phase-coherent constriction, the subgap spectrum is no longer continuous. Instead, coherent electron-hole trajectories become confined between successive Andreev reflections at the two superconducting interfaces. The resulting standing waves form discrete energy levels within the superconducting gap, known as Andreev bound states (ABS). These states constitute the fundamental microscopic origin of both the DC Josephson effect and the non-linear subgap transport observed at finite bias.

        The physical mechanism underlying ABS formation can be understood from the Andreev reflection process introduced in Section~\ref{subsec:meso:ar}. An electron entering the weak link from the left superconductor is retroreflected as a hole at the right interface, transferring a Cooper-pair into the right condensate. The hole then propagates back toward the left interface, where it undergoes a second Andreev reflection, converting back into an electron and transferring a second Cooper-pair to the left condensate. After these two reflections the system has returned to its original quasiparticle character, but the wavefunction has accumulated a phase. If the total phase gained over this round-trip is an integer multiple of $2\pi$, the cycle interferes constructively, yielding a stationary, bound quasiparticle mode.

        This quantization condition contains two contributions: (i) the Andreev reflection phase $\arccos(E/\Delta)$ acquired at each interface and (ii) the superconducting phase difference $\phi = \varphi_\mathrm{L} - \varphi_\mathrm{R}$ between the two electrodes. Solving the resulting phase-quantization condition leads to the characteristic energy--phase relation of a single transport channel with transmission $\tau$,
        \begin{equation}
            E_\pm(\phi)
            = \pm \Delta \sqrt{\,1 - \tau \sin^2\!\left(\frac{\phi}{2}\right)}\,.
            \label{eq:meso:abs-dispersion}
        \end{equation}
        Each channel supports a pair of particle-hole symmetric ABS, labeled by the signs $\pm$. The transmission $\tau$ governs the curvature of these levels: in the tunneling limit ($\tau \ll 1$), the energies remain close to the gap edges ($E \approx \pm \Delta$), whereas for highly transparent channels ($\tau \approx 1$), the lower branch approaches zero energy at $\phi = \pi$. This dispersion encapsulates the essential physics of ballistic superconducting point contacts.

        The bound states directly determine the phase-dependent supercurrent. At zero temperature, the negative branch $E_-(\phi)$ is occupied, and differentiation with respect to the phase yields
        \begin{equation}
            I(\phi)
            = \frac{2e}{\hbar}\,
            \frac{\partial E_-(\phi)}{\partial \phi}
            = \frac{e\Delta}{\hbar}\,
            \frac{\tau \sin\phi}{\sqrt{1 - \tau \sin^2(\phi/2)}}\,,
            \label{eq:meso:abs-current}
        \end{equation}
        which is the current-phase relation (CPR) of a short Josephson junction with a single transport channel. For $\tau \ll 1$, this expression reduces to the sinusoidal CPR known from tunnel junctions. For $\tau \approx 1$, the CPR becomes strongly forward-skewed, reflecting the enhanced susceptibility of the ABS to changes in the superconducting phase.

        Andreev bound states are not only responsible for the equilibrium supercurrent but also govern non-equilibrium transport at finite bias. When a DC voltage is applied across the junction, the superconducting phase evolves according to the Josephson relation,
        \begin{equation}
            \dot{\phi}(t) = \frac{2eV}{\hbar},
        \end{equation}
        rendering the ABS energies time-dependent. The bound states periodically traverse the superconducting gap, repeatedly crossing the continuum edges and allowing quasiparticles to enter or leave the ABS branches. This periodic population transfer gives rise to the highly non-linear subgap current known as multiple Andreev reflection (MAR), which is discussed in Section~\ref{subsec:meso:mar}. In this sense, MAR can be interpreted as the dynamical evolution of the ABS spectrum under a linearly time-varying superconducting phase.

        Finally, ABS play an increasingly important role in spectroscopic experiments on atomic contacts, proximitized nanowires, and Josephson quantum circuits. In these systems, ABS can be probed through microwave spectroscopy, quasiparticle injection, or gate-tunable transparency, and may serve as building blocks for novel qubit architectures. Their sensitivity to both phase and transmission makes them a uniquely versatile probe of mesoscopic superconductivity, bridging microscopic Bogoliubov quasiparticles and macroscopic Josephson dynamics within a single, unified framework.

        \subsubsection*{fractional Shapiro Steps}


    \subsection{Multiple Andreev Reflection}
    \label{subsec:meso:mar}

        When two superconductors are connected through a constriction of atomic dimensions, the quasi-particle transport at subgap voltages is governed by multiple Andreev reflection (MAR). In this regime, a quasi-particle incident on the interface cannot tunnel directly through the gap but undergoes successive Andreev reflections between the two superconducting electrodes. Each reflection converts an electron into a hole (or vice versa) while transferring a Cooper-pair to the condensate, effectively advancing the quasi-particle energy by $eV$ with every traversal of the junction.

        After $m$ such reflections, the quasi-particle gains an energy of $meV$ and can finally escape into the continuum when $meV = 2\Delta$, thus defining the characteristic subharmonic structure in the \textit{I-V} curve. Distinct features appearing at
        \begin{equation}
            eV_m=\frac{2\Delta}{m}\,,\quad(m\in\mathbb{N}^+)\,.
            \label{eq:meso:mar-voltage-onset}
        \end{equation}
        The process probability is given by
        \begin{equation}
            P_m \propto \tau^m\,,
        \end{equation}
        what implies a unique \textit{I-V} characteristics for each transmission. In the tunneling limit ($\tau \ll 1$), the subgap current is weak and dominated by single quasi-particle tunneling ($m=1$). As the transmission increases, higher-order MAR processes become more pronounced, producing a series of peaks in the differential conductance. In the fully transparent limit ($\tau \approx 1$), these discrete features merge into a smooth subgap current approaching the Andreev limit, where transport becomes dominated by successive pair transfers rather than discrete tunneling events.

        In contrast, multiple Andreev reflection cannot occur at a single N-S interface, since the normal electrode provides no second superconducting condensate to sustain repeated electron-hole conversions. After a single Andreev reflection, the reflected hole simply escapes into the normal reservoir instead of being reflected back toward the interface, limiting the process to one conversion event per incident quasi-particle.

        Obtaining the \textit{I-V} characteristics of multiple Andreev reflection (MAR) for arbitrary transmission represents a nontrivial problem, since the transport involves an infinite hierarchy of correlated two-particle processes occurring under nonequilibrium conditions. The first phenomenological descriptions were provided by BTK in 1982 and subsequently by Octavio et al. in 1983. These approaches treated MAR as a sequence of independent Andreev reflections within a semiclassical framework, successfully explaining the appearance of the subharmonic gap structure in the tunneling and weak-coupling limits. However, they relied on rate-equation or transmission-probability arguments and could not describe the full quantum coherence between successive reflections, nor the smooth crossover to the ballistic regime.

        A major theoretical step forward was achieved by Cuevas, Martín-Rodero, and Levy Yeyati (1996, 1998), who developed a fully microscopic theory of MAR based on the nonequilibrium Keldysh Green's-function formalism. Their Hamiltonian approach (HA) treated the applied voltage self-consistently as a time-dependent phase $\phi(t) = \phi_0 + 2eVt/\hbar$, rendering the system periodic in time. By solving this Floquet problem recursively, they obtained stationary solutions for the dc current that naturally include all orders of multiple Andreev reflections and remain valid for any channel transmission. 
        
        This formulation provides a continuous description linking the tunneling limit, where transport reduces to single-particle tunneling and reproduces the BCS density of states, with the fully transparent case, where coherent two-particle Andreev reflection dominates and yields a nearly linear subgap current. Intermediate transparencies show the gradual redistribution of spectral weight from the coherence peaks at the gap edge into the subgap region as successive Andreev processes become increasingly likely. This microscopic framework thus unifies the different transport regimes of superconducting point contacts within a single, quantitative model.

        A further conceptual development was introduced through the framework of full counting statistics (FCS), which extends the microscopic MAR theory to include the entire probability distribution of transmitted charge. Instead of describing only the mean current, FCS characterizes the stochastic sequence of charge transfer events by introducing a counting field that tracks the passage of discrete charge quanta during a measurement interval. The resulting cumulant generating function allows the evaluation of all current moments and cumulants, providing access to both the noise spectrum and higher-order correlations. 
        
        Within this picture, each MAR trajectory corresponds to the coherent transfer of a well-defined multiple of the electron charge, and the weight of each process is determined by its transmission dependent amplitude. This approach, pioneered by Belzig, Nazarov, and others, reveals that the subgap current in superconducting contacts is not continuous but built from discrete charge-transfer events whose effective charge increases as the bias is reduced. It thereby extends the Cuevas theory beyond the average current, offering a comprehensive, charge-resolved description of Andreev transport.

        \begin{figure}[t]
            \centering
            \import{theory/meso}{mar-iv.pgf}
            \caption{
                Numerical \textit{I-V} and \textit{dI-dV} characteristic calculated with the HA model by Cuevas ($\Delta_0 = 180\,\mu e\mathrm{V}$, $T=0\,\mathrm{K}$, $\Gamma = 0$). The curves illustrate the smooth crossover between the tunneling regime ($\tau\ll 1$) and the Andreev limit ($\tau\approx 1$).}
            \label{fig:meso:mar-iv}
        \end{figure}



    \subsection{Photon-Assisted Multiple Andreev Reflection}
    \label{subsec:meso:pamar}

        Before discussing the effect of microwaves on multiple Andreev reflection (MAR), it is helpful to restate the two key mechanisms on which PAMAR is built. Photon-assisted tunneling (PAT), introduced in Section~\ref{subsec:micro:pat}, arises whenever the applied voltage contains an AC component,
        \begin{equation}
            V(t) = V_{\mathrm{dc}} + A \sin(2\pi\nu t),
            \label{eq:meso:pamar-driving}
        \end{equation}
        which modulates the tunneling phase and generates a ladder of sidebands spaced by the photon energy $h\nu$. \cite{tien_multiphoton_1963}   
        In contrast, MAR (Section~\ref{subsec:meso:mar}) describes the coherent motion of
        a quasiparticle that undergoes $m$ Andreev reflections between two superconductors,
        gaining energy $eV$ on each traversal until $meV = 2\Delta$.\cite{Cuevas1996,Cuevas1998}
        Each MAR trajectory corresponds to the transfer of an effective charge
        \begin{equation}
            q_m = me,
        \end{equation}
        and the subgap current is a weighted sum of these elementary charge-transfer
        processes.

        When an AC voltage is applied, the coherent MAR trajectories remain operative,
        but the time dependence of Eq.~\eqref{eq:meso:pamar-driving} introduces an
        oscillatory phase into each order-$m$ process.  
        For a MAR trajectory transferring the charge $q_m$, the superconducting phase
        difference becomes
        \begin{equation}
            \phi_m(t)
            = \phi_0 + \frac{q_m V_{\mathrm{dc}}}{\hbar}\, t
            - \alpha_m \cos(2\pi\nu t),
            \qquad
            \alpha_m = \frac{q_m A}{h\nu}.
            \label{eq:meso:pamar-phase}
        \end{equation}
        The modulation amplitude $\alpha_m$ increases linearly with the effective
        charge $q_m$, implying that higher-order MAR processes couple more strongly to
        microwave irradiation.  
        This reflects that MAR transfers multiple electron charges in a single coherent
        sequence; the entire string of reflections is phase-modulated as a whole.

        The time-periodic phase renders the Hamiltonian a Floquet problem, as already
        discussed in the microscopic MAR theory of
        Cuevas~\textit{et al.}\cite{Cuevas1996,Cuevas1998}
        Expanding $e^{i\phi_m(t)}$ into its Fourier components gives
        \begin{equation}
            e^{i\phi_m(t)}
            = \sum_{n=-\infty}^{\infty}
                J_n(\alpha_m)
                \exp\!\left[
                    i\left(
                        \frac{q_m V_{\mathrm{dc}}}{\hbar}
                        + n 2\pi\nu
                    \right)t
                \right],
            \label{eq:meso:pamar-floquet}
        \end{equation}
        revealing that each MAR trajectory generates a ladder of photon-dressed sidebands.
        The Bessel function $J_n(\alpha_m)$ is the amplitude to absorb or emit $n$
        photons, and the observable current depends on the probability $J_n^2(\alpha_m)$
        of occupying the corresponding Floquet mode. \cite{tien_multiphoton_1963, kot}  
        This mechanism is identical to PAT but extended to charge-transfer processes of
        arbitrary order $m$.

        Averaging the microscopic current operator over one period of the drive yields
        a charge-resolved generalization of the Tien--Gordon expression,
        \begin{equation}
            I(V_{\mathrm{dc}},A)
            = \sum_{m=1}^{\infty}
                \sum_{n=-\infty}^{\infty}
                    J_n^2(\alpha_m)\,
                    I_m\!\left(
                        V_{\mathrm{dc}}
                        - \frac{n h\nu}{q_m}
                    \right),
            \label{eq:meso:pamar-unified}
        \end{equation}
        where $I_m(V)$ denotes the DC MAR contribution of order $m$ in the absence of
        irradiation.
        Equation~\eqref{eq:meso:pamar-unified} is the \emph{unified, charge-resolved
        Tien--Gordon formula} for superconducting point contacts.  
        It shows that microwaves do not alter the internal structure of MAR---as
        captured by $I_m(V)$---but merely generate shifted replicas of each MAR
        trajectory spaced by $\pm n h\nu/q_m$.  
        The strength of each replica is determined by $\alpha_m$ and thus grows with the
        MAR order.

        The resulting \textit{I--V} characteristics display a hierarchy of
        photon-assisted features at
        \begin{equation}
            eV_{m,n}
            = \frac{2\Delta}{m}
                \pm \frac{n h\nu}{m},
        \end{equation}
        corresponding to the intersections of the MAR thresholds with the photon
        sidebands.
        Changing the microwave frequency controls the spacing of these replicas,
        while changing the amplitude redistributes spectral weight among them.
        All structures occur at finite bias and arise from quasiparticle dynamics,
        distinguishing PAMAR from the Shapiro effect (Section~\ref{subsec:macro:shapiro}),
        which originates from phase locking in the zero-bias Josephson regime and
        follows a different (non-Tien--Gordon) mechanism.

        This framework naturally unifies photon-assisted transport across normal,
        N--S, and S--S junctions.
        PAT corresponds to the $m=1$ single-electron process ($q_1=e$).
        PAAR at an N--S interface arises from the $m=1$ and $m=2$ channels  
        ($q_1=e$, $q_2=2e$).
        PAMAR includes the entire hierarchy $q_m=me$ associated with MAR.
        Thus, the charge-resolved Tien--Gordon formalism provides a single,
        conceptually transparent mechanism describing PAT, PAAR, and PAMAR within the
        same theoretical structure, differing only by the effective charge transferred
        in the underlying microscopic process.