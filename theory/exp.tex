
            In real circuits, there is no clear current or voltage bias. The jump in voltage from switching current to the quasi-particle branch is perfect horizontal just for perfect current bias. For a perfect voltage bias, the current would jump to zero, upon reaching the switching current, and would follow the quasi-particle branch from there. In any dissipative bias conditions, the slope is finite and influenced by the ratio of sample resistance and bias resistance.

            In real measurements, neither perfect current nor perfect voltage bias exists. The wiring, filters, and source electronics present a finite impedance, which determines how the junction’s switching manifests in the measured \textit{I–V}. Perfect current bias produces a horizontal voltage jump at switching, while perfect voltage bias gives a vertical current drop. Real setups interpolate between these limits, resulting in characteristic tilt of the switching branch.
