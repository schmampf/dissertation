% !TEX root = ../thesis.tex

\section{Microscopic Description}
\label{sec:bcs}

    The microscopic description of superconductivity is provided by the Bardeen-Cooper-Schrieffer (BCS) theory, which assumes that electrons near the Fermi surface form bound pairs of opposite momentum and spin \cite{bardeen_microscopic_1957}.

    At its heart, this pairing mechanism reflects a subtle interplay between electrons and phonons. When an electron moves through the metal, it slightly displaces the positively charged ions, creating a momentary region of excess positive charge. A second electron passing nearby can be attracted to this distortion, leading to an effective-though very weak-attraction between the two. Below the critical temperature $T_\mathrm{C}$, these paired electrons form a collective quantum state known as the Cooper-pair condensate, where all pairs share a common phase. 

    The emergence of the superconducting gap $\Delta_0$ can be understood as the energy required to break a Cooper pair and excite its electrons back into normal quasiparticle states. 

    The following subsections outline the key aspects of this microscopic picture. 
    First, the temperature-dependent superconducting gap $\Delta(T)$ is introduced, describing how the collective pairing strength evolves with thermal excitation. 
    Second, the resulting modification of the electronic density of states is derived, highlighting how the superconducting energy gap reorganizes the quasiparticle spectrum around the Fermi level. 
    Third, the microscopic expressions for tunneling current are discussed, establishing the direct connection between these theoretical quantities and experimentally measurable $I(V)$ and $\mathrm{d}I/\mathrm{d}V(V)$ characteristics. 
    Finally, the extension to photon-assisted tunneling is presented, where an oscillating electromagnetic field enables quasi-particles to exchange discrete energy quanta during tunneling, linking the static picture to driven superconducting transport.

    \subsection{Superconducting Gap}
    \label{subsec:bcs:sc-gap}
        For conventional superconductors such as aluminum, the superconducting gap is very small compared to the Fermi energy. This regime is known as the weak-coupling limit. It implies that only a thin shell of electronic states around the Fermi surface participates in pairing, allowing the normal-state density of states $N_0$ to be treated as constant. Although the interaction is weak, the collective nature of the pairing amplifies its effect, transforming a small microscopic attraction into a macroscopic quantum order parameter. In this thesis, aluminum is well described within this weak-coupling framework, which relates $\Delta_0$ at zero temperature to $T_\mathrm{C}$ by

        \begin{equation}
            \Delta_0 \approx 1.764\, k_\mathrm{B} T_\mathrm{C}\,.
            \label{eq:Delta0}
        \end{equation}

        The superconducting gap does not remain constant with temperature. At absolute zero, all available electrons near the Fermi surface form Cooper pairs, and the condensate is perfectly ordered. As the temperature rises, thermal excitations begin to break some of these pairs, leaving fewer electrons bound in the superconducting state. With fewer pairs contributing to the collective order, the overall pairing strength weakens, and the energy required to break a pair the gap $\Delta(T)$ gradually decreases.

        This reduction continues smoothly until the critical temperature $T_\mathrm{C}$ is reached. At that point, thermal energy becomes strong enough to completely disrupt the pairing correlations, and the superconducting state collapses, leading to $\Delta(T_\mathrm{C}) = 0$. The temperature dependence of the gap is a direct reflection of this balance between thermal disorder and the pairing interaction.

        \begin{wrapfigure}[12]{r}{0.4\textwidth}
            \centering
            \vspace{-1em} % fine-tune vertical position
            %% Creator: Matplotlib, PGF backend
%%
%% To include the figure in your LaTeX document, write
%%   \input{<filename>.pgf}
%%
%% Make sure the required packages are loaded in your preamble
%%   \usepackage{pgf}
%%
%% Also ensure that all the required font packages are loaded; for instance,
%% the lmodern package is sometimes necessary when using math font.
%%   \usepackage{lmodern}
%%
%% Figures using additional raster images can only be included by \input if
%% they are in the same directory as the main LaTeX file. For loading figures
%% from other directories you can use the `import` package
%%   \usepackage{import}
%%
%% and then include the figures with
%%   \import{<path to file>}{<filename>.pgf}
%%
%% Matplotlib used the following preamble
%%   \def\mathdefault#1{#1}
%%   \everymath=\expandafter{\the\everymath\displaystyle}
%%   \IfFileExists{scrextend.sty}{
%%     \usepackage[fontsize=9.000000pt]{scrextend}
%%   }{
%%     \renewcommand{\normalsize}{\fontsize{9.000000}{10.800000}\selectfont}
%%     \normalsize
%%   }
%%   
%%   \ifdefined\pdftexversion\else  % non-pdftex case.
%%     \usepackage{fontspec}
%%     \setmainfont{DejaVuSerif.ttf}[Path=\detokenize{/Users/oliver/.pyenv/versions/3.13.3/lib/python3.13/site-packages/matplotlib/mpl-data/fonts/ttf/}]
%%     \setsansfont{DejaVuSans.ttf}[Path=\detokenize{/Users/oliver/.pyenv/versions/3.13.3/lib/python3.13/site-packages/matplotlib/mpl-data/fonts/ttf/}]
%%     \setmonofont{DejaVuSansMono.ttf}[Path=\detokenize{/Users/oliver/.pyenv/versions/3.13.3/lib/python3.13/site-packages/matplotlib/mpl-data/fonts/ttf/}]
%%   \fi
%%   \makeatletter\@ifpackageloaded{underscore}{}{\usepackage[strings]{underscore}}\makeatother
%%
\begingroup%
\makeatletter%
\begin{pgfpicture}%
\pgfpathrectangle{\pgfpointorigin}{\pgfqpoint{1.500000in}{1.200000in}}%
\pgfusepath{use as bounding box, clip}%
\begin{pgfscope}%
\pgfsetbuttcap%
\pgfsetmiterjoin%
\definecolor{currentfill}{rgb}{1.000000,1.000000,1.000000}%
\pgfsetfillcolor{currentfill}%
\pgfsetlinewidth{0.000000pt}%
\definecolor{currentstroke}{rgb}{1.000000,1.000000,1.000000}%
\pgfsetstrokecolor{currentstroke}%
\pgfsetdash{}{0pt}%
\pgfpathmoveto{\pgfqpoint{0.000000in}{0.000000in}}%
\pgfpathlineto{\pgfqpoint{1.500000in}{0.000000in}}%
\pgfpathlineto{\pgfqpoint{1.500000in}{1.200000in}}%
\pgfpathlineto{\pgfqpoint{0.000000in}{1.200000in}}%
\pgfpathlineto{\pgfqpoint{0.000000in}{0.000000in}}%
\pgfpathclose%
\pgfusepath{fill}%
\end{pgfscope}%
\begin{pgfscope}%
\pgfsetbuttcap%
\pgfsetmiterjoin%
\definecolor{currentfill}{rgb}{1.000000,1.000000,1.000000}%
\pgfsetfillcolor{currentfill}%
\pgfsetlinewidth{0.000000pt}%
\definecolor{currentstroke}{rgb}{0.000000,0.000000,0.000000}%
\pgfsetstrokecolor{currentstroke}%
\pgfsetstrokeopacity{0.000000}%
\pgfsetdash{}{0pt}%
\pgfpathmoveto{\pgfqpoint{0.388083in}{0.240778in}}%
\pgfpathlineto{\pgfqpoint{1.386562in}{0.240778in}}%
\pgfpathlineto{\pgfqpoint{1.386562in}{1.145833in}}%
\pgfpathlineto{\pgfqpoint{0.388083in}{1.145833in}}%
\pgfpathlineto{\pgfqpoint{0.388083in}{0.240778in}}%
\pgfpathclose%
\pgfusepath{fill}%
\end{pgfscope}%
\begin{pgfscope}%
\pgfsetbuttcap%
\pgfsetroundjoin%
\definecolor{currentfill}{rgb}{0.000000,0.000000,0.000000}%
\pgfsetfillcolor{currentfill}%
\pgfsetlinewidth{0.803000pt}%
\definecolor{currentstroke}{rgb}{0.000000,0.000000,0.000000}%
\pgfsetstrokecolor{currentstroke}%
\pgfsetdash{}{0pt}%
\pgfsys@defobject{currentmarker}{\pgfqpoint{0.000000in}{-0.041667in}}{\pgfqpoint{0.000000in}{0.000000in}}{%
\pgfpathmoveto{\pgfqpoint{0.000000in}{0.000000in}}%
\pgfpathlineto{\pgfqpoint{0.000000in}{-0.041667in}}%
\pgfusepath{stroke,fill}%
}%
\begin{pgfscope}%
\pgfsys@transformshift{0.388083in}{0.240778in}%
\pgfsys@useobject{currentmarker}{}%
\end{pgfscope}%
\end{pgfscope}%
\begin{pgfscope}%
\definecolor{textcolor}{rgb}{0.000000,0.000000,0.000000}%
\pgfsetstrokecolor{textcolor}%
\pgfsetfillcolor{textcolor}%
\pgftext[x=0.388083in,y=0.150500in,,top]{\color{textcolor}{\sffamily\fontsize{8.000000}{9.600000}\selectfont\catcode`\^=\active\def^{\ifmmode\sp\else\^{}\fi}\catcode`\%=\active\def%{\%}$0$}}%
\end{pgfscope}%
\begin{pgfscope}%
\pgfsetbuttcap%
\pgfsetroundjoin%
\definecolor{currentfill}{rgb}{0.000000,0.000000,0.000000}%
\pgfsetfillcolor{currentfill}%
\pgfsetlinewidth{0.803000pt}%
\definecolor{currentstroke}{rgb}{0.000000,0.000000,0.000000}%
\pgfsetstrokecolor{currentstroke}%
\pgfsetdash{}{0pt}%
\pgfsys@defobject{currentmarker}{\pgfqpoint{0.000000in}{-0.041667in}}{\pgfqpoint{0.000000in}{0.000000in}}{%
\pgfpathmoveto{\pgfqpoint{0.000000in}{0.000000in}}%
\pgfpathlineto{\pgfqpoint{0.000000in}{-0.041667in}}%
\pgfusepath{stroke,fill}%
}%
\begin{pgfscope}%
\pgfsys@transformshift{1.156144in}{0.240778in}%
\pgfsys@useobject{currentmarker}{}%
\end{pgfscope}%
\end{pgfscope}%
\begin{pgfscope}%
\definecolor{textcolor}{rgb}{0.000000,0.000000,0.000000}%
\pgfsetstrokecolor{textcolor}%
\pgfsetfillcolor{textcolor}%
\pgftext[x=1.156144in,y=0.150500in,,top]{\color{textcolor}{\sffamily\fontsize{8.000000}{9.600000}\selectfont\catcode`\^=\active\def^{\ifmmode\sp\else\^{}\fi}\catcode`\%=\active\def%{\%}$T_\mathrm{C}$}}%
\end{pgfscope}%
\begin{pgfscope}%
\pgfsetbuttcap%
\pgfsetroundjoin%
\definecolor{currentfill}{rgb}{0.000000,0.000000,0.000000}%
\pgfsetfillcolor{currentfill}%
\pgfsetlinewidth{0.803000pt}%
\definecolor{currentstroke}{rgb}{0.000000,0.000000,0.000000}%
\pgfsetstrokecolor{currentstroke}%
\pgfsetdash{}{0pt}%
\pgfsys@defobject{currentmarker}{\pgfqpoint{-0.041667in}{0.000000in}}{\pgfqpoint{-0.000000in}{0.000000in}}{%
\pgfpathmoveto{\pgfqpoint{-0.000000in}{0.000000in}}%
\pgfpathlineto{\pgfqpoint{-0.041667in}{0.000000in}}%
\pgfusepath{stroke,fill}%
}%
\begin{pgfscope}%
\pgfsys@transformshift{0.388083in}{0.240778in}%
\pgfsys@useobject{currentmarker}{}%
\end{pgfscope}%
\end{pgfscope}%
\begin{pgfscope}%
\definecolor{textcolor}{rgb}{0.000000,0.000000,0.000000}%
\pgfsetstrokecolor{textcolor}%
\pgfsetfillcolor{textcolor}%
\pgftext[x=0.238777in, y=0.198569in, left, base]{\color{textcolor}{\sffamily\fontsize{8.000000}{9.600000}\selectfont\catcode`\^=\active\def^{\ifmmode\sp\else\^{}\fi}\catcode`\%=\active\def%{\%}$0$}}%
\end{pgfscope}%
\begin{pgfscope}%
\pgfsetbuttcap%
\pgfsetroundjoin%
\definecolor{currentfill}{rgb}{0.000000,0.000000,0.000000}%
\pgfsetfillcolor{currentfill}%
\pgfsetlinewidth{0.803000pt}%
\definecolor{currentstroke}{rgb}{0.000000,0.000000,0.000000}%
\pgfsetstrokecolor{currentstroke}%
\pgfsetdash{}{0pt}%
\pgfsys@defobject{currentmarker}{\pgfqpoint{-0.041667in}{0.000000in}}{\pgfqpoint{-0.000000in}{0.000000in}}{%
\pgfpathmoveto{\pgfqpoint{-0.000000in}{0.000000in}}%
\pgfpathlineto{\pgfqpoint{-0.041667in}{0.000000in}}%
\pgfusepath{stroke,fill}%
}%
\begin{pgfscope}%
\pgfsys@transformshift{0.388083in}{0.994991in}%
\pgfsys@useobject{currentmarker}{}%
\end{pgfscope}%
\end{pgfscope}%
\begin{pgfscope}%
\definecolor{textcolor}{rgb}{0.000000,0.000000,0.000000}%
\pgfsetstrokecolor{textcolor}%
\pgfsetfillcolor{textcolor}%
\pgftext[x=0.238777in, y=0.952782in, left, base]{\color{textcolor}{\sffamily\fontsize{8.000000}{9.600000}\selectfont\catcode`\^=\active\def^{\ifmmode\sp\else\^{}\fi}\catcode`\%=\active\def%{\%}$1$}}%
\end{pgfscope}%
\begin{pgfscope}%
\definecolor{textcolor}{rgb}{0.000000,0.000000,0.000000}%
\pgfsetstrokecolor{textcolor}%
\pgfsetfillcolor{textcolor}%
\pgftext[x=0.183221in,y=0.693306in,,bottom,rotate=90.000000]{\color{textcolor}{\sffamily\fontsize{8.000000}{9.600000}\selectfont\catcode`\^=\active\def^{\ifmmode\sp\else\^{}\fi}\catcode`\%=\active\def%{\%}$\Delta(T)$ ($\Delta_0$)}}%
\end{pgfscope}%
\begin{pgfscope}%
\pgfpathrectangle{\pgfqpoint{0.388083in}{0.240778in}}{\pgfqpoint{0.998479in}{0.905056in}}%
\pgfusepath{clip}%
\pgfsetrectcap%
\pgfsetroundjoin%
\pgfsetlinewidth{1.505625pt}%
\definecolor{currentstroke}{rgb}{0.347656,0.777344,0.917969}%
\pgfsetstrokecolor{currentstroke}%
\pgfsetdash{}{0pt}%
\pgfpathmoveto{\pgfqpoint{0.388083in}{0.994991in}}%
\pgfpathlineto{\pgfqpoint{0.395764in}{0.994991in}}%
\pgfpathlineto{\pgfqpoint{0.403445in}{0.994991in}}%
\pgfpathlineto{\pgfqpoint{0.411125in}{0.994991in}}%
\pgfpathlineto{\pgfqpoint{0.418806in}{0.994991in}}%
\pgfpathlineto{\pgfqpoint{0.426486in}{0.994990in}}%
\pgfpathlineto{\pgfqpoint{0.434167in}{0.994989in}}%
\pgfpathlineto{\pgfqpoint{0.441848in}{0.994986in}}%
\pgfpathlineto{\pgfqpoint{0.449528in}{0.994979in}}%
\pgfpathlineto{\pgfqpoint{0.457209in}{0.994967in}}%
\pgfpathlineto{\pgfqpoint{0.464889in}{0.994947in}}%
\pgfpathlineto{\pgfqpoint{0.472570in}{0.994915in}}%
\pgfpathlineto{\pgfqpoint{0.480251in}{0.994869in}}%
\pgfpathlineto{\pgfqpoint{0.487931in}{0.994805in}}%
\pgfpathlineto{\pgfqpoint{0.495612in}{0.994720in}}%
\pgfpathlineto{\pgfqpoint{0.503292in}{0.994610in}}%
\pgfpathlineto{\pgfqpoint{0.510973in}{0.994471in}}%
\pgfpathlineto{\pgfqpoint{0.518654in}{0.994301in}}%
\pgfpathlineto{\pgfqpoint{0.526334in}{0.994094in}}%
\pgfpathlineto{\pgfqpoint{0.534015in}{0.993849in}}%
\pgfpathlineto{\pgfqpoint{0.541696in}{0.993560in}}%
\pgfpathlineto{\pgfqpoint{0.549376in}{0.993226in}}%
\pgfpathlineto{\pgfqpoint{0.557057in}{0.992842in}}%
\pgfpathlineto{\pgfqpoint{0.564737in}{0.992406in}}%
\pgfpathlineto{\pgfqpoint{0.572418in}{0.991913in}}%
\pgfpathlineto{\pgfqpoint{0.580099in}{0.991362in}}%
\pgfpathlineto{\pgfqpoint{0.587779in}{0.990749in}}%
\pgfpathlineto{\pgfqpoint{0.595460in}{0.990070in}}%
\pgfpathlineto{\pgfqpoint{0.603140in}{0.989324in}}%
\pgfpathlineto{\pgfqpoint{0.610821in}{0.988507in}}%
\pgfpathlineto{\pgfqpoint{0.618502in}{0.987615in}}%
\pgfpathlineto{\pgfqpoint{0.626182in}{0.986648in}}%
\pgfpathlineto{\pgfqpoint{0.633863in}{0.985601in}}%
\pgfpathlineto{\pgfqpoint{0.641543in}{0.984471in}}%
\pgfpathlineto{\pgfqpoint{0.649224in}{0.983257in}}%
\pgfpathlineto{\pgfqpoint{0.656905in}{0.981955in}}%
\pgfpathlineto{\pgfqpoint{0.664585in}{0.980562in}}%
\pgfpathlineto{\pgfqpoint{0.672266in}{0.979076in}}%
\pgfpathlineto{\pgfqpoint{0.679946in}{0.977494in}}%
\pgfpathlineto{\pgfqpoint{0.687627in}{0.975812in}}%
\pgfpathlineto{\pgfqpoint{0.695308in}{0.974029in}}%
\pgfpathlineto{\pgfqpoint{0.702988in}{0.972140in}}%
\pgfpathlineto{\pgfqpoint{0.710669in}{0.970144in}}%
\pgfpathlineto{\pgfqpoint{0.718350in}{0.968037in}}%
\pgfpathlineto{\pgfqpoint{0.726030in}{0.965815in}}%
\pgfpathlineto{\pgfqpoint{0.733711in}{0.963477in}}%
\pgfpathlineto{\pgfqpoint{0.741391in}{0.961018in}}%
\pgfpathlineto{\pgfqpoint{0.749072in}{0.958435in}}%
\pgfpathlineto{\pgfqpoint{0.756753in}{0.955725in}}%
\pgfpathlineto{\pgfqpoint{0.764433in}{0.952884in}}%
\pgfpathlineto{\pgfqpoint{0.772114in}{0.949909in}}%
\pgfpathlineto{\pgfqpoint{0.779794in}{0.946795in}}%
\pgfpathlineto{\pgfqpoint{0.787475in}{0.943539in}}%
\pgfpathlineto{\pgfqpoint{0.795156in}{0.940137in}}%
\pgfpathlineto{\pgfqpoint{0.802836in}{0.936584in}}%
\pgfpathlineto{\pgfqpoint{0.810517in}{0.932875in}}%
\pgfpathlineto{\pgfqpoint{0.818197in}{0.929007in}}%
\pgfpathlineto{\pgfqpoint{0.825878in}{0.924975in}}%
\pgfpathlineto{\pgfqpoint{0.833559in}{0.920772in}}%
\pgfpathlineto{\pgfqpoint{0.841239in}{0.916394in}}%
\pgfpathlineto{\pgfqpoint{0.848920in}{0.911836in}}%
\pgfpathlineto{\pgfqpoint{0.856600in}{0.907090in}}%
\pgfpathlineto{\pgfqpoint{0.864281in}{0.902151in}}%
\pgfpathlineto{\pgfqpoint{0.871962in}{0.897013in}}%
\pgfpathlineto{\pgfqpoint{0.879642in}{0.891667in}}%
\pgfpathlineto{\pgfqpoint{0.887323in}{0.886107in}}%
\pgfpathlineto{\pgfqpoint{0.895004in}{0.880325in}}%
\pgfpathlineto{\pgfqpoint{0.902684in}{0.874311in}}%
\pgfpathlineto{\pgfqpoint{0.910365in}{0.868058in}}%
\pgfpathlineto{\pgfqpoint{0.918045in}{0.861554in}}%
\pgfpathlineto{\pgfqpoint{0.925726in}{0.854790in}}%
\pgfpathlineto{\pgfqpoint{0.933407in}{0.847754in}}%
\pgfpathlineto{\pgfqpoint{0.941087in}{0.840434in}}%
\pgfpathlineto{\pgfqpoint{0.948768in}{0.832817in}}%
\pgfpathlineto{\pgfqpoint{0.956448in}{0.824887in}}%
\pgfpathlineto{\pgfqpoint{0.964129in}{0.816631in}}%
\pgfpathlineto{\pgfqpoint{0.971810in}{0.808029in}}%
\pgfpathlineto{\pgfqpoint{0.979490in}{0.799063in}}%
\pgfpathlineto{\pgfqpoint{0.987171in}{0.789713in}}%
\pgfpathlineto{\pgfqpoint{0.994851in}{0.779955in}}%
\pgfpathlineto{\pgfqpoint{1.002532in}{0.769763in}}%
\pgfpathlineto{\pgfqpoint{1.010213in}{0.759109in}}%
\pgfpathlineto{\pgfqpoint{1.017893in}{0.747959in}}%
\pgfpathlineto{\pgfqpoint{1.025574in}{0.736276in}}%
\pgfpathlineto{\pgfqpoint{1.033254in}{0.724019in}}%
\pgfpathlineto{\pgfqpoint{1.040935in}{0.711138in}}%
\pgfpathlineto{\pgfqpoint{1.048616in}{0.697577in}}%
\pgfpathlineto{\pgfqpoint{1.056296in}{0.683269in}}%
\pgfpathlineto{\pgfqpoint{1.063977in}{0.668134in}}%
\pgfpathlineto{\pgfqpoint{1.071658in}{0.652076in}}%
\pgfpathlineto{\pgfqpoint{1.079338in}{0.634979in}}%
\pgfpathlineto{\pgfqpoint{1.087019in}{0.616694in}}%
\pgfpathlineto{\pgfqpoint{1.094699in}{0.597034in}}%
\pgfpathlineto{\pgfqpoint{1.102380in}{0.575751in}}%
\pgfpathlineto{\pgfqpoint{1.110061in}{0.552506in}}%
\pgfpathlineto{\pgfqpoint{1.117741in}{0.526813in}}%
\pgfpathlineto{\pgfqpoint{1.125422in}{0.497933in}}%
\pgfpathlineto{\pgfqpoint{1.133102in}{0.464625in}}%
\pgfpathlineto{\pgfqpoint{1.140783in}{0.424485in}}%
\pgfpathlineto{\pgfqpoint{1.148464in}{0.371344in}}%
\pgfpathlineto{\pgfqpoint{1.156144in}{0.240778in}}%
\pgfusepath{stroke}%
\end{pgfscope}%
\begin{pgfscope}%
\pgfsetbuttcap%
\pgfsetmiterjoin%
\definecolor{currentfill}{rgb}{0.000000,0.000000,0.000000}%
\pgfsetfillcolor{currentfill}%
\pgfsetlinewidth{1.003750pt}%
\definecolor{currentstroke}{rgb}{0.000000,0.000000,0.000000}%
\pgfsetstrokecolor{currentstroke}%
\pgfsetdash{}{0pt}%
\pgfsys@defobject{currentmarker}{\pgfqpoint{-0.041667in}{-0.041667in}}{\pgfqpoint{0.041667in}{0.041667in}}{%
\pgfpathmoveto{\pgfqpoint{0.041667in}{-0.000000in}}%
\pgfpathlineto{\pgfqpoint{-0.041667in}{0.041667in}}%
\pgfpathlineto{\pgfqpoint{-0.041667in}{-0.041667in}}%
\pgfpathlineto{\pgfqpoint{0.041667in}{-0.000000in}}%
\pgfpathclose%
\pgfusepath{stroke,fill}%
}%
\begin{pgfscope}%
\pgfsys@transformshift{1.386562in}{0.240778in}%
\pgfsys@useobject{currentmarker}{}%
\end{pgfscope}%
\end{pgfscope}%
\begin{pgfscope}%
\pgfsetbuttcap%
\pgfsetmiterjoin%
\definecolor{currentfill}{rgb}{0.000000,0.000000,0.000000}%
\pgfsetfillcolor{currentfill}%
\pgfsetlinewidth{1.003750pt}%
\definecolor{currentstroke}{rgb}{0.000000,0.000000,0.000000}%
\pgfsetstrokecolor{currentstroke}%
\pgfsetdash{}{0pt}%
\pgfsys@defobject{currentmarker}{\pgfqpoint{-0.041667in}{-0.041667in}}{\pgfqpoint{0.041667in}{0.041667in}}{%
\pgfpathmoveto{\pgfqpoint{0.000000in}{0.041667in}}%
\pgfpathlineto{\pgfqpoint{-0.041667in}{-0.041667in}}%
\pgfpathlineto{\pgfqpoint{0.041667in}{-0.041667in}}%
\pgfpathlineto{\pgfqpoint{0.000000in}{0.041667in}}%
\pgfpathclose%
\pgfusepath{stroke,fill}%
}%
\begin{pgfscope}%
\pgfsys@transformshift{0.388083in}{1.145833in}%
\pgfsys@useobject{currentmarker}{}%
\end{pgfscope}%
\end{pgfscope}%
\begin{pgfscope}%
\pgfsetrectcap%
\pgfsetmiterjoin%
\pgfsetlinewidth{0.803000pt}%
\definecolor{currentstroke}{rgb}{0.000000,0.000000,0.000000}%
\pgfsetstrokecolor{currentstroke}%
\pgfsetdash{}{0pt}%
\pgfpathmoveto{\pgfqpoint{0.388083in}{0.240778in}}%
\pgfpathlineto{\pgfqpoint{0.388083in}{1.145833in}}%
\pgfusepath{stroke}%
\end{pgfscope}%
\begin{pgfscope}%
\pgfsetrectcap%
\pgfsetmiterjoin%
\pgfsetlinewidth{0.803000pt}%
\definecolor{currentstroke}{rgb}{0.000000,0.000000,0.000000}%
\pgfsetstrokecolor{currentstroke}%
\pgfsetdash{}{0pt}%
\pgfpathmoveto{\pgfqpoint{0.388083in}{0.240778in}}%
\pgfpathlineto{\pgfqpoint{1.386562in}{0.240778in}}%
\pgfusepath{stroke}%
\end{pgfscope}%
\begin{pgfscope}%
\definecolor{textcolor}{rgb}{0.000000,0.000000,0.000000}%
\pgfsetstrokecolor{textcolor}%
\pgfsetfillcolor{textcolor}%
\pgftext[x=1.386562in,y=0.089935in,left,]{\color{textcolor}{\sffamily\fontsize{8.000000}{9.600000}\selectfont\catcode`\^=\active\def^{\ifmmode\sp\else\^{}\fi}\catcode`\%=\active\def%{\%}$T$}}%
\end{pgfscope}%
\end{pgfpicture}%
\makeatother%
\endgroup%

            \caption{Temperature dependence of the superconducting gap $\Delta(T)$.}
            \label{fig:bcs:gap_suppression}
        \end{wrapfigure} 
        In the BCS framework, $\Delta(T)$ follows a universal curve that results from solving the self-consistent gap equation, meaning the same functional form applies to all weak-coupling superconductors. 
        Solving the underlying integrals over the Fermi distribution in the microscopic theory numerically, results in the following equation or shown in Figure \ref{fig:bcs:gap_suppression}.
        \begin{equation}
            \frac{\Delta(T)}{\Delta_0} \approx \tanh\left(1.74\,\sqrt{\frac{T_\mathrm{C}}{T}-1}\right)
            \label{eq:DeltaT}
        \end{equation}


    \subsection{Density of States}
    \label{subsec:bcs:dos}
        In the following, all energies are expressed relative to the Fermi energy $E_\mathrm{F}$, such that $E=0$ corresponds to the Fermi level around which superconducting correlations develop.

        In a normal metal, the density of states (DOS) can be approximated as constant near the Fermi level. Although the DOS of a free-electron gas increases as $\sqrt{E}$, this variation is negligible over the tiny energy range relevant to superconductivity\footnote{a few m$e$V compared to $E_\mathrm{F}$ of several $e$V}. We therefore treat it as a constant 
        \begin{equation}
            N_0 \equiv  N_\mathrm{N}(E_\mathrm{F}) = \frac{1}{2\pi^2} \left( \frac{2m}{\hbar^2}\right)^{3/2} \sqrt{E_\mathrm{F}}\,,
            \label{eq:DOS-N0}
        \end{equation}
        representing the normal-state DOS per spin at the Fermi energy. 

        When superconductivity sets in, pairing correlations reorganize this otherwise flat spectrum. A gap of width $2\Delta$ opens around $E_\mathrm{F}$ where single-particle excitations are absent in the ideal BCS limit, and the missing spectral weight is redistributed to the gap edges. These edges appear as sharp coherence peaks in the superconducting DOS, reflecting the high density of available quasiparticle states at the threshold for pair breaking. The resulting expression reads
        \begin{equation}
            \frac{N_\mathrm{S}(E)}{N_0} = 
                \left\{
                \begin{array}{@{}r@{\quad}l@{}}
                    0 & (|E| < \Delta)\\
                    \dfrac{|E|}{\sqrt{E^2-\Delta^2}} & (|E| \ge \Delta)
                \end{array}
                \right.\,.
            \label{eq:DOS-BCS}
        \end{equation}
        Thus, in the ideal BCS limit the quasiparticle DOS is strictly zero within the energy gap and diverges at its edges, $E=\pm\Delta$. At finite temperature, the gap magnitude $\Delta(T)$ decreases and the corresponding features of $N_\mathrm{S}(E)$ shift inward, as discussed previously.

        However, real spectra are never perfectly sharp. A simple and very effective phenomenology is the Dynes broadening, implemented by the substitution $E\to E+i\Gamma$, while just considering the real part of the whole expression
        \begin{equation}
            \frac{N_\mathrm{S}(E)}{N_0} = \Re\!\left(\frac{E+i\Gamma}{\sqrt{(E+i\Gamma)^2-\Delta^2}}\right) \quad (|E| \ge \Delta)\,.
            \label{eq:DOS-Dynes}
        \end{equation}
        Such broadening arises from finite quasiparticle lifetimes due to inelastic scattering, spatial inhomogeneity, pair breaking by magnetic impurities, or non-equilibrium effects, all of which smear the ideal BCS singularities.

        Whereas the DOS specifies where states exist, the Fermi--Dirac distribution encodes how single-particle states are occupied at a given temperature and chemical potential. In the context of metals and conventional superconductors treated in this thesis, the chemical potential can be identified with the Fermi energy to very good approximation, $\mu\approx E_\mathrm{F}$, because thermal corrections are small on the scale of $e$V. 

        In equilibrium the occupation probability of a state at energy $E$ is given by        
        \begin{equation}
            f(E) = \frac{1}{1+\exp\left(\frac{E}{k_\mathrm{B}T}\right)}\,.
            \label{eq:fermidirac}
        \end{equation}        
        At zero temperature the distribution reduces to a sharp step $ \theta(E)$.

        However, all states below the Fermi function are filled and all states above are empty. At finite temperature the step is thermally broadened over an energy scale of order $k_\mathrm{B}T$.

        \begin{figure}
            \centering
            %% Creator: Matplotlib, PGF backend
%%
%% To include the figure in your LaTeX document, write
%%   \input{<filename>.pgf}
%%
%% Make sure the required packages are loaded in your preamble
%%   \usepackage{pgf}
%%
%% Also ensure that all the required font packages are loaded; for instance,
%% the lmodern package is sometimes necessary when using math font.
%%   \usepackage{lmodern}
%%
%% Figures using additional raster images can only be included by \input if
%% they are in the same directory as the main LaTeX file. For loading figures
%% from other directories you can use the `import` package
%%   \usepackage{import}
%%
%% and then include the figures with
%%   \import{<path to file>}{<filename>.pgf}
%%
%% Matplotlib used the following preamble
%%   \def\mathdefault#1{#1}
%%   \everymath=\expandafter{\the\everymath\displaystyle}
%%   \IfFileExists{scrextend.sty}{
%%     \usepackage[fontsize=10.000000pt]{scrextend}
%%   }{
%%     \renewcommand{\normalsize}{\fontsize{10.000000}{12.000000}\selectfont}
%%     \normalsize
%%   }
%%   
%%   \ifdefined\pdftexversion\else  % non-pdftex case.
%%     \usepackage{fontspec}
%%     \setmainfont{DejaVuSerif.ttf}[Path=\detokenize{/Users/oliver/.pyenv/versions/3.13.3/lib/python3.13/site-packages/matplotlib/mpl-data/fonts/ttf/}]
%%     \setsansfont{DejaVuSans.ttf}[Path=\detokenize{/Users/oliver/.pyenv/versions/3.13.3/lib/python3.13/site-packages/matplotlib/mpl-data/fonts/ttf/}]
%%     \setmonofont{DejaVuSansMono.ttf}[Path=\detokenize{/Users/oliver/.pyenv/versions/3.13.3/lib/python3.13/site-packages/matplotlib/mpl-data/fonts/ttf/}]
%%   \fi
%%   \makeatletter\@ifpackageloaded{underscore}{}{\usepackage[strings]{underscore}}\makeatother
%%
\begingroup%
\makeatletter%
\begin{pgfpicture}%
\pgfpathrectangle{\pgfpointorigin}{\pgfqpoint{3.600000in}{2.000000in}}%
\pgfusepath{use as bounding box, clip}%
\begin{pgfscope}%
\pgfsetbuttcap%
\pgfsetmiterjoin%
\definecolor{currentfill}{rgb}{1.000000,1.000000,1.000000}%
\pgfsetfillcolor{currentfill}%
\pgfsetlinewidth{0.000000pt}%
\definecolor{currentstroke}{rgb}{1.000000,1.000000,1.000000}%
\pgfsetstrokecolor{currentstroke}%
\pgfsetdash{}{0pt}%
\pgfpathmoveto{\pgfqpoint{0.000000in}{0.000000in}}%
\pgfpathlineto{\pgfqpoint{3.600000in}{0.000000in}}%
\pgfpathlineto{\pgfqpoint{3.600000in}{2.000000in}}%
\pgfpathlineto{\pgfqpoint{0.000000in}{2.000000in}}%
\pgfpathlineto{\pgfqpoint{0.000000in}{0.000000in}}%
\pgfpathclose%
\pgfusepath{fill}%
\end{pgfscope}%
\begin{pgfscope}%
\pgfsetbuttcap%
\pgfsetmiterjoin%
\definecolor{currentfill}{rgb}{1.000000,1.000000,1.000000}%
\pgfsetfillcolor{currentfill}%
\pgfsetlinewidth{0.000000pt}%
\definecolor{currentstroke}{rgb}{0.000000,0.000000,0.000000}%
\pgfsetstrokecolor{currentstroke}%
\pgfsetstrokeopacity{0.000000}%
\pgfsetdash{}{0pt}%
\pgfpathmoveto{\pgfqpoint{0.554278in}{1.419683in}}%
\pgfpathlineto{\pgfqpoint{3.450000in}{1.419683in}}%
\pgfpathlineto{\pgfqpoint{3.450000in}{1.836184in}}%
\pgfpathlineto{\pgfqpoint{0.554278in}{1.836184in}}%
\pgfpathlineto{\pgfqpoint{0.554278in}{1.419683in}}%
\pgfpathclose%
\pgfusepath{fill}%
\end{pgfscope}%
\begin{pgfscope}%
\pgfsetbuttcap%
\pgfsetroundjoin%
\definecolor{currentfill}{rgb}{0.000000,0.000000,0.000000}%
\pgfsetfillcolor{currentfill}%
\pgfsetlinewidth{0.803000pt}%
\definecolor{currentstroke}{rgb}{0.000000,0.000000,0.000000}%
\pgfsetstrokecolor{currentstroke}%
\pgfsetdash{}{0pt}%
\pgfsys@defobject{currentmarker}{\pgfqpoint{0.000000in}{-0.048611in}}{\pgfqpoint{0.000000in}{0.000000in}}{%
\pgfpathmoveto{\pgfqpoint{0.000000in}{0.000000in}}%
\pgfpathlineto{\pgfqpoint{0.000000in}{-0.048611in}}%
\pgfusepath{stroke,fill}%
}%
\begin{pgfscope}%
\pgfsys@transformshift{1.124647in}{1.419683in}%
\pgfsys@useobject{currentmarker}{}%
\end{pgfscope}%
\end{pgfscope}%
\begin{pgfscope}%
\definecolor{textcolor}{rgb}{0.000000,0.000000,0.000000}%
\pgfsetstrokecolor{textcolor}%
\pgfsetfillcolor{textcolor}%
\pgftext[x=1.124647in,y=1.322461in,,top]{\color{textcolor}{\sffamily\fontsize{10.000000}{12.000000}\selectfont\catcode`\^=\active\def^{\ifmmode\sp\else\^{}\fi}\catcode`\%=\active\def%{\%}\ensuremath{-}2}}%
\end{pgfscope}%
\begin{pgfscope}%
\pgfsetbuttcap%
\pgfsetroundjoin%
\definecolor{currentfill}{rgb}{0.000000,0.000000,0.000000}%
\pgfsetfillcolor{currentfill}%
\pgfsetlinewidth{0.803000pt}%
\definecolor{currentstroke}{rgb}{0.000000,0.000000,0.000000}%
\pgfsetstrokecolor{currentstroke}%
\pgfsetdash{}{0pt}%
\pgfsys@defobject{currentmarker}{\pgfqpoint{0.000000in}{-0.048611in}}{\pgfqpoint{0.000000in}{0.000000in}}{%
\pgfpathmoveto{\pgfqpoint{0.000000in}{0.000000in}}%
\pgfpathlineto{\pgfqpoint{0.000000in}{-0.048611in}}%
\pgfusepath{stroke,fill}%
}%
\begin{pgfscope}%
\pgfsys@transformshift{2.002139in}{1.419683in}%
\pgfsys@useobject{currentmarker}{}%
\end{pgfscope}%
\end{pgfscope}%
\begin{pgfscope}%
\definecolor{textcolor}{rgb}{0.000000,0.000000,0.000000}%
\pgfsetstrokecolor{textcolor}%
\pgfsetfillcolor{textcolor}%
\pgftext[x=2.002139in,y=1.322461in,,top]{\color{textcolor}{\sffamily\fontsize{10.000000}{12.000000}\selectfont\catcode`\^=\active\def^{\ifmmode\sp\else\^{}\fi}\catcode`\%=\active\def%{\%}0}}%
\end{pgfscope}%
\begin{pgfscope}%
\pgfsetbuttcap%
\pgfsetroundjoin%
\definecolor{currentfill}{rgb}{0.000000,0.000000,0.000000}%
\pgfsetfillcolor{currentfill}%
\pgfsetlinewidth{0.803000pt}%
\definecolor{currentstroke}{rgb}{0.000000,0.000000,0.000000}%
\pgfsetstrokecolor{currentstroke}%
\pgfsetdash{}{0pt}%
\pgfsys@defobject{currentmarker}{\pgfqpoint{0.000000in}{-0.048611in}}{\pgfqpoint{0.000000in}{0.000000in}}{%
\pgfpathmoveto{\pgfqpoint{0.000000in}{0.000000in}}%
\pgfpathlineto{\pgfqpoint{0.000000in}{-0.048611in}}%
\pgfusepath{stroke,fill}%
}%
\begin{pgfscope}%
\pgfsys@transformshift{2.879630in}{1.419683in}%
\pgfsys@useobject{currentmarker}{}%
\end{pgfscope}%
\end{pgfscope}%
\begin{pgfscope}%
\definecolor{textcolor}{rgb}{0.000000,0.000000,0.000000}%
\pgfsetstrokecolor{textcolor}%
\pgfsetfillcolor{textcolor}%
\pgftext[x=2.879630in,y=1.322461in,,top]{\color{textcolor}{\sffamily\fontsize{10.000000}{12.000000}\selectfont\catcode`\^=\active\def^{\ifmmode\sp\else\^{}\fi}\catcode`\%=\active\def%{\%}2}}%
\end{pgfscope}%
\begin{pgfscope}%
\pgfsetbuttcap%
\pgfsetroundjoin%
\definecolor{currentfill}{rgb}{0.000000,0.000000,0.000000}%
\pgfsetfillcolor{currentfill}%
\pgfsetlinewidth{0.803000pt}%
\definecolor{currentstroke}{rgb}{0.000000,0.000000,0.000000}%
\pgfsetstrokecolor{currentstroke}%
\pgfsetdash{}{0pt}%
\pgfsys@defobject{currentmarker}{\pgfqpoint{-0.048611in}{0.000000in}}{\pgfqpoint{-0.000000in}{0.000000in}}{%
\pgfpathmoveto{\pgfqpoint{-0.000000in}{0.000000in}}%
\pgfpathlineto{\pgfqpoint{-0.048611in}{0.000000in}}%
\pgfusepath{stroke,fill}%
}%
\begin{pgfscope}%
\pgfsys@transformshift{0.554278in}{1.438615in}%
\pgfsys@useobject{currentmarker}{}%
\end{pgfscope}%
\end{pgfscope}%
\begin{pgfscope}%
\definecolor{textcolor}{rgb}{0.000000,0.000000,0.000000}%
\pgfsetstrokecolor{textcolor}%
\pgfsetfillcolor{textcolor}%
\pgftext[x=0.368690in, y=1.385853in, left, base]{\color{textcolor}{\sffamily\fontsize{10.000000}{12.000000}\selectfont\catcode`\^=\active\def^{\ifmmode\sp\else\^{}\fi}\catcode`\%=\active\def%{\%}0}}%
\end{pgfscope}%
\begin{pgfscope}%
\pgfsetbuttcap%
\pgfsetroundjoin%
\definecolor{currentfill}{rgb}{0.000000,0.000000,0.000000}%
\pgfsetfillcolor{currentfill}%
\pgfsetlinewidth{0.803000pt}%
\definecolor{currentstroke}{rgb}{0.000000,0.000000,0.000000}%
\pgfsetstrokecolor{currentstroke}%
\pgfsetdash{}{0pt}%
\pgfsys@defobject{currentmarker}{\pgfqpoint{-0.048611in}{0.000000in}}{\pgfqpoint{-0.000000in}{0.000000in}}{%
\pgfpathmoveto{\pgfqpoint{-0.000000in}{0.000000in}}%
\pgfpathlineto{\pgfqpoint{-0.048611in}{0.000000in}}%
\pgfusepath{stroke,fill}%
}%
\begin{pgfscope}%
\pgfsys@transformshift{0.554278in}{1.811680in}%
\pgfsys@useobject{currentmarker}{}%
\end{pgfscope}%
\end{pgfscope}%
\begin{pgfscope}%
\definecolor{textcolor}{rgb}{0.000000,0.000000,0.000000}%
\pgfsetstrokecolor{textcolor}%
\pgfsetfillcolor{textcolor}%
\pgftext[x=0.368690in, y=1.758918in, left, base]{\color{textcolor}{\sffamily\fontsize{10.000000}{12.000000}\selectfont\catcode`\^=\active\def^{\ifmmode\sp\else\^{}\fi}\catcode`\%=\active\def%{\%}5}}%
\end{pgfscope}%
\begin{pgfscope}%
\definecolor{textcolor}{rgb}{0.000000,0.000000,0.000000}%
\pgfsetstrokecolor{textcolor}%
\pgfsetfillcolor{textcolor}%
\pgftext[x=0.313135in,y=1.627934in,,bottom,rotate=90.000000]{\color{textcolor}{\sffamily\fontsize{10.000000}{12.000000}\selectfont\catcode`\^=\active\def^{\ifmmode\sp\else\^{}\fi}\catcode`\%=\active\def%{\%}$N_\mathrm{S}/N_0$}}%
\end{pgfscope}%
\begin{pgfscope}%
\pgfpathrectangle{\pgfqpoint{0.554278in}{1.419683in}}{\pgfqpoint{2.895722in}{0.416501in}}%
\pgfusepath{clip}%
\pgfsetrectcap%
\pgfsetroundjoin%
\pgfsetlinewidth{1.505625pt}%
\definecolor{currentstroke}{rgb}{0.121569,0.466667,0.705882}%
\pgfsetstrokecolor{currentstroke}%
\pgfsetdash{}{0pt}%
\pgfpathmoveto{\pgfqpoint{0.685902in}{1.517754in}}%
\pgfpathlineto{\pgfqpoint{0.712226in}{1.517959in}}%
\pgfpathlineto{\pgfqpoint{0.738551in}{1.518178in}}%
\pgfpathlineto{\pgfqpoint{0.764876in}{1.518414in}}%
\pgfpathlineto{\pgfqpoint{0.791201in}{1.518667in}}%
\pgfpathlineto{\pgfqpoint{0.817525in}{1.518940in}}%
\pgfpathlineto{\pgfqpoint{0.843850in}{1.519235in}}%
\pgfpathlineto{\pgfqpoint{0.870175in}{1.519555in}}%
\pgfpathlineto{\pgfqpoint{0.896499in}{1.519902in}}%
\pgfpathlineto{\pgfqpoint{0.922824in}{1.520280in}}%
\pgfpathlineto{\pgfqpoint{0.949149in}{1.520692in}}%
\pgfpathlineto{\pgfqpoint{0.975474in}{1.521143in}}%
\pgfpathlineto{\pgfqpoint{1.001798in}{1.521640in}}%
\pgfpathlineto{\pgfqpoint{1.028123in}{1.522187in}}%
\pgfpathlineto{\pgfqpoint{1.054448in}{1.522792in}}%
\pgfpathlineto{\pgfqpoint{1.080773in}{1.523466in}}%
\pgfpathlineto{\pgfqpoint{1.107097in}{1.524218in}}%
\pgfpathlineto{\pgfqpoint{1.133422in}{1.525064in}}%
\pgfpathlineto{\pgfqpoint{1.159747in}{1.526019in}}%
\pgfpathlineto{\pgfqpoint{1.186072in}{1.527105in}}%
\pgfpathlineto{\pgfqpoint{1.212396in}{1.528350in}}%
\pgfpathlineto{\pgfqpoint{1.238721in}{1.529789in}}%
\pgfpathlineto{\pgfqpoint{1.265046in}{1.531469in}}%
\pgfpathlineto{\pgfqpoint{1.291371in}{1.533453in}}%
\pgfpathlineto{\pgfqpoint{1.317695in}{1.535828in}}%
\pgfpathlineto{\pgfqpoint{1.344020in}{1.538719in}}%
\pgfpathlineto{\pgfqpoint{1.370345in}{1.542309in}}%
\pgfpathlineto{\pgfqpoint{1.396670in}{1.546886in}}%
\pgfpathlineto{\pgfqpoint{1.422994in}{1.552921in}}%
\pgfpathlineto{\pgfqpoint{1.449319in}{1.561258in}}%
\pgfpathlineto{\pgfqpoint{1.475644in}{1.573595in}}%
\pgfpathlineto{\pgfqpoint{1.501969in}{1.594014in}}%
\pgfpathlineto{\pgfqpoint{1.528293in}{1.636158in}}%
\pgfpathlineto{\pgfqpoint{1.554618in}{1.817252in}}%
\pgfpathlineto{\pgfqpoint{1.580943in}{1.438615in}}%
\pgfpathlineto{\pgfqpoint{1.607268in}{1.438615in}}%
\pgfpathlineto{\pgfqpoint{1.633592in}{1.438615in}}%
\pgfpathlineto{\pgfqpoint{1.659917in}{1.438615in}}%
\pgfpathlineto{\pgfqpoint{1.686242in}{1.438615in}}%
\pgfpathlineto{\pgfqpoint{1.712567in}{1.438615in}}%
\pgfpathlineto{\pgfqpoint{1.738891in}{1.438615in}}%
\pgfpathlineto{\pgfqpoint{1.765216in}{1.438615in}}%
\pgfpathlineto{\pgfqpoint{1.791541in}{1.438615in}}%
\pgfpathlineto{\pgfqpoint{1.817866in}{1.438615in}}%
\pgfpathlineto{\pgfqpoint{1.844190in}{1.438615in}}%
\pgfpathlineto{\pgfqpoint{1.870515in}{1.438615in}}%
\pgfpathlineto{\pgfqpoint{1.896840in}{1.438615in}}%
\pgfpathlineto{\pgfqpoint{1.923165in}{1.438615in}}%
\pgfpathlineto{\pgfqpoint{1.949489in}{1.438615in}}%
\pgfpathlineto{\pgfqpoint{1.975814in}{1.438615in}}%
\pgfpathlineto{\pgfqpoint{2.002139in}{1.438615in}}%
\pgfpathlineto{\pgfqpoint{2.028464in}{1.438615in}}%
\pgfpathlineto{\pgfqpoint{2.054788in}{1.438615in}}%
\pgfpathlineto{\pgfqpoint{2.081113in}{1.438615in}}%
\pgfpathlineto{\pgfqpoint{2.107438in}{1.438615in}}%
\pgfpathlineto{\pgfqpoint{2.133763in}{1.438615in}}%
\pgfpathlineto{\pgfqpoint{2.160087in}{1.438615in}}%
\pgfpathlineto{\pgfqpoint{2.186412in}{1.438615in}}%
\pgfpathlineto{\pgfqpoint{2.212737in}{1.438615in}}%
\pgfpathlineto{\pgfqpoint{2.239062in}{1.438615in}}%
\pgfpathlineto{\pgfqpoint{2.265386in}{1.438615in}}%
\pgfpathlineto{\pgfqpoint{2.291711in}{1.438615in}}%
\pgfpathlineto{\pgfqpoint{2.318036in}{1.438615in}}%
\pgfpathlineto{\pgfqpoint{2.344361in}{1.438615in}}%
\pgfpathlineto{\pgfqpoint{2.370685in}{1.438615in}}%
\pgfpathlineto{\pgfqpoint{2.397010in}{1.438615in}}%
\pgfpathlineto{\pgfqpoint{2.423335in}{1.438615in}}%
\pgfpathlineto{\pgfqpoint{2.449660in}{1.817252in}}%
\pgfpathlineto{\pgfqpoint{2.475984in}{1.636158in}}%
\pgfpathlineto{\pgfqpoint{2.502309in}{1.594014in}}%
\pgfpathlineto{\pgfqpoint{2.528634in}{1.573595in}}%
\pgfpathlineto{\pgfqpoint{2.554959in}{1.561258in}}%
\pgfpathlineto{\pgfqpoint{2.581283in}{1.552921in}}%
\pgfpathlineto{\pgfqpoint{2.607608in}{1.546886in}}%
\pgfpathlineto{\pgfqpoint{2.633933in}{1.542309in}}%
\pgfpathlineto{\pgfqpoint{2.660258in}{1.538719in}}%
\pgfpathlineto{\pgfqpoint{2.686582in}{1.535828in}}%
\pgfpathlineto{\pgfqpoint{2.712907in}{1.533453in}}%
\pgfpathlineto{\pgfqpoint{2.739232in}{1.531469in}}%
\pgfpathlineto{\pgfqpoint{2.765557in}{1.529789in}}%
\pgfpathlineto{\pgfqpoint{2.791881in}{1.528350in}}%
\pgfpathlineto{\pgfqpoint{2.818206in}{1.527105in}}%
\pgfpathlineto{\pgfqpoint{2.844531in}{1.526019in}}%
\pgfpathlineto{\pgfqpoint{2.870856in}{1.525064in}}%
\pgfpathlineto{\pgfqpoint{2.897180in}{1.524218in}}%
\pgfpathlineto{\pgfqpoint{2.923505in}{1.523466in}}%
\pgfpathlineto{\pgfqpoint{2.949830in}{1.522792in}}%
\pgfpathlineto{\pgfqpoint{2.976155in}{1.522187in}}%
\pgfpathlineto{\pgfqpoint{3.002479in}{1.521640in}}%
\pgfpathlineto{\pgfqpoint{3.028804in}{1.521143in}}%
\pgfpathlineto{\pgfqpoint{3.055129in}{1.520692in}}%
\pgfpathlineto{\pgfqpoint{3.081454in}{1.520280in}}%
\pgfpathlineto{\pgfqpoint{3.107778in}{1.519902in}}%
\pgfpathlineto{\pgfqpoint{3.134103in}{1.519555in}}%
\pgfpathlineto{\pgfqpoint{3.160428in}{1.519235in}}%
\pgfpathlineto{\pgfqpoint{3.186753in}{1.518940in}}%
\pgfpathlineto{\pgfqpoint{3.213077in}{1.518667in}}%
\pgfpathlineto{\pgfqpoint{3.239402in}{1.518414in}}%
\pgfpathlineto{\pgfqpoint{3.265727in}{1.518178in}}%
\pgfpathlineto{\pgfqpoint{3.292052in}{1.517959in}}%
\pgfpathlineto{\pgfqpoint{3.318376in}{1.517754in}}%
\pgfusepath{stroke}%
\end{pgfscope}%
\begin{pgfscope}%
\pgfpathrectangle{\pgfqpoint{0.554278in}{1.419683in}}{\pgfqpoint{2.895722in}{0.416501in}}%
\pgfusepath{clip}%
\pgfsetrectcap%
\pgfsetroundjoin%
\pgfsetlinewidth{1.505625pt}%
\definecolor{currentstroke}{rgb}{1.000000,0.498039,0.054902}%
\pgfsetstrokecolor{currentstroke}%
\pgfsetdash{}{0pt}%
\pgfpathmoveto{\pgfqpoint{0.685902in}{1.517754in}}%
\pgfpathlineto{\pgfqpoint{0.712226in}{1.517958in}}%
\pgfpathlineto{\pgfqpoint{0.738551in}{1.518178in}}%
\pgfpathlineto{\pgfqpoint{0.764876in}{1.518413in}}%
\pgfpathlineto{\pgfqpoint{0.791201in}{1.518667in}}%
\pgfpathlineto{\pgfqpoint{0.817525in}{1.518940in}}%
\pgfpathlineto{\pgfqpoint{0.843850in}{1.519235in}}%
\pgfpathlineto{\pgfqpoint{0.870175in}{1.519555in}}%
\pgfpathlineto{\pgfqpoint{0.896499in}{1.519902in}}%
\pgfpathlineto{\pgfqpoint{0.922824in}{1.520279in}}%
\pgfpathlineto{\pgfqpoint{0.949149in}{1.520691in}}%
\pgfpathlineto{\pgfqpoint{0.975474in}{1.521143in}}%
\pgfpathlineto{\pgfqpoint{1.001798in}{1.521639in}}%
\pgfpathlineto{\pgfqpoint{1.028123in}{1.522186in}}%
\pgfpathlineto{\pgfqpoint{1.054448in}{1.522791in}}%
\pgfpathlineto{\pgfqpoint{1.080773in}{1.523465in}}%
\pgfpathlineto{\pgfqpoint{1.107097in}{1.524217in}}%
\pgfpathlineto{\pgfqpoint{1.133422in}{1.525062in}}%
\pgfpathlineto{\pgfqpoint{1.159747in}{1.526017in}}%
\pgfpathlineto{\pgfqpoint{1.186072in}{1.527103in}}%
\pgfpathlineto{\pgfqpoint{1.212396in}{1.528347in}}%
\pgfpathlineto{\pgfqpoint{1.238721in}{1.529786in}}%
\pgfpathlineto{\pgfqpoint{1.265046in}{1.531465in}}%
\pgfpathlineto{\pgfqpoint{1.291371in}{1.533448in}}%
\pgfpathlineto{\pgfqpoint{1.317695in}{1.535821in}}%
\pgfpathlineto{\pgfqpoint{1.344020in}{1.538709in}}%
\pgfpathlineto{\pgfqpoint{1.370345in}{1.542296in}}%
\pgfpathlineto{\pgfqpoint{1.396670in}{1.546866in}}%
\pgfpathlineto{\pgfqpoint{1.422994in}{1.552890in}}%
\pgfpathlineto{\pgfqpoint{1.449319in}{1.561205in}}%
\pgfpathlineto{\pgfqpoint{1.475644in}{1.573490in}}%
\pgfpathlineto{\pgfqpoint{1.501969in}{1.593755in}}%
\pgfpathlineto{\pgfqpoint{1.528293in}{1.635100in}}%
\pgfpathlineto{\pgfqpoint{1.554618in}{1.787737in}}%
\pgfpathlineto{\pgfqpoint{1.580943in}{1.471369in}}%
\pgfpathlineto{\pgfqpoint{1.607268in}{1.447572in}}%
\pgfpathlineto{\pgfqpoint{1.633592in}{1.443276in}}%
\pgfpathlineto{\pgfqpoint{1.659917in}{1.441656in}}%
\pgfpathlineto{\pgfqpoint{1.686242in}{1.440846in}}%
\pgfpathlineto{\pgfqpoint{1.712567in}{1.440374in}}%
\pgfpathlineto{\pgfqpoint{1.738891in}{1.440072in}}%
\pgfpathlineto{\pgfqpoint{1.765216in}{1.439866in}}%
\pgfpathlineto{\pgfqpoint{1.791541in}{1.439720in}}%
\pgfpathlineto{\pgfqpoint{1.817866in}{1.439613in}}%
\pgfpathlineto{\pgfqpoint{1.844190in}{1.439534in}}%
\pgfpathlineto{\pgfqpoint{1.870515in}{1.439474in}}%
\pgfpathlineto{\pgfqpoint{1.896840in}{1.439430in}}%
\pgfpathlineto{\pgfqpoint{1.923165in}{1.439399in}}%
\pgfpathlineto{\pgfqpoint{1.949489in}{1.439377in}}%
\pgfpathlineto{\pgfqpoint{1.975814in}{1.439365in}}%
\pgfpathlineto{\pgfqpoint{2.002139in}{1.439361in}}%
\pgfpathlineto{\pgfqpoint{2.028464in}{1.439365in}}%
\pgfpathlineto{\pgfqpoint{2.054788in}{1.439377in}}%
\pgfpathlineto{\pgfqpoint{2.081113in}{1.439399in}}%
\pgfpathlineto{\pgfqpoint{2.107438in}{1.439430in}}%
\pgfpathlineto{\pgfqpoint{2.133763in}{1.439474in}}%
\pgfpathlineto{\pgfqpoint{2.160087in}{1.439534in}}%
\pgfpathlineto{\pgfqpoint{2.186412in}{1.439613in}}%
\pgfpathlineto{\pgfqpoint{2.212737in}{1.439720in}}%
\pgfpathlineto{\pgfqpoint{2.239062in}{1.439866in}}%
\pgfpathlineto{\pgfqpoint{2.265386in}{1.440072in}}%
\pgfpathlineto{\pgfqpoint{2.291711in}{1.440374in}}%
\pgfpathlineto{\pgfqpoint{2.318036in}{1.440846in}}%
\pgfpathlineto{\pgfqpoint{2.344361in}{1.441656in}}%
\pgfpathlineto{\pgfqpoint{2.370685in}{1.443276in}}%
\pgfpathlineto{\pgfqpoint{2.397010in}{1.447572in}}%
\pgfpathlineto{\pgfqpoint{2.423335in}{1.471369in}}%
\pgfpathlineto{\pgfqpoint{2.449660in}{1.787737in}}%
\pgfpathlineto{\pgfqpoint{2.475984in}{1.635100in}}%
\pgfpathlineto{\pgfqpoint{2.502309in}{1.593755in}}%
\pgfpathlineto{\pgfqpoint{2.528634in}{1.573490in}}%
\pgfpathlineto{\pgfqpoint{2.554959in}{1.561205in}}%
\pgfpathlineto{\pgfqpoint{2.581283in}{1.552890in}}%
\pgfpathlineto{\pgfqpoint{2.607608in}{1.546866in}}%
\pgfpathlineto{\pgfqpoint{2.633933in}{1.542296in}}%
\pgfpathlineto{\pgfqpoint{2.660258in}{1.538709in}}%
\pgfpathlineto{\pgfqpoint{2.686582in}{1.535821in}}%
\pgfpathlineto{\pgfqpoint{2.712907in}{1.533448in}}%
\pgfpathlineto{\pgfqpoint{2.739232in}{1.531465in}}%
\pgfpathlineto{\pgfqpoint{2.765557in}{1.529786in}}%
\pgfpathlineto{\pgfqpoint{2.791881in}{1.528347in}}%
\pgfpathlineto{\pgfqpoint{2.818206in}{1.527103in}}%
\pgfpathlineto{\pgfqpoint{2.844531in}{1.526017in}}%
\pgfpathlineto{\pgfqpoint{2.870856in}{1.525062in}}%
\pgfpathlineto{\pgfqpoint{2.897180in}{1.524217in}}%
\pgfpathlineto{\pgfqpoint{2.923505in}{1.523465in}}%
\pgfpathlineto{\pgfqpoint{2.949830in}{1.522791in}}%
\pgfpathlineto{\pgfqpoint{2.976155in}{1.522186in}}%
\pgfpathlineto{\pgfqpoint{3.002479in}{1.521639in}}%
\pgfpathlineto{\pgfqpoint{3.028804in}{1.521143in}}%
\pgfpathlineto{\pgfqpoint{3.055129in}{1.520691in}}%
\pgfpathlineto{\pgfqpoint{3.081454in}{1.520279in}}%
\pgfpathlineto{\pgfqpoint{3.107778in}{1.519902in}}%
\pgfpathlineto{\pgfqpoint{3.134103in}{1.519555in}}%
\pgfpathlineto{\pgfqpoint{3.160428in}{1.519235in}}%
\pgfpathlineto{\pgfqpoint{3.186753in}{1.518940in}}%
\pgfpathlineto{\pgfqpoint{3.213077in}{1.518667in}}%
\pgfpathlineto{\pgfqpoint{3.239402in}{1.518413in}}%
\pgfpathlineto{\pgfqpoint{3.265727in}{1.518178in}}%
\pgfpathlineto{\pgfqpoint{3.292052in}{1.517958in}}%
\pgfpathlineto{\pgfqpoint{3.318376in}{1.517754in}}%
\pgfusepath{stroke}%
\end{pgfscope}%
\begin{pgfscope}%
\pgfpathrectangle{\pgfqpoint{0.554278in}{1.419683in}}{\pgfqpoint{2.895722in}{0.416501in}}%
\pgfusepath{clip}%
\pgfsetrectcap%
\pgfsetroundjoin%
\pgfsetlinewidth{1.505625pt}%
\definecolor{currentstroke}{rgb}{0.172549,0.627451,0.172549}%
\pgfsetstrokecolor{currentstroke}%
\pgfsetdash{}{0pt}%
\pgfpathmoveto{\pgfqpoint{0.685902in}{1.517735in}}%
\pgfpathlineto{\pgfqpoint{0.712226in}{1.517938in}}%
\pgfpathlineto{\pgfqpoint{0.738551in}{1.518156in}}%
\pgfpathlineto{\pgfqpoint{0.764876in}{1.518389in}}%
\pgfpathlineto{\pgfqpoint{0.791201in}{1.518640in}}%
\pgfpathlineto{\pgfqpoint{0.817525in}{1.518910in}}%
\pgfpathlineto{\pgfqpoint{0.843850in}{1.519202in}}%
\pgfpathlineto{\pgfqpoint{0.870175in}{1.519517in}}%
\pgfpathlineto{\pgfqpoint{0.896499in}{1.519860in}}%
\pgfpathlineto{\pgfqpoint{0.922824in}{1.520232in}}%
\pgfpathlineto{\pgfqpoint{0.949149in}{1.520638in}}%
\pgfpathlineto{\pgfqpoint{0.975474in}{1.521082in}}%
\pgfpathlineto{\pgfqpoint{1.001798in}{1.521569in}}%
\pgfpathlineto{\pgfqpoint{1.028123in}{1.522106in}}%
\pgfpathlineto{\pgfqpoint{1.054448in}{1.522699in}}%
\pgfpathlineto{\pgfqpoint{1.080773in}{1.523357in}}%
\pgfpathlineto{\pgfqpoint{1.107097in}{1.524091in}}%
\pgfpathlineto{\pgfqpoint{1.133422in}{1.524913in}}%
\pgfpathlineto{\pgfqpoint{1.159747in}{1.525839in}}%
\pgfpathlineto{\pgfqpoint{1.186072in}{1.526888in}}%
\pgfpathlineto{\pgfqpoint{1.212396in}{1.528085in}}%
\pgfpathlineto{\pgfqpoint{1.238721in}{1.529462in}}%
\pgfpathlineto{\pgfqpoint{1.265046in}{1.531057in}}%
\pgfpathlineto{\pgfqpoint{1.291371in}{1.532925in}}%
\pgfpathlineto{\pgfqpoint{1.317695in}{1.535136in}}%
\pgfpathlineto{\pgfqpoint{1.344020in}{1.537787in}}%
\pgfpathlineto{\pgfqpoint{1.370345in}{1.541013in}}%
\pgfpathlineto{\pgfqpoint{1.396670in}{1.545002in}}%
\pgfpathlineto{\pgfqpoint{1.422994in}{1.550027in}}%
\pgfpathlineto{\pgfqpoint{1.449319in}{1.556466in}}%
\pgfpathlineto{\pgfqpoint{1.475644in}{1.564771in}}%
\pgfpathlineto{\pgfqpoint{1.501969in}{1.575024in}}%
\pgfpathlineto{\pgfqpoint{1.528293in}{1.584163in}}%
\pgfpathlineto{\pgfqpoint{1.554618in}{1.576269in}}%
\pgfpathlineto{\pgfqpoint{1.580943in}{1.536309in}}%
\pgfpathlineto{\pgfqpoint{1.607268in}{1.498423in}}%
\pgfpathlineto{\pgfqpoint{1.633592in}{1.477387in}}%
\pgfpathlineto{\pgfqpoint{1.659917in}{1.466119in}}%
\pgfpathlineto{\pgfqpoint{1.686242in}{1.459592in}}%
\pgfpathlineto{\pgfqpoint{1.712567in}{1.455501in}}%
\pgfpathlineto{\pgfqpoint{1.738891in}{1.452773in}}%
\pgfpathlineto{\pgfqpoint{1.765216in}{1.450868in}}%
\pgfpathlineto{\pgfqpoint{1.791541in}{1.449492in}}%
\pgfpathlineto{\pgfqpoint{1.817866in}{1.448475in}}%
\pgfpathlineto{\pgfqpoint{1.844190in}{1.447713in}}%
\pgfpathlineto{\pgfqpoint{1.870515in}{1.447140in}}%
\pgfpathlineto{\pgfqpoint{1.896840in}{1.446715in}}%
\pgfpathlineto{\pgfqpoint{1.923165in}{1.446407in}}%
\pgfpathlineto{\pgfqpoint{1.949489in}{1.446199in}}%
\pgfpathlineto{\pgfqpoint{1.975814in}{1.446079in}}%
\pgfpathlineto{\pgfqpoint{2.002139in}{1.446039in}}%
\pgfpathlineto{\pgfqpoint{2.028464in}{1.446079in}}%
\pgfpathlineto{\pgfqpoint{2.054788in}{1.446199in}}%
\pgfpathlineto{\pgfqpoint{2.081113in}{1.446407in}}%
\pgfpathlineto{\pgfqpoint{2.107438in}{1.446715in}}%
\pgfpathlineto{\pgfqpoint{2.133763in}{1.447140in}}%
\pgfpathlineto{\pgfqpoint{2.160087in}{1.447713in}}%
\pgfpathlineto{\pgfqpoint{2.186412in}{1.448475in}}%
\pgfpathlineto{\pgfqpoint{2.212737in}{1.449492in}}%
\pgfpathlineto{\pgfqpoint{2.239062in}{1.450868in}}%
\pgfpathlineto{\pgfqpoint{2.265386in}{1.452773in}}%
\pgfpathlineto{\pgfqpoint{2.291711in}{1.455501in}}%
\pgfpathlineto{\pgfqpoint{2.318036in}{1.459592in}}%
\pgfpathlineto{\pgfqpoint{2.344361in}{1.466119in}}%
\pgfpathlineto{\pgfqpoint{2.370685in}{1.477387in}}%
\pgfpathlineto{\pgfqpoint{2.397010in}{1.498423in}}%
\pgfpathlineto{\pgfqpoint{2.423335in}{1.536309in}}%
\pgfpathlineto{\pgfqpoint{2.449660in}{1.576269in}}%
\pgfpathlineto{\pgfqpoint{2.475984in}{1.584163in}}%
\pgfpathlineto{\pgfqpoint{2.502309in}{1.575024in}}%
\pgfpathlineto{\pgfqpoint{2.528634in}{1.564771in}}%
\pgfpathlineto{\pgfqpoint{2.554959in}{1.556466in}}%
\pgfpathlineto{\pgfqpoint{2.581283in}{1.550027in}}%
\pgfpathlineto{\pgfqpoint{2.607608in}{1.545002in}}%
\pgfpathlineto{\pgfqpoint{2.633933in}{1.541013in}}%
\pgfpathlineto{\pgfqpoint{2.660258in}{1.537787in}}%
\pgfpathlineto{\pgfqpoint{2.686582in}{1.535136in}}%
\pgfpathlineto{\pgfqpoint{2.712907in}{1.532925in}}%
\pgfpathlineto{\pgfqpoint{2.739232in}{1.531057in}}%
\pgfpathlineto{\pgfqpoint{2.765557in}{1.529462in}}%
\pgfpathlineto{\pgfqpoint{2.791881in}{1.528085in}}%
\pgfpathlineto{\pgfqpoint{2.818206in}{1.526888in}}%
\pgfpathlineto{\pgfqpoint{2.844531in}{1.525839in}}%
\pgfpathlineto{\pgfqpoint{2.870856in}{1.524913in}}%
\pgfpathlineto{\pgfqpoint{2.897180in}{1.524091in}}%
\pgfpathlineto{\pgfqpoint{2.923505in}{1.523357in}}%
\pgfpathlineto{\pgfqpoint{2.949830in}{1.522699in}}%
\pgfpathlineto{\pgfqpoint{2.976155in}{1.522106in}}%
\pgfpathlineto{\pgfqpoint{3.002479in}{1.521569in}}%
\pgfpathlineto{\pgfqpoint{3.028804in}{1.521082in}}%
\pgfpathlineto{\pgfqpoint{3.055129in}{1.520638in}}%
\pgfpathlineto{\pgfqpoint{3.081454in}{1.520232in}}%
\pgfpathlineto{\pgfqpoint{3.107778in}{1.519860in}}%
\pgfpathlineto{\pgfqpoint{3.134103in}{1.519517in}}%
\pgfpathlineto{\pgfqpoint{3.160428in}{1.519202in}}%
\pgfpathlineto{\pgfqpoint{3.186753in}{1.518910in}}%
\pgfpathlineto{\pgfqpoint{3.213077in}{1.518640in}}%
\pgfpathlineto{\pgfqpoint{3.239402in}{1.518389in}}%
\pgfpathlineto{\pgfqpoint{3.265727in}{1.518156in}}%
\pgfpathlineto{\pgfqpoint{3.292052in}{1.517938in}}%
\pgfpathlineto{\pgfqpoint{3.318376in}{1.517735in}}%
\pgfusepath{stroke}%
\end{pgfscope}%
\begin{pgfscope}%
\pgfsetrectcap%
\pgfsetmiterjoin%
\pgfsetlinewidth{0.803000pt}%
\definecolor{currentstroke}{rgb}{0.000000,0.000000,0.000000}%
\pgfsetstrokecolor{currentstroke}%
\pgfsetdash{}{0pt}%
\pgfpathmoveto{\pgfqpoint{0.554278in}{1.419683in}}%
\pgfpathlineto{\pgfqpoint{0.554278in}{1.836184in}}%
\pgfusepath{stroke}%
\end{pgfscope}%
\begin{pgfscope}%
\pgfsetrectcap%
\pgfsetmiterjoin%
\pgfsetlinewidth{0.803000pt}%
\definecolor{currentstroke}{rgb}{0.000000,0.000000,0.000000}%
\pgfsetstrokecolor{currentstroke}%
\pgfsetdash{}{0pt}%
\pgfpathmoveto{\pgfqpoint{3.450000in}{1.419683in}}%
\pgfpathlineto{\pgfqpoint{3.450000in}{1.836184in}}%
\pgfusepath{stroke}%
\end{pgfscope}%
\begin{pgfscope}%
\pgfsetrectcap%
\pgfsetmiterjoin%
\pgfsetlinewidth{0.803000pt}%
\definecolor{currentstroke}{rgb}{0.000000,0.000000,0.000000}%
\pgfsetstrokecolor{currentstroke}%
\pgfsetdash{}{0pt}%
\pgfpathmoveto{\pgfqpoint{0.554278in}{1.419683in}}%
\pgfpathlineto{\pgfqpoint{3.450000in}{1.419683in}}%
\pgfusepath{stroke}%
\end{pgfscope}%
\begin{pgfscope}%
\pgfsetrectcap%
\pgfsetmiterjoin%
\pgfsetlinewidth{0.803000pt}%
\definecolor{currentstroke}{rgb}{0.000000,0.000000,0.000000}%
\pgfsetstrokecolor{currentstroke}%
\pgfsetdash{}{0pt}%
\pgfpathmoveto{\pgfqpoint{0.554278in}{1.836184in}}%
\pgfpathlineto{\pgfqpoint{3.450000in}{1.836184in}}%
\pgfusepath{stroke}%
\end{pgfscope}%
\begin{pgfscope}%
\pgfsetbuttcap%
\pgfsetmiterjoin%
\definecolor{currentfill}{rgb}{1.000000,1.000000,1.000000}%
\pgfsetfillcolor{currentfill}%
\pgfsetlinewidth{0.000000pt}%
\definecolor{currentstroke}{rgb}{0.000000,0.000000,0.000000}%
\pgfsetstrokecolor{currentstroke}%
\pgfsetstrokeopacity{0.000000}%
\pgfsetdash{}{0pt}%
\pgfpathmoveto{\pgfqpoint{0.554278in}{0.592778in}}%
\pgfpathlineto{\pgfqpoint{3.450000in}{0.592778in}}%
\pgfpathlineto{\pgfqpoint{3.450000in}{1.009279in}}%
\pgfpathlineto{\pgfqpoint{0.554278in}{1.009279in}}%
\pgfpathlineto{\pgfqpoint{0.554278in}{0.592778in}}%
\pgfpathclose%
\pgfusepath{fill}%
\end{pgfscope}%
\begin{pgfscope}%
\pgfsetbuttcap%
\pgfsetroundjoin%
\definecolor{currentfill}{rgb}{0.000000,0.000000,0.000000}%
\pgfsetfillcolor{currentfill}%
\pgfsetlinewidth{0.803000pt}%
\definecolor{currentstroke}{rgb}{0.000000,0.000000,0.000000}%
\pgfsetstrokecolor{currentstroke}%
\pgfsetdash{}{0pt}%
\pgfsys@defobject{currentmarker}{\pgfqpoint{0.000000in}{-0.048611in}}{\pgfqpoint{0.000000in}{0.000000in}}{%
\pgfpathmoveto{\pgfqpoint{0.000000in}{0.000000in}}%
\pgfpathlineto{\pgfqpoint{0.000000in}{-0.048611in}}%
\pgfusepath{stroke,fill}%
}%
\begin{pgfscope}%
\pgfsys@transformshift{1.124647in}{0.592778in}%
\pgfsys@useobject{currentmarker}{}%
\end{pgfscope}%
\end{pgfscope}%
\begin{pgfscope}%
\definecolor{textcolor}{rgb}{0.000000,0.000000,0.000000}%
\pgfsetstrokecolor{textcolor}%
\pgfsetfillcolor{textcolor}%
\pgftext[x=1.124647in,y=0.495556in,,top]{\color{textcolor}{\sffamily\fontsize{10.000000}{12.000000}\selectfont\catcode`\^=\active\def^{\ifmmode\sp\else\^{}\fi}\catcode`\%=\active\def%{\%}\ensuremath{-}2}}%
\end{pgfscope}%
\begin{pgfscope}%
\pgfsetbuttcap%
\pgfsetroundjoin%
\definecolor{currentfill}{rgb}{0.000000,0.000000,0.000000}%
\pgfsetfillcolor{currentfill}%
\pgfsetlinewidth{0.803000pt}%
\definecolor{currentstroke}{rgb}{0.000000,0.000000,0.000000}%
\pgfsetstrokecolor{currentstroke}%
\pgfsetdash{}{0pt}%
\pgfsys@defobject{currentmarker}{\pgfqpoint{0.000000in}{-0.048611in}}{\pgfqpoint{0.000000in}{0.000000in}}{%
\pgfpathmoveto{\pgfqpoint{0.000000in}{0.000000in}}%
\pgfpathlineto{\pgfqpoint{0.000000in}{-0.048611in}}%
\pgfusepath{stroke,fill}%
}%
\begin{pgfscope}%
\pgfsys@transformshift{2.002139in}{0.592778in}%
\pgfsys@useobject{currentmarker}{}%
\end{pgfscope}%
\end{pgfscope}%
\begin{pgfscope}%
\definecolor{textcolor}{rgb}{0.000000,0.000000,0.000000}%
\pgfsetstrokecolor{textcolor}%
\pgfsetfillcolor{textcolor}%
\pgftext[x=2.002139in,y=0.495556in,,top]{\color{textcolor}{\sffamily\fontsize{10.000000}{12.000000}\selectfont\catcode`\^=\active\def^{\ifmmode\sp\else\^{}\fi}\catcode`\%=\active\def%{\%}0}}%
\end{pgfscope}%
\begin{pgfscope}%
\pgfsetbuttcap%
\pgfsetroundjoin%
\definecolor{currentfill}{rgb}{0.000000,0.000000,0.000000}%
\pgfsetfillcolor{currentfill}%
\pgfsetlinewidth{0.803000pt}%
\definecolor{currentstroke}{rgb}{0.000000,0.000000,0.000000}%
\pgfsetstrokecolor{currentstroke}%
\pgfsetdash{}{0pt}%
\pgfsys@defobject{currentmarker}{\pgfqpoint{0.000000in}{-0.048611in}}{\pgfqpoint{0.000000in}{0.000000in}}{%
\pgfpathmoveto{\pgfqpoint{0.000000in}{0.000000in}}%
\pgfpathlineto{\pgfqpoint{0.000000in}{-0.048611in}}%
\pgfusepath{stroke,fill}%
}%
\begin{pgfscope}%
\pgfsys@transformshift{2.879630in}{0.592778in}%
\pgfsys@useobject{currentmarker}{}%
\end{pgfscope}%
\end{pgfscope}%
\begin{pgfscope}%
\definecolor{textcolor}{rgb}{0.000000,0.000000,0.000000}%
\pgfsetstrokecolor{textcolor}%
\pgfsetfillcolor{textcolor}%
\pgftext[x=2.879630in,y=0.495556in,,top]{\color{textcolor}{\sffamily\fontsize{10.000000}{12.000000}\selectfont\catcode`\^=\active\def^{\ifmmode\sp\else\^{}\fi}\catcode`\%=\active\def%{\%}2}}%
\end{pgfscope}%
\begin{pgfscope}%
\definecolor{textcolor}{rgb}{0.000000,0.000000,0.000000}%
\pgfsetstrokecolor{textcolor}%
\pgfsetfillcolor{textcolor}%
\pgftext[x=2.002139in,y=0.305587in,,top]{\color{textcolor}{\sffamily\fontsize{10.000000}{12.000000}\selectfont\catcode`\^=\active\def^{\ifmmode\sp\else\^{}\fi}\catcode`\%=\active\def%{\%}$E/\Delta$}}%
\end{pgfscope}%
\begin{pgfscope}%
\pgfsetbuttcap%
\pgfsetroundjoin%
\definecolor{currentfill}{rgb}{0.000000,0.000000,0.000000}%
\pgfsetfillcolor{currentfill}%
\pgfsetlinewidth{0.803000pt}%
\definecolor{currentstroke}{rgb}{0.000000,0.000000,0.000000}%
\pgfsetstrokecolor{currentstroke}%
\pgfsetdash{}{0pt}%
\pgfsys@defobject{currentmarker}{\pgfqpoint{-0.048611in}{0.000000in}}{\pgfqpoint{-0.000000in}{0.000000in}}{%
\pgfpathmoveto{\pgfqpoint{-0.000000in}{0.000000in}}%
\pgfpathlineto{\pgfqpoint{-0.048611in}{0.000000in}}%
\pgfusepath{stroke,fill}%
}%
\begin{pgfscope}%
\pgfsys@transformshift{0.554278in}{0.611710in}%
\pgfsys@useobject{currentmarker}{}%
\end{pgfscope}%
\end{pgfscope}%
\begin{pgfscope}%
\definecolor{textcolor}{rgb}{0.000000,0.000000,0.000000}%
\pgfsetstrokecolor{textcolor}%
\pgfsetfillcolor{textcolor}%
\pgftext[x=0.368690in, y=0.558948in, left, base]{\color{textcolor}{\sffamily\fontsize{10.000000}{12.000000}\selectfont\catcode`\^=\active\def^{\ifmmode\sp\else\^{}\fi}\catcode`\%=\active\def%{\%}0}}%
\end{pgfscope}%
\begin{pgfscope}%
\pgfsetbuttcap%
\pgfsetroundjoin%
\definecolor{currentfill}{rgb}{0.000000,0.000000,0.000000}%
\pgfsetfillcolor{currentfill}%
\pgfsetlinewidth{0.803000pt}%
\definecolor{currentstroke}{rgb}{0.000000,0.000000,0.000000}%
\pgfsetstrokecolor{currentstroke}%
\pgfsetdash{}{0pt}%
\pgfsys@defobject{currentmarker}{\pgfqpoint{-0.048611in}{0.000000in}}{\pgfqpoint{-0.000000in}{0.000000in}}{%
\pgfpathmoveto{\pgfqpoint{-0.000000in}{0.000000in}}%
\pgfpathlineto{\pgfqpoint{-0.048611in}{0.000000in}}%
\pgfusepath{stroke,fill}%
}%
\begin{pgfscope}%
\pgfsys@transformshift{0.554278in}{0.990347in}%
\pgfsys@useobject{currentmarker}{}%
\end{pgfscope}%
\end{pgfscope}%
\begin{pgfscope}%
\definecolor{textcolor}{rgb}{0.000000,0.000000,0.000000}%
\pgfsetstrokecolor{textcolor}%
\pgfsetfillcolor{textcolor}%
\pgftext[x=0.368690in, y=0.937586in, left, base]{\color{textcolor}{\sffamily\fontsize{10.000000}{12.000000}\selectfont\catcode`\^=\active\def^{\ifmmode\sp\else\^{}\fi}\catcode`\%=\active\def%{\%}1}}%
\end{pgfscope}%
\begin{pgfscope}%
\definecolor{textcolor}{rgb}{0.000000,0.000000,0.000000}%
\pgfsetstrokecolor{textcolor}%
\pgfsetfillcolor{textcolor}%
\pgftext[x=0.313135in,y=0.801028in,,bottom,rotate=90.000000]{\color{textcolor}{\sffamily\fontsize{10.000000}{12.000000}\selectfont\catcode`\^=\active\def^{\ifmmode\sp\else\^{}\fi}\catcode`\%=\active\def%{\%}$f$}}%
\end{pgfscope}%
\begin{pgfscope}%
\pgfpathrectangle{\pgfqpoint{0.554278in}{0.592778in}}{\pgfqpoint{2.895722in}{0.416501in}}%
\pgfusepath{clip}%
\pgfsetrectcap%
\pgfsetroundjoin%
\pgfsetlinewidth{1.505625pt}%
\definecolor{currentstroke}{rgb}{0.121569,0.466667,0.705882}%
\pgfsetstrokecolor{currentstroke}%
\pgfsetdash{}{0pt}%
\pgfpathmoveto{\pgfqpoint{0.685902in}{0.990347in}}%
\pgfpathlineto{\pgfqpoint{0.712226in}{0.990347in}}%
\pgfpathlineto{\pgfqpoint{0.738551in}{0.990347in}}%
\pgfpathlineto{\pgfqpoint{0.764876in}{0.990347in}}%
\pgfpathlineto{\pgfqpoint{0.791201in}{0.990347in}}%
\pgfpathlineto{\pgfqpoint{0.817525in}{0.990347in}}%
\pgfpathlineto{\pgfqpoint{0.843850in}{0.990347in}}%
\pgfpathlineto{\pgfqpoint{0.870175in}{0.990347in}}%
\pgfpathlineto{\pgfqpoint{0.896499in}{0.990347in}}%
\pgfpathlineto{\pgfqpoint{0.922824in}{0.990347in}}%
\pgfpathlineto{\pgfqpoint{0.949149in}{0.990347in}}%
\pgfpathlineto{\pgfqpoint{0.975474in}{0.990347in}}%
\pgfpathlineto{\pgfqpoint{1.001798in}{0.990347in}}%
\pgfpathlineto{\pgfqpoint{1.028123in}{0.990347in}}%
\pgfpathlineto{\pgfqpoint{1.054448in}{0.990347in}}%
\pgfpathlineto{\pgfqpoint{1.080773in}{0.990347in}}%
\pgfpathlineto{\pgfqpoint{1.107097in}{0.990347in}}%
\pgfpathlineto{\pgfqpoint{1.133422in}{0.990347in}}%
\pgfpathlineto{\pgfqpoint{1.159747in}{0.990347in}}%
\pgfpathlineto{\pgfqpoint{1.186072in}{0.990347in}}%
\pgfpathlineto{\pgfqpoint{1.212396in}{0.990347in}}%
\pgfpathlineto{\pgfqpoint{1.238721in}{0.990347in}}%
\pgfpathlineto{\pgfqpoint{1.265046in}{0.990347in}}%
\pgfpathlineto{\pgfqpoint{1.291371in}{0.990347in}}%
\pgfpathlineto{\pgfqpoint{1.317695in}{0.990347in}}%
\pgfpathlineto{\pgfqpoint{1.344020in}{0.990347in}}%
\pgfpathlineto{\pgfqpoint{1.370345in}{0.990347in}}%
\pgfpathlineto{\pgfqpoint{1.396670in}{0.990347in}}%
\pgfpathlineto{\pgfqpoint{1.422994in}{0.990347in}}%
\pgfpathlineto{\pgfqpoint{1.449319in}{0.990347in}}%
\pgfpathlineto{\pgfqpoint{1.475644in}{0.990347in}}%
\pgfpathlineto{\pgfqpoint{1.501969in}{0.990347in}}%
\pgfpathlineto{\pgfqpoint{1.528293in}{0.990347in}}%
\pgfpathlineto{\pgfqpoint{1.554618in}{0.990347in}}%
\pgfpathlineto{\pgfqpoint{1.580943in}{0.990347in}}%
\pgfpathlineto{\pgfqpoint{1.607268in}{0.990347in}}%
\pgfpathlineto{\pgfqpoint{1.633592in}{0.990347in}}%
\pgfpathlineto{\pgfqpoint{1.659917in}{0.990347in}}%
\pgfpathlineto{\pgfqpoint{1.686242in}{0.990347in}}%
\pgfpathlineto{\pgfqpoint{1.712567in}{0.990347in}}%
\pgfpathlineto{\pgfqpoint{1.738891in}{0.990347in}}%
\pgfpathlineto{\pgfqpoint{1.765216in}{0.990347in}}%
\pgfpathlineto{\pgfqpoint{1.791541in}{0.990347in}}%
\pgfpathlineto{\pgfqpoint{1.817866in}{0.990347in}}%
\pgfpathlineto{\pgfqpoint{1.844190in}{0.990347in}}%
\pgfpathlineto{\pgfqpoint{1.870515in}{0.990347in}}%
\pgfpathlineto{\pgfqpoint{1.896840in}{0.990347in}}%
\pgfpathlineto{\pgfqpoint{1.923165in}{0.990347in}}%
\pgfpathlineto{\pgfqpoint{1.949489in}{0.990347in}}%
\pgfpathlineto{\pgfqpoint{1.975814in}{0.990347in}}%
\pgfpathlineto{\pgfqpoint{2.002139in}{0.611710in}}%
\pgfpathlineto{\pgfqpoint{2.028464in}{0.611710in}}%
\pgfpathlineto{\pgfqpoint{2.054788in}{0.611710in}}%
\pgfpathlineto{\pgfqpoint{2.081113in}{0.611710in}}%
\pgfpathlineto{\pgfqpoint{2.107438in}{0.611710in}}%
\pgfpathlineto{\pgfqpoint{2.133763in}{0.611710in}}%
\pgfpathlineto{\pgfqpoint{2.160087in}{0.611710in}}%
\pgfpathlineto{\pgfqpoint{2.186412in}{0.611710in}}%
\pgfpathlineto{\pgfqpoint{2.212737in}{0.611710in}}%
\pgfpathlineto{\pgfqpoint{2.239062in}{0.611710in}}%
\pgfpathlineto{\pgfqpoint{2.265386in}{0.611710in}}%
\pgfpathlineto{\pgfqpoint{2.291711in}{0.611710in}}%
\pgfpathlineto{\pgfqpoint{2.318036in}{0.611710in}}%
\pgfpathlineto{\pgfqpoint{2.344361in}{0.611710in}}%
\pgfpathlineto{\pgfqpoint{2.370685in}{0.611710in}}%
\pgfpathlineto{\pgfqpoint{2.397010in}{0.611710in}}%
\pgfpathlineto{\pgfqpoint{2.423335in}{0.611710in}}%
\pgfpathlineto{\pgfqpoint{2.449660in}{0.611710in}}%
\pgfpathlineto{\pgfqpoint{2.475984in}{0.611710in}}%
\pgfpathlineto{\pgfqpoint{2.502309in}{0.611710in}}%
\pgfpathlineto{\pgfqpoint{2.528634in}{0.611710in}}%
\pgfpathlineto{\pgfqpoint{2.554959in}{0.611710in}}%
\pgfpathlineto{\pgfqpoint{2.581283in}{0.611710in}}%
\pgfpathlineto{\pgfqpoint{2.607608in}{0.611710in}}%
\pgfpathlineto{\pgfqpoint{2.633933in}{0.611710in}}%
\pgfpathlineto{\pgfqpoint{2.660258in}{0.611710in}}%
\pgfpathlineto{\pgfqpoint{2.686582in}{0.611710in}}%
\pgfpathlineto{\pgfqpoint{2.712907in}{0.611710in}}%
\pgfpathlineto{\pgfqpoint{2.739232in}{0.611710in}}%
\pgfpathlineto{\pgfqpoint{2.765557in}{0.611710in}}%
\pgfpathlineto{\pgfqpoint{2.791881in}{0.611710in}}%
\pgfpathlineto{\pgfqpoint{2.818206in}{0.611710in}}%
\pgfpathlineto{\pgfqpoint{2.844531in}{0.611710in}}%
\pgfpathlineto{\pgfqpoint{2.870856in}{0.611710in}}%
\pgfpathlineto{\pgfqpoint{2.897180in}{0.611710in}}%
\pgfpathlineto{\pgfqpoint{2.923505in}{0.611710in}}%
\pgfpathlineto{\pgfqpoint{2.949830in}{0.611710in}}%
\pgfpathlineto{\pgfqpoint{2.976155in}{0.611710in}}%
\pgfpathlineto{\pgfqpoint{3.002479in}{0.611710in}}%
\pgfpathlineto{\pgfqpoint{3.028804in}{0.611710in}}%
\pgfpathlineto{\pgfqpoint{3.055129in}{0.611710in}}%
\pgfpathlineto{\pgfqpoint{3.081454in}{0.611710in}}%
\pgfpathlineto{\pgfqpoint{3.107778in}{0.611710in}}%
\pgfpathlineto{\pgfqpoint{3.134103in}{0.611710in}}%
\pgfpathlineto{\pgfqpoint{3.160428in}{0.611710in}}%
\pgfpathlineto{\pgfqpoint{3.186753in}{0.611710in}}%
\pgfpathlineto{\pgfqpoint{3.213077in}{0.611710in}}%
\pgfpathlineto{\pgfqpoint{3.239402in}{0.611710in}}%
\pgfpathlineto{\pgfqpoint{3.265727in}{0.611710in}}%
\pgfpathlineto{\pgfqpoint{3.292052in}{0.611710in}}%
\pgfpathlineto{\pgfqpoint{3.318376in}{0.611710in}}%
\pgfusepath{stroke}%
\end{pgfscope}%
\begin{pgfscope}%
\pgfpathrectangle{\pgfqpoint{0.554278in}{0.592778in}}{\pgfqpoint{2.895722in}{0.416501in}}%
\pgfusepath{clip}%
\pgfsetrectcap%
\pgfsetroundjoin%
\pgfsetlinewidth{1.505625pt}%
\definecolor{currentstroke}{rgb}{1.000000,0.498039,0.054902}%
\pgfsetstrokecolor{currentstroke}%
\pgfsetdash{}{0pt}%
\pgfpathmoveto{\pgfqpoint{0.685902in}{0.990347in}}%
\pgfpathlineto{\pgfqpoint{0.712226in}{0.990347in}}%
\pgfpathlineto{\pgfqpoint{0.738551in}{0.990347in}}%
\pgfpathlineto{\pgfqpoint{0.764876in}{0.990347in}}%
\pgfpathlineto{\pgfqpoint{0.791201in}{0.990347in}}%
\pgfpathlineto{\pgfqpoint{0.817525in}{0.990347in}}%
\pgfpathlineto{\pgfqpoint{0.843850in}{0.990347in}}%
\pgfpathlineto{\pgfqpoint{0.870175in}{0.990347in}}%
\pgfpathlineto{\pgfqpoint{0.896499in}{0.990347in}}%
\pgfpathlineto{\pgfqpoint{0.922824in}{0.990347in}}%
\pgfpathlineto{\pgfqpoint{0.949149in}{0.990347in}}%
\pgfpathlineto{\pgfqpoint{0.975474in}{0.990347in}}%
\pgfpathlineto{\pgfqpoint{1.001798in}{0.990347in}}%
\pgfpathlineto{\pgfqpoint{1.028123in}{0.990347in}}%
\pgfpathlineto{\pgfqpoint{1.054448in}{0.990347in}}%
\pgfpathlineto{\pgfqpoint{1.080773in}{0.990347in}}%
\pgfpathlineto{\pgfqpoint{1.107097in}{0.990347in}}%
\pgfpathlineto{\pgfqpoint{1.133422in}{0.990347in}}%
\pgfpathlineto{\pgfqpoint{1.159747in}{0.990347in}}%
\pgfpathlineto{\pgfqpoint{1.186072in}{0.990347in}}%
\pgfpathlineto{\pgfqpoint{1.212396in}{0.990347in}}%
\pgfpathlineto{\pgfqpoint{1.238721in}{0.990347in}}%
\pgfpathlineto{\pgfqpoint{1.265046in}{0.990347in}}%
\pgfpathlineto{\pgfqpoint{1.291371in}{0.990347in}}%
\pgfpathlineto{\pgfqpoint{1.317695in}{0.990347in}}%
\pgfpathlineto{\pgfqpoint{1.344020in}{0.990347in}}%
\pgfpathlineto{\pgfqpoint{1.370345in}{0.990347in}}%
\pgfpathlineto{\pgfqpoint{1.396670in}{0.990347in}}%
\pgfpathlineto{\pgfqpoint{1.422994in}{0.990347in}}%
\pgfpathlineto{\pgfqpoint{1.449319in}{0.990347in}}%
\pgfpathlineto{\pgfqpoint{1.475644in}{0.990347in}}%
\pgfpathlineto{\pgfqpoint{1.501969in}{0.990346in}}%
\pgfpathlineto{\pgfqpoint{1.528293in}{0.990346in}}%
\pgfpathlineto{\pgfqpoint{1.554618in}{0.990344in}}%
\pgfpathlineto{\pgfqpoint{1.580943in}{0.990342in}}%
\pgfpathlineto{\pgfqpoint{1.607268in}{0.990336in}}%
\pgfpathlineto{\pgfqpoint{1.633592in}{0.990325in}}%
\pgfpathlineto{\pgfqpoint{1.659917in}{0.990303in}}%
\pgfpathlineto{\pgfqpoint{1.686242in}{0.990258in}}%
\pgfpathlineto{\pgfqpoint{1.712567in}{0.990169in}}%
\pgfpathlineto{\pgfqpoint{1.738891in}{0.989989in}}%
\pgfpathlineto{\pgfqpoint{1.765216in}{0.989629in}}%
\pgfpathlineto{\pgfqpoint{1.791541in}{0.988910in}}%
\pgfpathlineto{\pgfqpoint{1.817866in}{0.987475in}}%
\pgfpathlineto{\pgfqpoint{1.844190in}{0.984628in}}%
\pgfpathlineto{\pgfqpoint{1.870515in}{0.979046in}}%
\pgfpathlineto{\pgfqpoint{1.896840in}{0.968335in}}%
\pgfpathlineto{\pgfqpoint{1.923165in}{0.948625in}}%
\pgfpathlineto{\pgfqpoint{1.949489in}{0.914997in}}%
\pgfpathlineto{\pgfqpoint{1.975814in}{0.864397in}}%
\pgfpathlineto{\pgfqpoint{2.002139in}{0.801028in}}%
\pgfpathlineto{\pgfqpoint{2.028464in}{0.737659in}}%
\pgfpathlineto{\pgfqpoint{2.054788in}{0.687059in}}%
\pgfpathlineto{\pgfqpoint{2.081113in}{0.653431in}}%
\pgfpathlineto{\pgfqpoint{2.107438in}{0.633722in}}%
\pgfpathlineto{\pgfqpoint{2.133763in}{0.623011in}}%
\pgfpathlineto{\pgfqpoint{2.160087in}{0.617428in}}%
\pgfpathlineto{\pgfqpoint{2.186412in}{0.614582in}}%
\pgfpathlineto{\pgfqpoint{2.212737in}{0.613147in}}%
\pgfpathlineto{\pgfqpoint{2.239062in}{0.612427in}}%
\pgfpathlineto{\pgfqpoint{2.265386in}{0.612068in}}%
\pgfpathlineto{\pgfqpoint{2.291711in}{0.611888in}}%
\pgfpathlineto{\pgfqpoint{2.318036in}{0.611799in}}%
\pgfpathlineto{\pgfqpoint{2.344361in}{0.611754in}}%
\pgfpathlineto{\pgfqpoint{2.370685in}{0.611732in}}%
\pgfpathlineto{\pgfqpoint{2.397010in}{0.611721in}}%
\pgfpathlineto{\pgfqpoint{2.423335in}{0.611715in}}%
\pgfpathlineto{\pgfqpoint{2.449660in}{0.611712in}}%
\pgfpathlineto{\pgfqpoint{2.475984in}{0.611711in}}%
\pgfpathlineto{\pgfqpoint{2.502309in}{0.611710in}}%
\pgfpathlineto{\pgfqpoint{2.528634in}{0.611710in}}%
\pgfpathlineto{\pgfqpoint{2.554959in}{0.611710in}}%
\pgfpathlineto{\pgfqpoint{2.581283in}{0.611710in}}%
\pgfpathlineto{\pgfqpoint{2.607608in}{0.611710in}}%
\pgfpathlineto{\pgfqpoint{2.633933in}{0.611710in}}%
\pgfpathlineto{\pgfqpoint{2.660258in}{0.611710in}}%
\pgfpathlineto{\pgfqpoint{2.686582in}{0.611710in}}%
\pgfpathlineto{\pgfqpoint{2.712907in}{0.611710in}}%
\pgfpathlineto{\pgfqpoint{2.739232in}{0.611710in}}%
\pgfpathlineto{\pgfqpoint{2.765557in}{0.611710in}}%
\pgfpathlineto{\pgfqpoint{2.791881in}{0.611710in}}%
\pgfpathlineto{\pgfqpoint{2.818206in}{0.611710in}}%
\pgfpathlineto{\pgfqpoint{2.844531in}{0.611710in}}%
\pgfpathlineto{\pgfqpoint{2.870856in}{0.611710in}}%
\pgfpathlineto{\pgfqpoint{2.897180in}{0.611710in}}%
\pgfpathlineto{\pgfqpoint{2.923505in}{0.611710in}}%
\pgfpathlineto{\pgfqpoint{2.949830in}{0.611710in}}%
\pgfpathlineto{\pgfqpoint{2.976155in}{0.611710in}}%
\pgfpathlineto{\pgfqpoint{3.002479in}{0.611710in}}%
\pgfpathlineto{\pgfqpoint{3.028804in}{0.611710in}}%
\pgfpathlineto{\pgfqpoint{3.055129in}{0.611710in}}%
\pgfpathlineto{\pgfqpoint{3.081454in}{0.611710in}}%
\pgfpathlineto{\pgfqpoint{3.107778in}{0.611710in}}%
\pgfpathlineto{\pgfqpoint{3.134103in}{0.611710in}}%
\pgfpathlineto{\pgfqpoint{3.160428in}{0.611710in}}%
\pgfpathlineto{\pgfqpoint{3.186753in}{0.611710in}}%
\pgfpathlineto{\pgfqpoint{3.213077in}{0.611710in}}%
\pgfpathlineto{\pgfqpoint{3.239402in}{0.611710in}}%
\pgfpathlineto{\pgfqpoint{3.265727in}{0.611710in}}%
\pgfpathlineto{\pgfqpoint{3.292052in}{0.611710in}}%
\pgfpathlineto{\pgfqpoint{3.318376in}{0.611710in}}%
\pgfusepath{stroke}%
\end{pgfscope}%
\begin{pgfscope}%
\pgfpathrectangle{\pgfqpoint{0.554278in}{0.592778in}}{\pgfqpoint{2.895722in}{0.416501in}}%
\pgfusepath{clip}%
\pgfsetrectcap%
\pgfsetroundjoin%
\pgfsetlinewidth{1.505625pt}%
\definecolor{currentstroke}{rgb}{0.172549,0.627451,0.172549}%
\pgfsetstrokecolor{currentstroke}%
\pgfsetdash{}{0pt}%
\pgfpathmoveto{\pgfqpoint{0.685902in}{0.979046in}}%
\pgfpathlineto{\pgfqpoint{0.712226in}{0.978257in}}%
\pgfpathlineto{\pgfqpoint{0.738551in}{0.977415in}}%
\pgfpathlineto{\pgfqpoint{0.764876in}{0.976516in}}%
\pgfpathlineto{\pgfqpoint{0.791201in}{0.975558in}}%
\pgfpathlineto{\pgfqpoint{0.817525in}{0.974536in}}%
\pgfpathlineto{\pgfqpoint{0.843850in}{0.973447in}}%
\pgfpathlineto{\pgfqpoint{0.870175in}{0.972286in}}%
\pgfpathlineto{\pgfqpoint{0.896499in}{0.971050in}}%
\pgfpathlineto{\pgfqpoint{0.922824in}{0.969735in}}%
\pgfpathlineto{\pgfqpoint{0.949149in}{0.968335in}}%
\pgfpathlineto{\pgfqpoint{0.975474in}{0.966846in}}%
\pgfpathlineto{\pgfqpoint{1.001798in}{0.965264in}}%
\pgfpathlineto{\pgfqpoint{1.028123in}{0.963583in}}%
\pgfpathlineto{\pgfqpoint{1.054448in}{0.961798in}}%
\pgfpathlineto{\pgfqpoint{1.080773in}{0.959905in}}%
\pgfpathlineto{\pgfqpoint{1.107097in}{0.957898in}}%
\pgfpathlineto{\pgfqpoint{1.133422in}{0.955772in}}%
\pgfpathlineto{\pgfqpoint{1.159747in}{0.953522in}}%
\pgfpathlineto{\pgfqpoint{1.186072in}{0.951141in}}%
\pgfpathlineto{\pgfqpoint{1.212396in}{0.948625in}}%
\pgfpathlineto{\pgfqpoint{1.238721in}{0.945970in}}%
\pgfpathlineto{\pgfqpoint{1.265046in}{0.943168in}}%
\pgfpathlineto{\pgfqpoint{1.291371in}{0.940217in}}%
\pgfpathlineto{\pgfqpoint{1.317695in}{0.937110in}}%
\pgfpathlineto{\pgfqpoint{1.344020in}{0.933844in}}%
\pgfpathlineto{\pgfqpoint{1.370345in}{0.930415in}}%
\pgfpathlineto{\pgfqpoint{1.396670in}{0.926818in}}%
\pgfpathlineto{\pgfqpoint{1.422994in}{0.923051in}}%
\pgfpathlineto{\pgfqpoint{1.449319in}{0.919112in}}%
\pgfpathlineto{\pgfqpoint{1.475644in}{0.914997in}}%
\pgfpathlineto{\pgfqpoint{1.501969in}{0.910707in}}%
\pgfpathlineto{\pgfqpoint{1.528293in}{0.906240in}}%
\pgfpathlineto{\pgfqpoint{1.554618in}{0.901597in}}%
\pgfpathlineto{\pgfqpoint{1.580943in}{0.896778in}}%
\pgfpathlineto{\pgfqpoint{1.607268in}{0.891788in}}%
\pgfpathlineto{\pgfqpoint{1.633592in}{0.886627in}}%
\pgfpathlineto{\pgfqpoint{1.659917in}{0.881302in}}%
\pgfpathlineto{\pgfqpoint{1.686242in}{0.875818in}}%
\pgfpathlineto{\pgfqpoint{1.712567in}{0.870180in}}%
\pgfpathlineto{\pgfqpoint{1.738891in}{0.864397in}}%
\pgfpathlineto{\pgfqpoint{1.765216in}{0.858478in}}%
\pgfpathlineto{\pgfqpoint{1.791541in}{0.852433in}}%
\pgfpathlineto{\pgfqpoint{1.817866in}{0.846272in}}%
\pgfpathlineto{\pgfqpoint{1.844190in}{0.840008in}}%
\pgfpathlineto{\pgfqpoint{1.870515in}{0.833654in}}%
\pgfpathlineto{\pgfqpoint{1.896840in}{0.827223in}}%
\pgfpathlineto{\pgfqpoint{1.923165in}{0.820729in}}%
\pgfpathlineto{\pgfqpoint{1.949489in}{0.814189in}}%
\pgfpathlineto{\pgfqpoint{1.975814in}{0.807617in}}%
\pgfpathlineto{\pgfqpoint{2.002139in}{0.801028in}}%
\pgfpathlineto{\pgfqpoint{2.028464in}{0.794440in}}%
\pgfpathlineto{\pgfqpoint{2.054788in}{0.787868in}}%
\pgfpathlineto{\pgfqpoint{2.081113in}{0.781327in}}%
\pgfpathlineto{\pgfqpoint{2.107438in}{0.774834in}}%
\pgfpathlineto{\pgfqpoint{2.133763in}{0.768403in}}%
\pgfpathlineto{\pgfqpoint{2.160087in}{0.762049in}}%
\pgfpathlineto{\pgfqpoint{2.186412in}{0.755784in}}%
\pgfpathlineto{\pgfqpoint{2.212737in}{0.749624in}}%
\pgfpathlineto{\pgfqpoint{2.239062in}{0.743578in}}%
\pgfpathlineto{\pgfqpoint{2.265386in}{0.737659in}}%
\pgfpathlineto{\pgfqpoint{2.291711in}{0.731877in}}%
\pgfpathlineto{\pgfqpoint{2.318036in}{0.726239in}}%
\pgfpathlineto{\pgfqpoint{2.344361in}{0.720755in}}%
\pgfpathlineto{\pgfqpoint{2.370685in}{0.715429in}}%
\pgfpathlineto{\pgfqpoint{2.397010in}{0.710269in}}%
\pgfpathlineto{\pgfqpoint{2.423335in}{0.705278in}}%
\pgfpathlineto{\pgfqpoint{2.449660in}{0.700460in}}%
\pgfpathlineto{\pgfqpoint{2.475984in}{0.695817in}}%
\pgfpathlineto{\pgfqpoint{2.502309in}{0.691350in}}%
\pgfpathlineto{\pgfqpoint{2.528634in}{0.687059in}}%
\pgfpathlineto{\pgfqpoint{2.554959in}{0.682945in}}%
\pgfpathlineto{\pgfqpoint{2.581283in}{0.679005in}}%
\pgfpathlineto{\pgfqpoint{2.607608in}{0.675238in}}%
\pgfpathlineto{\pgfqpoint{2.633933in}{0.671642in}}%
\pgfpathlineto{\pgfqpoint{2.660258in}{0.668213in}}%
\pgfpathlineto{\pgfqpoint{2.686582in}{0.664947in}}%
\pgfpathlineto{\pgfqpoint{2.712907in}{0.661840in}}%
\pgfpathlineto{\pgfqpoint{2.739232in}{0.658888in}}%
\pgfpathlineto{\pgfqpoint{2.765557in}{0.656087in}}%
\pgfpathlineto{\pgfqpoint{2.791881in}{0.653431in}}%
\pgfpathlineto{\pgfqpoint{2.818206in}{0.650916in}}%
\pgfpathlineto{\pgfqpoint{2.844531in}{0.648535in}}%
\pgfpathlineto{\pgfqpoint{2.870856in}{0.646285in}}%
\pgfpathlineto{\pgfqpoint{2.897180in}{0.644158in}}%
\pgfpathlineto{\pgfqpoint{2.923505in}{0.642151in}}%
\pgfpathlineto{\pgfqpoint{2.949830in}{0.640258in}}%
\pgfpathlineto{\pgfqpoint{2.976155in}{0.638474in}}%
\pgfpathlineto{\pgfqpoint{3.002479in}{0.636793in}}%
\pgfpathlineto{\pgfqpoint{3.028804in}{0.635211in}}%
\pgfpathlineto{\pgfqpoint{3.055129in}{0.633722in}}%
\pgfpathlineto{\pgfqpoint{3.081454in}{0.632322in}}%
\pgfpathlineto{\pgfqpoint{3.107778in}{0.631006in}}%
\pgfpathlineto{\pgfqpoint{3.134103in}{0.629770in}}%
\pgfpathlineto{\pgfqpoint{3.160428in}{0.628610in}}%
\pgfpathlineto{\pgfqpoint{3.186753in}{0.627521in}}%
\pgfpathlineto{\pgfqpoint{3.213077in}{0.626499in}}%
\pgfpathlineto{\pgfqpoint{3.239402in}{0.625540in}}%
\pgfpathlineto{\pgfqpoint{3.265727in}{0.624642in}}%
\pgfpathlineto{\pgfqpoint{3.292052in}{0.623800in}}%
\pgfpathlineto{\pgfqpoint{3.318376in}{0.623011in}}%
\pgfusepath{stroke}%
\end{pgfscope}%
\begin{pgfscope}%
\pgfsetrectcap%
\pgfsetmiterjoin%
\pgfsetlinewidth{0.803000pt}%
\definecolor{currentstroke}{rgb}{0.000000,0.000000,0.000000}%
\pgfsetstrokecolor{currentstroke}%
\pgfsetdash{}{0pt}%
\pgfpathmoveto{\pgfqpoint{0.554278in}{0.592778in}}%
\pgfpathlineto{\pgfqpoint{0.554278in}{1.009279in}}%
\pgfusepath{stroke}%
\end{pgfscope}%
\begin{pgfscope}%
\pgfsetrectcap%
\pgfsetmiterjoin%
\pgfsetlinewidth{0.803000pt}%
\definecolor{currentstroke}{rgb}{0.000000,0.000000,0.000000}%
\pgfsetstrokecolor{currentstroke}%
\pgfsetdash{}{0pt}%
\pgfpathmoveto{\pgfqpoint{3.450000in}{0.592778in}}%
\pgfpathlineto{\pgfqpoint{3.450000in}{1.009279in}}%
\pgfusepath{stroke}%
\end{pgfscope}%
\begin{pgfscope}%
\pgfsetrectcap%
\pgfsetmiterjoin%
\pgfsetlinewidth{0.803000pt}%
\definecolor{currentstroke}{rgb}{0.000000,0.000000,0.000000}%
\pgfsetstrokecolor{currentstroke}%
\pgfsetdash{}{0pt}%
\pgfpathmoveto{\pgfqpoint{0.554278in}{0.592778in}}%
\pgfpathlineto{\pgfqpoint{3.450000in}{0.592778in}}%
\pgfusepath{stroke}%
\end{pgfscope}%
\begin{pgfscope}%
\pgfsetrectcap%
\pgfsetmiterjoin%
\pgfsetlinewidth{0.803000pt}%
\definecolor{currentstroke}{rgb}{0.000000,0.000000,0.000000}%
\pgfsetstrokecolor{currentstroke}%
\pgfsetdash{}{0pt}%
\pgfpathmoveto{\pgfqpoint{0.554278in}{1.009279in}}%
\pgfpathlineto{\pgfqpoint{3.450000in}{1.009279in}}%
\pgfusepath{stroke}%
\end{pgfscope}%
\end{pgfpicture}%
\makeatother%
\endgroup%

            \caption{smearing of $N_S$ due to increasing $\Gamma$, smearing of $f$ due to increasing $T$}
            \label{fig:bcs:dos-fermi}
        \end{figure}

    
    \subsection{Tunnel Current}
    \label{subsec:bcs:tunnel-current}

        Tunneling spectroscopy offers a powerful means to probe the quasi-particle excitation spectrum of superconductors in a controlled and conceptually simple way. The key idea is that if two electrodes are separated by a sufficiently thin insulating barrier, quasi-particles can quantum-mechanically tunnel between them even though classically forbidden. 
        
        In the tunneling limit, the barrier is high and wide enough that the process is incoherent and each quasi-particle tunnels independently, so momentum conservation is effectively relaxed. 
        
        Under these conditions, the tunnel current from material $1$ to $2$ is given by
        \begin{equation}
            I_{1\to2}(V) \propto \int_{-\infty}^\infty \left(\frac{N_1(E)}{N_0} f_1(E) \right)\cdot \left(\frac{N_2(E+eV)}{N_0} \left(1-f_2(E+eV)\right)\right)\mathrm{d}E\,,
            \label{eq:tunnel-1to2}
        \end{equation}
        where $eV$ is an externally applied voltage bias. The first part in Equation \ref{eq:tunnel-1to2}, is given by the occupied states in material $1$, the second part is given by the unoccupied states in material $2$. However, in order to get the total tunnel current, one have to substract the reverse tunnel current $I=I_{1\to2}-I_{2\to 1}$. 

        Considering any combination of normal- and superconducting materials, we get the following formulas.
        \begin{align}
            I_\mathrm{NN}(V) &= G_\mathrm{N}\cdot V
            \label{eq:tunnel-nn}\\
            I_\mathrm{NS}(V) &= G_\mathrm{N} \int_{-\infty}^{\infty}\frac{N_\mathrm{S}(E)}{N_0}\left[f(E) - f(E + eV)\right] \mathrm{d}E\
            \label{eq:tunnel-ns}\\
            I_\mathrm{SS}(V) &= G_\mathrm{N}\int_{-\infty}^{\infty} \frac{N_\mathrm{S}(E)}{N_0} \cdot \frac{N_\mathrm{S}(E+eV)}{N_0} \cdot \left[f(E) - f(E + eV)\right] \mathrm{d}E
            \label{eq:tunnel-ss}
        \end{align}
        All geometric factors of the tunnel barrier, along with $N_0$ are collapsing into the normal conductance $G_\mathrm{N}$.
        
        The difference $f_1(E)-f_2(E+eV)$ accounts for the imbalance in occupation between the two electrodes induced by the applied bias voltage $V$, ensuring that current flows only when filled states on one side overlap with empty states on the other. In this picture, the densities of states $N_1(E)$ and $N_2(E)$ define \emph{where} electrons can tunnel, while the Fermi--Dirac distributions define \emph{which} of those states are populated. The convolution of these terms thus directly connects the microscopic electronic structure to the measurable current--voltage characteristics.

        The expressions derived above not only describe how tunneling currents arise but also provide a direct route to interpret experimental data. In scanning tunneling microscopy (STM), for example, a normal-metal tip above a superconducting surface realizes an NIS junction. At sufficiently low temperature, the differential conductance $\mathrm{d}I/\mathrm{d}V(V)$ is proportional to the superconducting density of states $N_\mathrm{S}(E)$, evaluated at $E = eV$. Thus, by measuring $I_\mathrm{NS}(V)$ or $\mathrm{d}I/\mathrm{d}V$, one directly maps out the energy-resolved quasiparticle spectrum of the superconductor.

        In SIS or SS junctions, both electrodes contribute gapped densities of states. Their convolution, together with thermal and lifetime broadening, determines the observed shape of the $I(V)$ and $\mathrm{d}I/\mathrm{d}V(V)$ curves. Importantly, temperature $T$ and Dynes broadening $\Gamma$ influence the spectra in distinct ways: finite temperature broadens the Fermi edges via the Fermi--Dirac distribution, while $\Gamma$ introduces intrinsic smearing of the quasiparticle DOS itself. By fitting measured $I(V)$ or $\mathrm{d}I/\mathrm{d}V(V)$ data with both parameters as variables, one can disentangle thermal effects from genuine lifetime or inelastic processes.

        These interpretations form the foundation of tunneling spectroscopy as a quantitative probe of superconductivity, allowing the extraction of $\Delta$, $\Gamma$, and $T$ from experimental data with high accuracy.

        \begin{figure}
            \centering
            %% Creator: Matplotlib, PGF backend
%%
%% To include the figure in your LaTeX document, write
%%   \input{<filename>.pgf}
%%
%% Make sure the required packages are loaded in your preamble
%%   \usepackage{pgf}
%%
%% Also ensure that all the required font packages are loaded; for instance,
%% the lmodern package is sometimes necessary when using math font.
%%   \usepackage{lmodern}
%%
%% Figures using additional raster images can only be included by \input if
%% they are in the same directory as the main LaTeX file. For loading figures
%% from other directories you can use the `import` package
%%   \usepackage{import}
%%
%% and then include the figures with
%%   \import{<path to file>}{<filename>.pgf}
%%
%% Matplotlib used the following preamble
%%   \def\mathdefault#1{#1}
%%   \everymath=\expandafter{\the\everymath\displaystyle}
%%   \IfFileExists{scrextend.sty}{
%%     \usepackage[fontsize=10.000000pt]{scrextend}
%%   }{
%%     \renewcommand{\normalsize}{\fontsize{10.000000}{12.000000}\selectfont}
%%     \normalsize
%%   }
%%   
%%   \ifdefined\pdftexversion\else  % non-pdftex case.
%%     \usepackage{fontspec}
%%     \setmainfont{DejaVuSerif.ttf}[Path=\detokenize{/Users/oliver/.pyenv/versions/3.13.3/lib/python3.13/site-packages/matplotlib/mpl-data/fonts/ttf/}]
%%     \setsansfont{DejaVuSans.ttf}[Path=\detokenize{/Users/oliver/.pyenv/versions/3.13.3/lib/python3.13/site-packages/matplotlib/mpl-data/fonts/ttf/}]
%%     \setmonofont{DejaVuSansMono.ttf}[Path=\detokenize{/Users/oliver/.pyenv/versions/3.13.3/lib/python3.13/site-packages/matplotlib/mpl-data/fonts/ttf/}]
%%   \fi
%%   \makeatletter\@ifpackageloaded{underscore}{}{\usepackage[strings]{underscore}}\makeatother
%%
\begingroup%
\makeatletter%
\begin{pgfpicture}%
\pgfpathrectangle{\pgfpointorigin}{\pgfqpoint{3.000000in}{1.000000in}}%
\pgfusepath{use as bounding box, clip}%
\begin{pgfscope}%
\pgfsetbuttcap%
\pgfsetmiterjoin%
\definecolor{currentfill}{rgb}{1.000000,1.000000,1.000000}%
\pgfsetfillcolor{currentfill}%
\pgfsetlinewidth{0.000000pt}%
\definecolor{currentstroke}{rgb}{1.000000,1.000000,1.000000}%
\pgfsetstrokecolor{currentstroke}%
\pgfsetdash{}{0pt}%
\pgfpathmoveto{\pgfqpoint{0.000000in}{0.000000in}}%
\pgfpathlineto{\pgfqpoint{3.000000in}{0.000000in}}%
\pgfpathlineto{\pgfqpoint{3.000000in}{1.000000in}}%
\pgfpathlineto{\pgfqpoint{0.000000in}{1.000000in}}%
\pgfpathlineto{\pgfqpoint{0.000000in}{0.000000in}}%
\pgfpathclose%
\pgfusepath{fill}%
\end{pgfscope}%
\begin{pgfscope}%
\pgfsetbuttcap%
\pgfsetmiterjoin%
\definecolor{currentfill}{rgb}{1.000000,1.000000,1.000000}%
\pgfsetfillcolor{currentfill}%
\pgfsetlinewidth{0.000000pt}%
\definecolor{currentstroke}{rgb}{0.000000,0.000000,0.000000}%
\pgfsetstrokecolor{currentstroke}%
\pgfsetstrokeopacity{0.000000}%
\pgfsetdash{}{0pt}%
\pgfpathmoveto{\pgfqpoint{0.565528in}{0.616778in}}%
\pgfpathlineto{\pgfqpoint{2.801932in}{0.616778in}}%
\pgfpathlineto{\pgfqpoint{2.801932in}{0.830000in}}%
\pgfpathlineto{\pgfqpoint{0.565528in}{0.830000in}}%
\pgfpathlineto{\pgfqpoint{0.565528in}{0.616778in}}%
\pgfpathclose%
\pgfusepath{fill}%
\end{pgfscope}%
\begin{pgfscope}%
\pgfsetbuttcap%
\pgfsetroundjoin%
\definecolor{currentfill}{rgb}{0.000000,0.000000,0.000000}%
\pgfsetfillcolor{currentfill}%
\pgfsetlinewidth{0.803000pt}%
\definecolor{currentstroke}{rgb}{0.000000,0.000000,0.000000}%
\pgfsetstrokecolor{currentstroke}%
\pgfsetdash{}{0pt}%
\pgfsys@defobject{currentmarker}{\pgfqpoint{0.000000in}{-0.048611in}}{\pgfqpoint{0.000000in}{0.000000in}}{%
\pgfpathmoveto{\pgfqpoint{0.000000in}{0.000000in}}%
\pgfpathlineto{\pgfqpoint{0.000000in}{-0.048611in}}%
\pgfusepath{stroke,fill}%
}%
\begin{pgfscope}%
\pgfsys@transformshift{0.667183in}{0.616778in}%
\pgfsys@useobject{currentmarker}{}%
\end{pgfscope}%
\end{pgfscope}%
\begin{pgfscope}%
\definecolor{textcolor}{rgb}{0.000000,0.000000,0.000000}%
\pgfsetstrokecolor{textcolor}%
\pgfsetfillcolor{textcolor}%
\pgftext[x=0.667183in,y=0.519556in,,top]{\color{textcolor}{\sffamily\fontsize{10.000000}{12.000000}\selectfont\catcode`\^=\active\def^{\ifmmode\sp\else\^{}\fi}\catcode`\%=\active\def%{\%}0.00}}%
\end{pgfscope}%
\begin{pgfscope}%
\pgfsetbuttcap%
\pgfsetroundjoin%
\definecolor{currentfill}{rgb}{0.000000,0.000000,0.000000}%
\pgfsetfillcolor{currentfill}%
\pgfsetlinewidth{0.803000pt}%
\definecolor{currentstroke}{rgb}{0.000000,0.000000,0.000000}%
\pgfsetstrokecolor{currentstroke}%
\pgfsetdash{}{0pt}%
\pgfsys@defobject{currentmarker}{\pgfqpoint{0.000000in}{-0.048611in}}{\pgfqpoint{0.000000in}{0.000000in}}{%
\pgfpathmoveto{\pgfqpoint{0.000000in}{0.000000in}}%
\pgfpathlineto{\pgfqpoint{0.000000in}{-0.048611in}}%
\pgfusepath{stroke,fill}%
}%
\begin{pgfscope}%
\pgfsys@transformshift{1.175456in}{0.616778in}%
\pgfsys@useobject{currentmarker}{}%
\end{pgfscope}%
\end{pgfscope}%
\begin{pgfscope}%
\definecolor{textcolor}{rgb}{0.000000,0.000000,0.000000}%
\pgfsetstrokecolor{textcolor}%
\pgfsetfillcolor{textcolor}%
\pgftext[x=1.175456in,y=0.519556in,,top]{\color{textcolor}{\sffamily\fontsize{10.000000}{12.000000}\selectfont\catcode`\^=\active\def^{\ifmmode\sp\else\^{}\fi}\catcode`\%=\active\def%{\%}0.25}}%
\end{pgfscope}%
\begin{pgfscope}%
\pgfsetbuttcap%
\pgfsetroundjoin%
\definecolor{currentfill}{rgb}{0.000000,0.000000,0.000000}%
\pgfsetfillcolor{currentfill}%
\pgfsetlinewidth{0.803000pt}%
\definecolor{currentstroke}{rgb}{0.000000,0.000000,0.000000}%
\pgfsetstrokecolor{currentstroke}%
\pgfsetdash{}{0pt}%
\pgfsys@defobject{currentmarker}{\pgfqpoint{0.000000in}{-0.048611in}}{\pgfqpoint{0.000000in}{0.000000in}}{%
\pgfpathmoveto{\pgfqpoint{0.000000in}{0.000000in}}%
\pgfpathlineto{\pgfqpoint{0.000000in}{-0.048611in}}%
\pgfusepath{stroke,fill}%
}%
\begin{pgfscope}%
\pgfsys@transformshift{1.683730in}{0.616778in}%
\pgfsys@useobject{currentmarker}{}%
\end{pgfscope}%
\end{pgfscope}%
\begin{pgfscope}%
\definecolor{textcolor}{rgb}{0.000000,0.000000,0.000000}%
\pgfsetstrokecolor{textcolor}%
\pgfsetfillcolor{textcolor}%
\pgftext[x=1.683730in,y=0.519556in,,top]{\color{textcolor}{\sffamily\fontsize{10.000000}{12.000000}\selectfont\catcode`\^=\active\def^{\ifmmode\sp\else\^{}\fi}\catcode`\%=\active\def%{\%}0.50}}%
\end{pgfscope}%
\begin{pgfscope}%
\pgfsetbuttcap%
\pgfsetroundjoin%
\definecolor{currentfill}{rgb}{0.000000,0.000000,0.000000}%
\pgfsetfillcolor{currentfill}%
\pgfsetlinewidth{0.803000pt}%
\definecolor{currentstroke}{rgb}{0.000000,0.000000,0.000000}%
\pgfsetstrokecolor{currentstroke}%
\pgfsetdash{}{0pt}%
\pgfsys@defobject{currentmarker}{\pgfqpoint{0.000000in}{-0.048611in}}{\pgfqpoint{0.000000in}{0.000000in}}{%
\pgfpathmoveto{\pgfqpoint{0.000000in}{0.000000in}}%
\pgfpathlineto{\pgfqpoint{0.000000in}{-0.048611in}}%
\pgfusepath{stroke,fill}%
}%
\begin{pgfscope}%
\pgfsys@transformshift{2.192003in}{0.616778in}%
\pgfsys@useobject{currentmarker}{}%
\end{pgfscope}%
\end{pgfscope}%
\begin{pgfscope}%
\definecolor{textcolor}{rgb}{0.000000,0.000000,0.000000}%
\pgfsetstrokecolor{textcolor}%
\pgfsetfillcolor{textcolor}%
\pgftext[x=2.192003in,y=0.519556in,,top]{\color{textcolor}{\sffamily\fontsize{10.000000}{12.000000}\selectfont\catcode`\^=\active\def^{\ifmmode\sp\else\^{}\fi}\catcode`\%=\active\def%{\%}0.75}}%
\end{pgfscope}%
\begin{pgfscope}%
\pgfsetbuttcap%
\pgfsetroundjoin%
\definecolor{currentfill}{rgb}{0.000000,0.000000,0.000000}%
\pgfsetfillcolor{currentfill}%
\pgfsetlinewidth{0.803000pt}%
\definecolor{currentstroke}{rgb}{0.000000,0.000000,0.000000}%
\pgfsetstrokecolor{currentstroke}%
\pgfsetdash{}{0pt}%
\pgfsys@defobject{currentmarker}{\pgfqpoint{0.000000in}{-0.048611in}}{\pgfqpoint{0.000000in}{0.000000in}}{%
\pgfpathmoveto{\pgfqpoint{0.000000in}{0.000000in}}%
\pgfpathlineto{\pgfqpoint{0.000000in}{-0.048611in}}%
\pgfusepath{stroke,fill}%
}%
\begin{pgfscope}%
\pgfsys@transformshift{2.700277in}{0.616778in}%
\pgfsys@useobject{currentmarker}{}%
\end{pgfscope}%
\end{pgfscope}%
\begin{pgfscope}%
\definecolor{textcolor}{rgb}{0.000000,0.000000,0.000000}%
\pgfsetstrokecolor{textcolor}%
\pgfsetfillcolor{textcolor}%
\pgftext[x=2.700277in,y=0.519556in,,top]{\color{textcolor}{\sffamily\fontsize{10.000000}{12.000000}\selectfont\catcode`\^=\active\def^{\ifmmode\sp\else\^{}\fi}\catcode`\%=\active\def%{\%}1.00}}%
\end{pgfscope}%
\begin{pgfscope}%
\definecolor{textcolor}{rgb}{0.000000,0.000000,0.000000}%
\pgfsetstrokecolor{textcolor}%
\pgfsetfillcolor{textcolor}%
\pgftext[x=1.683730in,y=0.329587in,,top]{\color{textcolor}{\sffamily\fontsize{10.000000}{12.000000}\selectfont\catcode`\^=\active\def^{\ifmmode\sp\else\^{}\fi}\catcode`\%=\active\def%{\%}$T$ ($T_\mathrm{C}$)}}%
\end{pgfscope}%
\begin{pgfscope}%
\pgfsetbuttcap%
\pgfsetroundjoin%
\definecolor{currentfill}{rgb}{0.000000,0.000000,0.000000}%
\pgfsetfillcolor{currentfill}%
\pgfsetlinewidth{0.803000pt}%
\definecolor{currentstroke}{rgb}{0.000000,0.000000,0.000000}%
\pgfsetstrokecolor{currentstroke}%
\pgfsetdash{}{0pt}%
\pgfsys@defobject{currentmarker}{\pgfqpoint{-0.048611in}{0.000000in}}{\pgfqpoint{-0.000000in}{0.000000in}}{%
\pgfpathmoveto{\pgfqpoint{-0.000000in}{0.000000in}}%
\pgfpathlineto{\pgfqpoint{-0.048611in}{0.000000in}}%
\pgfusepath{stroke,fill}%
}%
\begin{pgfscope}%
\pgfsys@transformshift{0.565528in}{0.626470in}%
\pgfsys@useobject{currentmarker}{}%
\end{pgfscope}%
\end{pgfscope}%
\begin{pgfscope}%
\definecolor{textcolor}{rgb}{0.000000,0.000000,0.000000}%
\pgfsetstrokecolor{textcolor}%
\pgfsetfillcolor{textcolor}%
\pgftext[x=0.379940in, y=0.573708in, left, base]{\color{textcolor}{\sffamily\fontsize{10.000000}{12.000000}\selectfont\catcode`\^=\active\def^{\ifmmode\sp\else\^{}\fi}\catcode`\%=\active\def%{\%}0}}%
\end{pgfscope}%
\begin{pgfscope}%
\pgfsetbuttcap%
\pgfsetroundjoin%
\definecolor{currentfill}{rgb}{0.000000,0.000000,0.000000}%
\pgfsetfillcolor{currentfill}%
\pgfsetlinewidth{0.803000pt}%
\definecolor{currentstroke}{rgb}{0.000000,0.000000,0.000000}%
\pgfsetstrokecolor{currentstroke}%
\pgfsetdash{}{0pt}%
\pgfsys@defobject{currentmarker}{\pgfqpoint{-0.048611in}{0.000000in}}{\pgfqpoint{-0.000000in}{0.000000in}}{%
\pgfpathmoveto{\pgfqpoint{-0.000000in}{0.000000in}}%
\pgfpathlineto{\pgfqpoint{-0.048611in}{0.000000in}}%
\pgfusepath{stroke,fill}%
}%
\begin{pgfscope}%
\pgfsys@transformshift{0.565528in}{0.820308in}%
\pgfsys@useobject{currentmarker}{}%
\end{pgfscope}%
\end{pgfscope}%
\begin{pgfscope}%
\definecolor{textcolor}{rgb}{0.000000,0.000000,0.000000}%
\pgfsetstrokecolor{textcolor}%
\pgfsetfillcolor{textcolor}%
\pgftext[x=0.379940in, y=0.767547in, left, base]{\color{textcolor}{\sffamily\fontsize{10.000000}{12.000000}\selectfont\catcode`\^=\active\def^{\ifmmode\sp\else\^{}\fi}\catcode`\%=\active\def%{\%}1}}%
\end{pgfscope}%
\begin{pgfscope}%
\definecolor{textcolor}{rgb}{0.000000,0.000000,0.000000}%
\pgfsetstrokecolor{textcolor}%
\pgfsetfillcolor{textcolor}%
\pgftext[x=0.324385in,y=0.723389in,,bottom,rotate=90.000000]{\color{textcolor}{\sffamily\fontsize{10.000000}{12.000000}\selectfont\catcode`\^=\active\def^{\ifmmode\sp\else\^{}\fi}\catcode`\%=\active\def%{\%}$\Delta$ ($\Delta_0$)}}%
\end{pgfscope}%
\begin{pgfscope}%
\pgfpathrectangle{\pgfqpoint{0.565528in}{0.616778in}}{\pgfqpoint{2.236404in}{0.213222in}}%
\pgfusepath{clip}%
\pgfsetrectcap%
\pgfsetroundjoin%
\pgfsetlinewidth{1.505625pt}%
\definecolor{currentstroke}{rgb}{0.121569,0.466667,0.705882}%
\pgfsetstrokecolor{currentstroke}%
\pgfsetdash{}{0pt}%
\pgfpathmoveto{\pgfqpoint{0.667183in}{0.820308in}}%
\pgfpathlineto{\pgfqpoint{0.687513in}{0.820308in}}%
\pgfpathlineto{\pgfqpoint{0.707844in}{0.820308in}}%
\pgfpathlineto{\pgfqpoint{0.728175in}{0.820308in}}%
\pgfpathlineto{\pgfqpoint{0.748506in}{0.820308in}}%
\pgfpathlineto{\pgfqpoint{0.768837in}{0.820308in}}%
\pgfpathlineto{\pgfqpoint{0.789168in}{0.820308in}}%
\pgfpathlineto{\pgfqpoint{0.809499in}{0.820307in}}%
\pgfpathlineto{\pgfqpoint{0.829830in}{0.820305in}}%
\pgfpathlineto{\pgfqpoint{0.850161in}{0.820302in}}%
\pgfpathlineto{\pgfqpoint{0.870492in}{0.820297in}}%
\pgfpathlineto{\pgfqpoint{0.890823in}{0.820289in}}%
\pgfpathlineto{\pgfqpoint{0.911154in}{0.820277in}}%
\pgfpathlineto{\pgfqpoint{0.931485in}{0.820260in}}%
\pgfpathlineto{\pgfqpoint{0.951816in}{0.820238in}}%
\pgfpathlineto{\pgfqpoint{0.972147in}{0.820210in}}%
\pgfpathlineto{\pgfqpoint{0.992478in}{0.820175in}}%
\pgfpathlineto{\pgfqpoint{1.012809in}{0.820131in}}%
\pgfpathlineto{\pgfqpoint{1.033140in}{0.820078in}}%
\pgfpathlineto{\pgfqpoint{1.053470in}{0.820015in}}%
\pgfpathlineto{\pgfqpoint{1.073801in}{0.819940in}}%
\pgfpathlineto{\pgfqpoint{1.094132in}{0.819855in}}%
\pgfpathlineto{\pgfqpoint{1.114463in}{0.819756in}}%
\pgfpathlineto{\pgfqpoint{1.134794in}{0.819644in}}%
\pgfpathlineto{\pgfqpoint{1.155125in}{0.819517in}}%
\pgfpathlineto{\pgfqpoint{1.175456in}{0.819375in}}%
\pgfpathlineto{\pgfqpoint{1.195787in}{0.819218in}}%
\pgfpathlineto{\pgfqpoint{1.216118in}{0.819043in}}%
\pgfpathlineto{\pgfqpoint{1.236449in}{0.818852in}}%
\pgfpathlineto{\pgfqpoint{1.256780in}{0.818642in}}%
\pgfpathlineto{\pgfqpoint{1.277111in}{0.818413in}}%
\pgfpathlineto{\pgfqpoint{1.297442in}{0.818164in}}%
\pgfpathlineto{\pgfqpoint{1.317773in}{0.817895in}}%
\pgfpathlineto{\pgfqpoint{1.338104in}{0.817605in}}%
\pgfpathlineto{\pgfqpoint{1.358435in}{0.817292in}}%
\pgfpathlineto{\pgfqpoint{1.378766in}{0.816958in}}%
\pgfpathlineto{\pgfqpoint{1.399097in}{0.816600in}}%
\pgfpathlineto{\pgfqpoint{1.419428in}{0.816218in}}%
\pgfpathlineto{\pgfqpoint{1.439758in}{0.815811in}}%
\pgfpathlineto{\pgfqpoint{1.460089in}{0.815379in}}%
\pgfpathlineto{\pgfqpoint{1.480420in}{0.814921in}}%
\pgfpathlineto{\pgfqpoint{1.500751in}{0.814435in}}%
\pgfpathlineto{\pgfqpoint{1.521082in}{0.813922in}}%
\pgfpathlineto{\pgfqpoint{1.541413in}{0.813381in}}%
\pgfpathlineto{\pgfqpoint{1.561744in}{0.812810in}}%
\pgfpathlineto{\pgfqpoint{1.582075in}{0.812209in}}%
\pgfpathlineto{\pgfqpoint{1.602406in}{0.811577in}}%
\pgfpathlineto{\pgfqpoint{1.622737in}{0.810913in}}%
\pgfpathlineto{\pgfqpoint{1.643068in}{0.810217in}}%
\pgfpathlineto{\pgfqpoint{1.663399in}{0.809486in}}%
\pgfpathlineto{\pgfqpoint{1.683730in}{0.808722in}}%
\pgfpathlineto{\pgfqpoint{1.704061in}{0.807921in}}%
\pgfpathlineto{\pgfqpoint{1.724392in}{0.807085in}}%
\pgfpathlineto{\pgfqpoint{1.744723in}{0.806210in}}%
\pgfpathlineto{\pgfqpoint{1.765054in}{0.805297in}}%
\pgfpathlineto{\pgfqpoint{1.785385in}{0.804344in}}%
\pgfpathlineto{\pgfqpoint{1.805715in}{0.803350in}}%
\pgfpathlineto{\pgfqpoint{1.826046in}{0.802313in}}%
\pgfpathlineto{\pgfqpoint{1.846377in}{0.801233in}}%
\pgfpathlineto{\pgfqpoint{1.866708in}{0.800108in}}%
\pgfpathlineto{\pgfqpoint{1.887039in}{0.798937in}}%
\pgfpathlineto{\pgfqpoint{1.907370in}{0.797717in}}%
\pgfpathlineto{\pgfqpoint{1.927701in}{0.796448in}}%
\pgfpathlineto{\pgfqpoint{1.948032in}{0.795127in}}%
\pgfpathlineto{\pgfqpoint{1.968363in}{0.793753in}}%
\pgfpathlineto{\pgfqpoint{1.988694in}{0.792324in}}%
\pgfpathlineto{\pgfqpoint{2.009025in}{0.790838in}}%
\pgfpathlineto{\pgfqpoint{2.029356in}{0.789293in}}%
\pgfpathlineto{\pgfqpoint{2.049687in}{0.787685in}}%
\pgfpathlineto{\pgfqpoint{2.070018in}{0.786014in}}%
\pgfpathlineto{\pgfqpoint{2.090349in}{0.784275in}}%
\pgfpathlineto{\pgfqpoint{2.110680in}{0.782467in}}%
\pgfpathlineto{\pgfqpoint{2.131011in}{0.780586in}}%
\pgfpathlineto{\pgfqpoint{2.151342in}{0.778628in}}%
\pgfpathlineto{\pgfqpoint{2.171672in}{0.776590in}}%
\pgfpathlineto{\pgfqpoint{2.192003in}{0.774468in}}%
\pgfpathlineto{\pgfqpoint{2.212334in}{0.772257in}}%
\pgfpathlineto{\pgfqpoint{2.232665in}{0.769953in}}%
\pgfpathlineto{\pgfqpoint{2.252996in}{0.767550in}}%
\pgfpathlineto{\pgfqpoint{2.273327in}{0.765042in}}%
\pgfpathlineto{\pgfqpoint{2.293658in}{0.762423in}}%
\pgfpathlineto{\pgfqpoint{2.313989in}{0.759685in}}%
\pgfpathlineto{\pgfqpoint{2.334320in}{0.756819in}}%
\pgfpathlineto{\pgfqpoint{2.354651in}{0.753816in}}%
\pgfpathlineto{\pgfqpoint{2.374982in}{0.750666in}}%
\pgfpathlineto{\pgfqpoint{2.395313in}{0.747356in}}%
\pgfpathlineto{\pgfqpoint{2.415644in}{0.743870in}}%
\pgfpathlineto{\pgfqpoint{2.435975in}{0.740193in}}%
\pgfpathlineto{\pgfqpoint{2.456306in}{0.736303in}}%
\pgfpathlineto{\pgfqpoint{2.476637in}{0.732176in}}%
\pgfpathlineto{\pgfqpoint{2.496968in}{0.727782in}}%
\pgfpathlineto{\pgfqpoint{2.517299in}{0.723083in}}%
\pgfpathlineto{\pgfqpoint{2.537630in}{0.718030in}}%
\pgfpathlineto{\pgfqpoint{2.557960in}{0.712560in}}%
\pgfpathlineto{\pgfqpoint{2.578291in}{0.706586in}}%
\pgfpathlineto{\pgfqpoint{2.598622in}{0.699983in}}%
\pgfpathlineto{\pgfqpoint{2.618953in}{0.692560in}}%
\pgfpathlineto{\pgfqpoint{2.639284in}{0.684000in}}%
\pgfpathlineto{\pgfqpoint{2.659615in}{0.673684in}}%
\pgfpathlineto{\pgfqpoint{2.679946in}{0.660026in}}%
\pgfpathlineto{\pgfqpoint{2.700277in}{0.626470in}}%
\pgfusepath{stroke}%
\end{pgfscope}%
\begin{pgfscope}%
\pgfsetrectcap%
\pgfsetmiterjoin%
\pgfsetlinewidth{0.803000pt}%
\definecolor{currentstroke}{rgb}{0.000000,0.000000,0.000000}%
\pgfsetstrokecolor{currentstroke}%
\pgfsetdash{}{0pt}%
\pgfpathmoveto{\pgfqpoint{0.565528in}{0.616778in}}%
\pgfpathlineto{\pgfqpoint{0.565528in}{0.830000in}}%
\pgfusepath{stroke}%
\end{pgfscope}%
\begin{pgfscope}%
\pgfsetrectcap%
\pgfsetmiterjoin%
\pgfsetlinewidth{0.803000pt}%
\definecolor{currentstroke}{rgb}{0.000000,0.000000,0.000000}%
\pgfsetstrokecolor{currentstroke}%
\pgfsetdash{}{0pt}%
\pgfpathmoveto{\pgfqpoint{2.801932in}{0.616778in}}%
\pgfpathlineto{\pgfqpoint{2.801932in}{0.830000in}}%
\pgfusepath{stroke}%
\end{pgfscope}%
\begin{pgfscope}%
\pgfsetrectcap%
\pgfsetmiterjoin%
\pgfsetlinewidth{0.803000pt}%
\definecolor{currentstroke}{rgb}{0.000000,0.000000,0.000000}%
\pgfsetstrokecolor{currentstroke}%
\pgfsetdash{}{0pt}%
\pgfpathmoveto{\pgfqpoint{0.565528in}{0.616778in}}%
\pgfpathlineto{\pgfqpoint{2.801932in}{0.616778in}}%
\pgfusepath{stroke}%
\end{pgfscope}%
\begin{pgfscope}%
\pgfsetrectcap%
\pgfsetmiterjoin%
\pgfsetlinewidth{0.803000pt}%
\definecolor{currentstroke}{rgb}{0.000000,0.000000,0.000000}%
\pgfsetstrokecolor{currentstroke}%
\pgfsetdash{}{0pt}%
\pgfpathmoveto{\pgfqpoint{0.565528in}{0.830000in}}%
\pgfpathlineto{\pgfqpoint{2.801932in}{0.830000in}}%
\pgfusepath{stroke}%
\end{pgfscope}%
\end{pgfpicture}%
\makeatother%
\endgroup%

            \caption{Tunnel current, smearing, dos, fermidirac}
            \label{fig:bcs:tunnel-current}
        \end{figure}
        \textbf{cite: Giaever}

    \subsection{Photon-Assisted Tunneling (PAT)}
    \label{subsec:bcs:pat}

        While the previous section described static tunneling, the application of a time-dependent voltage enables quasi-particles to exchange discrete energy quanta with an external electromagnetic field, giving rise to photon-assisted tunneling (PAT).

        In practice, this effect is typically studied using electromagnetic radiation in the microwave range, since photon energies $h\nu$ in this regime are comparable to the superconducting energy gap $\Delta$ and can therefore induce measurable sidebands without breaking Cooper pairs or overheating the junction.

        Photon-assisted tunneling describes the exchange of photon energies between tunneling charge carriers and the microwave field. Possible energies as given by,
        \begin{equation}
            E_{\nu}(n) = n h\nu\quad (n \in \mathbb{Z})\,,
        \end{equation}
        where each photon carries an energy $h\nu$ and $n$ is the number of photons absorbed ($n>0$) or emitted ($n<0$). As a result, additional tunneling channels open at these energies, producing characteristic replicas of the original spectral features in $I(V)$.

        The Tien-Gordon model provides a simple and intuitive framework to describe this effect. It assumes that the junction operates in the tunneling limit, meaning that the barrier is sufficiently opaque so that electrons tunnel independently without preserving momentum or phase coherence. The electromagnetic field is treated classically and is represented by a spatially uniform, time-dependent voltage across the junction,
        \begin{equation}
            V(t) = V_0 + A \cos (2\pi\nu t)\,,
            \label{eq:pat-sinusoidal}
        \end{equation}
        where $V_0$ is the applied voltage bias, $\nu$ the  frequency, and $A$ the amplitude of the microwave field across the junction. The field is assumed to remain unaltered by the tunneling current, meaning no cavity or backaction effects, and the drive is weak enough not to heat the electrodes, allowing both to stay in thermal equilibrium at temperature $T$. The voltage couples linearly through the electrostatic potential $V(t)$ and introduces a time-dependent phase to the tunneling amplitude. The model therefore applies to incoherent quasiparticle tunneling and excludes phase-coherent phenomena such as the Josephson supercurrent, which are treated separately in Section~\ref{sec:josephson}.

        The microwaves continuously modulates the potential difference between the electrodes and hence the phase of the tunneling electrons. This periodic modulation generates a series of harmonic components in the tunneling amplitude, corresponding to processes in which electrons exchange discrete quanta of energy with the oscillating field. When averaged over many cycles, these contributions combine into a stationary current that can be written as a weighted sum of shifted copies of the static $I$-$V$ curve, with each shift by $n h\nu / e$ corresponding to the net energy exchange. The resulting formula is given by
        \begin{equation}
            I(V_0) = \sum_{n=-\infty}^{\infty} J_n^2\!\left( \frac{eA}{h\nu}\right) \cdot I_0\!\left(V_0 + \frac{n h\nu}{e}\right)\,.
            \label{eq:pat-tien-gordon}
        \end{equation}
        The Bessel functions of the first kind $J_n$ arise from the periodic phase modulation induced by the oscillating potential and quantify the probability for an electron to absorb or emit $n$ photons. Each term in the sum corresponds to a replica of the static $I_0(V_0)$ curve, shifted in voltage by $n h\nu / e$ and weighted by $J_n^2(eA/h\nu)$. The resulting current therefore reflects the superposition of all photon-assisted tunneling processes, which manifest as evenly spaced sidebands of the superconducting features in the measured $I(V_0)$.

        The same relation also holds for the differential conductance $\mathrm{d}I/\mathrm{d}V(V_0)$, meaning that the photon-assisted replicas $\mathrm{d}I_0/\mathrm{d}V(V_0)$ appear identically as in the current, providing a direct experimental link to the superconducting density of states.

        In practice, the amplitude $A$ can be determined experimentally by comparing the relative heights of the sidebands with the expected Bessel-function weights $J_n^2(eA/h\nu)$. The spacing $h\nu$ between sidebands provides a direct and robust calibration of the frequency. Photon-assisted tunneling thus offers a straightforward semiclassical description of how an oscillating field modifies quasi-particle transport, serving as a bridge between static tunneling spectroscopy and driven quantum dynamics.

    \newpage
    \subsection*{TODO}
    \textbf{
    \begin{itemize}
        \item $f(E), N_S(E)$
        \item schemata tunneling, nin, nis, sis
        \item $I(V, T, Gamma)$
        \item schemata pat nin, nis, sis
        \item $I(V, A, \nu)$
        \item reference the figures
        \item do the citations
    \end{itemize}
    }
    \newpage
