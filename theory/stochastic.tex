% !TEX root = ../thesis.tex

\section{Stochastic Description}
\label{sec:stochastic}
    
    In the previous sections superconducting transport was described in three complementary ways. The microscopic picture treats it as incoherent quasiparticle tunneling, the macroscopic picture describes it as coherent Cooper-pair dynamics governed by the superconducting phase, and the mesoscopic picture considers coherent electron-hole processes in high-transmission weak links. All of these descriptions rely on a well-defined superconducting phase and on the coherence of successive tunneling events. When the electromagnetic environment generates strong voltage fluctuations, this phase coherence is lost. The junction then enters the stochastic transport regime.

    In the stochastic regime, the loss of phase coherence plays a central role. The superconducting phase $\phi$ is conjugate to the charge $q$ on the junction,
    \begin{equation}
        \left[\phi, q\right] = 2e\ima\,,
        \label{eq:stochastic:phase-charge}
    \end{equation}
    implying that a well-defined phase requires charge to be delocalized, whereas localized charge leads to strong phase fluctuations. Any electromagnetic environment connected to the junction produces voltage fluctuations $\delta V(t)$, which translate into fluctuations of the phase via the AC Josephson relation (Eq.~\ref{eq:macro:ac}),
    \begin{equation}
        \delta \phi(t) = \frac{2e}{\hbar}\int_0^t \delta V(t')\,\mathrm{d}t'\,.
    \end{equation}

    Large phase fluctuations randomize the phase so strongly that the average of the phase-dependent factor vanishes
    \begin{equation}
        \left\langle \exp(\ima\phi(t))\right\rangle = 0\,.
    \end{equation}
    This loss of a non-zero phase average signals the destruction of long-range phase coherence. Consequently, the coherent Josephson effect, Andreev bound states, and multiple Andreev reflections no longer exist. Transport becomes a sequence of  incoherent tunneling events whose rates are governed by the energy exchanged with the electromagnetic environment. The crossover between coherent and incoherent regimes is controlled by the strength of the environment. Weak damping ($\mathrm{Re}\,Z(\omega) \ll R_\mathrm{Q}$) preserves phase coherence, whereas strong damping ($\mathrm{Re}\,Z(\omega) \sim R_\mathrm{Q}$) leads to phase diffusion and marks the onset of the stochastic transport regime.

    So, charge transfer no longer proceeds through coherent condensate dynamics or well-defined quasiparticle trajectories, but instead through discrete and statistically independent tunneling events. The electromagnetic environment can absorb or emit energy during each event, so that tunneling rates are determined not only by the electronic density of states but also by the impedance of the surrounding circuit. This interplay is captured by the \textit{P(E)}--theory, which provides the universal framework for describing energy exchange between a tunnel junction and its environment.

    The stochastic description therefore complements the microscopic, macroscopic, and mesoscopic frameworks by covering the fully incoherent limit of superconducting transport. The following sections introduce the origin of phase fluctuations, the \textit{P(E)}--formalism, and the resulting phenomena: dynamical Coulomb blockade of single-electron tunneling\footnote{In the microscopic description, tunneling is elastic and proceeds between well-defined BCS quasi-particle states, which we refer to as quasi-particle tunneling. In contrast, the stochastic description considers single-electron tunneling events dressed by environmental fluctuations, where energy exchange with the environment renders the process inelastic and probabilistic.}, incoherent Cooper-pair tunneling and its photon-assisted counterpart, incoherent Andreev reflection in the absence of phase coherence, and finally the superconducting single-electron transistor as a device in which these processes combine in a controlled and experimentally relevant manner.
    

    \subsection[P(E)--Theory]{\textit{P(E)}--Theory}
    \label{subsec:stochastic:pe-theory}

        In the presence of strong phase fluctuations, charge transport through a junction occurs as a sequence of independent tunneling events. Because the electromagnetic environment can absorb or emit energy during such a process, the tunneling rate is determined not only by the electronic density of states but also by the probability $P(E)$ that the environment exchanges an energy $E$ with the junction. This renders the tunneling process inelastic even at zero temperature and forms the basis of the stochastic description of superconducting transport.

        The statistical properties of these fluctuations are determined by the environmental impedance $Z(\omega)$. Within linear-response theory, the phase correlation function can be written as
        \begin{equation}
            J(t) = \frac{2}{R_\mathrm{Q}} \int_0^\infty \frac{\mathrm{Re}\,Z(\omega)}{\omega}
            \coth\!\left(\frac{\hbar\omega}{2k_\mathrm{B}T}\right) \left(\cos(\omega t) - 1\right)
            - \ima\sin(\omega t)\,\mathrm{d}\omega\,,
            \label{eq:stochastic:Jt}
        \end{equation}
        where $R_\mathrm{Q}$ is the superconducting resistance quantum\footnote{The superconducting resistance quantum is $R_\mathrm{Q} = h/(2e)^2 \approx 6.453\,\mathrm{k\Omega}$, which appears when phase fluctuations couple to Cooper-pair charge $2e$. The quantity $R_0 = h/2e^2 = 12.9\,\mathrm{k\Omega}$ is the resistance quantum for a single spin-resolved electronic channel. In this thesis the conductance quantum is defined as $G_0 = 2e^2/h = 77.48\,\mu\mathrm{S}$, which includes spin degeneracy and corresponds to the conductance of a fully transmitting normal-state channel. $R_\mathrm{Q}$ governs the strength of phase fluctuations in the stochastic description, while $R_0$ and $G_0$ appear in mesoscopic transport and Landauer-type expressions.}.

        The structure of Eq.~\ref{eq:stochastic:Jt} reflects how the electromagnetic environment shapes the phase dynamics. The factor $\mathrm{Re}\,Z(\omega)$ captures the dissipative part of the environment, which determines how strongly voltage fluctuations couple to the junction. It appears divided by $\omega$ because the phase is the time integral of the voltage, so that low-frequency components contribute most strongly to its fluctuations. The thermal occupation of each environmental mode enters through the factor $\coth(\beta\hbar\omega/2)$, ensuring that both quantum and thermal noise are included. The combination $\cos(\omega t)-1$ forms the real part of $J(t)$ and describes phase diffusion, reflecting the loss of phase memory induced by fluctuating voltages. In contrast, the term $-\ima\sin(\omega t)$ yields the imaginary part of $J(t)$ and encodes the phase winding generated by the environment. Altogether, these ingredients ensure that the long-time or low-frequency behavior of the environmental impedance dominates the asymptotic form of $J(t)$.

        The function $P(E)$ is defined as the Fourier transform of $J(t)$, 
        \begin{equation}
            P(E) = \frac{1}{2\pi\hbar} \int_{-\infty}^{\infty} \exp\left(J(t) + \ima E t/\hbar\right)\,\mathrm{d}t\,.
            \label{eq:stochastic:pedef}
        \end{equation}
        This formulation relies on the assumption of a linear, Gaussian environment, so that all phase fluctuations are fully captured by the correlator $J(t)$.
        The shape of $P(E)$ therefore reflects the spectral properties of the environment in a universal way.
        
        An important universal property of the $P(E)$ function is its normalization,
        \begin{equation}
            \int_{-\infty}^{\infty} P(E)\,\mathrm{d}E = 1\,,
            \label{eq:stochastic:pe-normalization}
        \end{equation}
        which follows directly from the definition in Eq.~\ref{eq:stochastic:pedef} and the condition $J(t\!\to\!0)=0$. This ensures that environmental fluctuations redistribute spectral weight among different energy-exchange channels without altering the total tunneling probability.

        In thermal equilibrium the environment additionally satisfies the detailed-balance relation
        \begin{equation}
            P(-E) = e^{-E/k_\mathrm{B}T} P(E)\,,
            \label{eq:stochastic:pe-detailedbalance}
        \end{equation}
        which guarantees thermodynamic consistency of energy exchange processes.

        The rate for a tunneling event\footnote{Throughout this work we distinguish between the Dynes broadening parameter $\gamma$, which enters the quasi-particle density of states in the microscopic tunneling description, and the tunneling rates $\Gamma$ that appear in the stochastic description of incoherent charge transfer. The two quantities are unrelated and refer to different physical mechanisms.},
        \begin{equation}
            \Gamma(V) = \int_{-\infty}^\infty P(E)\,F(E, V)\,\mathrm{d}E\,,
            \label{eq:stochastic:rate}
        \end{equation}
        is obtained from Fermi's golden rule as a convolution of $P(E)$ with the electronic part of the problem $F(E, V)$. It collects the relevant density of states and Fermi functions.
        This expression is completely general and applies to single-electron tunneling, Cooper-pair tunneling, and Andreev processes alike\footnote{The only distinction is the transferred charge $q$ in a tunneling event. The environment couples to the energy $qV$, with $q=e$ for single-electron tunneling and $q=2e$ for Cooper-pair tunneling and Andreev reflection.}.

        Several limiting forms of $P(E)$ are particularly useful for understanding the stochastic transport regime. The \textit{P(E)}-framework applies strictly to tunneling processes, where individual transfer events are well separated and described by Fermi's golden rule. It does not capture coherent multi-particle trajectories such as multiple Andreev reflections, which require a mesoscopic description and finite transparency.
        For a weak electromagnetic environment with, phase fluctuations are small and $P(E)$ becomes sharply peaked around $E=0$, approaching
        \begin{equation}
            P(E) \approx \delta(E) \qquad (\mathrm{Re}\,Z(\omega) \ll R_\mathrm{Q})\,.
            \label{eq:stochastic:pe-coherent}
        \end{equation}
        In this limit tunneling is effectively elastic and coherent transport is recovered.

        In contrast, a strong electromagnetic environment with $\mathrm{Re}\,Z(\omega)\sim R_\mathrm{Q}$ produces large phase fluctuations and a broad $P(E)$, such that tunneling events must exchange energy with the environment. This regime marks the onset of incoherent, environment-assisted charge transfer and underlies phenomena such as dynamical Coulomb blockade and incoherent Cooper-pair tunneling.

        For an Ohmic environment, the low-energy behavior of $P(E)$ exhibits a universal power law,
        \begin{equation}
            P(E) \propto E^{2R/R_\mathrm{Q} - 1} \qquad (E > 0,\ \mathrm{Re}\,Z(\omega) = R)\,,
            \label{eq:stochastic:pe-ohmic}
        \end{equation}
        which reflects the suppression of small-energy exchange by quantum fluctuations. 
        This result follows from the low-frequency limit of an environment with a frequency-independent real impedance, $\mathrm{Re}\,Z(\omega)=R$, and is therefore a property of the $P(E)$ kernel itself rather than a feature of any specific transport process.
        
        The specific consequences of these limiting forms for single-electron tunneling, Cooper-pair tunneling, and subgap Andreev processes are discussed in the subsequent sections.


    \subsection{Dynamical Coulomb Blockade}
    \label{subsec:stochastic:dcb}

        Dynamical Coulomb blockade (DCB) describes the suppression of single-electron tunneling at low bias due to the combined effects of charge quantization and the electromagnetic environment. In contrast to the microscopic description, where quasi-particle tunneling is elastic and governed solely by the electronic density of states, DCB arises when the environment possesses a sufficiently large real impedance such that the transfer of an electron across the junction requires the environment to absorb a finite amount of energy. If this energy is not available, the tunneling event is suppressed.

        The physical origin of dynamical Coulomb blockade can be understood by considering the energetic requirements of a single tunneling event. When an electron traverses the junction, it must transiently raise the voltage across the junction capacitor, which requires an electrostatic energy of the order of $E_\mathrm{C} = e^2/2C$. Because this charge transfer takes place within a quantum circuit, the required energy cannot arise from the junction itself, but must instead be supplied by the surrounding electromagnetic environment. Whether the environment can provide this energy is determined by the probability distribution $P(E)$, whose low-energy weight reflects the extent to which environmental modes can exchange small amounts of energy.
        
        If the real part of the environmental impedance is appreciable at low frequencies, the corresponding suppression of $P(E\!\approx\!0)$ makes it unlikely that the environment can provide the small energy quanta required for low-energy tunneling processes. This not only reduces the differential conductance near zero voltage but more generally suppresses all tunneling events that rely on small energy exchange with the environment. The resulting modification of the \textit{I-V} characteristics is therefore governed entirely by the low-frequency properties of the impedance. This mechanism, rooted in the discrete transfer of charge across the junction and the energetic constraints imposed by the environment, captures the universal and modelindependent essence of dynamical Coulomb blockade.
        
        Whenever the low-frequency part of the environmental impedance is Ohmic, $\mathrm{Re}\,Z(\omega)=R$ for $\omega \ll \omega_c$, the suppression of small-energy exchange produces a characteristic power-law dependence of the differential conductance,
        \begin{equation}
            \frac{\mathrm{d}I(V)}{\mathrm{d}V} \propto V^{2R/R_\mathrm{Q}} \qquad (eV \ll \hbar\omega_\mathrm{c})\,,
            \label{eq:stochastic:dcb-powerlaw}
        \end{equation}
        where $\omega_\mathrm{c}$ denotes the effective high-frequency cutoff of the environment. For a simple RC model one obtains $\omega_\mathrm{c}=1/RC$, but in general $\omega_\mathrm{c}$ is set by the fastest environmental mode for which the Ohmic approximation remains valid. This ''zero-bias anomaly'' directly reflects the low-energy behavior of $P(E)$ and constitutes the most prominent experimental signature of DCB. While Eq.~\ref{eq:stochastic:dcb-powerlaw} applies to a purely Ohmic environment, a zero-bias suppression of the conductance is a generic feature whenever the low-frequency part of the environmental impedance provides dissipation, i.e.\ whenever $\mathrm{Re}\,Z(\omega\to 0)>0$. In this case $P(E\to 0)\to 0$ and a zero-bias anomaly emerges with an exponent determined by the low-frequency behavior of $\mathrm{Re}\,Z(\omega)$.

        For environments with a real impedance that exceeds $R_\mathrm{Q}$ at low frequencies the tunneling electron cannot draw the small amounts of energy needed for charge transfer. The conductance then becomes exponentially small at low bias. In this strong damping limit the charge remains localized on the junction capacitor and the junction behaves as an effective insulator.

        DCB is not restricted to normal-metal junctions. The expression for the tunneling current remains
        \begin{equation}
            I(V) \propto \int_{-\infty}^\infty P(E)\, \frac{N_1(E)}{N_0}\, \frac{N_2(E+eV)}{N_0}\, \left(f_1(E)-f_2(E+eV)\right)\,\mathrm{d}E\,.
            \label{eq:stochastic:dcb-current}
        \end{equation}
        but the electronic factor inherits the density of states of the electrodes. For NN junctions the density of states is constant, for NS junctions it contains the superconducting BCS form on one side, and for SS junctions both electrodes contribute superconducting densities of states. Regardless of the microscopic details, the environment always suppresses single-electron tunneling at low bias through the same mechanism.
        In SIS junctions the single-electron current appears only above the pair-breaking threshold $eV \gtrsim 2\Delta$, but once quasi-particle states are available, the environmental suppression of small-energy exchange acts on the onset in the same universal manner.

        We now apply this general formalism to single-electron tunneling before turning to the corresponding two-electron processes.


        \subsubsection*{Dynamical Coulomb Blockade with Environmental Resonances}

            If the impedance contains discrete resonances that originate from inductive elements or cavity modes, the real part of the impedance acquires sharp features at the corresponding frequencies. These features imprint themselves onto the $P(E)$ function, which develops peaks at the energies of the environmental modes. The resulting $I(V)$ curves contain satellite structures at the same energies. These resonant features arise purely from the structure of the electromagnetic environment and should not be confused with photon-assisted dynamical Coulomb blockade, discussed below, which requires an externally applied microwave drive and produces sidebands at integer multiples of the drive frequency.

            A discrete environmental resonance at frequency $\omega_0$ produces a characteristic structure in the $P(E)$ function. The oscillatory contribution to the phase correlator leads to a Poisson series of sidebands in the energy-exchange probability,  
            \begin{equation}
                P(E) = e^{-\alpha_0} \sum_{n=0}^{\infty} \frac{\alpha_0^{\,n}}{n!}\, P_{\mathrm{Ohmic}}(E - n\hbar\omega_0)\,,
            \end{equation}
            where $\alpha_0 = R/R_\mathrm{Q}$ quantifies the coupling strength to the mode. Each term represents the absorption or emission of $n$ quanta of the environmental resonance. The measurable current inherits the same structure through the convolution with the electronic factor,  
            \begin{equation}
                I(V) = e^{-\alpha_0} \sum_{n=0}^{\infty} \frac{\alpha_0^{\,n}}{n!}\, I_0\!\left(V - \frac{n\hbar\omega_0}{e}\right),
            \end{equation}
            so that satellite peaks appear at voltages shifted by $n\hbar\omega_0/e$. These features arise solely from the internal mode of the environment and do not require an external microwave drive.


        \subsubsection*{Photon-Assisted Dynamical Coulomb Blockade}

            In the presence of microwave irradiation, a tunneling electron may absorb or emit integer multiples of the photon energy $h\nu$. In the stochastic regime this photon-assisted tunneling does not arise from a coherent phase modulation as in the Tien--Gordon description, but instead from additional energy channels in the inelastic $P(E)$-process.  
            In this regime the microwave field is treated as a classical, deterministic voltage modulation superposed on the stochastic environmental fluctuations; no phase-coherent mixing between sidebands occurs. The classical treatment of the microwave field is valid whenever the applied drive contains many photons per cycle, such that quantum fluctuations of the field are negligible compared to the deterministic modulation.

            A tunneling event may therefore exchange an energy $nh\nu$, with both the environment and the microwave field. The corresponding probability is obtained by dressing the environmental
            probability with Bessel weights,
            \begin{equation}
                P(E) = \sum_{n=-\infty}^{\infty} J_n^2\left(\frac{eA}{h\nu}\right)\, P_0(E - nh\nu)\,,
                \label{eq:stochastic:pe-padcb}
            \end{equation}
            where $A$ and $\nu$ are the amplitude and frequency of the applied microwave drive.

            The resulting photon-assisted DCB current generalizes Eq.~\ref{eq:stochastic:dcb-current} by incorporating the additional photon sidebands,
            \begin{equation}
                I(V_0) = \sum_{n=-\infty}^{\infty} J_n^2\!\left( \frac{eA}{h\nu}\right) \cdot I_0\!\left(V_0 - \frac{n h\nu}{e}\right)\,.
                \label{eq:stochastic:padcb}
            \end{equation}
            Photon-assisted DCB therefore combines environmental energy exchange with photon-assisted processes in a fully incoherent manner. In contrast to coherent PAT in the microscopic Tien--Gordon picture, no phase-coherent sideband mixing occurs. Instead, microwaves redistribute weight among the inelastic channels of the $P(E)$ kernel. This produces microwave-induced replicas of the DCB-suppressed current without generating coherent Shapiro-like features.

            DCB constitutes the single-electron $q=e$ counterpart of incoherent Cooper-pair tunneling, which is discussed next. Whereas DCB suppresses the tunneling of individual electrons, the same environmental interaction enables incoherent $q=2e$ charge transfer in superconducting junctions. This forms the basis of incoherent Cooper-pair tunneling, discussed in the following section.


    \subsection{Incoherent Two-Electron Tunneling}
    \label{subsec:stochastic:2e}

        In the stochastic regime, the superconducting phase becomes fully randomized by the electromagnetic environment, and any phase-coherent mechanism—Josephson tunneling, Andreev bound states, and multiple Andreev reflections—is suppressed. Nevertheless, the transfer of a charge $(2e)$ across a voltage-biased junction remains possible through inelastic, stochastic tunneling events. These processes form a unified class of incoherent two-electron tunneling, in which the junction exchanges an energy $(2e)V$ with its environment. Their rates are governed by the same $P(E)$ kernel as in dynamical Coulomb blockade, but evaluated at the doubled energy scale associated with the transferred charge.

        Despite their different microscopic origins, incoherent Cooper-pair tunneling (SS junctions) and incoherent Andreev reflection (NS junctions) share the same universal structure:
        \begin{equation}
            \Gamma_{2e}(V) = \int_{-\infty}^{\infty} P(E)\,F_{2e}(E,V)\,\mathrm{d}E\,,
            \label{eq:stochastic:2e-general}
        \end{equation}
        with the environment coupling to the charge $(2e)$ through the argument of the $P(E)$ function. The only distinction lies in the electronic prefactor $F_{2e}(E,V)$, which encodes the microscopic structure of the junction.
        
        The stochastic regime treats all $(2e)$ processes on equal footing: both incoherent Cooper-pair tunneling in SS junctions and incoherent Andreev reflection in NS junctions are described by the same charge-$(2e)$ energy-exchange kernel $P(E)$, with only their electronic prefactors differentiating them. The following subsections describe these two cases in detail.

        \subsubsection*{Incoherent Cooper-Pair Tunneling}

            In a SS tunnel junction the coherent Josephson effect is destroyed once environmental phase fluctuations become strong enough that the phase-dependent part of the Cooper-pair tunneling amplitude averages to zero
            \begin{equation}
                \left\langle \exp(\ima\phi(t))\right\rangle = 0\,.
            \end{equation}
            The Josephson coupling energy $E_\mathrm{J}=\hbar I_\mathrm{C}/2e$ then no longer produces a DC supercurrent but enters the tunneling rate in second order. Since a Cooper pair of charge $(2e)$ is transferred between the two condensates, the corresponding electronic factor takes the simple form
            \begin{equation}
                F_\mathrm{ICPT}(E,V) = \frac{\pi E_\mathrm{J}^{2}}{2\hbar} \left(\delta(E-2eV) - \delta(E+2eV)\right)\,,
                \label{eq:stochastic:icpt-F2e}
            \end{equation}
            which, inserted into Eq.~\ref{eq:stochastic:2e-general}, yields the standard expression for the incoherent Cooper-pair tunneling rate and by multiplying by the transferred charge gives the ICPT current,
            \begin{equation}
                I_\mathrm{ICPT}(V)
                = \frac{\pi E_\mathrm{J}^{2}}{\hbar}
                \left(1 - \exp(-2eV/k_\mathrm{B}T)\right) P(2eV)\,.
                \label{eq:stochastic:icpt-current}
            \end{equation}

            The low-bias suppression of $I_\mathrm{ICPT}(V)$ is therefore entirely governed by the behavior of $P(E)$ near $E=0$. For Ohmic environments this leads to the universal power law $I\propto V^{2R/R_\mathrm{Q}-1}$, identical to the dynamical Coulomb blockade of single-electron tunneling but evaluated at the doubled charge $(2e)$.


        \subsubsection*{Incoherent Andreev Reflection}

            At an NS interface, subgap transport is governed microscopically by Andreev reflection: an incoming electron from the normal electrode is retroreflected as a hole, while a Cooper pair is injected into the superconductor. In the coherent BTK description this process relies on well-defined electron-hole amplitudes. Once environmental phase fluctuations destroy coherence, Andreev reflection remains possible but becomes a stochastic, inelastic two-electron tunneling process.

            In the tunneling limit, the coherent BTK structure collapses to a single effective second-order tunneling parameter, and the electronic factor becomes
            \begin{equation}
                F_\mathrm{IAR}(E,V) = \frac{\tau^2 G_0}{e^{2}} \left(f(E-eV) - f(E+eV)\right)\,,
                \label{eq:stochastic:iar-F2e}
            \end{equation}
            where $f(E)$ is the Fermi function of the normal electrode. Inserting Eq.~\ref{eq:stochastic:iar-F2e} into the general 2e formula \ref{eq:stochastic:2e-general} results in the corresponding current,
            \begin{equation}
                I_\mathrm{IAR}(V) = \frac{2\tau^2 G_0}{e} \int_{-\infty}^{\infty} P(E)\,\left(f(E-eV)-f(E+eV)\right)\,\mathrm{d}E\,.
                \label{eq:stochastic:iar-current}
            \end{equation}

            As in ICPT, low-bias IAR is suppressed by $P(E\to 0)$. The only difference between IAR and ICPT lies in the electronic factor: the normal metal provides the Fermi functions $f(E\pm eV)$, whereas ICPT involves only the condensate wave functions encoded in $E_\mathrm{J}^{2}$.

            Finally, it is important to clarify why neither coherent Andreev reflection nor multiple Andreev reflections survive in a SS junction once the system enters the stochastic regime. In an SS junction both electrodes possess gapped BCS densities of states, so no normal-metal continuum is available to support elastic electron-hole conversion inside the gap. Coherent Andreev reflection relies on well-defined electron and hole amplitudes with a fixed superconducting phase, and multiple Andreev reflections require an extended sequence of such coherent conversions. Strong environmental phase fluctuations destroy this phase coherence, making the amplitudes of successive conversion events uncorrelated. As a result, MAR processes are fully suppressed, and even single Andreev reflection has no independent meaning in SS junctions. The only remaining subgap transport channel is therefore the incoherent transfer of a Cooper pair between the two condensates, described by incoherent Cooper-pair tunneling (ICPT).


        \subsubsection*{Photon-Assisted Incoherent Two-Electron Tunneling}

            When a microwave drive is applied, the phase remains fully randomized by the environment, so no Shapiro-like coherent interference occurs. Instead, the drive opens additional inelastic channels. The probability to exchange energy with both environment and microwave field becomes
            \begin{equation}
                P(E) = \sum_{n=-\infty}^{\infty} J_{n}^{2}\!\left(\frac{2eA}{h\nu}\right)P_{0}(E-nh\nu).
                \label{eq:stochastic:pe-pa2e}
            \end{equation}

            Inserting Eq.~\ref{eq:stochastic:pe-pa2e} into the 2e tunneling rate yields the photon-assisted current,
            \begin{equation}
                I_{2e}(V_{0}) = \sum_{n=-\infty}^{\infty} J_{n}^{2}\!\left(\frac{2eA}{h\nu}\right)\, I_{2e,0}\!\left(V_{0}-\frac{nh\nu}{2e}\right),
                \label{eq:stochastic:pa2e}
            \end{equation}
            where $I_{2e,0}(V)$ is either Eqs.~\ref{eq:stochastic:icpt-current} (ICPT) or \ref{eq:stochastic:iar-current} (IAR), depending on the junction type. Thus photon-assisted ICPT and photon-assisted IAR share the same structure, only their electronic prefactors differ.


    \subsection{Superconducting Single-Electron Transistor}
    \label{subsec:stochastic:sset}

        The superconducting single-electron transistor (SSET) combines two fundamental ingredients of mesoscopic charge transport: Coulomb blockade due to charge quantization on a small island, and superconductivity in the source, drain, and island electrodes. To understand its transport characteristics, it is instructive to first recall the basic concepts of static Coulomb blockade and the operation of a normal-state single-electron transistor (SET), before discussing the qualitative differences that arise once all electrodes are superconducting.

        \subsubsection*{Static Coulomb Blockade}

            When a metallic island is isolated by tunnel junctions of capacitance $C_1$ and $C_2$, adding an excess electron requires the electrostatic charging energy 
            \begin{equation}
                E_\mathrm{C} = e^2/(2C_\Sigma)\,, \qquad C_\Sigma = C_1 + C_2 + C_\mathrm{G}\,
            \end{equation}
            where $C_\mathrm{G}$ denotes the gate capacitance. 

            For the static Coulomb blockade, one need to suppress both, thermal and charge fluctuation.
            Thermal fluctuation are suppressed, when the charging energy exceeds the thermal energy $E_\mathrm{C} \gg k_\mathrm{B}T$. Charge Fluctuations are considered to be suppressed usually, when the normal state tunneling resistance exceep the resistance quantum $R_\mathrm{T} \gg R_Q$.
            Only when both criteria are met does the island retain a well-defined integer charge. Adding or removing an electron is both energetically unfavorable and quantum-mechanically unlikely. In this regime charge becomes localized on the island.

            This suppression of charge motion is known as static Coulomb blockade. The island is then characterized by well-defined charge states $n e$ and its electrostatic energy
            \begin{equation}
                E_n(V_\mathrm{G}) = E_\mathrm{C} (n - n_\mathrm{G})^2,
            \end{equation}
            with $n_\mathrm{G} = C_\mathrm{G}V_\mathrm{G}/e$ the dimensionless gate-induced charge. Degeneracy between two charge states occurs at half-integer values of $n_\mathrm{G}$.

        \subsubsection*{Normal-State SET}

            A normal-state single-electron transistor consists of two tunnel junctions in series, each with tunneling resistance $R_{T,i}$ and capacitance $C_i$, and a metallic island between them.  Transport occurs through sequential single-electron transitions $n\!\rightarrow n\pm1$ whenever the electrostatic energy decreases during the tunneling event.

            Instead of writing the general condition $\Delta E<0$, it is more transparent to express the onset of conduction directly through the Coulomb-diamond boundaries.  The island has electrostatic energy $E_n = E_C (n - n_g)^2$, and a tunneling transition becomes energetically allowed when the applied bias provides enough energy to overcome the charging cost. This yields the diamond edges
            \begin{equation}
                |eV| = 2E_C\,\left|\,2n_g - (2n+1)\,\right|,
            \end{equation}
            which delimit regions of blocked and allowed transport in the $(V,V_g)$ plane.  Inside these diamonds no sequential tunneling is possible and the SET is in Coulomb blockade; on the diamond edges, the charge states $n$ and $(n\pm1)$ become degenerate, giving rise to conductance peaks; and outside the diamonds sequential single-electron tunneling provides a finite conductance.

            Outside the Coulomb-blockaded region the SET behaves as two independent tunnel junctions in series. The sequential-tunneling conductance therefore approaches $G_{\mathrm{seq}} = (R_{T,1}+R_{T,2})^{-1}$, which reduces to $G_T/2$ for symmetric junctions.

            Away from the degeneracy points, only higher-order cotunneling processes remain, and the conductance is strongly suppressed.

        \subsubsection*{Superconducting SET (SSET): Why It Is Special}

            When the leads and the island become superconducting, the SET acquires qualitatively new transport channels:
            \begin{enumerate}
                \item \textbf{Cooper-pair tunneling} allows $2e$ charge transfer between neighboring charge states $n\!\to\!n\pm2$. In the absence of phase coherence (as treated in this chapter), these processes occur as incoherent Cooper-pair tunneling (ICPT) with rates governed by the same $P(E)$ kernel that enters single-junction ICPT.
                \item \textbf{Quasiparticle tunneling} becomes possible only above the superconducting energy gap. Single-electron transitions $n\!\to\!n\pm1$ require an energy cost of at least $\Delta$ per broken pair, modifying the Coulomb blockade diagram into superconducting Coulomb diamonds.
                \item \textbf{Parity effects} arise because the island prefers to host an even number of electrons. Near charge degeneracy, odd charge states acquire an additional energy cost of $\Delta$, which modifies the periodicity of the stability diagram and the accessible charge states at low temperature.
            \end{enumerate}
            Together, these ingredients produce a device whose transport is governed not simply by sequential tunneling, but by competing single-electron and two-electron processes, each dressed by the electromagnetic environment through stochastic energy exchange. The SSET therefore naturally combines all mechanisms introduced in this chapter: DCB of quasiparticles, incoherent Cooper-pair tunneling, and incoherent Andreev processes at the NS interfaces between island and leads.

        \subsubsection*{Transport Cycles in the SSET}

            The coexistence of quasiparticle and Cooper-pair processes gives rise to distinct transport cycles, the most prominent being the Josephson--quasiparticle (JQP) cycle, the double Josephson--quasiparticle (DJQP) cycle, and subgap cycles involving incoherent Andreev reflection. These cycles arise when a sequence of energetically allowed transitions forms a closed loop in the $(n,V)$ plane, enabling a steady-state current through the device. In the stochastic regime, the rates of the Cooper-pair steps are governed by ICPT, the quasiparticle steps follow the DCB-modified single-electron tunneling expression, and the cycle current is set by the slowest transition.

            A quantitative description is obtained by writing rate equations for the charge-state occupations and computing the steady-state current, discussed in the following sections.

        \subsubsection*{Coherent Multiple Andreev Reflections in SSETs}

In addition to incoherent processes described by the $P(E)$-framework, 
superconducting single-electron transistors can access a qualitatively 
different regime in which transport is governed by coherent multiple 
Andreev reflections (MAR). This regime appears when the electromagnetic 
environment is weak and the junction transparencies are sufficiently 
large so that the superconducting phase remains well defined over many 
successive electron-hole conversion events.

In a superconducting weak link with transmission $\tau\lesssim 1$, an 
electron incident at subgap energy undergoes repeated Andreev reflections 
at the two NS interfaces. These coherent trajectories gain energy in 
steps of $eV$ and allow charge transfer at bias voltages 
$eV<2\Delta$. The resulting MAR current produces characteristic 
subharmonic gap structures at
\[
    eV = \frac{2\Delta}{n}, \qquad n = 1,2,3,\ldots
\]
which appear as step-like onsets in the current-voltage characteristics 
and as pronounced lines of enhanced conductance in the SSET stability diagram.

\subsubsection*{Experimental observations (Lorenz and Sprenger).}
Coherent MAR in aluminium SSETs was first observed in detail by 
Lorenz~\cite{LorenzPhD} and Sprenger~\cite{SprengerPhD}. Their devices 
featured moderately transparent tunnel junctions and an electromagnetic 
environment with low dissipation, allowing the phase to remain coherent 
over several Andreev cycles. Both theses report clear subharmonic gap 
structure, MAR-induced steps in the $I(V)$ characteristics, and coherent 
interference features indicative of well-preserved superconducting 
phase dynamics.

\subsubsection*{Theoretical and hybrid-regime simulations (Sobral-Rey).}
More recently, Sobral-Rey extended this picture by studying the 
interplay of MAR and Coulomb blockade in hybrid SSET devices in a regime 
that interpolates between pure tunneling and coherent Andreev transport. 
Her simulations and measurements~\cite{SobralReyPRL2024} demonstrate 
how MAR survives in the presence of charging effects, how Coulomb 
blockade modifies the MAR subharmonic structure, and how the onset of 
higher-order MAR processes depends sensitively on junction transparency, 
parity effects, and the detuning from charge degeneracy. These results 
show that MAR remains robust well beyond the simplest weak-coupling 
limit and that charging energy and superconductivity can coexist in a 
nontrivial way in the intermediate-transparency regime.

\subsubsection*{Relation to the stochastic regime.}
Coherent MAR processes rely on a well-defined superconducting phase and 
on the correlated nature of successive electron-hole conversions. 
Therefore, they lie fundamentally outside the applicability of the 
$P(E)$-framework, which assumes statistically independent tunneling 
events. As environmental damping increases or the junction transparency 
is reduced, phase fluctuations destroy this coherence and MAR gradually 
gives way to incoherent two-electron processes (ICPT in SS junctions and 
IAR in NS junctions). MAR therefore marks the opposite, mesoscopic 
extreme of superconducting transport compared to the stochastic regime 
discussed in this chapter.