
    \section{Data Acquisition}
        In this section i want to fucus on the used electronics. I scheme the measurement principle and the work on grounding. I'll also talk about the self-developed measurement program and go through the written drivers. A great deal of developing is done by Florian Kayatz, within his Hiwi (cite: Flo). Finally I will also talk about the data treatment and processing.

        \subsection{DC Measurement Concept}
            As mentioned in Subsection \ref{subsec:dc cabling} a 4-point measurement principle is used. Given, by the former described setup, tiny voltages should be measurable. Instead an analog to digital converter ADwin Gold 2, in combination with two femto preamplifier, is used. 
            
            Previosly for SSET experiments at the Scheer2 cryostat a combintation of NF, SRT and a similar ADC (by NI) was used by References \cite{lorenz_wechselspiel_2018, sprenger_interplay_2019,sobral-rey_interplay_2022}. On the Bluefors a combination of Femto amplifier and keysight multimeter was used by Reference \cite{prestel_magnetic_2020}.

            All combinations of DAC Keysight1, Keysight2 and Adwin Gold 2 with Preamplifiers Femto, NF und SRT has been tested, regarding the electrical noise. I found, that the combination of ADwin Gold 2 and Femto preamplifier provided the lowest noise floor. This combination offers some further advantages. The Femto pre-amplifiers has a high bandwidth and still works analog opposite to SRT pre-amplifier. Also, it can be remote controlled, what is super beneficial, when operating with a script or not on-sight for different amplification settings. The Femto amplifier can be operated with 4 different amplifications: 10, 100, 1000, 10000. The DAC measures continously, with high bandwidth and on two channels simultaniously in opposition of the Keysight multimeters. All this, helps to correlate the two measured channels with each other in order to reliably measure I-V curves.

            As a source, the function generator, used by References \cite{lorenz_wechselspiel_2018, sprenger_interplay_2019,sobral-rey_interplay_2022}, KeysightB, used by \cite{prestel_magnetic_2020} and the DAC of the already used ADwin. Again, the usage of ADwin was best for the lowest noise floor. However, to enhance the resolution, part of the 
            - {layout}
            - {AD converter}
            - {Pre Amplifier}

        \subsection{Measurement Software}
        \label{subsec:measurement_software}

            A central component of the experimental workflow is the self–developed
            measurement framework \texttt{p5control}, which provides a unified 
            interface for device communication, data acquisition and automated 
            measurement routines. The framework was originally created by 
            Florian Kayatz during his HiWi work in the Scheer group and has since
            been substantially extended and adapted for the requirements of
            superconducting transport experiments on the Bluefors dilution
            refrigerator. In the following, the conceptual architecture of 
            \texttt{p5control} is outlined, followed by a description of the 
            drivers and graphical user interfaces implemented in the context of 
            this thesis.

            \subsubsection*{Concept of \texttt{p5control}.}
            The design of \texttt{p5control} follows a modular and device-oriented 
            philosophy. Each instrument (voltage source, preamplifier, lock-in 
            amplifier, multimeter, temperature controller, etc.) is represented 
            as a Python class encapsulating:
            \begin{itemize}
                \item the low-level communication (SCPI, serial, socket, custom protocols),
                \item an internal state representation,
                \item device-specific safety checks and limits,
                \item and high-level convenience methods for experimental control.
            \end{itemize}

            Devices can be instantiated independently or registered in a 
            \textit{session}, which provides synchronized access, logging, and
            coordinated shutdown procedures. By decoupling communication,
            configuration and measurement logic, \texttt{p5control} enables:
            \begin{itemize}
                \item reproducible experiment scripts,
                \item simple substitution of hardware components,
                \item multi-threaded polling of device data,
                \item and integration of automated feedback routines.
            \end{itemize}

            Measurement routines are written as independent Python scripts making
            use of these device abstractions. This avoids monolithic measurement
            programs and enables a clean separation between data acquisition, 
            hardware control, analysis and plotting. The underlying philosophy is 
            that a measurement should be fully reproducible and executable 
            non-interactively, while still providing interactive GUI control for
            debugging, alignment and sample conditioning.

            \subsubsection*{Implemented drivers and GUI extensions.}
            For the experiments in this thesis, \texttt{p5control} was extended by
            several device drivers and graphical interfaces tailored to the Bluefors
            setup. The most relevant components are:
            \begin{itemize}
                \item \textbf{Femto preamplifier driver:}  
                Remote adjustment of gain (10–10000), reading of amplifier state, 
                overload detection and integration into automated measurement runs.
                Useful for switching gain dynamically in high-bias ranges.

                \item \textbf{ADwin Gold~2 driver:}  
                Controls the DAC outputs, configures ADC sampling, synchronizes 
                two-channel voltage acquisition and provides ring-buffer data 
                retrieval for high-bandwidth I--V measurements.

                \item \textbf{Keysight/NI multimeter drivers:}  
                Implement basic SCPI routines for slow, high-accuracy voltage 
                readout, used primarily for cross-checks and calibration.

                \item \textbf{SRT and NF preamplifier integration:}  
                Implemented for legacy compatibility (Scheer2 cryostat) and used
                for noise-comparison measurements.

                \item \textbf{Bluefors temperature controller driver:}  
                Provides read/write access to PID settings, heater power, and
                thermometry channels on the BF-FL DC board as well as the Lakeshore 
                372 for sample-stage temperature readout.

                \item \textbf{MW setup and antenna control:}  
                Drivers for the microwave source, attenuator settings and switch
                boxes used in photon-assisted measurements.

                \item \textbf{GUI widgets:}  
                A collection of PyQt widgets was created to allow:
                \begin{itemize}
                    \item live plotting of I--V curves,
                    \item real-time monitoring of ADwin channels,
                    \item gain switching on the Femto amplifier,
                    \item microwave power and frequency control,
                    \item and temperature monitoring.
                \end{itemize}
                The GUI is optional but extremely useful for debugging and aligning
                the junction prior to automated measurements.
            \end{itemize}

            The resulting software environment allows reproducible control over 
            all instruments required for the acquisition of high-quality I--V curves,
            MAR traces, Shapiro maps, and temperature-dependent measurements.
            Together with the evaluation scripts described in the following subsection,
            this framework forms the backbone of the data pipeline used throughout 
            this thesis.
        

        \subsection{Measurement Script}
            The automated acquisition of \textit{I--V} characteristics, MAR traces and
            photon-assisted transport maps is performed using the dedicated measurement
            script \texttt{iv\_script\_v2.py}. This script builds directly on the device
            abstractions provided by \texttt{p5control} (see
            Subsection~\ref{subsec:measurement_software}) and implements a fully
            reproducible pipeline for bias sweeps, synchronized data collection and
            metadata logging.

                    \subsubsection*{Structure and workflow.}
        The script follows a linear but modular workflow:
        \begin{enumerate}
            \item \textbf{Session initialization:}  
            All instruments required for a measurement (ADwin, Femto preamplifiers,
            microwave source, and optional temperature controller) are instantiated
            and registered in a common \texttt{p5.Session}. This ensures globally
            synchronized access, automatic shutdown behaviour and uniform logging.
            \item \textbf{Bias sweep definition:}  
            The voltage sweep is generated on the ADwin DAC using user-defined
            parameters such as start and stop voltage, step size, dwell time and
            optional overshoot or pre-biasing routines used to condition the
            junction. Additional features include symmetric sweeps, repeated sweeps
            for averaging, and embedded stabilization periods.         
            \item \textbf{Data acquisition:}  
            During the sweep the ADwin records voltages on both ADC channels at a
            high sampling rate. The Femto preamplifier outputs (amplified sample
            current) are digitized simultaneously, which allows point-wise
            correlation of the two channels and reliable reconstruction of
            \textit{I--V} curves. Optional gain switching routines can be activated
            when entering high-bias regimes.
                        \item \textbf{Metadata tracking:}  
            Every measurement run is stored together with temperature readings,
            amplifier gain, bias history, filter configuration, microwave settings,
            timestamp and device states. The script automatically generates a
            metadata dictionary that is saved alongside the numerical data.
                        \item \textbf{Saving and integrity checks:}  
            After each sweep, data is written to disk in a structured file format,
            including an internal hash for integrity control. Additional quick-look
            plots can be generated for immediate visual inspection.
        \end{enumerate}
                \subsubsection*{Advanced features.}
        The script also provides:
        \begin{itemize}
            \item \textbf{Microwave control:}  
            Automated stepping of frequency and power for Shapiro maps and
            photon-assisted MAR measurements.
                        \item \textbf{Stability checks:}  
            Monitoring of amplifier overload, noise bursts or sudden mechanical
            instabilities of the junction. Measurements are paused or repeated if
            thresholds are exceeded.           
             \item \textbf{Automated segmentation:}  
            Large sweeps can be segmented into blocks to avoid ADwin buffer
            overflows and to allow real-time analysis.
                        \item \textbf{Automated segmentation:}  
            Large sweeps can be segmented into blocks to avoid ADwin buffer
            overflows and to allow real-time analysis.           
             \item \textbf{Reproducibility mode:}  
            Using the metadata and device-state logs, any measurement can be
            reproduced exactly, including DAC timing, gain settings and MW
            configuration.
        \end{itemize}
                    
            
        \subsubsection*{Purpose within the experimental workflow.}
            The measurement script acts as the operational layer between the modular
            device drivers of \texttt{p5control} and the later evaluation routines
            described in the next subsection. Its design ensures that complex
            measurement sequences remain reproducible, transparent and robust against
            hardware-specific subtleties, forming an essential part of the data
            pipeline used in this thesis.

    \subsection{Evaluation Script}
        The raw data acquired with \texttt{iv\_script\_v2.py} is processed using a
        dedicated evaluation script that reconstructs calibrated \textit{I--V} curves,
        extracts subgap features and prepares the data for MAR and PAMAR analysis.  
        Each measurement is stored together with a complete metadata dictionary,  
        which allows the evaluation pipeline to treat every dataset in a fully  
        reproducible and hardware–independent manner.
        
        \subsubsection*{Data loading and metadata handling.}
        The evaluation script begins by parsing the numerical data and its associated
        metadata. All quantities relevant to reconstruction—Femto gain, ADC range,
        temperature, sweep direction, overshoot history, microwave settings and  
        timestamp—are automatically extracted. This ensures that the evaluation  
        procedure does not rely on implicit experimental knowledge but remains  
        entirely self–contained.
        
                \subsubsection*{Binning and resampling.}
        The ADwin records data at a fixed sampling rate that is independent of the
        actual DAC stepping speed. As a consequence, the voltage points of each sweep
        are not evenly spaced. To obtain smooth and differentiable \textit{I--V}
        curves, the script performs:
        \begin{itemize}
            \item \textbf{voltage binning:} grouping samples into narrow voltage
            intervals,
            \item \textbf{median or mean aggregation:} suppressing noise bursts or
            occasional digitizer outliers,
            \item \textbf{optional symmetric averaging:} combining forward and
            backward sweeps.
        \end{itemize}
        This procedure yields robust, reproducible voltage–current pairs that are
        suitable for MAR indexing and further quantitative analysis.

        \subsubsection*{Calibration and current reconstruction.}
        Because the ADwin measures voltage drops across the Femto preamplifier,
        the current is reconstructed using the amplifier gain stored in metadata.
        The script applies:
        \begin{itemize}
            \item gain calibration,
            \item sign correction (depending on wiring orientation),
            \item conversion to absolute current,
            \item and optional offset removal.
        \end{itemize}
        The resulting \textit{I--V} curve is then interpolated onto a uniform voltage
        grid for plotting and further processing.

                \subsubsection*{Filtering and data treatment.}
        The script implements several cleaning operations:
        \begin{itemize}
            \item removal of discontinuities associated with mechanical switching
            events in the MCBJ,
            \item spike filtering using rolling quantiles,
            \item low-pass smoothing for visualization (never applied to raw data),
            \item normalization routines for temperature or gain comparisons.
        \end{itemize}

                \subsubsection*{Extraction of MAR and photon-assisted features.}
        For MAR and PAMAR analysis, the script annotates:
        \begin{itemize}
            \item subharmonic gap features at $2\Delta/me$,
            \item photon-assisted replicas at shifted voltages
            $n h\nu / (me)$,
            \item and differential-conductance extrema used for comparison with
            simulations.
        \end{itemize}
        These annotations allow automated alignment of experimental data with the
        theoretical MAR and PAMAR frameworks described in 
        Chapter~\ref{sec:mesoscopic_theory}.
                \subsubsection*{Output and reproducibility.}
        The final processed dataset contains:
        \begin{itemize}
                    \item the cleaned and binned \textit{I--V} curve,
            \item the original raw trace,
            \item the metadata dictionary,
            \item and diagnostic quantities such as sweep speed, stability flags and
            amplifier states.
        \end{itemize}
                All processed data is written with a versioned filename, enabling complete
        reproducibility of any figure or analysis presented in this thesis.