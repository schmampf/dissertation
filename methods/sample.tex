% !TEX root = ../thesis.tex
\section{Sample Preparation}
\label{section:sample-preparation}

    Single-electron transistors (SETs) consist of two junctions forming a small island, which is capacitively coupled to a DC gate electrode. Typically, this configuration is achieved using two oxide barriers in a wire, fabricated through shadow evaporation \cite{fulton_observation_1987,henning_charging_1999,krupenin_metallic_2001}.

    In our approach, however, one oxide barrier is replaced by a mechanically controlled break junction (MCBJ). Aluminum is chosen as the material due to its superconducting properties at low temperatures \cite{lorenz_wechselspiel_2018, sprenger_interplay_2019,sobral-rey_interplay_2022}. Additionally, an AC gate or strip line is incorporated to ensure efficient and controlled coupling of microwave fields.

    This section describes work conducted together with Patrick Raif as part of his project practical\footnote{A component of the master's curriculum completed within a semester.} \cite{raif_electronic_2024}.

    The following subsections detail the sample preparation process from various perspectives. First, the conceptual framework is outlined. Next, I delve into the design choices and the rationale behind the procedural decisions. Finally, I present a step-by-step guide to the process, sharing practical tips and addressing challenges encountered in the laboratory.    

    \subsection{Concept} 
    \label{subsection:concept}

        In this subsection, I explain the general concept of the sample preparation process and its connection to the underlying physics.

        A MCBJ is implemented using a freestanding bridge-like structure on a bendable substrate. Figure 1 illustrates a schematic of the MCBJ used in this work. The substrate can be bent either by pushing a stamp from below or by pressing counter rods from above. This bending action elongates the freestanding bridge, enabling the MCBJ to form atomic contacts.

        The physical setup of the MCBJ is described in more detail in Subsection \ref{subsec:mcbj}.

        \begin{figure}
            \includegraphics{methods/sample/mcbj/image.png}
            \caption{Here comes some text}
            \label{fig:methods:mcbj}
        \end{figure}

        A polyimide (PI) layer insulates the substrate from the structures produced in the subsequent steps. More importantly, the PI layer serves as a sacrificial layer. During later processing, it is partially etched, allowing the structure to become freestanding.

        Next, two layers of positive electron-sensitive resist are applied. Electron beam lithography (EBL) is then used for precise exposure. After development, the exposed resist is dissolved, leaving the desired pattern.

        For the SET, beside the MCBJ, a second weak link is formed by creating an oxide barrier. A convenient approach for this is the shadow evaporation technique \cite{niemeyer_observation_1976,dolan_offset_1977}. This process involves two evaporation steps performed at different incident angles, which slightly offset the structure in each step. Between these steps, a controlled oxidation process can be performed without breaking the vacuum. This creates overlapping areas with a metal-oxide-metal cross-section, as illustrated in Figure \ref{fig:methods:sem}.

        \begin{figure}
            \includegraphics[width=0.4\textwidth]{methods/sample/stamps/stamps.png}
            \includegraphics[width=0.4\textwidth]{methods/sample/stamps/image.png}
            \caption{Here comes some text}
            \label{fig:methods:sem}
        \end{figure}

        Since a superconducting single-electron transistor (SSET) is planned, aluminum (Al) is selected for both evaporation steps. This results in an Al break junction (BJ) and an Al-Al2O3-Al oxide barrier, as described Thomas Lorenz \cite{lorenz_wechselspiel_2018} and Susanne Sprenger \cite{sprenger_interplay_2019}. Alternatively, the second material can be different; for instance, Al-Al2O3-Cu has been successfully realized by Laura Sobral Rey \cite{sobral-rey_interplay_2022}.

        As the final step, approximately 500 nm of the polyimide (PI) layer is etched using oxygen plasma. The homogeneous etching process creates a significant undercut, as illustrated in Figure \ref{fig:methods:sem}. This final step renders the break junction freestanding and fully functional.

        \begin{figure}
            \includegraphics[width=0.9\textwidth]{methods/sample/flowchart/image.png}
            \caption{Here comes some text}
            \label{fig:methods:flow}
        \end{figure}

    \subsection{Design} 
    \label{subsection:design}
        This subsection explains the sample preparation process in terms of design decisions. While there is no single definitive path to a finished sample, different approaches can have subtle effects. Here, I outline the decision-making process behind the individual steps.

        Bronze has been used as a substrate by many generations of doctoral students in this group. One key reason is that it requires relatively low force to bend while maintaining excellent elastic behavior. This is crucial for MCBJ, where the configuration needs to be adjusted repeatedly. Other materials, such as copper, exhibit less homogeneous bending and tend to deform primarily in the middle, directly beneath the stamp. This leads to a greater change in angle and poorer elastic properties. Additionally, copper often displays non-monotonic bending behavior, making fine adjustments in the atomic contact regime impractical. On the other hand, PI substrates\footnote{PI substrates are also called Kapton.} bend uniformly across their length but fail to achieve sufficient bending for the MCBJ to function effectively. Materials like copper-beryllium, although mechanically suitable, are toxic and therefore unsuitable for use. Another critical requirement is that the substrate material must be non-magnetic and have minimal magnetic impurities. Considering all these factors, bronze has proven to be the most suitable substrate for MCBJ applications over the years. \cite{schirm_uss_2009}
        
        A polyamide (PA) layer is applied by spin-coating onto an entire bronze wafer. Following a hard-baking process, the PA is converted into PI. Subsequently, two layers of resist are applied. The thin top layer, optimized for high-resolution lithography, ensures the resulting structure has sharp and well-defined edges. The thicker bottom layer serves a dual purpose: elevating the top layer and acting as a sacrificial layer that forms an undercut during development. Two mechanisms contribute to the formation of this undercut. First, the bottom layer receives a higher electron dose because electrons diffuse through the resist, and back-scattered electrons further enhance exposure in the bottom layer. Second, an extended time in isopropyl alcohol (IPA) after development with diluted methyl isobutyl ketone (MIBK) amplifies the undercut effect.

        EBL is performed using a Zeiss Crossbeam system, which has been modified to achieve significantly faster beam blanking. The beam blanker operates in synchronization with the beam deflection, allowing precise exposure of any desired structure within the specified field of view. Typically, a 100mum field is used for small, high-resolution structures, while a 1000mum field is employed for coarser, larger features. For structures exceeding these field sizes, the sample stage can be moved to enable exposure of multiple adjacent fields. To ensure seamless transitions between fields, an overlap is planned to facilitate stitching of the different areas.

        At the start of my doctoral studies, only one of the two available EBL programs, \texttt{Elphy Plus}  by Raith, was functional and properly supported. As a result, I had to implement the new sample design from scratch. The dimensions of the break junction (BJ) and gate are consistent with those shown in Figure \ref{fig:methods:ebeam} and as previously implemented by References \cite{lorenz_wechselspiel_2018, sprenger_interplay_2019,sobral-rey_interplay_2022}. In contrast, the design of the finger forming the oxide barrier was modified to be thicker towards its lower part. This adjustment reduces the freestanding length of the evaporation mask, thereby increasing its stability. However, an issue arose with older resist material, which tends to sag during evaporation. Consequently, the finger does not reliably reach the island during the second evaporation step, leading to potential inconsistencies in the structure.
        
        \begin{figure}
            \centering
            \import{methods/sample/ebeamdesign}{test.pdf_tex}
            \caption{blablabla.}
            \label{fig:methods:ebeam}
        \end{figure}

        The larger leads and pads are arranged and optimized to facilitate easy electrical contacting. An AC gate or stripline is included at an 8 mu distance to ensure maximum microwave irradiation. Its length is maximized while maintaining a reasonable writing time during electron beam lithography. Additionally, a shorting structure is integrated into the design to protect the sensitive tunnel barrier from static discharge. The central vertical lead is made thicker to enhance its contrast, which aids in precisely aligning the BJ with the stamp during setup.
        % The resonance frequency is about 10 GHz.           

        The evaporator used in this work features a long distance between the evaporation source and the sample. This setup results in aluminum being evaporated in a highly directional manner but also significantly reduces the effective cross-section of the deposition. The evaporator provides two methods for aluminum deposition: thermal evaporation and electron beam evaporation. In thermal evaporation, a large current heats a small crucible, causing the aluminum to evaporate. However, the crucible used for thermal evaporation is delicate and must be loaded and run empty after each use to avoid crack formation. This process is tedious, prone to errors, and offers no significant advantages over electron beam evaporation. In electron beam evaporation, a focused electron beam heats the aluminum crucible locally, allowing for a more efficient and reliable deposition process. Therefore, in contrast to References \cite{lorenz_wechselspiel_2018, sprenger_interplay_2019,sobral-rey_interplay_2022}, electron beam evaporation was used in this work.

        The oxidation step between the two evaporation processes is carried out in the load lock of the evaporator at a low oxygen pressure of approximately 3.0(2) mbar for 3 minutes \cite{sobral-rey_interplay_2022}. In the past, this step was inconsistent and difficult to reproduce. As a result, it is now closely monitored to ensure reliability \cite{sprenger_interplay_2019, conzelmann_optimierung_nodate}.

        Reactive ion etching (RIE) is performed using the \textit{PlasmaPro 80 ICP RIE system} from Oxford Instruments. Although this machine is not originally designed for homogeneous etching, acceptable uniformity can be achieved by operating it with low table radio frequency (RF) power, high inductively coupled plasma (ICP) RF power, and high chamber pressure. The ICP system generates a high-density plasma by coupling energy into the chamber via an electromagnetic field, while the RF power applied to the table controls the ion energy impacting the sample surface. A heated table further enhances the homogeneity of the etching process. The etched depth is monitored in real-time using laser interferometry, ensuring precise control over the process.

        The final step is contacting the sample. Wedge bonding is unsuitable for this purpose because the bonding wire can press through the pad into the soft PI layer. Instead, thin copper wires are attached to the pads using conductive silver paste. This method ensures reliable electrical contact without damaging the underlying structure.

        % Polished up to here.
